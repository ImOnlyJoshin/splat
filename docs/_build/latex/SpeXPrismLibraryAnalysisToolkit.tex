% Generated by Sphinx.
\def\sphinxdocclass{report}
\newif\ifsphinxKeepOldNames \sphinxKeepOldNamestrue
\documentclass[letterpaper,10pt,english]{sphinxmanual}
\usepackage{iftex}

\ifPDFTeX
  \usepackage[utf8]{inputenc}
\fi
\ifdefined\DeclareUnicodeCharacter
  \DeclareUnicodeCharacter{00A0}{\nobreakspace}
\fi
\usepackage{cmap}
\usepackage[T1]{fontenc}
\usepackage{amsmath,amssymb,amstext}
\usepackage{babel}
\usepackage{times}
\usepackage[Bjarne]{fncychap}
\usepackage{longtable}
\usepackage{sphinx}
\usepackage{multirow}
\usepackage{eqparbox}


\addto\captionsenglish{\renewcommand{\figurename}{Fig.\@ }}
\addto\captionsenglish{\renewcommand{\tablename}{Table }}
\SetupFloatingEnvironment{literal-block}{name=Listing }

\addto\extrasenglish{\def\pageautorefname{page}}

\setcounter{tocdepth}{2}


\title{SpeX Prism Library Analysis Toolkit Documentation}
\date{Jul 27, 2016}
\release{0.5}
\author{Adam Burgasser}
\newcommand{\sphinxlogo}{\sphinxincludegraphics{logo.jpg}\par}
\renewcommand{\releasename}{Release}
\makeindex

\makeatletter
\def\PYG@reset{\let\PYG@it=\relax \let\PYG@bf=\relax%
    \let\PYG@ul=\relax \let\PYG@tc=\relax%
    \let\PYG@bc=\relax \let\PYG@ff=\relax}
\def\PYG@tok#1{\csname PYG@tok@#1\endcsname}
\def\PYG@toks#1+{\ifx\relax#1\empty\else%
    \PYG@tok{#1}\expandafter\PYG@toks\fi}
\def\PYG@do#1{\PYG@bc{\PYG@tc{\PYG@ul{%
    \PYG@it{\PYG@bf{\PYG@ff{#1}}}}}}}
\def\PYG#1#2{\PYG@reset\PYG@toks#1+\relax+\PYG@do{#2}}

\expandafter\def\csname PYG@tok@kt\endcsname{\def\PYG@tc##1{\textcolor[rgb]{0.56,0.13,0.00}{##1}}}
\expandafter\def\csname PYG@tok@o\endcsname{\def\PYG@tc##1{\textcolor[rgb]{0.40,0.40,0.40}{##1}}}
\expandafter\def\csname PYG@tok@mf\endcsname{\def\PYG@tc##1{\textcolor[rgb]{0.13,0.50,0.31}{##1}}}
\expandafter\def\csname PYG@tok@ge\endcsname{\let\PYG@it=\textit}
\expandafter\def\csname PYG@tok@nc\endcsname{\let\PYG@bf=\textbf\def\PYG@tc##1{\textcolor[rgb]{0.05,0.52,0.71}{##1}}}
\expandafter\def\csname PYG@tok@sh\endcsname{\def\PYG@tc##1{\textcolor[rgb]{0.25,0.44,0.63}{##1}}}
\expandafter\def\csname PYG@tok@kd\endcsname{\let\PYG@bf=\textbf\def\PYG@tc##1{\textcolor[rgb]{0.00,0.44,0.13}{##1}}}
\expandafter\def\csname PYG@tok@cp\endcsname{\def\PYG@tc##1{\textcolor[rgb]{0.00,0.44,0.13}{##1}}}
\expandafter\def\csname PYG@tok@gp\endcsname{\let\PYG@bf=\textbf\def\PYG@tc##1{\textcolor[rgb]{0.78,0.36,0.04}{##1}}}
\expandafter\def\csname PYG@tok@mo\endcsname{\def\PYG@tc##1{\textcolor[rgb]{0.13,0.50,0.31}{##1}}}
\expandafter\def\csname PYG@tok@cm\endcsname{\let\PYG@it=\textit\def\PYG@tc##1{\textcolor[rgb]{0.25,0.50,0.56}{##1}}}
\expandafter\def\csname PYG@tok@nb\endcsname{\def\PYG@tc##1{\textcolor[rgb]{0.00,0.44,0.13}{##1}}}
\expandafter\def\csname PYG@tok@sd\endcsname{\let\PYG@it=\textit\def\PYG@tc##1{\textcolor[rgb]{0.25,0.44,0.63}{##1}}}
\expandafter\def\csname PYG@tok@ss\endcsname{\def\PYG@tc##1{\textcolor[rgb]{0.32,0.47,0.09}{##1}}}
\expandafter\def\csname PYG@tok@s1\endcsname{\def\PYG@tc##1{\textcolor[rgb]{0.25,0.44,0.63}{##1}}}
\expandafter\def\csname PYG@tok@ch\endcsname{\let\PYG@it=\textit\def\PYG@tc##1{\textcolor[rgb]{0.25,0.50,0.56}{##1}}}
\expandafter\def\csname PYG@tok@mh\endcsname{\def\PYG@tc##1{\textcolor[rgb]{0.13,0.50,0.31}{##1}}}
\expandafter\def\csname PYG@tok@c\endcsname{\let\PYG@it=\textit\def\PYG@tc##1{\textcolor[rgb]{0.25,0.50,0.56}{##1}}}
\expandafter\def\csname PYG@tok@gu\endcsname{\let\PYG@bf=\textbf\def\PYG@tc##1{\textcolor[rgb]{0.50,0.00,0.50}{##1}}}
\expandafter\def\csname PYG@tok@nl\endcsname{\let\PYG@bf=\textbf\def\PYG@tc##1{\textcolor[rgb]{0.00,0.13,0.44}{##1}}}
\expandafter\def\csname PYG@tok@ow\endcsname{\let\PYG@bf=\textbf\def\PYG@tc##1{\textcolor[rgb]{0.00,0.44,0.13}{##1}}}
\expandafter\def\csname PYG@tok@k\endcsname{\let\PYG@bf=\textbf\def\PYG@tc##1{\textcolor[rgb]{0.00,0.44,0.13}{##1}}}
\expandafter\def\csname PYG@tok@c1\endcsname{\let\PYG@it=\textit\def\PYG@tc##1{\textcolor[rgb]{0.25,0.50,0.56}{##1}}}
\expandafter\def\csname PYG@tok@nd\endcsname{\let\PYG@bf=\textbf\def\PYG@tc##1{\textcolor[rgb]{0.33,0.33,0.33}{##1}}}
\expandafter\def\csname PYG@tok@sr\endcsname{\def\PYG@tc##1{\textcolor[rgb]{0.14,0.33,0.53}{##1}}}
\expandafter\def\csname PYG@tok@sx\endcsname{\def\PYG@tc##1{\textcolor[rgb]{0.78,0.36,0.04}{##1}}}
\expandafter\def\csname PYG@tok@s\endcsname{\def\PYG@tc##1{\textcolor[rgb]{0.25,0.44,0.63}{##1}}}
\expandafter\def\csname PYG@tok@mb\endcsname{\def\PYG@tc##1{\textcolor[rgb]{0.13,0.50,0.31}{##1}}}
\expandafter\def\csname PYG@tok@nf\endcsname{\def\PYG@tc##1{\textcolor[rgb]{0.02,0.16,0.49}{##1}}}
\expandafter\def\csname PYG@tok@il\endcsname{\def\PYG@tc##1{\textcolor[rgb]{0.13,0.50,0.31}{##1}}}
\expandafter\def\csname PYG@tok@vc\endcsname{\def\PYG@tc##1{\textcolor[rgb]{0.73,0.38,0.84}{##1}}}
\expandafter\def\csname PYG@tok@sb\endcsname{\def\PYG@tc##1{\textcolor[rgb]{0.25,0.44,0.63}{##1}}}
\expandafter\def\csname PYG@tok@nt\endcsname{\let\PYG@bf=\textbf\def\PYG@tc##1{\textcolor[rgb]{0.02,0.16,0.45}{##1}}}
\expandafter\def\csname PYG@tok@nv\endcsname{\def\PYG@tc##1{\textcolor[rgb]{0.73,0.38,0.84}{##1}}}
\expandafter\def\csname PYG@tok@gs\endcsname{\let\PYG@bf=\textbf}
\expandafter\def\csname PYG@tok@gh\endcsname{\let\PYG@bf=\textbf\def\PYG@tc##1{\textcolor[rgb]{0.00,0.00,0.50}{##1}}}
\expandafter\def\csname PYG@tok@w\endcsname{\def\PYG@tc##1{\textcolor[rgb]{0.73,0.73,0.73}{##1}}}
\expandafter\def\csname PYG@tok@kp\endcsname{\def\PYG@tc##1{\textcolor[rgb]{0.00,0.44,0.13}{##1}}}
\expandafter\def\csname PYG@tok@ne\endcsname{\def\PYG@tc##1{\textcolor[rgb]{0.00,0.44,0.13}{##1}}}
\expandafter\def\csname PYG@tok@se\endcsname{\let\PYG@bf=\textbf\def\PYG@tc##1{\textcolor[rgb]{0.25,0.44,0.63}{##1}}}
\expandafter\def\csname PYG@tok@vg\endcsname{\def\PYG@tc##1{\textcolor[rgb]{0.73,0.38,0.84}{##1}}}
\expandafter\def\csname PYG@tok@mi\endcsname{\def\PYG@tc##1{\textcolor[rgb]{0.13,0.50,0.31}{##1}}}
\expandafter\def\csname PYG@tok@ni\endcsname{\let\PYG@bf=\textbf\def\PYG@tc##1{\textcolor[rgb]{0.84,0.33,0.22}{##1}}}
\expandafter\def\csname PYG@tok@cs\endcsname{\def\PYG@tc##1{\textcolor[rgb]{0.25,0.50,0.56}{##1}}\def\PYG@bc##1{\setlength{\fboxsep}{0pt}\colorbox[rgb]{1.00,0.94,0.94}{\strut ##1}}}
\expandafter\def\csname PYG@tok@gd\endcsname{\def\PYG@tc##1{\textcolor[rgb]{0.63,0.00,0.00}{##1}}}
\expandafter\def\csname PYG@tok@err\endcsname{\def\PYG@bc##1{\setlength{\fboxsep}{0pt}\fcolorbox[rgb]{1.00,0.00,0.00}{1,1,1}{\strut ##1}}}
\expandafter\def\csname PYG@tok@kc\endcsname{\let\PYG@bf=\textbf\def\PYG@tc##1{\textcolor[rgb]{0.00,0.44,0.13}{##1}}}
\expandafter\def\csname PYG@tok@si\endcsname{\let\PYG@it=\textit\def\PYG@tc##1{\textcolor[rgb]{0.44,0.63,0.82}{##1}}}
\expandafter\def\csname PYG@tok@na\endcsname{\def\PYG@tc##1{\textcolor[rgb]{0.25,0.44,0.63}{##1}}}
\expandafter\def\csname PYG@tok@no\endcsname{\def\PYG@tc##1{\textcolor[rgb]{0.38,0.68,0.84}{##1}}}
\expandafter\def\csname PYG@tok@bp\endcsname{\def\PYG@tc##1{\textcolor[rgb]{0.00,0.44,0.13}{##1}}}
\expandafter\def\csname PYG@tok@kn\endcsname{\let\PYG@bf=\textbf\def\PYG@tc##1{\textcolor[rgb]{0.00,0.44,0.13}{##1}}}
\expandafter\def\csname PYG@tok@gr\endcsname{\def\PYG@tc##1{\textcolor[rgb]{1.00,0.00,0.00}{##1}}}
\expandafter\def\csname PYG@tok@s2\endcsname{\def\PYG@tc##1{\textcolor[rgb]{0.25,0.44,0.63}{##1}}}
\expandafter\def\csname PYG@tok@cpf\endcsname{\let\PYG@it=\textit\def\PYG@tc##1{\textcolor[rgb]{0.25,0.50,0.56}{##1}}}
\expandafter\def\csname PYG@tok@gt\endcsname{\def\PYG@tc##1{\textcolor[rgb]{0.00,0.27,0.87}{##1}}}
\expandafter\def\csname PYG@tok@vi\endcsname{\def\PYG@tc##1{\textcolor[rgb]{0.73,0.38,0.84}{##1}}}
\expandafter\def\csname PYG@tok@kr\endcsname{\let\PYG@bf=\textbf\def\PYG@tc##1{\textcolor[rgb]{0.00,0.44,0.13}{##1}}}
\expandafter\def\csname PYG@tok@m\endcsname{\def\PYG@tc##1{\textcolor[rgb]{0.13,0.50,0.31}{##1}}}
\expandafter\def\csname PYG@tok@go\endcsname{\def\PYG@tc##1{\textcolor[rgb]{0.20,0.20,0.20}{##1}}}
\expandafter\def\csname PYG@tok@sc\endcsname{\def\PYG@tc##1{\textcolor[rgb]{0.25,0.44,0.63}{##1}}}
\expandafter\def\csname PYG@tok@gi\endcsname{\def\PYG@tc##1{\textcolor[rgb]{0.00,0.63,0.00}{##1}}}
\expandafter\def\csname PYG@tok@nn\endcsname{\let\PYG@bf=\textbf\def\PYG@tc##1{\textcolor[rgb]{0.05,0.52,0.71}{##1}}}

\def\PYGZbs{\char`\\}
\def\PYGZus{\char`\_}
\def\PYGZob{\char`\{}
\def\PYGZcb{\char`\}}
\def\PYGZca{\char`\^}
\def\PYGZam{\char`\&}
\def\PYGZlt{\char`\<}
\def\PYGZgt{\char`\>}
\def\PYGZsh{\char`\#}
\def\PYGZpc{\char`\%}
\def\PYGZdl{\char`\$}
\def\PYGZhy{\char`\-}
\def\PYGZsq{\char`\'}
\def\PYGZdq{\char`\"}
\def\PYGZti{\char`\~}
% for compatibility with earlier versions
\def\PYGZat{@}
\def\PYGZlb{[}
\def\PYGZrb{]}
\makeatother

\renewcommand\PYGZsq{\textquotesingle}

\begin{document}

\maketitle
\tableofcontents
\phantomsection\label{index::doc}


SPLAT is a python-based spectral access and analysis package designed to interface
with the SpeX Prism Library (SPL: \url{http://www.browndwarfs.org/spexprism}),
an online repository of over
1500 low-resolution, near-infrared spectra of low-temperature stars and brown dwarfs.
It is built on common python packages such as astropy, matplotlib, numpy, pandas and scipy.
\begin{description}
\item[{SPLAT tools allow you to:}] \leavevmode\begin{itemize}
\item {} 
search the SPL for data and source information;

\item {} 
access the publically-available (published) spectra contained in it;

\item {} 
compare your near-infrared spectrum to these data;

\item {} 
make use of published empirical trends in absolute magnitudes and effective temperatures;

\item {} 
perform basic spectral analyses such as spectral classification, gravity classification, index measurement, spectrophotometry, reddening, robust comparison statistics, basic math operations;

\item {} 
perform advanced analyses such as MCMC spectral model fitting;

\item {} 
transform observables using empirical trends;

\item {} 
transform observable to physical parameters using evolutionary models; and

\item {} 
plot/tabulate/publish your results.

\end{itemize}

\end{description}

\textbf{Note that many of these features are currently under development.}


\chapter{Installation and Dependencies}
\label{index:installation-and-dependencies}\label{index:splat-the-spex-prism-library-analysis-toolkit}
SPLAT is best forked from the github site \url{http://github.org/aburgasser/splat},
which is updated on a regular basis.
SPLAT has not yet reached v1.0, so bugs are common. Please help us squish them by
sending bug reports to \href{mailto:aburgasser@ucsd.edu}{aburgasser@ucsd.edu} or start an issue on the github site.

You may also obtain splat using {\color{red}\bfseries{}pip\_}:

..\_pip: \url{https://pip.pypa.io/en/stable/}

\begin{Verbatim}[commandchars=\\\{\}]
\PYG{g+gp}{\PYGZgt{}\PYGZgt{}\PYGZgt{} }\PYG{n}{pip} \PYG{n}{install} \PYG{n}{splat}
\end{Verbatim}

Instructions on setting up and using SPLAT are maintained at \url{http://www.browndwarfs.org/splat}.

Copy the file \sphinxcode{.splat\_access} into your home directory - this is your access key
if you have priveleged access to unpublished data in the SPL.


\chapter{Using SPLAT}
\label{index:using-splat}
SPLAT is best used in the \textbf{ipython} or \textbf{ipython notebook}; all of the necessary data is
included in the github/pip install, so you don't need to be online to run most programs.

Here are some examples:
\begin{itemize}
\item {} 
The best way to read in a spectrum is to use getSpectrum:

\end{itemize}

\begin{Verbatim}[commandchars=\\\{\}]
\PYG{g+gp}{\PYGZgt{}\PYGZgt{}\PYGZgt{} }\PYG{k+kn}{import} \PYG{n+nn}{splat}
\PYG{g+gp}{\PYGZgt{}\PYGZgt{}\PYGZgt{} }\PYG{n}{splist} \PYG{o}{=} \PYG{n}{splat}\PYG{o}{.}\PYG{n}{getSpectrum}\PYG{p}{(}\PYG{n}{shortname}\PYG{o}{=}\PYG{l+s+s1}{\PYGZsq{}}\PYG{l+s+s1}{0415\PYGZhy{}0935}\PYG{l+s+s1}{\PYGZsq{}}\PYG{p}{)}
\PYG{g+gp}{\PYGZgt{}\PYGZgt{}\PYGZgt{} }\PYG{n}{splist} \PYG{o}{=} \PYG{n}{splat}\PYG{o}{.}\PYG{n}{getSpectrum}\PYG{p}{(}\PYG{n}{young}\PYG{o}{=}\PYG{k+kc}{True}\PYG{p}{)}
\PYG{g+gp}{\PYGZgt{}\PYGZgt{}\PYGZgt{} }\PYG{n}{splist} \PYG{o}{=} \PYG{n}{splat}\PYG{o}{.}\PYG{n}{getSpectrum}\PYG{p}{(}\PYG{n}{spt}\PYG{o}{=}\PYG{p}{[}\PYG{l+s+s1}{\PYGZsq{}}\PYG{l+s+s1}{M7}\PYG{l+s+s1}{\PYGZsq{}}\PYG{p}{,}\PYG{l+s+s1}{\PYGZsq{}}\PYG{l+s+s1}{L5}\PYG{l+s+s1}{\PYGZsq{}}\PYG{p}{]}\PYG{p}{,}\PYG{n}{jmag}\PYG{o}{=}\PYG{p}{[}\PYG{l+m+mf}{14.}\PYG{p}{,}\PYG{l+m+mf}{99.}\PYG{p}{]}\PYG{p}{)}
\end{Verbatim}

In each case, splist is a list of Spectrum objects, which is the container of various
aspects of the spectrum and it source properties. For example, selecting the first spectrum,

\begin{Verbatim}[commandchars=\\\{\}]
\PYG{g+gp}{\PYGZgt{}\PYGZgt{}\PYGZgt{} }\PYG{n}{sp} \PYG{o}{=} \PYG{n}{splist}\PYG{p}{[}\PYG{l+m+mi}{0}\PYG{p}{]}
\end{Verbatim}

\sphinxcode{sp.wave} gives the wavelengths of this spectrum, \sphinxcode{sp.flux} the flux values, and \sphinxcode{sp.noise} the
flux uncertainty. There are several other elements to the Spectrum object that can be accessed using \sphinxcode{sp.info()}.

You can also read in your own spectrum by passing a filename

\begin{Verbatim}[commandchars=\\\{\}]
\PYG{g+gp}{\PYGZgt{}\PYGZgt{}\PYGZgt{} }\PYG{n}{sp} \PYG{o}{=} \PYG{n}{splat}\PYG{o}{.}\PYG{n}{Spectrum}\PYG{p}{(}\PYG{n}{filename}\PYG{o}{=}\PYG{l+s+s1}{\PYGZsq{}}\PYG{l+s+s1}{PATH\PYGZus{}TO/myspectrum.fits}\PYG{l+s+s1}{\PYGZsq{}}\PYG{p}{)}
\end{Verbatim}

Note that this file must conform to the following standard: the first column is
wavelength in microns, second column flux in f\_lambda units, third column (optional) is
flux uncertainty in \(f_{\lambda}\) units.
\begin{itemize}
\item {} 
To flux calibrate the spectrum, use the object's built in fluxCalibrate method:

\end{itemize}

\begin{Verbatim}[commandchars=\\\{\}]
\PYG{g+gp}{\PYGZgt{}\PYGZgt{}\PYGZgt{} }\PYG{n}{sp} \PYG{o}{=} \PYG{n}{splat}\PYG{o}{.}\PYG{n}{getSpectrum}\PYG{p}{(}\PYG{n}{shortname}\PYG{o}{=}\PYG{l+s+s1}{\PYGZsq{}}\PYG{l+s+s1}{0415\PYGZhy{}0935}\PYG{l+s+s1}{\PYGZsq{}}\PYG{p}{)}\PYG{p}{[}\PYG{l+m+mi}{0}\PYG{p}{]}
\PYG{g+gp}{\PYGZgt{}\PYGZgt{}\PYGZgt{} }\PYG{n}{sp}\PYG{o}{.}\PYG{n}{fluxCalibrate}\PYG{p}{(}\PYG{l+s+s1}{\PYGZsq{}}\PYG{l+s+s1}{2MASS J}\PYG{l+s+s1}{\PYGZsq{}}\PYG{p}{,}\PYG{l+m+mf}{14.0}\PYG{p}{)}
\end{Verbatim}
\begin{itemize}
\item {} 
To display the spectrum, use the Spectrum object's plot function or plotSpectrum :

\end{itemize}

\begin{Verbatim}[commandchars=\\\{\}]
\PYG{g+gp}{\PYGZgt{}\PYGZgt{}\PYGZgt{} }\PYG{n}{sp}\PYG{o}{.}\PYG{n}{plot}\PYG{p}{(}\PYG{p}{)}
\PYG{g+gp}{\PYGZgt{}\PYGZgt{}\PYGZgt{} }\PYG{n}{splat}\PYG{o}{.}\PYG{n}{plotSpectrum}\PYG{p}{(}\PYG{n}{sp}\PYG{p}{)}
\end{Verbatim}

which will pop up a window displaying flux vs. wavelength.
You can save this display by adding a filename:

\begin{Verbatim}[commandchars=\\\{\}]
\PYG{g+gp}{\PYGZgt{}\PYGZgt{}\PYGZgt{} }\PYG{n}{splat}\PYG{o}{.}\PYG{n}{plotSpectrum}\PYG{p}{(}\PYG{n}{sp}\PYG{p}{,}\PYG{n}{file}\PYG{o}{=}\PYG{l+s+s1}{\PYGZsq{}}\PYG{l+s+s1}{spectrum.png}\PYG{l+s+s1}{\PYGZsq{}}\PYG{p}{)}
\end{Verbatim}

You can also compare multiple spectra:

\begin{Verbatim}[commandchars=\\\{\}]
\PYG{g+gp}{\PYGZgt{}\PYGZgt{}\PYGZgt{} }\PYG{n}{sp1} \PYG{o}{=} \PYG{n}{splat}\PYG{o}{.}\PYG{n}{getSpectrum}\PYG{p}{(}\PYG{n}{shortname}\PYG{o}{=}\PYG{l+s+s1}{\PYGZsq{}}\PYG{l+s+s1}{0415\PYGZhy{}0935}\PYG{l+s+s1}{\PYGZsq{}}\PYG{p}{)}\PYG{p}{[}\PYG{l+m+mi}{0}\PYG{p}{]}
\PYG{g+gp}{\PYGZgt{}\PYGZgt{}\PYGZgt{} }\PYG{n}{sp2} \PYG{o}{=} \PYG{n}{splat}\PYG{o}{.}\PYG{n}{getSpectrum}\PYG{p}{(}\PYG{n}{shortname}\PYG{o}{=}\PYG{l+s+s1}{\PYGZsq{}}\PYG{l+s+s1}{1217\PYGZhy{}0311}\PYG{l+s+s1}{\PYGZsq{}}\PYG{p}{)}\PYG{p}{[}\PYG{l+m+mi}{0}\PYG{p}{]}
\PYG{g+gp}{\PYGZgt{}\PYGZgt{}\PYGZgt{} }\PYG{n}{splat}\PYG{o}{.}\PYG{n}{plotSpectrum}\PYG{p}{(}\PYG{n}{sp1}\PYG{p}{,}\PYG{n}{sp2}\PYG{p}{,}\PYG{n}{colors}\PYG{o}{=}\PYG{p}{[}\PYG{l+s+s1}{\PYGZsq{}}\PYG{l+s+s1}{black}\PYG{l+s+s1}{\PYGZsq{}}\PYG{p}{,}\PYG{l+s+s1}{\PYGZsq{}}\PYG{l+s+s1}{red}\PYG{l+s+s1}{\PYGZsq{}}\PYG{p}{]}\PYG{p}{)}
\end{Verbatim}

You can add several extras to this to label features, plot uncertainties,
indicate telluric absorption regions, make multi-panel and multi-page plots
of lists of spectra, etc. Be sure to look through the plotting
subpackage for more details.

SPLAT can analyze and compare an arbitrary number of spectra.
\begin{itemize}
\item {} 
To measure spectral indices, use measureIndex or measureIndexSet:

\end{itemize}

\begin{Verbatim}[commandchars=\\\{\}]
\PYG{g+gp}{\PYGZgt{}\PYGZgt{}\PYGZgt{} }\PYG{n}{sp} \PYG{o}{=} \PYG{n}{splat}\PYG{o}{.}\PYG{n}{getSpectrum}\PYG{p}{(}\PYG{n}{shortname}\PYG{o}{=}\PYG{l+s+s1}{\PYGZsq{}}\PYG{l+s+s1}{0415\PYGZhy{}0935}\PYG{l+s+s1}{\PYGZsq{}}\PYG{p}{)}\PYG{p}{[}\PYG{l+m+mi}{0}\PYG{p}{]}
\PYG{g+gp}{\PYGZgt{}\PYGZgt{}\PYGZgt{} }\PYG{n}{value}\PYG{p}{,} \PYG{n}{error} \PYG{o}{=} \PYG{n}{splat}\PYG{o}{.}\PYG{n}{measureIndex}\PYG{p}{(}\PYG{n}{sp}\PYG{p}{,}\PYG{p}{[}\PYG{l+m+mf}{1.14}\PYG{p}{,}\PYG{l+m+mf}{1.165}\PYG{p}{]}\PYG{p}{,}\PYG{p}{[}\PYG{l+m+mf}{1.21}\PYG{p}{,}\PYG{l+m+mf}{1.235}\PYG{p}{]}\PYG{p}{,}\PYG{n}{method}\PYG{o}{=}\PYG{l+s+s1}{\PYGZsq{}}\PYG{l+s+s1}{integrate}\PYG{l+s+s1}{\PYGZsq{}}\PYG{p}{)}
\PYG{g+gp}{\PYGZgt{}\PYGZgt{}\PYGZgt{} }\PYG{n}{indices} \PYG{o}{=} \PYG{n}{splat}\PYG{o}{.}\PYG{n}{measureIndexSet}\PYG{p}{(}\PYG{n}{sp}\PYG{p}{,}\PYG{n+nb}{set}\PYG{o}{=}\PYG{l+s+s1}{\PYGZsq{}}\PYG{l+s+s1}{testi}\PYG{l+s+s1}{\PYGZsq{}}\PYG{p}{)}
\end{Verbatim}

The last line returns a dictionary, whose value,error pair can be accessed by the name
of the index:

\begin{Verbatim}[commandchars=\\\{\}]
\PYG{g+gp}{\PYGZgt{}\PYGZgt{}\PYGZgt{} }\PYG{n+nb}{print} \PYG{n}{indices}\PYG{p}{[}\PYG{l+s+s1}{\PYGZsq{}}\PYG{l+s+s1}{sH2O\PYGZhy{}J}\PYG{l+s+s1}{\PYGZsq{}}\PYG{p}{]}             \PYG{c+c1}{\PYGZsh{} returns value, error}
\end{Verbatim}
\begin{itemize}
\item {} 
You can also determine the gravity classification of a source via \href{http://adsabs.harvard.edu/abs/2013ApJ...772...79A}{Allers \& Liu (2013)} using classifyGravity:

\end{itemize}

\begin{Verbatim}[commandchars=\\\{\}]
\PYG{g+gp}{\PYGZgt{}\PYGZgt{}\PYGZgt{} }\PYG{n}{sp} \PYG{o}{=} \PYG{n}{splat}\PYG{o}{.}\PYG{n}{getSpectrum}\PYG{p}{(}\PYG{n}{young}\PYG{o}{=}\PYG{k+kc}{True}\PYG{p}{,} \PYG{n}{lucky}\PYG{o}{=}\PYG{k+kc}{True}\PYG{p}{)}\PYG{p}{[}\PYG{l+m+mi}{0}\PYG{p}{]}
\PYG{g+gp}{\PYGZgt{}\PYGZgt{}\PYGZgt{} }\PYG{n+nb}{print} \PYG{n}{splat}\PYG{o}{.}\PYG{n}{classifyGravity}\PYG{p}{(}\PYG{n}{sp}\PYG{p}{)}   \PYG{c+c1}{\PYGZsh{} returned \PYGZsq{}VL\PYGZhy{}G\PYGZsq{}}
\end{Verbatim}
\begin{itemize}
\item {} 
To classify a spectrum, use the various classifyByXXX methods:

\end{itemize}

\begin{Verbatim}[commandchars=\\\{\}]
\PYG{g+gp}{\PYGZgt{}\PYGZgt{}\PYGZgt{} }\PYG{n}{sp} \PYG{o}{=} \PYG{n}{splat}\PYG{o}{.}\PYG{n}{getSpectrum}\PYG{p}{(}\PYG{n}{shortname}\PYG{o}{=}\PYG{l+s+s1}{\PYGZsq{}}\PYG{l+s+s1}{0415\PYGZhy{}0935}\PYG{l+s+s1}{\PYGZsq{}}\PYG{p}{)}\PYG{p}{[}\PYG{l+m+mi}{0}\PYG{p}{]}
\PYG{g+gp}{\PYGZgt{}\PYGZgt{}\PYGZgt{} }\PYG{n}{spt}\PYG{p}{,}\PYG{n}{unc} \PYG{o}{=} \PYG{n}{splat}\PYG{o}{.}\PYG{n}{classifyByIndex}\PYG{p}{(}\PYG{n}{sp}\PYG{p}{,}\PYG{n+nb}{set}\PYG{o}{=}\PYG{l+s+s1}{\PYGZsq{}}\PYG{l+s+s1}{burgasser}\PYG{l+s+s1}{\PYGZsq{}}\PYG{p}{)}
\PYG{g+gp}{\PYGZgt{}\PYGZgt{}\PYGZgt{} }\PYG{n}{spt}\PYG{p}{,}\PYG{n}{unc} \PYG{o}{=} \PYG{n}{splat}\PYG{o}{.}\PYG{n}{classifyByStandard}\PYG{p}{(}\PYG{n}{sp}\PYG{p}{,}\PYG{n}{spt}\PYG{o}{=}\PYG{p}{[}\PYG{l+s+s1}{\PYGZsq{}}\PYG{l+s+s1}{T5}\PYG{l+s+s1}{\PYGZsq{}}\PYG{p}{,}\PYG{l+s+s1}{\PYGZsq{}}\PYG{l+s+s1}{T9}\PYG{l+s+s1}{\PYGZsq{}}\PYG{p}{]}\PYG{p}{)}
\PYG{g+gp}{\PYGZgt{}\PYGZgt{}\PYGZgt{} }\PYG{n}{result} \PYG{o}{=} \PYG{n}{splat}\PYG{o}{.}\PYG{n}{classifyByTemplate}\PYG{p}{(}\PYG{n}{sp}\PYG{p}{,}\PYG{n}{spt}\PYG{o}{=}\PYG{p}{[}\PYG{l+s+s1}{\PYGZsq{}}\PYG{l+s+s1}{T6}\PYG{l+s+s1}{\PYGZsq{}}\PYG{p}{,}\PYG{l+s+s1}{\PYGZsq{}}\PYG{l+s+s1}{T9}\PYG{l+s+s1}{\PYGZsq{}}\PYG{p}{]}\PYG{p}{,}\PYG{n}{nbest}\PYG{o}{=}\PYG{l+m+mi}{5}\PYG{p}{)}
\end{Verbatim}

The last line returns a dictionary containing the best 5 template matches to the Spectrum \sphinxcode{sp}.
\begin{itemize}
\item {} 
To compare a spectrum to another spectrum or a model, use compareSpectra :

\end{itemize}

\begin{Verbatim}[commandchars=\\\{\}]
\PYG{g+gp}{\PYGZgt{}\PYGZgt{}\PYGZgt{} }\PYG{n}{sp} \PYG{o}{=} \PYG{n}{splat}\PYG{o}{.}\PYG{n}{getSpectrum}\PYG{p}{(}\PYG{n}{shortname}\PYG{o}{=}\PYG{l+s+s1}{\PYGZsq{}}\PYG{l+s+s1}{0415\PYGZhy{}0935}\PYG{l+s+s1}{\PYGZsq{}}\PYG{p}{)}\PYG{p}{[}\PYG{l+m+mi}{0}\PYG{p}{]}
\PYG{g+gp}{\PYGZgt{}\PYGZgt{}\PYGZgt{} }\PYG{n}{mdl} \PYG{o}{=} \PYG{n}{splat}\PYG{o}{.}\PYG{n}{loadModel}\PYG{p}{(}\PYG{n}{teff}\PYG{o}{=}\PYG{l+m+mi}{700}\PYG{p}{,}\PYG{n}{logg}\PYG{o}{=}\PYG{l+m+mf}{5.0}\PYG{p}{)}                    \PYG{c+c1}{\PYGZsh{} loads a BTSettl08 model by default}
\PYG{g+gp}{\PYGZgt{}\PYGZgt{}\PYGZgt{} }\PYG{n}{chi}\PYG{p}{,}\PYG{n}{scale} \PYG{o}{=} \PYG{n}{splat}\PYG{o}{.}\PYG{n}{compareSpectra}\PYG{p}{(}\PYG{n}{sp}\PYG{p}{,}\PYG{n}{mdl}\PYG{p}{)}
\PYG{g+gp}{\PYGZgt{}\PYGZgt{}\PYGZgt{} }\PYG{n}{mdl}\PYG{o}{.}\PYG{n}{scale}\PYG{p}{(}\PYG{n}{scale}\PYG{p}{)}
\PYG{g+gp}{\PYGZgt{}\PYGZgt{}\PYGZgt{} }\PYG{n}{splat}\PYG{o}{.}\PYG{n}{plotSpectrum}\PYG{p}{(}\PYG{n}{sp}\PYG{p}{,}\PYG{n}{mdl}\PYG{p}{,}\PYG{n}{colors}\PYG{o}{=}\PYG{p}{[}\PYG{l+s+s1}{\PYGZsq{}}\PYG{l+s+s1}{black}\PYG{l+s+s1}{\PYGZsq{}}\PYG{p}{,}\PYG{l+s+s1}{\PYGZsq{}}\PYG{l+s+s1}{red}\PYG{l+s+s1}{\PYGZsq{}}\PYG{p}{]}\PYG{p}{,}\PYG{n}{legend}\PYG{o}{=}\PYG{p}{[}\PYG{n}{sp}\PYG{o}{.}\PYG{n}{name}\PYG{p}{,}\PYG{n}{mdl}\PYG{o}{.}\PYG{n}{name}\PYG{p}{]}\PYG{p}{)}
\end{Verbatim}

\# There is also a basic Markov Chain Monte Carlo code to compare models to spectra called modelFitMCMC:

\begin{Verbatim}[commandchars=\\\{\}]
\PYG{g+gp}{\PYGZgt{}\PYGZgt{}\PYGZgt{} }\PYG{n}{sp} \PYG{o}{=} \PYG{n}{splat}\PYG{o}{.}\PYG{n}{getSpectrum}\PYG{p}{(}\PYG{n}{shortname}\PYG{o}{=}\PYG{l+s+s1}{\PYGZsq{}}\PYG{l+s+s1}{0415\PYGZhy{}0935}\PYG{l+s+s1}{\PYGZsq{}}\PYG{p}{)}\PYG{p}{[}\PYG{l+m+mi}{0}\PYG{p}{]}
\PYG{g+gp}{\PYGZgt{}\PYGZgt{}\PYGZgt{} }\PYG{n}{sp}\PYG{o}{.}\PYG{n}{fluxCalibrate}\PYG{p}{(}\PYG{l+s+s1}{\PYGZsq{}}\PYG{l+s+s1}{2MASS J}\PYG{l+s+s1}{\PYGZsq{}}\PYG{p}{,}\PYG{l+m+mf}{14.49}\PYG{p}{,}\PYG{n}{absolute}\PYG{o}{=}\PYG{k+kc}{True}\PYG{p}{)}
\PYG{g+gp}{\PYGZgt{}\PYGZgt{}\PYGZgt{} }\PYG{n}{table} \PYG{o}{=} \PYG{n}{splat}\PYG{o}{.}\PYG{n}{modelFitMCMC}\PYG{p}{(}\PYG{n}{sp}\PYG{p}{,}\PYG{n}{initial\PYGZus{}guess}\PYG{o}{=}\PYG{p}{[}\PYG{l+m+mi}{800}\PYG{p}{,}\PYG{l+m+mf}{5.0}\PYG{p}{,}\PYG{l+m+mf}{0.}\PYG{p}{]}\PYG{p}{,}\PYG{n}{nsamples}\PYG{o}{=}\PYG{l+m+mi}{300}\PYG{p}{,}\PYG{n}{step\PYGZus{}sizes}\PYG{o}{=}\PYG{p}{[}\PYG{l+m+mf}{50.}\PYG{p}{,}\PYG{l+m+mf}{0.5}\PYG{p}{,}\PYG{l+m+mf}{0.}\PYG{p}{]}\PYG{p}{)}
\end{Verbatim}

All of these routines have many options worth exploring, and which are (increasingly) documented on this website. If there are capabilities
you need, please suggest them to \href{mailto:aburgasser@ucsd.edu}{aburgasser@ucsd.edu}, or note it in the ``Issues'' link on our \href{https://github.com/aburgasser/splat}{github site}.


\chapter{Acknowledgements}
\label{index:acknowledgements}
SPLAT is an experimental, collaborative project of research students in \href{http://www.coolstarlab.org}{Adam Burgasser's
UCSD Cool Star Lab}, aimed at teaching students how to do research by building their own analysis tools.  Contributors to SPLAT have included Christian Aganze, Daniella Bardalez Gagliuffi, Adam Burgasser (PI), Caleb Choban, Ivanna Escala, Aishwarya Iyer, Yuhui Jin, Mike Lopez, Alex Mendez, Gretel Mercado, Johnny Parra, Maitrayee Sahi, Adrian Suarez, Melisa Tallis, Tomoki Tamiya and Chris Theissen.

This project is supported by the National Aeronautics and Space Administration under Grant No. NNX15AI75G.

\emph{Contents}


\section{Installation and Dependencies}
\label{installation:installation-and-dependencies}\label{installation::doc}
SPLAT is best forked from the github site \url{http://github.org/aburgasser/splat},
which is updated on a regular basis. Make sure the folder you download the package to is either in your \sphinxcode{PATH} or \sphinxcode{PYTHON\_PATH} system variable.

Once you have this in place, You should copy the file \sphinxcode{.splat\_access} into your home directory - this is your access key if you have priveleged access to unpublished data.

SPLAT has been tested in Python 2.7.X and 3.5.X. It works best in ipython and ipython notebook.

The SPLAT code uses the following external packages:
\begin{itemize}
\item {} 
\href{http://www.astropy.org/}{astropy}

\item {} 
\href{http://matplotlib.org/}{matplotlib}

\item {} 
\href{http://www.numpy.org/}{numpy}

\item {} 
\href{http://docs.python-requests.org/en/master/}{requests}

\item {} 
\href{https://www.scipy.org/}{scipy}

\item {} 
\href{https://pypi.python.org/pypi/triangle\_plot}{triangle}

\end{itemize}

and the following internal (python) packages:
\begin{itemize}
\item {} 
base64

\item {} 
copy

\item {} 
datetime

\item {} 
math

\item {} 
os

\item {} 
random

\item {} 
re

\item {} 
sys

\item {} 
warnings

\end{itemize}

SPLAT has not yet reached v1.0, so bugs are common. Please help us squish them by
sending bug reports to \href{mailto:aburgasser@ucsd.edu}{aburgasser@ucsd.edu} or start an issue on the github site.
\begin{itemize}
\item {} 
\DUrole{xref,std,std-ref}{genindex}

\item {} 
\DUrole{xref,std,std-ref}{modindex}

\item {} 
\DUrole{xref,std,std-ref}{search}

\end{itemize}


\section{Quickstart}
\label{quickstart::doc}\label{quickstart:quickstart}
SPLAT is best used in the ipython or ipython notebook; all of the necessary data is
included in the github install, so you shouldn't need to be online to run anything
unless you are using proprietary data (these are not distributed with the package).

Here are some examples:
\begin{itemize}
\item {} 
The best way to read in a spectrum is to use getSpectrum:

\end{itemize}

\begin{Verbatim}[commandchars=\\\{\}]
\PYG{g+gp}{\PYGZgt{}\PYGZgt{}\PYGZgt{} }\PYG{k+kn}{import} \PYG{n+nn}{splat}
\PYG{g+gp}{\PYGZgt{}\PYGZgt{}\PYGZgt{} }\PYG{n}{splist} \PYG{o}{=} \PYG{n}{splat}\PYG{o}{.}\PYG{n}{getSpectrum}\PYG{p}{(}\PYG{n}{shortname}\PYG{o}{=}\PYG{l+s+s1}{\PYGZsq{}}\PYG{l+s+s1}{0415\PYGZhy{}0935}\PYG{l+s+s1}{\PYGZsq{}}\PYG{p}{)}
\PYG{g+gp}{\PYGZgt{}\PYGZgt{}\PYGZgt{} }\PYG{n}{splist} \PYG{o}{=} \PYG{n}{splat}\PYG{o}{.}\PYG{n}{getSpectrum}\PYG{p}{(}\PYG{n}{young}\PYG{o}{=}\PYG{k+kc}{True}\PYG{p}{)}
\PYG{g+gp}{\PYGZgt{}\PYGZgt{}\PYGZgt{} }\PYG{n}{splist} \PYG{o}{=} \PYG{n}{splat}\PYG{o}{.}\PYG{n}{getSpectrum}\PYG{p}{(}\PYG{n}{spt}\PYG{o}{=}\PYG{p}{[}\PYG{l+s+s1}{\PYGZsq{}}\PYG{l+s+s1}{M7}\PYG{l+s+s1}{\PYGZsq{}}\PYG{p}{,}\PYG{l+s+s1}{\PYGZsq{}}\PYG{l+s+s1}{L5}\PYG{l+s+s1}{\PYGZsq{}}\PYG{p}{]}\PYG{p}{,}\PYG{n}{jmag}\PYG{o}{=}\PYG{p}{[}\PYG{l+m+mf}{14.}\PYG{p}{,}\PYG{l+m+mf}{99.}\PYG{p}{]}\PYG{p}{)}
\end{Verbatim}

In each case, splist is a list of Spectrum objects, which is the container of various
aspects of the spectrum and it source properties. For example, selecting the first spectrum,

\begin{Verbatim}[commandchars=\\\{\}]
\PYG{g+gp}{\PYGZgt{}\PYGZgt{}\PYGZgt{} }\PYG{n}{sp} \PYG{o}{=} \PYG{n}{splist}\PYG{p}{[}\PYG{l+m+mi}{0}\PYG{p}{]}
\end{Verbatim}

\sphinxcode{sp.wave} gives the wavelengths of this spectrum, \sphinxcode{sp.flux} the flux values, and \sphinxcode{sp.noise} the
flux uncertainty.

You can also read in your own spectrum by passing a filename

\begin{Verbatim}[commandchars=\\\{\}]
\PYG{g+gp}{\PYGZgt{}\PYGZgt{}\PYGZgt{} }\PYG{n}{sp} \PYG{o}{=} \PYG{n}{splat}\PYG{o}{.}\PYG{n}{Spectrum}\PYG{p}{(}\PYG{n}{filename}\PYG{o}{=}\PYG{l+s+s1}{\PYGZsq{}}\PYG{l+s+s1}{PATH\PYGZus{}TO/myspectrum.fits}\PYG{l+s+s1}{\PYGZsq{}}\PYG{p}{)}
\end{Verbatim}

Note that this file must conform to the standard of the SPL data: the first column is
wavelength in microns, second column flux in f\_lambda units, third column (optional) is
flux uncertainty.
\begin{itemize}
\item {} 
To flux calibrate the spectrum, use the object's built in \sphinxcode{fluxCalibrate()} method:

\end{itemize}

\begin{Verbatim}[commandchars=\\\{\}]
\PYG{g+gp}{\PYGZgt{}\PYGZgt{}\PYGZgt{} }\PYG{n}{sp} \PYG{o}{=} \PYG{n}{splat}\PYG{o}{.}\PYG{n}{getSpectrum}\PYG{p}{(}\PYG{n}{shortname}\PYG{o}{=}\PYG{l+s+s1}{\PYGZsq{}}\PYG{l+s+s1}{0415\PYGZhy{}0935}\PYG{l+s+s1}{\PYGZsq{}}\PYG{p}{)}\PYG{p}{[}\PYG{l+m+mi}{0}\PYG{p}{]}
\PYG{g+gp}{\PYGZgt{}\PYGZgt{}\PYGZgt{} }\PYG{n}{sp}\PYG{o}{.}\PYG{n}{fluxCalibrate}\PYG{p}{(}\PYG{l+s+s1}{\PYGZsq{}}\PYG{l+s+s1}{2MASS J}\PYG{l+s+s1}{\PYGZsq{}}\PYG{p}{,}\PYG{l+m+mf}{14.0}\PYG{p}{)}
\end{Verbatim}
\begin{itemize}
\item {} 
To display the spectrum, use the Spectrum object's plot function or plotSpectrum

\end{itemize}

\begin{Verbatim}[commandchars=\\\{\}]
\PYG{g+gp}{\PYGZgt{}\PYGZgt{}\PYGZgt{} }\PYG{n}{sp}\PYG{o}{.}\PYG{n}{plot}\PYG{p}{(}\PYG{p}{)}
\PYG{g+gp}{\PYGZgt{}\PYGZgt{}\PYGZgt{} }\PYG{n}{splat}\PYG{o}{.}\PYG{n}{plotSpectrum}\PYG{p}{(}\PYG{n}{sp}\PYG{p}{)}
\end{Verbatim}

which will pop up a window displaying flux vs. wavelength.
You can save this display by adding a filename:

\begin{Verbatim}[commandchars=\\\{\}]
\PYG{g+gp}{\PYGZgt{}\PYGZgt{}\PYGZgt{} }\PYG{n}{splat}\PYG{o}{.}\PYG{n}{plotSpectrum}\PYG{p}{(}\PYG{n}{sp}\PYG{p}{,}\PYG{n}{file}\PYG{o}{=}\PYG{l+s+s1}{\PYGZsq{}}\PYG{l+s+s1}{spectrum.png}\PYG{l+s+s1}{\PYGZsq{}}\PYG{p}{)}
\end{Verbatim}

You can also compare multiple spectra:

\begin{Verbatim}[commandchars=\\\{\}]
\PYG{g+gp}{\PYGZgt{}\PYGZgt{}\PYGZgt{} }\PYG{n}{sp1} \PYG{o}{=} \PYG{n}{splat}\PYG{o}{.}\PYG{n}{getSpectrum}\PYG{p}{(}\PYG{n}{shortname}\PYG{o}{=}\PYG{l+s+s1}{\PYGZsq{}}\PYG{l+s+s1}{0415\PYGZhy{}0935}\PYG{l+s+s1}{\PYGZsq{}}\PYG{p}{)}\PYG{p}{[}\PYG{l+m+mi}{0}\PYG{p}{]}
\PYG{g+gp}{\PYGZgt{}\PYGZgt{}\PYGZgt{} }\PYG{n}{sp2} \PYG{o}{=} \PYG{n}{splat}\PYG{o}{.}\PYG{n}{getSpectrum}\PYG{p}{(}\PYG{n}{shortname}\PYG{o}{=}\PYG{l+s+s1}{\PYGZsq{}}\PYG{l+s+s1}{1217\PYGZhy{}0311}\PYG{l+s+s1}{\PYGZsq{}}\PYG{p}{)}\PYG{p}{[}\PYG{l+m+mi}{0}\PYG{p}{]}
\PYG{g+gp}{\PYGZgt{}\PYGZgt{}\PYGZgt{} }\PYG{n}{splat}\PYG{o}{.}\PYG{n}{plotSpectrum}\PYG{p}{(}\PYG{n}{sp1}\PYG{p}{,}\PYG{n}{sp2}\PYG{p}{,}\PYG{n}{colors}\PYG{o}{=}\PYG{p}{[}\PYG{l+s+s1}{\PYGZsq{}}\PYG{l+s+s1}{black}\PYG{l+s+s1}{\PYGZsq{}}\PYG{p}{,}\PYG{l+s+s1}{\PYGZsq{}}\PYG{l+s+s1}{red}\PYG{l+s+s1}{\PYGZsq{}}\PYG{p}{]}\PYG{p}{)}
\end{Verbatim}

You can add several extras to this to label features, plot uncertainties,
indicate telluric absorption regions, make multi-panel and multi-page plots
of lists of spectra, etc. Be sure to look through the SPLAT plotting
subpackage for more details.

SPLAT can analyze and compare an arbitrary number of spectra.
\begin{itemize}
\item {} 
To measure spectral indices, use measureIndex or measureIndexSet:

\end{itemize}

\begin{Verbatim}[commandchars=\\\{\}]
\PYG{g+gp}{\PYGZgt{}\PYGZgt{}\PYGZgt{} }\PYG{n}{sp} \PYG{o}{=} \PYG{n}{splat}\PYG{o}{.}\PYG{n}{getSpectrum}\PYG{p}{(}\PYG{n}{shortname}\PYG{o}{=}\PYG{l+s+s1}{\PYGZsq{}}\PYG{l+s+s1}{0415\PYGZhy{}0935}\PYG{l+s+s1}{\PYGZsq{}}\PYG{p}{)}\PYG{p}{[}\PYG{l+m+mi}{0}\PYG{p}{]}
\PYG{g+gp}{\PYGZgt{}\PYGZgt{}\PYGZgt{} }\PYG{n}{value}\PYG{p}{,} \PYG{n}{error} \PYG{o}{=} \PYG{n}{splat}\PYG{o}{.}\PYG{n}{measureIndex}\PYG{p}{(}\PYG{n}{sp}\PYG{p}{,}\PYG{p}{[}\PYG{l+m+mf}{1.14}\PYG{p}{,}\PYG{l+m+mf}{1.165}\PYG{p}{]}\PYG{p}{,}\PYG{p}{[}\PYG{l+m+mf}{1.21}\PYG{p}{,}\PYG{l+m+mf}{1.235}\PYG{p}{]}\PYG{p}{,}\PYG{n}{method}\PYG{o}{=}\PYG{l+s+s1}{\PYGZsq{}}\PYG{l+s+s1}{integrate}\PYG{l+s+s1}{\PYGZsq{}}\PYG{p}{)}
\PYG{g+gp}{\PYGZgt{}\PYGZgt{}\PYGZgt{} }\PYG{n}{indices} \PYG{o}{=} \PYG{n}{splat}\PYG{o}{.}\PYG{n}{measureIndexSet}\PYG{p}{(}\PYG{n}{sp}\PYG{p}{,}\PYG{n+nb}{set}\PYG{o}{=}\PYG{l+s+s1}{\PYGZsq{}}\PYG{l+s+s1}{testi}\PYG{l+s+s1}{\PYGZsq{}}\PYG{p}{)}
\end{Verbatim}

The last line returns a dictionary, whose value,error pair can be accessed by the name
of the index:

\begin{Verbatim}[commandchars=\\\{\}]
\PYG{g+gp}{\PYGZgt{}\PYGZgt{}\PYGZgt{} }\PYG{n+nb}{print} \PYG{n}{indices}\PYG{p}{[}\PYG{l+s+s1}{\PYGZsq{}}\PYG{l+s+s1}{sH2O\PYGZhy{}J}\PYG{l+s+s1}{\PYGZsq{}}\PYG{p}{]}             \PYG{c+c1}{\PYGZsh{} returns value, error}
\end{Verbatim}
\begin{itemize}
\item {} 
You can also determine the gravity classification of a source via Allers \& Liu (2013):

\end{itemize}

\begin{Verbatim}[commandchars=\\\{\}]
\PYG{g+gp}{\PYGZgt{}\PYGZgt{}\PYGZgt{} }\PYG{n}{sp} \PYG{o}{=} \PYG{n}{splat}\PYG{o}{.}\PYG{n}{getSpectrum}\PYG{p}{(}\PYG{n}{young}\PYG{o}{=}\PYG{k+kc}{True}\PYG{p}{,} \PYG{n}{lucky}\PYG{o}{=}\PYG{k+kc}{True}\PYG{p}{)}\PYG{p}{[}\PYG{l+m+mi}{0}\PYG{p}{]}
\PYG{g+gp}{\PYGZgt{}\PYGZgt{}\PYGZgt{} }\PYG{n}{splat}\PYG{o}{.}\PYG{n}{classifyGravity}\PYG{p}{(}\PYG{n}{sp}\PYG{p}{,}\PYG{n}{verbose}\PYG{o}{=}\PYG{k+kc}{True}\PYG{p}{)}
\end{Verbatim}

This returns:

\begin{Verbatim}[commandchars=\\\{\}]
\PYG{g+gp}{\PYGZgt{}\PYGZgt{}\PYGZgt{} }\PYG{n}{Gravity} \PYG{n}{Classification}\PYG{p}{:}
\PYG{g+gp}{\PYGZgt{}\PYGZgt{}\PYGZgt{} }  \PYG{n}{SpT} \PYG{o}{=} \PYG{n}{L1}\PYG{o}{.}\PYG{l+m+mi}{0}
\PYG{g+gp}{\PYGZgt{}\PYGZgt{}\PYGZgt{} }  \PYG{n}{VO}\PYG{o}{\PYGZhy{}}\PYG{n}{z}\PYG{p}{:} \PYG{l+m+mf}{1.193}\PYG{o}{+}\PYG{o}{/}\PYG{o}{\PYGZhy{}}\PYG{l+m+mf}{0.018} \PYG{o}{=}\PYG{o}{\PYGZgt{}} \PYG{l+m+mf}{1.0}
\PYG{g+gp}{\PYGZgt{}\PYGZgt{}\PYGZgt{} }  \PYG{n}{FeH}\PYG{o}{\PYGZhy{}}\PYG{n}{z}\PYG{p}{:} \PYG{l+m+mf}{1.096}\PYG{o}{+}\PYG{o}{/}\PYG{o}{\PYGZhy{}}\PYG{l+m+mf}{0.026} \PYG{o}{=}\PYG{o}{\PYGZgt{}} \PYG{l+m+mf}{2.0}
\PYG{g+gp}{\PYGZgt{}\PYGZgt{}\PYGZgt{} }  \PYG{n}{H}\PYG{o}{\PYGZhy{}}\PYG{n}{cont}\PYG{p}{:} \PYG{l+m+mf}{0.973}\PYG{o}{+}\PYG{o}{/}\PYG{o}{\PYGZhy{}}\PYG{l+m+mf}{0.010} \PYG{o}{=}\PYG{o}{\PYGZgt{}} \PYG{l+m+mf}{2.0}
\PYG{g+gp}{\PYGZgt{}\PYGZgt{}\PYGZgt{} }  \PYG{n}{KI}\PYG{o}{\PYGZhy{}}\PYG{n}{J}\PYG{p}{:} \PYG{l+m+mf}{1.044}\PYG{o}{+}\PYG{o}{/}\PYG{o}{\PYGZhy{}}\PYG{l+m+mf}{0.008} \PYG{o}{=}\PYG{o}{\PYGZgt{}} \PYG{l+m+mf}{2.0}
\PYG{g+gp}{\PYGZgt{}\PYGZgt{}\PYGZgt{} }  \PYG{n}{Gravity} \PYG{n}{Class} \PYG{o}{=} \PYG{n}{VL}\PYG{o}{\PYGZhy{}}\PYG{n}{G}
\end{Verbatim}
\begin{itemize}
\item {} 
To classify a spectrum, use the classifyByXXX methods:

\end{itemize}

\begin{Verbatim}[commandchars=\\\{\}]
\PYG{g+gp}{\PYGZgt{}\PYGZgt{}\PYGZgt{} }\PYG{n}{sp} \PYG{o}{=} \PYG{n}{splat}\PYG{o}{.}\PYG{n}{getSpectrum}\PYG{p}{(}\PYG{n}{shortname}\PYG{o}{=}\PYG{l+s+s1}{\PYGZsq{}}\PYG{l+s+s1}{0415\PYGZhy{}0935}\PYG{l+s+s1}{\PYGZsq{}}\PYG{p}{)}\PYG{p}{[}\PYG{l+m+mi}{0}\PYG{p}{]}
\PYG{g+gp}{\PYGZgt{}\PYGZgt{}\PYGZgt{} }\PYG{n}{spt}\PYG{p}{,}\PYG{n}{unc} \PYG{o}{=} \PYG{n}{splat}\PYG{o}{.}\PYG{n}{classifyByIndex}\PYG{p}{(}\PYG{n}{sp}\PYG{p}{,}\PYG{n+nb}{set}\PYG{o}{=}\PYG{l+s+s1}{\PYGZsq{}}\PYG{l+s+s1}{burgasser}\PYG{l+s+s1}{\PYGZsq{}}\PYG{p}{)}
\PYG{g+gp}{\PYGZgt{}\PYGZgt{}\PYGZgt{} }\PYG{n}{spt}\PYG{p}{,}\PYG{n}{unc} \PYG{o}{=} \PYG{n}{splat}\PYG{o}{.}\PYG{n}{classifyByStandard}\PYG{p}{(}\PYG{n}{sp}\PYG{p}{,}\PYG{n}{spt}\PYG{o}{=}\PYG{p}{[}\PYG{l+s+s1}{\PYGZsq{}}\PYG{l+s+s1}{T5}\PYG{l+s+s1}{\PYGZsq{}}\PYG{p}{,}\PYG{l+s+s1}{\PYGZsq{}}\PYG{l+s+s1}{T9}\PYG{l+s+s1}{\PYGZsq{}}\PYG{p}{]}\PYG{p}{)}
\PYG{g+gp}{\PYGZgt{}\PYGZgt{}\PYGZgt{} }\PYG{n}{bestMatches} \PYG{o}{=} \PYG{n}{splat}\PYG{o}{.}\PYG{n}{classifyByTemplate}\PYG{p}{(}\PYG{n}{sp}\PYG{p}{,}\PYG{n}{spt}\PYG{o}{=}\PYG{p}{[}\PYG{l+s+s1}{\PYGZsq{}}\PYG{l+s+s1}{T6}\PYG{l+s+s1}{\PYGZsq{}}\PYG{p}{,}\PYG{l+s+s1}{\PYGZsq{}}\PYG{l+s+s1}{T9}\PYG{l+s+s1}{\PYGZsq{}}\PYG{p}{]}\PYG{p}{,}\PYG{n}{nbest}\PYG{o}{=}\PYG{l+m+mi}{5}\PYG{p}{)}
\end{Verbatim}

The last line returns a dictionary containing the best 5 template matches to the Spectrum sp.
Note that this can take a long time to run!
\begin{itemize}
\item {} 
To compare a spectrum to another spectrum or a model, use compareSpectra:

\end{itemize}

\begin{Verbatim}[commandchars=\\\{\}]
\PYG{g+gp}{\PYGZgt{}\PYGZgt{}\PYGZgt{} }\PYG{n}{sp} \PYG{o}{=} \PYG{n}{splat}\PYG{o}{.}\PYG{n}{getSpectrum}\PYG{p}{(}\PYG{n}{shortname}\PYG{o}{=}\PYG{l+s+s1}{\PYGZsq{}}\PYG{l+s+s1}{0415\PYGZhy{}0935}\PYG{l+s+s1}{\PYGZsq{}}\PYG{p}{)}\PYG{p}{[}\PYG{l+m+mi}{0}\PYG{p}{]}
\PYG{g+gp}{\PYGZgt{}\PYGZgt{}\PYGZgt{} }\PYG{n}{mdl} \PYG{o}{=} \PYG{n}{splat}\PYG{o}{.}\PYG{n}{loadModel}\PYG{p}{(}\PYG{n}{teff}\PYG{o}{=}\PYG{l+m+mi}{700}\PYG{p}{,}\PYG{n}{logg}\PYG{o}{=}\PYG{l+m+mf}{5.0}\PYG{p}{)}                    \PYG{c+c1}{\PYGZsh{} BTSettl08 model by default}
\PYG{g+gp}{\PYGZgt{}\PYGZgt{}\PYGZgt{} }\PYG{n}{chi}\PYG{p}{,}\PYG{n}{scale} \PYG{o}{=} \PYG{n}{splat}\PYG{o}{.}\PYG{n}{compareSpectra}\PYG{p}{(}\PYG{n}{sp}\PYG{p}{,}\PYG{n}{mdl}\PYG{p}{)}
\PYG{g+gp}{\PYGZgt{}\PYGZgt{}\PYGZgt{} }\PYG{n}{mdl}\PYG{o}{.}\PYG{n}{scale}\PYG{p}{(}\PYG{n}{scale}\PYG{p}{)}
\PYG{g+gp}{\PYGZgt{}\PYGZgt{}\PYGZgt{} }\PYG{n}{splat}\PYG{o}{.}\PYG{n}{plotSpectrum}\PYG{p}{(}\PYG{n}{sp}\PYG{p}{,}\PYG{n}{mdl}\PYG{p}{,}\PYG{n}{colors}\PYG{o}{=}\PYG{p}{[}\PYG{l+s+s1}{\PYGZsq{}}\PYG{l+s+s1}{black}\PYG{l+s+s1}{\PYGZsq{}}\PYG{p}{,}\PYG{l+s+s1}{\PYGZsq{}}\PYG{l+s+s1}{red}\PYG{l+s+s1}{\PYGZsq{}}\PYG{p}{]}\PYG{p}{,}\PYG{n}{legend}\PYG{o}{=}\PYG{p}{[}\PYG{n}{sp}\PYG{o}{.}\PYG{n}{name}\PYG{p}{,}\PYG{n}{mdl}\PYG{o}{.}\PYG{n}{name}\PYG{p}{]}\PYG{p}{)}
\end{Verbatim}

The available spectral models are
\begin{itemize}
\item {} 
`BTSettl08' (Allard et al. 2008)

\item {} 
`drift' (Witte et al. 2008)

\item {} 
`burrows06' (Burrows et al. 2006)

\item {} 
`saumon12' (Saumon \& Marley 2012)

\item {} 
`morley12' (Morley et al. 2012)

\item {} 
`morley14; (Morley et al. 2014)

\end{itemize}
\begin{itemize}
\item {} 
There is also a basic Markov Chain Monte Carlo code to compare models to spectra (Note: still in development)

\end{itemize}

\begin{Verbatim}[commandchars=\\\{\}]
\PYG{g+gp}{\PYGZgt{}\PYGZgt{}\PYGZgt{} }\PYG{n}{sp} \PYG{o}{=} \PYG{n}{splat}\PYG{o}{.}\PYG{n}{getSpectrum}\PYG{p}{(}\PYG{n}{shortname}\PYG{o}{=}\PYG{l+s+s1}{\PYGZsq{}}\PYG{l+s+s1}{0415\PYGZhy{}0935}\PYG{l+s+s1}{\PYGZsq{}}\PYG{p}{)}\PYG{p}{[}\PYG{l+m+mi}{0}\PYG{p}{]}
\PYG{g+gp}{\PYGZgt{}\PYGZgt{}\PYGZgt{} }\PYG{n}{sp}\PYG{o}{.}\PYG{n}{fluxCalibrate}\PYG{p}{(}\PYG{l+s+s1}{\PYGZsq{}}\PYG{l+s+s1}{2MASS J}\PYG{l+s+s1}{\PYGZsq{}}\PYG{p}{,}\PYG{l+m+mf}{14.49}\PYG{p}{,}\PYG{n}{absolute}\PYG{o}{=}\PYG{k+kc}{True}\PYG{p}{)}
\PYG{g+gp}{\PYGZgt{}\PYGZgt{}\PYGZgt{} }\PYG{n}{table} \PYG{o}{=} \PYG{n}{splat}\PYG{o}{.}\PYG{n}{modelFitMCMC}\PYG{p}{(}\PYG{n}{sp}\PYG{p}{,}\PYG{n}{initial\PYGZus{}guess}\PYG{o}{=}\PYG{p}{[}\PYG{l+m+mi}{800}\PYG{p}{,}\PYG{l+m+mf}{5.0}\PYG{p}{,}\PYG{l+m+mf}{0.}\PYG{p}{]}\PYG{p}{,}\PYG{n}{nsamples}\PYG{o}{=}\PYG{l+m+mi}{300}\PYG{p}{,}\PYG{n}{step\PYGZus{}sizes}\PYG{o}{=}\PYG{p}{[}\PYG{l+m+mf}{50.}\PYG{p}{,}\PYG{l+m+mf}{0.5}\PYG{p}{,}\PYG{l+m+mf}{0.}\PYG{p}{]}\PYG{p}{)}
\end{Verbatim}

All of these routines have many options worth exploring, and which are (partially) documented
in the following pages. If there are other capabilities
you need, please suggest them, or note it in the ``Issues'' link on our github site

\emph{Search}
\begin{itemize}
\item {} 
\DUrole{xref,std,std-ref}{genindex}

\item {} 
\DUrole{xref,std,std-ref}{modindex}

\item {} 
\DUrole{xref,std,std-ref}{search}

\end{itemize}


\section{Main SPLAT module}
\label{splat::doc}\label{splat:main-splat-module}
The primary SPLAT module contains the definition of the core Spectrum object and associated function,
standard classification and spectral analysis routines, and helper functions.


\subsection{The SPLAT Spectrum object}
\label{splat:the-splat-spectrum-object}
Spectal data are manipulated as a Spectrum object, which contains the relevant data (wavelength,
flux, uncertainty) and additional information on the source and/or observation.  A Spectrum object is
the output of the various database access routines:

\begin{Verbatim}[commandchars=\\\{\}]
\PYG{g+gp}{\PYGZgt{}\PYGZgt{}\PYGZgt{} }\PYG{n}{sp} \PYG{o}{=} \PYG{n}{splat}\PYG{o}{.}\PYG{n}{getSpectrum}\PYG{p}{(}\PYG{n}{shortname}\PYG{o}{=}\PYG{l+s+s1}{\PYGZsq{}}\PYG{l+s+s1}{0415\PYGZhy{}0935}\PYG{l+s+s1}{\PYGZsq{}}\PYG{p}{)}\PYG{p}{[}\PYG{l+m+mi}{0}\PYG{p}{]}    \PYG{c+c1}{\PYGZsh{} note that getSpectrum returns a list}
\PYG{g+gp}{\PYGZgt{}\PYGZgt{}\PYGZgt{} }\PYG{n}{sp} \PYG{o}{=} \PYG{n}{splat}\PYG{o}{.}\PYG{n}{getStandard}\PYG{p}{(}\PYG{l+s+s1}{\PYGZsq{}}\PYG{l+s+s1}{M0}\PYG{l+s+s1}{\PYGZsq{}}\PYG{p}{)}\PYG{p}{[}\PYG{l+m+mi}{0}\PYG{p}{]}
\PYG{g+gp}{\PYGZgt{}\PYGZgt{}\PYGZgt{} }\PYG{n}{sp} \PYG{o}{=} \PYG{n}{splat}\PYG{o}{.}\PYG{n}{getModel}\PYG{p}{(}\PYG{n}{teff} \PYG{o}{=} \PYG{l+m+mi}{700}\PYG{p}{,} \PYG{n}{logg}\PYG{o}{=}\PYG{l+m+mf}{4.5}\PYG{p}{)}
\PYG{g+gp}{\PYGZgt{}\PYGZgt{}\PYGZgt{} }\PYG{n}{splist} \PYG{o}{=} \PYG{n}{splat}\PYG{o}{.}\PYG{n}{getSpectrum}\PYG{p}{(}\PYG{n}{spt}\PYG{o}{=}\PYG{p}{[}\PYG{l+s+s1}{\PYGZsq{}}\PYG{l+s+s1}{M7}\PYG{l+s+s1}{\PYGZsq{}}\PYG{p}{,}\PYG{l+s+s1}{\PYGZsq{}}\PYG{l+s+s1}{L5}\PYG{l+s+s1}{\PYGZsq{}}\PYG{p}{]}\PYG{p}{,}\PYG{n}{jmag}\PYG{o}{=}\PYG{p}{[}\PYG{l+m+mf}{14.}\PYG{p}{,}\PYG{l+m+mf}{99.}\PYG{p}{]}\PYG{p}{)}
\end{Verbatim}

You can also read in your own spectrum by passing a filename

\begin{Verbatim}[commandchars=\\\{\}]
\PYG{g+gp}{\PYGZgt{}\PYGZgt{}\PYGZgt{} }\PYG{n}{sp} \PYG{o}{=} \PYG{n}{splat}\PYG{o}{.}\PYG{n}{Spectrum}\PYG{p}{(}\PYG{n}{filename}\PYG{o}{=}\PYG{l+s+s1}{\PYGZsq{}}\PYG{l+s+s1}{PATH\PYGZus{}TO/myspectrum.fits}\PYG{l+s+s1}{\PYGZsq{}}\PYG{p}{)}
\end{Verbatim}

Note that this file must conform to the standard of the SPL data: the first column is
wavelength in microns, second column flux in \(F_{\lambda}\) units, third column (optional) is
flux uncertainty. The file can be a fits or ascii file.

You can also access a file based on its unique spectum key

\begin{Verbatim}[commandchars=\\\{\}]
\PYG{g+gp}{\PYGZgt{}\PYGZgt{}\PYGZgt{} }\PYG{n}{sp} \PYG{o}{=} \PYG{n}{splat}\PYG{o}{.}\PYG{n}{Spectrum}\PYG{p}{(}\PYG{l+m+mi}{10002}\PYG{p}{)}
\end{Verbatim}

There are many aspects of the Spectrum object that can be referenced or set, all of which are
described in the API entry. Some primary examples:
\begin{description}
\item[{\sphinxcode{sp.wave}, \sphinxcode{sp.flux}, \sphinxcode{sp.noise}}] \leavevmode
Wavelengths, flux density (per wavelength) values, and flux uncertainty of the spectrum (uncertainty = NaN for models)

\item[{\sphinxcode{sp.wunit}, \sphinxcode{sp.funit}}] \leavevmode
Astropy units for wavelength and flux, by default microns and erg/cm:sup: \sphinxtitleref{2}/s/micron

\item[{\sphinxcode{sp.nu}, \sphinxcode{sp.fnu}, \sphinxcode{sp.fnu\_unit}}] \leavevmode
Frequencies, flux density (per frequency), and fnu units, by default Jy

\item[{\sphinxcode{sp.snr}}] \leavevmode
Median estimate of spectrum signal-to-noise

\item[{\sphinxcode{sp.header}}] \leavevmode
Fits header (dictionary) from original file, if present

\item[{\sphinxcode{sp.teff}, \sphinxcode{sp.logg}, \sphinxcode{sp.z}, \sphinxcode{sp.fsed}, \sphinxcode{sp.cld}, \sphinxcode{sp.kzz}}] \leavevmode
If Spectrum is a model, these values give the effective temperature (in K), log surface gravity (cm/s$^{\text{2}}$),
log metallicity (relative to Sun), sedimentation efficient f:sub: sed(for Saumon \& Marley 2012 models),
cloud coverage fraction (for Morley models) and non-equilibrium chemistry diffusion constant

\end{description}


\subsubsection{Built-in Commands}
\label{splat:built-in-commands}\begin{itemize}
\item {} 
Information on the Spectrum object

\end{itemize}

The \sphinxcode{info()} command produces a summary of the Spectrum object's primary information:

\begin{Verbatim}[commandchars=\\\{\}]
\PYG{g+gp}{\PYGZgt{}\PYGZgt{}\PYGZgt{} }\PYG{n}{sp}\PYG{o}{.}\PYG{n}{info}\PYG{p}{(}\PYG{p}{)}
\end{Verbatim}

The \sphinxcode{showHistory()} command provides a summary of actions taken to maniupate a Spectrum object:

\begin{Verbatim}[commandchars=\\\{\}]
\PYG{g+gp}{\PYGZgt{}\PYGZgt{}\PYGZgt{} }\PYG{n}{sp}\PYG{o}{.}\PYG{n}{showHistory}\PYG{p}{(}\PYG{p}{)}
\end{Verbatim}

If you make changes to your Spectrum object, you can in many cases return it to its original state using the \sphinxcode{reset()} function

\begin{Verbatim}[commandchars=\\\{\}]
\PYG{g+gp}{\PYGZgt{}\PYGZgt{}\PYGZgt{} }\PYG{n}{sp}\PYG{o}{.}\PYG{n}{reset}\PYG{p}{(}\PYG{p}{)}
\end{Verbatim}
\begin{itemize}
\item {} 
To display the spectrum, use the Spectrum object's plot function, which makes use of the many options available in the plotSpectrum routine

\end{itemize}

\begin{Verbatim}[commandchars=\\\{\}]
\PYG{g+gp}{\PYGZgt{}\PYGZgt{}\PYGZgt{} }\PYG{n}{sp}\PYG{o}{.}\PYG{n}{plot}\PYG{p}{(}\PYG{p}{)}
\PYG{g+gp}{\PYGZgt{}\PYGZgt{}\PYGZgt{} }\PYG{n}{sp}\PYG{o}{.}\PYG{n}{plot}\PYG{p}{(}\PYG{n}{label}\PYG{o}{=}\PYG{l+s+s1}{\PYGZsq{}}\PYG{l+s+s1}{Awesome source}\PYG{l+s+s1}{\PYGZsq{}}\PYG{p}{,} \PYG{n}{telluric}\PYG{o}{=}\PYG{k+kc}{True}\PYG{p}{)}
\PYG{g+gp}{\PYGZgt{}\PYGZgt{}\PYGZgt{} }\PYG{n}{sp}\PYG{o}{.}\PYG{n}{plot}\PYG{p}{(}\PYG{n}{file}\PYG{o}{=}\PYG{l+s+s1}{\PYGZsq{}}\PYG{l+s+s1}{plot\PYGZus{}output.eps}\PYG{l+s+s1}{\PYGZsq{}}\PYG{p}{)}
\end{Verbatim}

You can save this display by adding a filename:

\begin{Verbatim}[commandchars=\\\{\}]
\PYG{g+gp}{\PYGZgt{}\PYGZgt{}\PYGZgt{} }\PYG{n}{sp}\PYG{o}{.}\PYG{n}{plot}\PYG{p}{(}\PYG{n}{sp}\PYG{p}{,}\PYG{n}{file}\PYG{o}{=}\PYG{l+s+s1}{\PYGZsq{}}\PYG{l+s+s1}{spectrum.png}\PYG{l+s+s1}{\PYGZsq{}}\PYG{p}{)}
\end{Verbatim}
\begin{itemize}
\item {} 
Saving a spectrum

\end{itemize}

A spectrum contained in a Spectrum object can be output to a file using the built-in export() or save() commands; both fits and tab-delimited ascii outputs are supported:

\begin{Verbatim}[commandchars=\\\{\}]
\PYG{g+gp}{\PYGZgt{}\PYGZgt{}\PYGZgt{} }\PYG{n}{sp}\PYG{o}{.}\PYG{n}{save}\PYG{p}{(}\PYG{l+s+s1}{\PYGZsq{}}\PYG{l+s+s1}{myspectrum.fits}\PYG{l+s+s1}{\PYGZsq{}}\PYG{p}{)}
\end{Verbatim}


\subsection{Spectral Analysis}
\label{splat:spectral-analysis}
SPLAT has several routines to do basic spectral analysis and combining of spectra.
\begin{itemize}
\item {} 
Scaling a spectrum

\end{itemize}

Spectra can be scaled by an arbitrary factor:

\begin{Verbatim}[commandchars=\\\{\}]
\PYG{g+gp}{\PYGZgt{}\PYGZgt{}\PYGZgt{} }\PYG{n}{sp}\PYG{o}{.}\PYG{n}{scale}\PYG{p}{(}\PYG{l+m+mf}{1.e9}\PYG{p}{)}
\end{Verbatim}

Or simply normalized:

\begin{Verbatim}[commandchars=\\\{\}]
\PYG{g+gp}{\PYGZgt{}\PYGZgt{}\PYGZgt{} }\PYG{n}{sp}\PYG{o}{.}\PYG{n}{normalize}\PYG{p}{(}\PYG{p}{)}
\end{Verbatim}
\begin{itemize}
\item {} 
Spectral math

\end{itemize}

Spectrum objects can be manipulated through normal arithmetic operations, which function on a wavelength-by-wavelength scale and properly propogate uncertainties

\begin{Verbatim}[commandchars=\\\{\}]
\PYG{g+gp}{\PYGZgt{}\PYGZgt{}\PYGZgt{} }\PYG{n}{sp3} \PYG{o}{=} \PYG{n}{sp1}\PYG{o}{+}\PYG{n}{sp2}
\PYG{g+gp}{\PYGZgt{}\PYGZgt{}\PYGZgt{} }\PYG{n}{sp3} \PYG{o}{=} \PYG{n}{sp1}\PYG{o}{\PYGZhy{}}\PYG{n}{sp2}
\PYG{g+gp}{\PYGZgt{}\PYGZgt{}\PYGZgt{} }\PYG{n}{sp3} \PYG{o}{=} \PYG{n}{sp1}\PYG{o}{*}\PYG{n}{sp2}
\PYG{g+gp}{\PYGZgt{}\PYGZgt{}\PYGZgt{} }\PYG{n}{sp3} \PYG{o}{=} \PYG{n}{sp1}\PYG{o}{/}\PYG{n}{sp2}
\end{Verbatim}


\subsection{Spectrophotometry}
\label{splat:spectrophotometry}
SPLAT allows spectrophotometry of spectra using common filters in the red optical and near-infrared. The filter transmission files are stored in the SPLAT reference library, and are accessed by name.  A list of current filters can be made by through the {\color{red}\bfseries{}{}`{}`}filterInfo(){}`{}`\_ routine:

\begin{Verbatim}[commandchars=\\\{\}]
\PYG{g+gp}{\PYGZgt{}\PYGZgt{}\PYGZgt{} }\PYG{n}{splat}\PYG{o}{.}\PYG{n}{filterInfo}\PYG{p}{(}\PYG{p}{)}
\PYG{g+go}{  2MASS H: 2MASS H\PYGZhy{}band}
\PYG{g+go}{  2MASS J: 2MASS J\PYGZhy{}band}
\PYG{g+go}{  2MASS KS: 2MASS Ks\PYGZhy{}band}
\PYG{g+go}{  BESSEL I: Bessel I\PYGZhy{}band}
\PYG{g+go}{  FOURSTAR H: FOURSTAR H\PYGZhy{}band}
\PYG{g+go}{  FOURSTAR H LONG: FOURSTAR H long}
\PYG{g+go}{  ...}
\end{Verbatim}

You can access specific information about a given filter profile with the {\color{red}\bfseries{}{}`{}`}filterProperties(){}`{}`\_ routine

\begin{Verbatim}[commandchars=\\\{\}]
\PYG{g+gp}{\PYGZgt{}\PYGZgt{}\PYGZgt{} }\PYG{n}{result} \PYG{o}{=} \PYG{n}{splat}\PYG{o}{.}\PYG{n}{filterProperties}\PYG{p}{(}\PYG{l+s+s1}{\PYGZsq{}}\PYG{l+s+s1}{2MASS J}\PYG{l+s+s1}{\PYGZsq{}}\PYG{p}{)}
\PYG{g+go}{        Filter 2MASS J: 2MASS J\PYGZhy{}band}
\PYG{g+go}{        Zeropoint = 1594.0 Jy}
\PYG{g+go}{        Pivot point: = 1.252 micron}
\PYG{g+go}{        FWHM = 0.280 micron}
\PYG{g+go}{        Wavelength range = 1.075 to 1.416 micron}
\end{Verbatim}

The {\color{red}\bfseries{}{}`{}`}filterMag(){}`{}`\_ routine determines the photometric magnitude of a source based on its spectrum, by convolving fluxes with a defined filter profile:

\begin{Verbatim}[commandchars=\\\{\}]
\PYG{g+gp}{\PYGZgt{}\PYGZgt{}\PYGZgt{} }\PYG{n}{sp} \PYG{o}{=} \PYG{n}{splat}\PYG{o}{.}\PYG{n}{getSpectrum}\PYG{p}{(}\PYG{n}{shortname}\PYG{o}{=}\PYG{l+s+s1}{\PYGZsq{}}\PYG{l+s+s1}{1507\PYGZhy{}1627}\PYG{l+s+s1}{\PYGZsq{}}\PYG{p}{)}\PYG{p}{[}\PYG{l+m+mi}{0}\PYG{p}{]}
\PYG{g+gp}{\PYGZgt{}\PYGZgt{}\PYGZgt{} }\PYG{n}{sp}\PYG{o}{.}\PYG{n}{fluxCalibrate}\PYG{p}{(}\PYG{l+s+s1}{\PYGZsq{}}\PYG{l+s+s1}{2MASS J}\PYG{l+s+s1}{\PYGZsq{}}\PYG{p}{,}\PYG{l+m+mf}{14.5}\PYG{p}{)}
\PYG{g+gp}{\PYGZgt{}\PYGZgt{}\PYGZgt{} }\PYG{n}{splat}\PYG{o}{.}\PYG{n}{filterMag}\PYG{p}{(}\PYG{n}{sp}\PYG{p}{,}\PYG{l+s+s1}{\PYGZsq{}}\PYG{l+s+s1}{MKO J}\PYG{l+s+s1}{\PYGZsq{}}\PYG{p}{)}
\PYG{g+go}{    (14.346586427733005, 0.032091919093387822)}
\end{Verbatim}

By default the filter is convolved with a model of Vega to extract Vega magnitudes, but the user can also set the \sphinxcode{ab} parameter to get AB magnitudes, the \sphinxcode{photon} parameter to get photon flux, or the \sphinxcode{energy} parameter to get total energy flux:

\begin{Verbatim}[commandchars=\\\{\}]
\PYG{g+gp}{\PYGZgt{}\PYGZgt{}\PYGZgt{} }\PYG{n}{splat}\PYG{o}{.}\PYG{n}{filterMag}\PYG{p}{(}\PYG{n}{sp}\PYG{p}{,}\PYG{l+s+s1}{\PYGZsq{}}\PYG{l+s+s1}{MKO J}\PYG{l+s+s1}{\PYGZsq{}}\PYG{p}{,}\PYG{n}{ab}\PYG{o}{=}\PYG{k+kc}{True}\PYG{p}{)}
\PYG{g+go}{    (15.245064259793901, 0.031168695728282524)}
\PYG{g+gp}{\PYGZgt{}\PYGZgt{}\PYGZgt{} }\PYG{n}{splat}\PYG{o}{.}\PYG{n}{filterMag}\PYG{p}{(}\PYG{n}{sp}\PYG{p}{,}\PYG{l+s+s1}{\PYGZsq{}}\PYG{l+s+s1}{MKO J}\PYG{l+s+s1}{\PYGZsq{}}\PYG{p}{,}\PYG{n}{energy}\PYG{o}{=}\PYG{k+kc}{True}\PYG{p}{)}
\PYG{g+go}{        (\PYGZlt{}Quantity 7.907663172914481e\PYGZhy{}13 erg / (cm2 s)\PYGZgt{},}
\PYG{g+go}{         \PYGZlt{}Quantity 2.090970538372485e\PYGZhy{}14 erg / (cm2 s)\PYGZgt{})}
\PYG{g+gp}{\PYGZgt{}\PYGZgt{}\PYGZgt{} }\PYG{n}{splat}\PYG{o}{.}\PYG{n}{filterMag}\PYG{p}{(}\PYG{n}{sp}\PYG{p}{,}\PYG{l+s+s1}{\PYGZsq{}}\PYG{l+s+s1}{MKO J}\PYG{l+s+s1}{\PYGZsq{}}\PYG{p}{,}\PYG{n}{photon}\PYG{o}{=}\PYG{k+kc}{True}\PYG{p}{)}
\PYG{g+go}{        (\PYGZlt{}Quantity 1.954421499626954e\PYGZhy{}24 1 / (cm2 s)\PYGZgt{},}
\PYG{g+go}{         \PYGZlt{}Quantity 5.53673880346918e\PYGZhy{}26 1 / (cm2 s)\PYGZgt{})}
\end{Verbatim}

One can measure photometry for custom filters using the \sphinxcode{custom} parameter:

\begin{Verbatim}[commandchars=\\\{\}]
\PYG{g+gp}{\PYGZgt{}\PYGZgt{}\PYGZgt{} }\PYG{k+kn}{import} \PYG{n+nn}{numpy}
\PYG{g+gp}{\PYGZgt{}\PYGZgt{}\PYGZgt{} }\PYG{n}{fwave}\PYG{p}{,}\PYG{n}{ftrans} \PYG{o}{=} \PYG{n}{numpy}\PYG{o}{.}\PYG{n}{genfromtxt}\PYG{p}{(}\PYG{l+s+s1}{\PYGZsq{}}\PYG{l+s+s1}{my\PYGZus{}custom\PYGZus{}filter.txt}\PYG{l+s+s1}{\PYGZsq{}}\PYG{p}{,}\PYG{n}{unpack}\PYG{o}{=}\PYG{k+kc}{True}\PYG{p}{)}
\PYG{g+gp}{\PYGZgt{}\PYGZgt{}\PYGZgt{} }\PYG{n}{filt} \PYG{o}{=} \PYG{n}{numpy}\PYG{o}{.}\PYG{n}{vstack}\PYG{p}{(}\PYG{p}{(}\PYG{n}{fwave}\PYG{p}{,}\PYG{n}{ftans}\PYG{p}{)}\PYG{p}{)}
\PYG{g+gp}{\PYGZgt{}\PYGZgt{}\PYGZgt{} }\PYG{n}{splat}\PYG{o}{.}\PYG{n}{filterMag}\PYG{p}{(}\PYG{n}{sp}\PYG{p}{,}\PYG{l+s+s1}{\PYGZsq{}}\PYG{l+s+s1}{Custom}\PYG{l+s+s1}{\PYGZsq{}}\PYG{p}{,}\PYG{n}{custom} \PYG{o}{=} \PYG{n}{filt}\PYG{p}{)}
\PYG{g+go}{        (13.097348489365396, 0.046530636178618558)}
\end{Verbatim}

or define a simple notch filter with the two end wavelengthts:

\begin{Verbatim}[commandchars=\\\{\}]
\PYG{g+gp}{\PYGZgt{}\PYGZgt{}\PYGZgt{} }\PYG{n}{splat}\PYG{o}{.}\PYG{n}{filterMag}\PYG{p}{(}\PYG{n}{sp}\PYG{p}{,}\PYG{l+s+s1}{\PYGZsq{}}\PYG{l+s+s1}{Custom}\PYG{l+s+s1}{\PYGZsq{}}\PYG{p}{,}\PYG{n}{notch}\PYG{o}{=}\PYG{p}{[}\PYG{l+m+mf}{1.2}\PYG{p}{,}\PYG{l+m+mf}{1.3}\PYG{p}{]}\PYG{p}{)}
\PYG{g+go}{        (14.301864415761377, 0.031774478113182188)}
\end{Verbatim}

Finally, to flux calibrate a spectrum to a given magnitude, use the Spectrum object's built in {\color{red}\bfseries{}{}`{}`}fluxCalibrate(){}`{}`\_ method:

\begin{Verbatim}[commandchars=\\\{\}]
\PYG{g+gp}{\PYGZgt{}\PYGZgt{}\PYGZgt{} }\PYG{n}{sp}\PYG{o}{.}\PYG{n}{fluxCalibrate}\PYG{p}{(}\PYG{l+s+s1}{\PYGZsq{}}\PYG{l+s+s1}{2MASS J}\PYG{l+s+s1}{\PYGZsq{}}\PYG{p}{,}\PYG{l+m+mf}{14.0}\PYG{p}{)}
\end{Verbatim}

This routine can take \sphinxcode{absolute} as a parameter flag to indicate that the spectra are absolutely flux calibrated:

\begin{Verbatim}[commandchars=\\\{\}]
\PYG{g+gp}{\PYGZgt{}\PYGZgt{}\PYGZgt{} }\PYG{n}{sp}\PYG{o}{.}\PYG{n}{fluxCalibrate}\PYG{p}{(}\PYG{l+s+s1}{\PYGZsq{}}\PYG{l+s+s1}{2MASS J}\PYG{l+s+s1}{\PYGZsq{}}\PYG{p}{,}\PYG{n}{splat}\PYG{o}{.}\PYG{n}{typeToMag}\PYG{p}{(}\PYG{l+s+s1}{\PYGZsq{}}\PYG{l+s+s1}{L5}\PYG{l+s+s1}{\PYGZsq{}}\PYG{p}{,}\PYG{l+s+s1}{\PYGZsq{}}\PYG{l+s+s1}{2MASS J}\PYG{l+s+s1}{\PYGZsq{}}\PYG{p}{)}\PYG{p}{[}\PYG{l+m+mi}{0}\PYG{p}{]}\PYG{p}{,}\PYG{n}{absolute}\PYG{o}{=}\PYG{k+kc}{True}\PYG{p}{)}
\PYG{g+gp}{\PYGZgt{}\PYGZgt{}\PYGZgt{} }\PYG{n}{sp}\PYG{o}{.}\PYG{n}{fscale}
\PYG{g+go}{        \PYGZsq{}Absolute\PYGZsq{}}
\end{Verbatim}


\subsection{Classification}
\label{splat:classification}
SPLAT contains several different methods for classifying a spectrum:
\begin{itemize}
\item {} 
Classifying by Indices

\end{itemize}

SPLAT contains the spectral index/spectral type classification relations from the following studies:
\begin{itemize}
\item {} 
\href{http://adsabs.harvard.edu/abs/2001AJ....121.1710R}{Reid et al. (2001)}

\item {} 
\href{http://adsabs.harvard.edu/abs/2001ApJ...552L.147T}{Testi et al. (2001)}

\item {} 
\href{http://adsabs.harvard.edu/abs/2007ApJ...657..511A}{Allers et al. (2007)}

\item {} 
\href{http://adsabs.harvard.edu/abs/2007ApJ...659..655B}{Burgasser (2007)}.

\end{itemize}

These can be accessed through the {\color{red}\bfseries{}{}`{}`}classifyByIndices{}`{}`\_ routine, which returns the average subtype and uncertainty:

\begin{Verbatim}[commandchars=\\\{\}]
\PYG{g+gp}{\PYGZgt{}\PYGZgt{}\PYGZgt{} }\PYG{n}{sp} \PYG{o}{=} \PYG{n}{splat}\PYG{o}{.}\PYG{n}{getSpectrum}\PYG{p}{(}\PYG{n}{shortname}\PYG{o}{=}\PYG{l+s+s1}{\PYGZsq{}}\PYG{l+s+s1}{0559\PYGZhy{}1404}\PYG{l+s+s1}{\PYGZsq{}}\PYG{p}{)}\PYG{p}{[}\PYG{l+m+mi}{0}\PYG{p}{]}
\PYG{g+gp}{\PYGZgt{}\PYGZgt{}\PYGZgt{} }\PYG{n}{splat}\PYG{o}{.}\PYG{n}{classifyByIndex}\PYG{p}{(}\PYG{n}{sp}\PYG{p}{,} \PYG{n}{string}\PYG{o}{=}\PYG{k+kc}{True}\PYG{p}{,} \PYG{n+nb}{set}\PYG{o}{=}\PYG{l+s+s1}{\PYGZsq{}}\PYG{l+s+s1}{burgasser}\PYG{l+s+s1}{\PYGZsq{}}\PYG{p}{,} \PYG{n+nb}{round}\PYG{o}{=}\PYG{k+kc}{True}\PYG{p}{)}
\PYG{g+go}{        (\PYGZsq{}T4.5\PYGZsq{}, 0.2562934083414341)}
\end{Verbatim}

Using the \sphinxcode{allmeasures} parameter provides the index values and individual index spectral types in a dictionary:

\begin{Verbatim}[commandchars=\\\{\}]
\PYG{g+gp}{\PYGZgt{}\PYGZgt{}\PYGZgt{} }\PYG{n}{sp} \PYG{o}{=} \PYG{n}{splat}\PYG{o}{.}\PYG{n}{getSpectrum}\PYG{p}{(}\PYG{n}{shortname}\PYG{o}{=}\PYG{l+s+s1}{\PYGZsq{}}\PYG{l+s+s1}{2320+4123}\PYG{l+s+s1}{\PYGZsq{}}\PYG{p}{)}\PYG{p}{[}\PYG{l+m+mi}{0}\PYG{p}{]}
\PYG{g+gp}{\PYGZgt{}\PYGZgt{}\PYGZgt{} }\PYG{n}{splat}\PYG{o}{.}\PYG{n}{classifyByIndex}\PYG{p}{(}\PYG{n}{sp}\PYG{p}{,} \PYG{n+nb}{set}\PYG{o}{=}\PYG{l+s+s1}{\PYGZsq{}}\PYG{l+s+s1}{reid}\PYG{l+s+s1}{\PYGZsq{}}\PYG{p}{,} \PYG{n}{allmeasures}\PYG{o}{=}\PYG{k+kc}{True}\PYG{p}{)}
\PYG{g+go}{        \PYGZob{}\PYGZsq{}H2O\PYGZhy{}A\PYGZsq{}: \PYGZob{}\PYGZsq{}index\PYGZsq{}: 0.76670417987511119,}
\PYG{g+go}{          \PYGZsq{}index\PYGZus{}e\PYGZsq{}: 0.76670417987511119,}
\PYG{g+go}{          \PYGZsq{}spt\PYGZsq{}: 18.791162413674424,}
\PYG{g+go}{          \PYGZsq{}spt\PYGZus{}e\PYGZsq{}: 1.1944901925036935\PYGZcb{},}
\PYG{g+go}{         \PYGZsq{}H2O\PYGZhy{}B\PYGZsq{}: \PYGZob{}\PYGZsq{}index\PYGZsq{}: 0.83184397268498511,}
\PYG{g+go}{          \PYGZsq{}index\PYGZus{}e\PYGZsq{}: 0.83184397268498511,}
\PYG{g+go}{          \PYGZsq{}spt\PYGZsq{}: 19.956823648632948,}
\PYG{g+go}{          \PYGZsq{}spt\PYGZus{}e\PYGZsq{}: 1.0460714823631427\PYGZcb{},}
\PYG{g+go}{         \PYGZsq{}result\PYGZsq{}: (\PYGZsq{}M9.5\PYGZsq{}, 0.78695752933890462)\PYGZcb{}}
\end{Verbatim}
\begin{itemize}
\item {} 
Classifying by Standards

\end{itemize}

SPLAT contains spectral standards for dwarf classes M0 through T9, drawn from \href{http://adsabs.harvard.edu/abs/2006ApJ...637.1067B}{Burgasser et al. (2006)}, \href{http://adsabs.harvard.edu/abs/2010ApJS..190..100K}{Kirkpatrick et al. (2010)} and \href{http://adsabs.harvard.edu/abs/2011ApJ...743...50C}{Cushing et al. (2011)}. There are also M and L subdwarf and M extreme subdwarf standards.  These may be used to infer spectral classifications by ``closest match'', using all or part of the near-infrared spectrum.

The routine for this is {\color{red}\bfseries{}{}`{}`}classifyByStandard{}`{}`\_, which by default simply matches to the best-fitting standard:

\begin{Verbatim}[commandchars=\\\{\}]
\PYG{g+gp}{\PYGZgt{}\PYGZgt{}\PYGZgt{} }\PYG{n}{sp} \PYG{o}{=} \PYG{n}{splat}\PYG{o}{.}\PYG{n}{getSpectrum}\PYG{p}{(}\PYG{n}{shortname}\PYG{o}{=}\PYG{l+s+s1}{\PYGZsq{}}\PYG{l+s+s1}{0805+4812}\PYG{l+s+s1}{\PYGZsq{}}\PYG{p}{)}\PYG{p}{[}\PYG{l+m+mi}{0}\PYG{p}{]}
\PYG{g+gp}{\PYGZgt{}\PYGZgt{}\PYGZgt{} }\PYG{n}{splat}\PYG{o}{.}\PYG{n}{classifyByStandard}\PYG{p}{(}\PYG{n}{sp}\PYG{p}{)}
\PYG{g+go}{        (\PYGZsq{}T0.0\PYGZsq{}, 0.5)}
\end{Verbatim}

You can also return an uncertainty-weighted mean classifiction using average=True:

\begin{Verbatim}[commandchars=\\\{\}]
\PYG{g+gp}{\PYGZgt{}\PYGZgt{}\PYGZgt{} }\PYG{n}{splat}\PYG{o}{.}\PYG{n}{classifyByStandard}\PYG{p}{(}\PYG{n}{sp}\PYG{p}{,}\PYG{n}{average}\PYG{o}{=}\PYG{k+kc}{True}\PYG{p}{)}
\PYG{g+go}{    (\PYGZsq{}L7.0::\PYGZsq{}, 2.1064575737396338)}
\end{Verbatim}

and fit to specific regions using either the \sphinxcode{fit\_ranges} parameter or setting \sphinxcode{method='kirkpatrick'} to conform with the \href{http://adsabs.harvard.edu/abs/2010ApJS..190..100K}{Kirkpatrick et al. (2010)} method of near-infrared spectral classification:

\begin{Verbatim}[commandchars=\\\{\}]
\PYG{g+gp}{\PYGZgt{}\PYGZgt{}\PYGZgt{} }\PYG{n}{splat}\PYG{o}{.}\PYG{n}{classifyByStandard}\PYG{p}{(}\PYG{n}{sp}\PYG{p}{,}\PYG{n}{method}\PYG{o}{=}\PYG{l+s+s1}{\PYGZsq{}}\PYG{l+s+s1}{kirkpatrick}\PYG{l+s+s1}{\PYGZsq{}}\PYG{p}{)}
\PYG{g+go}{    (\PYGZsq{}L7.0\PYGZsq{}, 0.5)}
\end{Verbatim}

Subdwarf and extreme subdwarf standards can be accessed by setting the \sphinxcode{sd} or \sphinxcode{esd} parameters to True.  Finally, setting \sphinxcode{plot} to True will bring up a comparison plot between the source and best fit standard.

\begin{Verbatim}[commandchars=\\\{\}]
\PYG{g+gp}{\PYGZgt{}\PYGZgt{}\PYGZgt{} }\PYG{n}{splat}\PYG{o}{.}\PYG{n}{classifyByStandard}\PYG{p}{(}\PYG{n}{sp}\PYG{p}{,}\PYG{n}{method}\PYG{o}{=}\PYG{l+s+s1}{\PYGZsq{}}\PYG{l+s+s1}{kirkpatrick}\PYG{l+s+s1}{\PYGZsq{}}\PYG{p}{,}\PYG{n}{plot}\PYG{o}{=}\PYG{k+kc}{True}\PYG{p}{)}
\PYG{g+go}{    (\PYGZsq{}L7.0\PYGZsq{}, 0.5)}
\end{Verbatim}

\noindent{\hspace*{\fill}\sphinxincludegraphics{{_images/classifyByStandard_example1}.png}\hspace*{\fill}}

Note that the first time you run classifyByStandard, the standards must be initially read in to the dictionaries \sphinxcode{splat.SPEX\_STDS}, \sphinxcode{splat.SPEX\_SD\_STDS} and \sphinxcode{splat.SPEX\_ESD\_ STDS}. This can be prompted using the \sphinxcode{initiateStandards()} routine:

\begin{Verbatim}[commandchars=\\\{\}]
\PYG{g+gp}{\PYGZgt{}\PYGZgt{}\PYGZgt{} }\PYG{n}{splat}\PYG{o}{.}\PYG{n}{initiateStandards}\PYG{p}{(}\PYG{p}{)}
\end{Verbatim}

One the standards are loaded, subsequent calls to \sphinxcode{classifyByStandard} are much faster.
\begin{itemize}
\item {} 
Classifying by Templates

\end{itemize}

You can also classify sources by comparing to individual template spectra in the library. The \sphinxcode{classifyByTemplate{}`{}`\_ routine behaves similarly to {}`{}`classifyByStandard{}`{}`\_, but has the option of returning a dictionary of the {}`{}`nbest} best matches sorted by whatever statistic is desired (set with the \sphinxcode{statistic} parameter; see {\color{red}\bfseries{}{}`{}`}compareSpectra{}`{}`\_).  Because each template must be read in, it is strongly recommended that users downselect the templates using keywords associated with {\color{red}\bfseries{}{}`{}`}searchLibrary{}`{}`\_:

\begin{Verbatim}[commandchars=\\\{\}]
\PYG{g+gp}{\PYGZgt{}\PYGZgt{}\PYGZgt{} }\PYG{n}{sp} \PYG{o}{=} \PYG{n}{splat}\PYG{o}{.}\PYG{n}{getSpectrum}\PYG{p}{(}\PYG{n}{shortname}\PYG{o}{=}\PYG{l+s+s1}{\PYGZsq{}}\PYG{l+s+s1}{1507\PYGZhy{}1627}\PYG{l+s+s1}{\PYGZsq{}}\PYG{p}{)}\PYG{p}{[}\PYG{l+m+mi}{0}\PYG{p}{]}
\PYG{g+gp}{\PYGZgt{}\PYGZgt{}\PYGZgt{} }\PYG{n}{result} \PYG{o}{=} \PYG{n}{splat}\PYG{o}{.}\PYG{n}{classifyByTemplate}\PYG{p}{(}\PYG{n}{sp}\PYG{p}{,}\PYG{n}{spt}\PYG{o}{=}\PYG{p}{[}\PYG{l+m+mi}{24}\PYG{p}{,}\PYG{l+m+mi}{26}\PYG{p}{]}\PYG{p}{,}\PYG{n}{nbest}\PYG{o}{=}\PYG{l+m+mi}{5}\PYG{p}{)}
\PYG{g+go}{        Too many templates (1819) for classifyByTemplate; set force=True to override this}
\PYG{g+gp}{\PYGZgt{}\PYGZgt{}\PYGZgt{} }\PYG{n}{result} \PYG{o}{=} \PYG{n}{splat}\PYG{o}{.}\PYG{n}{classifyByTemplate}\PYG{p}{(}\PYG{n}{sp}\PYG{p}{,}\PYG{n}{spt}\PYG{o}{=}\PYG{p}{[}\PYG{l+m+mi}{24}\PYG{p}{,}\PYG{l+m+mi}{26}\PYG{p}{]}\PYG{p}{,}\PYG{n}{nbest}\PYG{o}{=}\PYG{l+m+mi}{5}\PYG{p}{)}
\PYG{g+go}{    Too many templates (210) for classifyByTemplate; set force=True to override this}
\PYG{g+gp}{\PYGZgt{}\PYGZgt{}\PYGZgt{} }\PYG{n}{result} \PYG{o}{=} \PYG{n}{splat}\PYG{o}{.}\PYG{n}{classifyByTemplate}\PYG{p}{(}\PYG{n}{sp}\PYG{p}{,}\PYG{n}{spt}\PYG{o}{=}\PYG{p}{[}\PYG{l+m+mi}{24}\PYG{p}{,}\PYG{l+m+mi}{26}\PYG{p}{]}\PYG{p}{,}\PYG{n}{snr}\PYG{o}{=}\PYG{l+m+mi}{80}\PYG{p}{,}\PYG{n}{nbest}\PYG{o}{=}\PYG{l+m+mi}{5}\PYG{p}{,}\PYG{n}{verbose}\PYG{o}{=}\PYG{k+kc}{True}\PYG{p}{)}
\PYG{g+go}{    Comparing to 58 templates}
\PYG{g+go}{    LHS 102B L5.0 10488.1100432 11.0947838116}
\PYG{g+go}{    SDSS J001608.44\PYGZhy{}004302.3 L5.5 15468.6209466 274.797693706}
\PYG{g+go}{        2MASS J00250365+4759191AB L4.0 28458.3112163 4.19176819291}
\PYG{g+go}{        2MASS J00332386\PYGZhy{}1521309 L4.0 29141.2681221 2.2567421444e\PYGZhy{}14}
\PYG{g+go}{    ...}
\PYG{g+go}{    Best match = DENIS\PYGZhy{}P J153941.96\PYGZhy{}052042.4 with spectral type L4:}
\PYG{g+go}{    Mean spectral type = L4.5+/\PYGZhy{}0.724296125146}
\end{Verbatim}

Note that the program doesn't proceed automatically if there are more than 100 templates; you can override this using the \sphinxcode{force} parameter:

\begin{Verbatim}[commandchars=\\\{\}]
\PYG{g+gp}{\PYGZgt{}\PYGZgt{}\PYGZgt{} }\PYG{n}{result} \PYG{o}{=} \PYG{n}{splat}\PYG{o}{.}\PYG{n}{classifyByTemplate}\PYG{p}{(}\PYG{n}{sp}\PYG{p}{,}\PYG{n}{spt}\PYG{o}{=}\PYG{p}{[}\PYG{l+m+mi}{24}\PYG{p}{,}\PYG{l+m+mi}{26}\PYG{p}{]}\PYG{p}{,}\PYG{n}{nbest}\PYG{o}{=}\PYG{l+m+mi}{5}\PYG{p}{,}\PYG{n}{force}\PYG{o}{=}\PYG{k+kc}{True}\PYG{p}{,}\PYG{n}{verbose}\PYG{o}{=}\PYG{k+kc}{True}\PYG{p}{)}
\PYG{g+go}{        Comparing to 210 templates}
\PYG{g+go}{        This may take some time!}
\PYG{g+go}{        SDSS J000112.18+153535.5 L4.0 24551.836698 14.3533608111}
\PYG{g+go}{        SDSS J000250.98+245413.8 L5.5 15517.679593 51.274551132}
\PYG{g+go}{        ...}
\PYG{g+go}{        Best match = 2MASS J17461199+5034036 with spectral type L5}
\PYG{g+go}{        Mean spectral type = L5.0+/\PYGZhy{}0.42094300506}
\end{Verbatim}
\begin{description}
\item[{You can also downselect templates using the \sphinxcode{select} parameter for the following predefined template sets:}] \leavevmode\begin{itemize}
\item {} 
select = \sphinxtitleref{m dwarf}: fit to M dwarfs only

\item {} 
select = \sphinxtitleref{l dwarf}: fit to M dwarfs only

\item {} 
select = \sphinxtitleref{t dwarf}: fit to M dwarfs only

\item {} 
select = \sphinxtitleref{vlm}: fit to M7-T9 dwarfs

\item {} 
select = \sphinxtitleref{optical}: only optical classifications

\item {} 
select = \sphinxtitleref{high sn}: median S/N greater than 100

\item {} 
select = \sphinxtitleref{young}: only young/low surface gravity dwarfs

\item {} 
select = \sphinxtitleref{companion}: only companion dwarfs

\item {} 
select = \sphinxtitleref{subdwarf}: only subdwarfs

\item {} 
select = \sphinxtitleref{single}: only dwarfs not indicated a binaries

\item {} 
select = \sphinxtitleref{spectral binaries}: only dwarfs indicated to be spectral binaries

\item {} 
select = \sphinxtitleref{standard}: only spectral standards (in this case it is better to use the {\color{red}\bfseries{}{}`{}`}classifyByStandard{}`{}`\_ routine instead)

\end{itemize}

\end{description}

These sets can be combined:

\begin{Verbatim}[commandchars=\\\{\}]
\PYG{g+gp}{\PYGZgt{}\PYGZgt{}\PYGZgt{} }\PYG{n}{result} \PYG{o}{=} \PYG{n}{splat}\PYG{o}{.}\PYG{n}{classifyByTemplate}\PYG{p}{(}\PYG{n}{sp}\PYG{p}{,}\PYG{n}{select}\PYG{o}{=}\PYG{l+s+s1}{\PYGZsq{}}\PYG{l+s+s1}{l dwarfs, young}\PYG{l+s+s1}{\PYGZsq{}}\PYG{p}{,}\PYG{n}{nbest}\PYG{o}{=}\PYG{l+m+mi}{5}\PYG{p}{,}\PYG{n}{verbose}\PYG{o}{=}\PYG{k+kc}{True}\PYG{p}{)}
\PYG{g+go}{        Comparing to 79 templates}
\PYG{g+go}{        SDSS J000112.18+153535.5 L4.0 24551.836698 14.3533608111}
\PYG{g+go}{        2MASS J00193927\PYGZhy{}3724392 L3.0 10299.0508807 16.2807901643}
\PYG{g+go}{        2MASS J0028208+224905 L5.0 19350.1803596 12.281257449}
\PYG{g+go}{        ...}
\PYG{g+go}{        Best match = 2MASS J10224821+5825453 with spectral type L1beta}
\PYG{g+go}{        Mean spectral type = L0.5+/\PYGZhy{}0.86022832423}
\end{Verbatim}
\begin{itemize}
\item {} 
Gravity Classification

\end{itemize}

The {\color{red}\bfseries{}{}`{}`}classifyGravity{}`{}`\_ routine uses the index-based method of \href{http://adsabs.harvard.edu/abs/2013ApJ...772...79A}{Allers \& Liu (2013)} to determine gravity scores from VO, FeH, K I and H-band continuum indices.

\begin{Verbatim}[commandchars=\\\{\}]
\PYG{g+gp}{\PYGZgt{}\PYGZgt{}\PYGZgt{} }\PYG{n}{sp} \PYG{o}{=} \PYG{n}{splat}\PYG{o}{.}\PYG{n}{getSpectrum}\PYG{p}{(}\PYG{n}{shortname}\PYG{o}{=}\PYG{l+s+s1}{\PYGZsq{}}\PYG{l+s+s1}{1507\PYGZhy{}1627}\PYG{l+s+s1}{\PYGZsq{}}\PYG{p}{)}\PYG{p}{[}\PYG{l+m+mi}{0}\PYG{p}{]}
\PYG{g+gp}{\PYGZgt{}\PYGZgt{}\PYGZgt{} }\PYG{n}{splat}\PYG{o}{.}\PYG{n}{classifyGravity}\PYG{p}{(}\PYG{n}{sp}\PYG{p}{)}
\PYG{g+go}{    FLD\PYGZhy{}G}
\end{Verbatim}

In its default mode it also determines the classification of the source using the
\href{http://adsabs.harvard.edu/abs/2007ApJ...657..511A}{Allers et al. (2007)} index-based scheme, but you can also force an spectral type by setting the \sphinxcode{spt} parameter:

\begin{Verbatim}[commandchars=\\\{\}]
\PYG{g+gp}{\PYGZgt{}\PYGZgt{}\PYGZgt{} }\PYG{n}{splat}\PYG{o}{.}\PYG{n}{classifyGravity}\PYG{p}{(}\PYG{n}{sp}\PYG{p}{,}\PYG{n}{spt}\PYG{o}{=}\PYG{l+s+s1}{\PYGZsq{}}\PYG{l+s+s1}{L5}\PYG{l+s+s1}{\PYGZsq{}}\PYG{p}{)}
\PYG{g+go}{    FLD\PYGZhy{}G}
\end{Verbatim}

Finally, the routine will return a dictionary of all index scores by setting the \sphinxcode{allscores} parameter to True:

\begin{Verbatim}[commandchars=\\\{\}]
\PYG{g+gp}{\PYGZgt{}\PYGZgt{}\PYGZgt{} }\PYG{n}{result} \PYG{o}{=} \PYG{n}{splat}\PYG{o}{.}\PYG{n}{classifyGravity}\PYG{p}{(}\PYG{n}{sp}\PYG{p}{,} \PYG{n}{allscores} \PYG{o}{=} \PYG{k+kc}{True}\PYG{p}{,} \PYG{n}{verbose}\PYG{o}{=}\PYG{k+kc}{True}\PYG{p}{)}
\PYG{g+go}{    Gravity Classification:}
\PYG{g+go}{        SpT = L4.0}
\PYG{g+go}{        VO\PYGZhy{}z: 1.012+/\PYGZhy{}0.029 =\PYGZgt{} 0.0}
\PYG{g+go}{        FeH\PYGZhy{}z: 1.299+/\PYGZhy{}0.031 =\PYGZgt{} 1.0}
\PYG{g+go}{        H\PYGZhy{}cont: 0.859+/\PYGZhy{}0.032 =\PYGZgt{} 0.0}
\PYG{g+go}{        KI\PYGZhy{}J: 1.114+/\PYGZhy{}0.038 =\PYGZgt{} 1.0}
\PYG{g+go}{        Gravity Class = FLD\PYGZhy{}G}
\PYG{g+gp}{\PYGZgt{}\PYGZgt{}\PYGZgt{} }\PYG{n}{result}
\PYG{g+go}{    \PYGZob{}\PYGZsq{}FeH\PYGZhy{}z\PYGZsq{}: 1.0,}
\PYG{g+go}{     \PYGZsq{}H\PYGZhy{}cont\PYGZsq{}: 0.0,}
\PYG{g+go}{     \PYGZsq{}KI\PYGZhy{}J\PYGZsq{}: 1.0,}
\PYG{g+go}{     \PYGZsq{}VO\PYGZhy{}z\PYGZsq{}: 0.0,}
\PYG{g+go}{     \PYGZsq{}gravity\PYGZus{}class\PYGZsq{}: \PYGZsq{}FLD\PYGZhy{}G\PYGZsq{},}
\PYG{g+go}{     \PYGZsq{}score\PYGZsq{}: 0.5,}
\PYG{g+go}{     \PYGZsq{}spt\PYGZsq{}: \PYGZsq{}L4.0\PYGZsq{}\PYGZcb{}}
\end{Verbatim}


\subsection{Potentially Useful Program Constants}
\label{splat:potentially-useful-program-constants}\begin{description}
\item[{\sphinxcode{splat.DB\_SOURCES}}] \leavevmode
An Astropy Table object containing the Source Database

\item[{\sphinxcode{splat.DB\_SPECTRA}}] \leavevmode
An Astropy Table object containing the Spectrum Database

\item[{\sphinxcode{splat.SPEX\_STDS}}] \leavevmode
A dictionary containing Spectrum objects of the M0-T9 dwarf standards; this dictionary is
populated through calls to \sphinxcode{splat.getStandard}. A standard Spectrum object can be accessed
by using the spectral type as the referring key:

\end{description}

\begin{Verbatim}[commandchars=\\\{\}]
\PYG{g+gp}{\PYGZgt{}\PYGZgt{}\PYGZgt{} }\PYG{n}{sp} \PYG{o}{=} \PYG{n}{splat}\PYG{o}{.}\PYG{n}{getStandard}\PYG{p}{(}\PYG{l+s+s1}{\PYGZsq{}}\PYG{l+s+s1}{M0}\PYG{l+s+s1}{\PYGZsq{}}\PYG{p}{)}\PYG{p}{[}\PYG{l+m+mi}{0}\PYG{p}{]}             \PYG{c+c1}{\PYGZsh{} both are Spectrum objects of Gliese 270}
\PYG{g+gp}{\PYGZgt{}\PYGZgt{}\PYGZgt{} }\PYG{n}{sp} \PYG{o}{=} \PYG{n}{splat}\PYG{o}{.}\PYG{n}{SPEX\PYGZus{}STDS}\PYG{p}{[}\PYG{l+s+s1}{\PYGZsq{}}\PYG{l+s+s1}{M0.0}\PYG{l+s+s1}{\PYGZsq{}}\PYG{p}{]}                \PYG{c+c1}{\PYGZsh{} note the mandatory decimal}
\end{Verbatim}
\begin{quote}

Available standards can be accessed through the command:
\end{quote}

\begin{Verbatim}[commandchars=\\\{\}]
\PYG{g+gp}{\PYGZgt{}\PYGZgt{}\PYGZgt{} }\PYG{n}{splat}\PYG{o}{.}\PYG{n}{SPEX\PYGZus{}STDS}\PYG{o}{.}\PYG{n}{keys}\PYG{p}{(}\PYG{p}{)}
\end{Verbatim}
\begin{description}
\item[{\sphinxcode{splat.SPEX\_SD\_STDS}}] \leavevmode
Same as \sphinxcode{splat.SPEX\_STDS} for subdwarf standards

\item[{\sphinxcode{splat.SPEX\_ESD\_STDS}}] \leavevmode
Same as \sphinxcode{splat.SPEX\_STDS} for extreme subdwarf standards

\item[{\sphinxcode{splat.FILTERS}}] \leavevmode
A dictionary containing information on all of the filters used in SPLAT photometry. The command:

\end{description}


\subsection{Additional Programs}
\label{splat:additional-programs}\begin{itemize}
\item {} 
\DUrole{xref,std,std-ref}{genindex}

\item {} 
\DUrole{xref,std,std-ref}{modindex}

\item {} 
\DUrole{xref,std,std-ref}{search}

\end{itemize}


\section{SPLAT Plotting Routines}
\label{splat_plot::doc}\label{splat_plot:splat-plotting-routines}
These are the primary plotting routines for SPLAT, which allow visualization of spectral data and comparison to other templates and models. Additional routines to visualize indices on the spectral data and index values themselves are currently under development


\subsection{plotSpectrum}
\label{splat_plot:plotspectrum}
SPLAT allows spectrophotometry of spectra using common filters in the red optical and near-infrared. The filter transmission files are stored in the SPLAT reference library, and are accessed by name.  A list of current filters can be made by through the {\color{red}\bfseries{}{}`{}`}plotSpectrum(){}`{}`\_ routine:

The core plotting function is {\color{red}\bfseries{}{}`{}`}plotSpectrum(){}`{}`\_, which allows flexible methods for displaying single or groups of spectra, and comparing spectra to each other. This codes are built around the routines in matplotlib, and the API has the full list of options.


\subsubsection{Simple plots}
\label{splat_plot:simple-plots}
Simple


\subsubsection{Plotting multiple spectra}
\label{splat_plot:plotting-multiple-spectra}
Spectra can be stacked on top of each other using the \sphinxcode{stack} parameter, which is a numerical value that indicates the offset between


\subsubsection{Comparison plots}
\label{splat_plot:comparison-plots}

\subsubsection{Legends, feature labels and other additions}
\label{splat_plot:legends-feature-labels-and-other-additions}\begin{description}
\item[{Specific absorption features can be labeled on the plot setting the \sphinxcode{features} to a list of atoms and molecules. The features currently contained in the code include:}] \leavevmode\begin{itemize}
\item {} 
r'H\$\_2\$O': bands at 0.92-0.95, 1.08-1.20, 1.33-1.55, and 1.72-2.14 micron

\item {} 
r'CH\$\_4\$': bands at 1.10-1.24, 1.28-1.44, 1.60-1.76, and 2.20-2.35 micron

\item {} 
r'CO': band at 2.29-2.39

\item {} 
r'TiO': bands at 0.76-0.80 and 0.825-0.831 micron

\item {} 
r'VO': band at 1.04-1.08 micron

\item {} 
r'FeH': bands at 0.98-1.03, 1.19-1.25, and 1.57-1.64 micron

\item {} 
r'H\$\_2\$': broad absorption over 1.5-2.4 micron

\item {} 
r'H I': lines at 1.004, 1.093, 1.281, 1.944, and 2.166 micron

\item {} 
r'Na I': lines at 0.819, 1.136, and 2.21 micron

\end{itemize}

`nai': \{`label': r'Na I', `type': `line', `wavelengths': {[}{[}0.8186,0.8195{]},{[}1.136,1.137{]},{[}2.206,2.209{]}{]}\}, `na1': \{`label': r'Na I', `type': `line', `wavelengths': {[}{[}0.8186,0.8195{]},{[}1.136,1.137{]},{[}2.206,2.209{]}{]}\}, `mg': \{`label': r'Mg I', `type': `line', `wavelengths': {[}{[}1.7113336,1.7113336{]},{[}1.5745017,1.5770150{]},{[}1.4881595,1.4881847,1.5029098,1.5044356{]},{[}1.1831422,1.2086969{]},{]}\}, `mgi': \{`label': r'Mg I', `type': `line', `wavelengths': {[}{[}1.7113336,1.7113336{]},{[}1.5745017,1.5770150{]},{[}1.4881595,1.4881847,1.5029098,1.5044356{]},{[}1.1831422,1.2086969{]},{]}\}, `mg1': \{`label': r'Mg I', `type': `line', `wavelengths': {[}{[}1.7113336,1.7113336{]},{[}1.5745017,1.5770150{]},{[}1.4881595,1.4881847,1.5029098,1.5044356{]},{[}1.1831422,1.2086969{]},{]}\}, `ca': \{`label': r'Ca I', `type': `line', `wavelengths': {[}{[}2.263110,2.265741{]},{[}1.978219,1.985852,1.986764{]},{[}1.931447,1.945830,1.951105{]}{]}\}, `cai': \{`label': r'Ca I', `type': `line', `wavelengths': {[}{[}2.263110,2.265741{]},{[}1.978219,1.985852,1.986764{]},{[}1.931447,1.945830,1.951105{]}{]}\}, `ca1': \{`label': r'Ca I', `type': `line', `wavelengths': {[}{[}2.263110,2.265741{]},{[}1.978219,1.985852,1.986764{]},{[}1.931447,1.945830,1.951105{]}{]}\}, `caii': \{`label': r'Ca II', `type': `line', `wavelengths': {[}{[}1.184224,1.195301{]},{[}0.985746,0.993409{]}{]}\}, `ca2': \{`label': r'Ca II', `type': `line', `wavelengths': {[}{[}1.184224,1.195301{]},{[}0.985746,0.993409{]}{]}\}, `al': \{`label': r'Al I', `type': `line', `wavelengths': {[}{[}1.672351,1.675511{]},{[}1.3127006,1.3154345{]}{]}\}, `ali': \{`label': r'Al I', `type': `line', `wavelengths': {[}{[}1.672351,1.675511{]},{[}1.3127006,1.3154345{]}{]}\}, `al1': \{`label': r'Al I', `type': `line', `wavelengths': {[}{[}1.672351,1.675511{]},{[}1.3127006,1.3154345{]}{]}\}, `fe': \{`label': r'Fe I', `type': `line', `wavelengths': {[}{[}1.5081407,1.5494570{]},{[}1.25604314,1.28832892{]},{[}1.14254467,1.15967616,1.16107501,1.16414462,1.16931726,1.18860965,1.18873357,1.19763233{]}{]}\}, `fei': \{`label': r'Fe I', `type': `line', `wavelengths': {[}{[}1.5081407,1.5494570{]},{[}1.25604314,1.28832892{]},{[}1.14254467,1.15967616,1.16107501,1.16414462,1.16931726,1.18860965,1.18873357,1.19763233{]}{]}\}, `fe1': \{`label': r'Fe I', `type': `line', `wavelengths': {[}{[}1.5081407,1.5494570{]},{[}1.25604314,1.28832892{]},{[}1.14254467,1.15967616,1.16107501,1.16414462,1.16931726,1.18860965,1.18873357,1.19763233{]}{]}\}, `k': \{`label': r'K I', `type': `line', `wavelengths': {[}{[}0.7699,0.7665{]},{[}1.169,1.177{]},{[}1.244,1.252{]}{]}\}, `ki': \{`label': r'K I', `type': `line', `wavelengths': {[}{[}0.7699,0.7665{]},{[}1.169,1.177{]},{[}1.244,1.252{]}{]}\}, `k1': \{`label': r'K I', `type': `line', `wavelengths': {[}{[}0.7699,0.7665{]},{[}1.169,1.177{]},{[}1.244,1.252{]}{]}\}\}
`sb': \{`label': r'*', `type': `band', `wavelengths': {[}{[}1.6,1.64{]}{]}\}, 

\end{description}
\begin{itemize}
\item {} 
\sphinxcode{telluric} - labels the regions of strong telluric absorption

\end{itemize}

Legends are handled in the same manner


\subsection{Examples}
\label{splat_plot:examples}\begin{quote}
\begin{description}
\item[{\textbf{Example 1: A simple view of a random spectrum}}] \leavevmode
This example shows various ways of displaying a random spectrum in the library

\begin{Verbatim}[commandchars=\\\{\}]
\PYG{g+gp}{\PYGZgt{}\PYGZgt{}\PYGZgt{} }\PYG{k+kn}{import} \PYG{n+nn}{splat}
\PYG{g+gp}{\PYGZgt{}\PYGZgt{}\PYGZgt{} }\PYG{n}{spc} \PYG{o}{=} \PYG{n}{splat}\PYG{o}{.}\PYG{n}{getSpectrum}\PYG{p}{(}\PYG{n}{spt} \PYG{o}{=} \PYG{l+s+s1}{\PYGZsq{}}\PYG{l+s+s1}{T5}\PYG{l+s+s1}{\PYGZsq{}}\PYG{p}{,} \PYG{n}{lucky}\PYG{o}{=}\PYG{k+kc}{True}\PYG{p}{)}\PYG{p}{[}\PYG{l+m+mi}{0}\PYG{p}{]}   \PYG{c+c1}{\PYGZsh{} select random spectrum}
\PYG{g+gp}{\PYGZgt{}\PYGZgt{}\PYGZgt{} }\PYG{n}{spc}\PYG{o}{.}\PYG{n}{plot}\PYG{p}{(}\PYG{p}{)}                                                   \PYG{c+c1}{\PYGZsh{} this automatically generates a \PYGZdq{}quicklook\PYGZdq{} plot}
\PYG{g+gp}{\PYGZgt{}\PYGZgt{}\PYGZgt{} }\PYG{n}{splat}\PYG{o}{.}\PYG{n}{plotSpectrum}\PYG{p}{(}\PYG{n}{spc}\PYG{p}{)}                                      \PYG{c+c1}{\PYGZsh{} does the same thing}
\PYG{g+gp}{\PYGZgt{}\PYGZgt{}\PYGZgt{} }\PYG{n}{splat}\PYG{o}{.}\PYG{n}{plotSpectrum}\PYG{p}{(}\PYG{n}{spc}\PYG{p}{,}\PYG{n}{uncertainty}\PYG{o}{=}\PYG{k+kc}{True}\PYG{p}{,}\PYG{n}{tdwarf}\PYG{o}{=}\PYG{k+kc}{True}\PYG{p}{)}     \PYG{c+c1}{\PYGZsh{} show the spectrum uncertainty and T dwarf absorption features}
\end{Verbatim}

The last plot should look like the following:

\end{description}

\noindent{\hspace*{\fill}\sphinxincludegraphics{{plot_example1}.png}\hspace*{\fill}}
\begin{description}
\item[{\textbf{Example 2: Compare two spectra}}] \leavevmode
Optimally scale and compare two spectra.

\begin{Verbatim}[commandchars=\\\{\}]
\PYG{g+gp}{\PYGZgt{}\PYGZgt{}\PYGZgt{} }\PYG{k+kn}{import} \PYG{n+nn}{splat}
\PYG{g+gp}{\PYGZgt{}\PYGZgt{}\PYGZgt{} }\PYG{n}{spc} \PYG{o}{=} \PYG{n}{splat}\PYG{o}{.}\PYG{n}{getSpectrum}\PYG{p}{(}\PYG{n}{spt} \PYG{o}{=} \PYG{l+s+s1}{\PYGZsq{}}\PYG{l+s+s1}{T5}\PYG{l+s+s1}{\PYGZsq{}}\PYG{p}{,} \PYG{n}{lucky}\PYG{o}{=}\PYG{k+kc}{True}\PYG{p}{)}\PYG{p}{[}\PYG{l+m+mi}{0}\PYG{p}{]}   \PYG{c+c1}{\PYGZsh{} select random spectrum}
\PYG{g+gp}{\PYGZgt{}\PYGZgt{}\PYGZgt{} }\PYG{n}{spc2} \PYG{o}{=} \PYG{n}{splat}\PYG{o}{.}\PYG{n}{getSpectrum}\PYG{p}{(}\PYG{n}{spt} \PYG{o}{=} \PYG{l+s+s1}{\PYGZsq{}}\PYG{l+s+s1}{T4}\PYG{l+s+s1}{\PYGZsq{}}\PYG{p}{,} \PYG{n}{lucky}\PYG{o}{=}\PYG{k+kc}{True}\PYG{p}{)}\PYG{p}{[}\PYG{l+m+mi}{0}\PYG{p}{]}  \PYG{c+c1}{\PYGZsh{} read in another random spectrum}
\PYG{g+gp}{\PYGZgt{}\PYGZgt{}\PYGZgt{} }\PYG{n}{comp} \PYG{o}{=} \PYG{n}{splat}\PYG{o}{.}\PYG{n}{compareSpectra}\PYG{p}{(}\PYG{n}{spc}\PYG{p}{,}\PYG{n}{spc2}\PYG{p}{)}        \PYG{c+c1}{\PYGZsh{} compare spectra to get optimal scaling}
\PYG{g+gp}{\PYGZgt{}\PYGZgt{}\PYGZgt{} }\PYG{n}{spc2}\PYG{o}{.}\PYG{n}{scale}\PYG{p}{(}\PYG{n}{comp}\PYG{p}{[}\PYG{l+m+mi}{1}\PYG{p}{]}\PYG{p}{)}                  \PYG{c+c1}{\PYGZsh{} apply optimal scaling}
\PYG{g+gp}{\PYGZgt{}\PYGZgt{}\PYGZgt{} }\PYG{n}{splat}\PYG{o}{.}\PYG{n}{plotSpectrum}\PYG{p}{(}\PYG{n}{spc}\PYG{p}{,}\PYG{n}{spc2}\PYG{p}{,}\PYG{n}{colors}\PYG{o}{=}\PYG{p}{[}\PYG{l+s+s1}{\PYGZsq{}}\PYG{l+s+s1}{black}\PYG{l+s+s1}{\PYGZsq{}}\PYG{p}{,}\PYG{l+s+s1}{\PYGZsq{}}\PYG{l+s+s1}{red}\PYG{l+s+s1}{\PYGZsq{}}\PYG{p}{]}\PYG{p}{,}\PYG{n}{labels}\PYG{o}{=}\PYG{p}{[}\PYG{n}{spc}\PYG{o}{.}\PYG{n}{name}\PYG{p}{,}\PYG{n}{spc2}\PYG{o}{.}\PYG{n}{name}\PYG{p}{]}\PYG{p}{)}     \PYG{c+c1}{\PYGZsh{} show the spectrum uncertainty and T dwarf absorption features}
\end{Verbatim}

\end{description}

\noindent{\hspace*{\fill}\sphinxincludegraphics{{plot_example2}.png}\hspace*{\fill}}
\begin{description}
\item[{\textbf{Example 3: Compare several spectra for a given object}}] \leavevmode
In this case we'll look at all of the spectra of TWA 30B in the library, sorted by year and compare each to the first epoch data. This is an example of using both multiplot and multipage.

\begin{Verbatim}[commandchars=\\\{\}]
\PYG{g+gp}{\PYGZgt{}\PYGZgt{}\PYGZgt{} }\PYG{n}{splist} \PYG{o}{=} \PYG{n}{splat}\PYG{o}{.}\PYG{n}{getSpectrum}\PYG{p}{(}\PYG{n}{name} \PYG{o}{=} \PYG{l+s+s1}{\PYGZsq{}}\PYG{l+s+s1}{TWA 30B}\PYG{l+s+s1}{\PYGZsq{}}\PYG{p}{)}         \PYG{c+c1}{\PYGZsh{} get all spectra of TWA 30B}
\PYG{g+gp}{\PYGZgt{}\PYGZgt{}\PYGZgt{} }\PYG{n}{junk} \PYG{o}{=} \PYG{p}{[}\PYG{n}{sp}\PYG{o}{.}\PYG{n}{normalize}\PYG{p}{(}\PYG{p}{)} \PYG{k}{for} \PYG{n}{sp} \PYG{o+ow}{in} \PYG{n}{splist}\PYG{p}{]}             \PYG{c+c1}{\PYGZsh{} normalize the spectra}
\PYG{g+gp}{\PYGZgt{}\PYGZgt{}\PYGZgt{} }\PYG{n}{dates} \PYG{o}{=} \PYG{p}{[}\PYG{n}{sp}\PYG{o}{.}\PYG{n}{date} \PYG{k}{for} \PYG{n}{sp} \PYG{o+ow}{in} \PYG{n}{splist}\PYG{p}{]}                   \PYG{c+c1}{\PYGZsh{} observation dates}
\PYG{g+gp}{\PYGZgt{}\PYGZgt{}\PYGZgt{} }\PYG{n}{spsort} \PYG{o}{=} \PYG{p}{[}\PYG{n}{s} \PYG{k}{for} \PYG{p}{(}\PYG{n}{d}\PYG{p}{,}\PYG{n}{s}\PYG{p}{)} \PYG{o+ow}{in} \PYG{n+nb}{sorted}\PYG{p}{(}\PYG{n+nb}{zip}\PYG{p}{(}\PYG{n}{dates}\PYG{p}{,}\PYG{n}{splist}\PYG{p}{)}\PYG{p}{)}\PYG{p}{]}   \PYG{c+c1}{\PYGZsh{} sort spectra by dates}
\PYG{g+gp}{\PYGZgt{}\PYGZgt{}\PYGZgt{} }\PYG{n}{dates}\PYG{o}{.}\PYG{n}{sort}\PYG{p}{(}\PYG{p}{)}                                         \PYG{c+c1}{\PYGZsh{} don\PYGZsq{}t forget to sort dates!}
\PYG{g+gp}{\PYGZgt{}\PYGZgt{}\PYGZgt{} }\PYG{n}{splat}\PYG{o}{.}\PYG{n}{plotSpectrum}\PYG{p}{(}\PYG{n}{spsort}\PYG{p}{,}\PYG{n}{multiplot}\PYG{o}{=}\PYG{k+kc}{True}\PYG{p}{,}\PYG{n}{layout}\PYG{o}{=}\PYG{p}{[}\PYG{l+m+mi}{2}\PYG{p}{,}\PYG{l+m+mi}{2}\PYG{p}{]}\PYG{p}{,}\PYG{n}{multipage}\PYG{o}{=}\PYG{k+kc}{True}\PYG{p}{,}\PYGZbs{}   \PYG{c+c1}{\PYGZsh{} here\PYGZsq{}s our plot statement}
\PYG{g+go}{    comparison=spsort[0],uncertainty=True,mdwarf=True,telluric=True,legends=dates,\PYGZbs{}}
\PYG{g+go}{    legendLocation=\PYGZsq{}lower left\PYGZsq{},output=\PYGZsq{}TWA30B.pdf\PYGZsq{})}
\end{Verbatim}

Here is the first page of the resulting 5 page pdf file

\end{description}

\noindent{\hspace*{\fill}\sphinxincludegraphics{{plot_example3}.png}\hspace*{\fill}}
\begin{description}
\item[{\textbf{Example 4: Display the spectra sequence of L dwarfs}}] \leavevmode\begin{quote}

This example uses the list of standard files contained in SPLAT, and illustrates the stack feature
\end{quote}

\begin{Verbatim}[commandchars=\\\{\}]
\PYG{g+gp}{\PYGZgt{}\PYGZgt{}\PYGZgt{} }\PYG{n}{spt} \PYG{o}{=} \PYG{p}{[}\PYG{n}{splat}\PYG{o}{.}\PYG{n}{typeToNum}\PYG{p}{(}\PYG{n}{i}\PYG{o}{+}\PYG{l+m+mi}{20}\PYG{p}{)} \PYG{k}{for} \PYG{n}{i} \PYG{o+ow}{in} \PYG{n+nb}{range}\PYG{p}{(}\PYG{l+m+mi}{10}\PYG{p}{)}\PYG{p}{]} \PYG{c+c1}{\PYGZsh{} generate list of L spectral types}
\PYG{g+gp}{\PYGZgt{}\PYGZgt{}\PYGZgt{} }\PYG{n}{splat}\PYG{o}{.}\PYG{n}{initiateStandards}\PYG{p}{(}\PYG{p}{)}                        \PYG{c+c1}{\PYGZsh{} initiate standards}
\PYG{g+gp}{\PYGZgt{}\PYGZgt{}\PYGZgt{} }\PYG{n}{splist} \PYG{o}{=} \PYG{p}{[}\PYG{n}{splat}\PYG{o}{.}\PYG{n}{SPEX\PYGZus{}STDS}\PYG{p}{[}\PYG{n}{s}\PYG{p}{]} \PYG{k}{for} \PYG{n}{s} \PYG{o+ow}{in} \PYG{n}{spt}\PYG{p}{]}       \PYG{c+c1}{\PYGZsh{} extact just L dwarfs}
\PYG{g+gp}{\PYGZgt{}\PYGZgt{}\PYGZgt{} }\PYG{n}{junk} \PYG{o}{=} \PYG{p}{[}\PYG{n}{sp}\PYG{o}{.}\PYG{n}{normalize}\PYG{p}{(}\PYG{p}{)} \PYG{k}{for} \PYG{n}{sp} \PYG{o+ow}{in} \PYG{n}{splist}\PYG{p}{]}         \PYG{c+c1}{\PYGZsh{} normalize the spectra}
\PYG{g+gp}{\PYGZgt{}\PYGZgt{}\PYGZgt{} }\PYG{n}{labels} \PYG{o}{=} \PYG{p}{[}\PYG{n}{sp}\PYG{o}{.}\PYG{n}{shortname} \PYG{k}{for} \PYG{n}{sp} \PYG{o+ow}{in} \PYG{n}{splist}\PYG{p}{]}         \PYG{c+c1}{\PYGZsh{} set labels to be names}
\PYG{g+gp}{\PYGZgt{}\PYGZgt{}\PYGZgt{} }\PYG{n}{splat}\PYG{o}{.}\PYG{n}{plotSpectrum}\PYG{p}{(}\PYG{n}{splist}\PYG{p}{,}\PYG{n}{figsize}\PYG{o}{=}\PYG{p}{[}\PYG{l+m+mi}{10}\PYG{p}{,}\PYG{l+m+mi}{20}\PYG{p}{]}\PYG{p}{,}\PYG{n}{labels}\PYG{o}{=}\PYG{n}{labels}\PYG{p}{,}\PYG{n}{stack}\PYG{o}{=}\PYG{l+m+mf}{0.5}\PYG{p}{,}\PYGZbs{}  \PYG{c+c1}{\PYGZsh{} here\PYGZsq{}s our plot statement}
\PYG{g+go}{    colorScheme=\PYGZsq{}copper\PYGZsq{},legendLocation=\PYGZsq{}outside\PYGZsq{},telluric=True,output=\PYGZsq{}lstandards.pdf\PYGZsq{})}
\end{Verbatim}

\end{description}

\noindent{\hspace*{\fill}\sphinxincludegraphics{{plot_example4}.png}\hspace*{\fill}}
\end{quote}


\subsection{Routines}
\label{splat_plot:routines}\index{plotSpectrum() (in module splat)}

\begin{fulllineitems}
\phantomsection\label{splat_plot:splat.plotSpectrum}\pysiglinewithargsret{\sphinxcode{splat.}\sphinxbfcode{plotSpectrum}}{\emph{*args}, \emph{**kwargs}}{}~\begin{quote}\begin{description}
\item[{Purpose}] \leavevmode
\sphinxcode{Primary plotting program for Spectrum objects.}

\end{description}\end{quote}

:Input
Spectrum objects, either sequentially, in list, or in list of lists
\begin{itemize}
\item {} 
Spec1, Spec2, ...: plot multiple spectra together, or separately if multiplot = True

\item {} 
{[}Spec1, Spec2, ...{]}: plot multiple spectra together, or separately if multiplot = True

\item {} 
{[}{[}Spec1, Spec2{]}, {[}Spec3, Spec4{]}, ..{]}: plot multiple sets of spectra (multiplot forced to be True)

\end{itemize}

:Parameters
title = `'
\begin{quote}

string giving plot title
\end{quote}
\begin{description}
\item[{xrange = {[}0.85,2.42{]}:}] \leavevmode
plot range for wavelength axis

\item[{yrange = {[}-0.02,1.2{]}*fluxMax:}] \leavevmode
plot range for wavelength axis

\item[{xlabel:}] \leavevmode
wavelength axis label; by default set by wlabel and wunit keywords in first spectrum object

\item[{ylabel:}] \leavevmode
flux axis label; by default set by fscale, flabel and funit keywords in first spectrum object

\item[{features:}] \leavevmode
a list of strings indicating chemical features to label on the spectra
options include H2O, CH4, CO, TiO, VO, FeH, H2, HI, KI, NaI, SB (for spectral binary)

\item[{mdwarf, ldwarf, tdwarf, young, binary = False:}] \leavevmode
add in features characteristic of these classes

\item[{telluric = False:}] \leavevmode
mark telluric absorption features

\item[{legend, legends, label or labels:}] \leavevmode
list of strings providing legend-style labels for each spectrum plotted

\item[{legendLocation or labelLocation = `upper right':}] \leavevmode
place of legend; options are `upper left', `center middle', `lower right' (variations thereof) and `outside'

\item[{legendfontscale = 1:}] \leavevmode
sets the scale factor for the legend fontsize (defaults to fontscale)

\item[{grid = False:}] \leavevmode
add a grid

\item[{stack = 0:}] \leavevmode
set to a numerical offset to stack spectra on top of each other

\item[{zeropoint = {[}0,...{]}:}] \leavevmode
list of offsets for each spectrum, giving finer control than stack

\item[{showZero = True:}] \leavevmode
plot the zeropoint(s) of the spectra

\item[{comparison:}] \leavevmode
a comparison Spectrum to compare in each plot, useful for common reference standard

\item[{noise, showNoise or uncertainty = False:}] \leavevmode
plot the uncertainty for each spectrum

\item[{residual = False:}] \leavevmode
plots the residual between two spectra

\item[{color or colors:}] \leavevmode
color of plot lines; by default all black

\item[{colorUnc or colorsUnc:}] \leavevmode
color of uncertainty lines; by default same as line color but reduced opacity

\item[{colorScheme or colorMap:}] \leavevmode
color map to apply based on matplotlib colormaps;
see \url{http://matplotlib.org/api/pyplot\_summary.html?highlight=colormaps\#matplotlib.pyplot.colormaps}

\item[{linestyle:}] \leavevmode
line style of plot lines; by default all solid

\item[{fontscale = 1:}] \leavevmode
sets a scale factor for the fontsize

\item[{inset = False:}] \leavevmode
place an inset panel showing a close up region of the spectral data

\item[{inset\_xrange = False:}] \leavevmode
wavelength range for inset panel

\item[{inset\_position = {[}0.65,0.60,0.20,0.20{]}}] \leavevmode
position of inset planet in normalized units, in order left, bottom, width, height

\item[{inset\_features = False}] \leavevmode
list of features to label in inset plot

\item[{file or filename or output:}] \leavevmode
filename or filename base for output

\item[{filetype = `pdf':}] \leavevmode
output filetype, generally determined from filename

\item[{multiplot = False:}] \leavevmode
creates multiple plots, depending on format of input (optional)

\item[{multipage = False:}] \leavevmode
spreads plots across multiple pages; output file format must be PDF
if not set and plots span multiple pages, these pages are output sequentially as separate files

\item[{layout or multilayout = {[}1,1{]}:}] \leavevmode
defines how multiple plots are laid out on a page

\item[{figsize:}] \leavevmode
set the figure size; set to default size if not indicated

\item[{interactive = False:}] \leavevmode
if plotting to window, set this to make window interactive

\end{description}
\begin{quote}\begin{description}
\item[{Example 1}] \leavevmode
A simple view of a random spectrum
\textgreater{}\textgreater{}\textgreater{} import splat
\textgreater{}\textgreater{}\textgreater{} spc = splat.getSpectrum(spt = `T5', lucky=True){[}0{]}
\textgreater{}\textgreater{}\textgreater{} spc.plot()                       \# this automatically generates a ``quicklook'' plot
\textgreater{}\textgreater{}\textgreater{} splat.plotSpectrum(spc)          \# does the same thing
\textgreater{}\textgreater{}\textgreater{} splat.plotSpectrum(spc,uncertainty=True,tdwarf=True)     \# show the spectrum uncertainty and T dwarf absorption features

\item[{Example 2}] \leavevmode\begin{description}
\item[{Viewing a set of spectra for a given object}] \leavevmode
In this case we'll look at all of the spectra of TWA 30B in the library, sorted by year and compared to the first epoch data
This is an example of using multiplot and multipage

\end{description}

\begin{Verbatim}[commandchars=\\\{\}]
\PYG{g+gp}{\PYGZgt{}\PYGZgt{}\PYGZgt{} }\PYG{n}{splist} \PYG{o}{=} \PYG{n}{splat}\PYG{o}{.}\PYG{n}{getSpectrum}\PYG{p}{(}\PYG{n}{name} \PYG{o}{=} \PYG{l+s+s1}{\PYGZsq{}}\PYG{l+s+s1}{TWA 30B}\PYG{l+s+s1}{\PYGZsq{}}\PYG{p}{)}         \PYG{c+c1}{\PYGZsh{} get all spectra of TWA 30B}
\PYG{g+gp}{\PYGZgt{}\PYGZgt{}\PYGZgt{} }\PYG{n}{junk} \PYG{o}{=} \PYG{p}{[}\PYG{n}{sp}\PYG{o}{.}\PYG{n}{normalize}\PYG{p}{(}\PYG{p}{)} \PYG{k}{for} \PYG{n}{sp} \PYG{o+ow}{in} \PYG{n}{splist}\PYG{p}{]}             \PYG{c+c1}{\PYGZsh{} normalize the spectra}
\PYG{g+gp}{\PYGZgt{}\PYGZgt{}\PYGZgt{} }\PYG{n}{dates} \PYG{o}{=} \PYG{p}{[}\PYG{n}{sp}\PYG{o}{.}\PYG{n}{date} \PYG{k}{for} \PYG{n}{sp} \PYG{o+ow}{in} \PYG{n}{splist}\PYG{p}{]}                   \PYG{c+c1}{\PYGZsh{} observation dates}
\PYG{g+gp}{\PYGZgt{}\PYGZgt{}\PYGZgt{} }\PYG{n}{spsort} \PYG{o}{=} \PYG{p}{[}\PYG{n}{s} \PYG{k}{for} \PYG{p}{(}\PYG{n}{s}\PYG{p}{,}\PYG{n}{d}\PYG{p}{)} \PYG{o+ow}{in} \PYG{n+nb}{sorted}\PYG{p}{(}\PYG{n+nb}{zip}\PYG{p}{(}\PYG{n}{dates}\PYG{p}{,}\PYG{n}{splis}\PYG{p}{)}\PYG{p}{)}\PYG{p}{]}   \PYG{c+c1}{\PYGZsh{} sort spectra by dates}
\PYG{g+gp}{\PYGZgt{}\PYGZgt{}\PYGZgt{} }\PYG{n}{dates}\PYG{o}{.}\PYG{n}{sort}\PYG{p}{(}\PYG{p}{)}                                         \PYG{c+c1}{\PYGZsh{} don\PYGZsq{}t forget to sort dates!}
\PYG{g+gp}{\PYGZgt{}\PYGZgt{}\PYGZgt{} }\PYG{n}{splat}\PYG{o}{.}\PYG{n}{plotSpectrum}\PYG{p}{(}\PYG{n}{spsort}\PYG{p}{,}\PYG{n}{multiplot}\PYG{o}{=}\PYG{k+kc}{True}\PYG{p}{,}\PYG{n}{layout}\PYG{o}{=}\PYG{p}{[}\PYG{l+m+mi}{2}\PYG{p}{,}\PYG{l+m+mi}{2}\PYG{p}{]}\PYG{p}{,}\PYG{n}{multipage}\PYG{o}{=}\PYG{k+kc}{True}\PYG{p}{,}\PYGZbs{}   \PYG{c+c1}{\PYGZsh{} here\PYGZsq{}s our plot statement}
\PYG{g+go}{    comparison=spsort[0],uncertainty=True,mdwarf=True,telluric=True,legends=dates,           legendLocation=\PYGZsq{}lower left\PYGZsq{},output=\PYGZsq{}TWA30B.pdf\PYGZsq{})}
\end{Verbatim}

\item[{Example 3}] \leavevmode\begin{description}
\item[{Display the spectra sequence of L dwarfs}] \leavevmode
This example uses the list of standard files contained in SPLAT, and illustrates the stack feature

\end{description}

\begin{Verbatim}[commandchars=\\\{\}]
\PYG{g+gp}{\PYGZgt{}\PYGZgt{}\PYGZgt{} }\PYG{n}{spt} \PYG{o}{=} \PYG{p}{[}\PYG{n}{splat}\PYG{o}{.}\PYG{n}{typeToNum}\PYG{p}{(}\PYG{n}{i}\PYG{o}{+}\PYG{l+m+mi}{20}\PYG{p}{)} \PYG{k}{for} \PYG{n}{i} \PYG{o+ow}{in} \PYG{n+nb}{range}\PYG{p}{(}\PYG{l+m+mi}{10}\PYG{p}{)}\PYG{p}{]} \PYG{c+c1}{\PYGZsh{} generate list of L spectral types}
\PYG{g+gp}{\PYGZgt{}\PYGZgt{}\PYGZgt{} }\PYG{n}{splat}\PYG{o}{.}\PYG{n}{initiateStandards}\PYG{p}{(}\PYG{p}{)}                        \PYG{c+c1}{\PYGZsh{} initiate standards}
\PYG{g+gp}{\PYGZgt{}\PYGZgt{}\PYGZgt{} }\PYG{n}{splist} \PYG{o}{=} \PYG{p}{[}\PYG{n}{splat}\PYG{o}{.}\PYG{n}{SPEX\PYGZus{}STDS}\PYG{p}{[}\PYG{n}{s}\PYG{p}{]} \PYG{k}{for} \PYG{n}{s} \PYG{o+ow}{in} \PYG{n}{spt}\PYG{p}{]}       \PYG{c+c1}{\PYGZsh{} extact just L dwarfs}
\PYG{g+gp}{\PYGZgt{}\PYGZgt{}\PYGZgt{} }\PYG{n}{junk} \PYG{o}{=} \PYG{p}{[}\PYG{n}{sp}\PYG{o}{.}\PYG{n}{normalize}\PYG{p}{(}\PYG{p}{)} \PYG{k}{for} \PYG{n}{sp} \PYG{o+ow}{in} \PYG{n}{splist}\PYG{p}{]}         \PYG{c+c1}{\PYGZsh{} normalize the spectra}
\PYG{g+gp}{\PYGZgt{}\PYGZgt{}\PYGZgt{} }\PYG{n}{labels} \PYG{o}{=} \PYG{p}{[}\PYG{n}{sp}\PYG{o}{.}\PYG{n}{shortname} \PYG{k}{for} \PYG{n}{sp} \PYG{o+ow}{in} \PYG{n}{splist}\PYG{p}{]}         \PYG{c+c1}{\PYGZsh{} set labels to be names}
\PYG{g+gp}{\PYGZgt{}\PYGZgt{}\PYGZgt{} }\PYG{n}{splat}\PYG{o}{.}\PYG{n}{plotSpectrum}\PYG{p}{(}\PYG{n}{splist}\PYG{p}{,}\PYG{n}{figsize}\PYG{o}{=}\PYG{p}{[}\PYG{l+m+mi}{10}\PYG{p}{,}\PYG{l+m+mi}{20}\PYG{p}{]}\PYG{p}{,}\PYG{n}{labels}\PYG{o}{=}\PYG{n}{labels}\PYG{p}{,}\PYG{n}{stack}\PYG{o}{=}\PYG{l+m+mf}{0.5}\PYG{p}{,}\PYGZbs{}  \PYG{c+c1}{\PYGZsh{} here\PYGZsq{}s our plot statement}
\PYG{g+go}{    colorScheme=\PYGZsq{}copper\PYGZsq{},legendLocation=\PYGZsq{}outside\PYGZsq{},telluric=True,output=\PYGZsq{}lstandards.pdf\PYGZsq{})}
\end{Verbatim}

\end{description}\end{quote}

\end{fulllineitems}

\begin{itemize}
\item {} 
\DUrole{xref,std,std-ref}{genindex}

\item {} 
\DUrole{xref,std,std-ref}{modindex}

\item {} 
\DUrole{xref,std,std-ref}{search}

\end{itemize}


\section{SPLAT Spectral Modeling}
\label{splat_model::doc}\label{splat_model:splat-spectral-modeling}
The SPLAT spectral modeling package provides tools for reading in and
\phantomsection\label{splat_model:module-splat_model}\index{splat\_model (module)}\index{calcLuminosity() (in module splat\_model)}

\begin{fulllineitems}
\phantomsection\label{splat_model:splat_model.calcLuminosity}\pysiglinewithargsret{\sphinxcode{splat\_model.}\sphinxbfcode{calcLuminosity}}{\emph{sp}, \emph{mdl=False}, \emph{absmags=False}, \emph{**kwargs}}{}~\begin{quote}\begin{description}
\item[{Purpose}] \leavevmode
Calculate luminosity from photometry and stitching models.

\end{description}\end{quote}

THIS IS CURRENTLY BEING WRITTEN - DO NOT USE!
\begin{quote}\begin{description}
\item[{Parameters}] \leavevmode\begin{itemize}
\item {} 
\textbf{\texttt{sp}} -- Spectrum class object, which should contain wave, flux and
noise array elements.

\item {} 
\textbf{\texttt{mdl}} (\emph{\texttt{default = False}}) -- model spectrum loaded using \sphinxcode{loadModel}

\item {} 
\textbf{\texttt{absmags}} (\emph{\texttt{default = False}}) -- a dictionary whose keys are one of the following filters: `SDSS Z',
`2MASS J', `2MASS H', `2MASS KS', `MKO J', `MKO H', `MKO K', `SDSS R',
`SDSS I', `WISE W1', `WISE W2', `WISE W3', `WISE W4', `IRAC CH1',
`IRAC CH2', `IRAC CH3', `IRAC CH4'

\end{itemize}

\end{description}\end{quote}

\end{fulllineitems}

\index{distributionStats() (in module splat\_model)}

\begin{fulllineitems}
\phantomsection\label{splat_model:splat_model.distributionStats}\pysiglinewithargsret{\sphinxcode{splat\_model.}\sphinxbfcode{distributionStats}}{\emph{x, q={[}0.16, 0.5, 0.84{]}, weights=None, sigma=None, **kwargs}}{}~\begin{quote}\begin{description}
\item[{Purpose}] \leavevmode
Find key values along distributions based on quantile steps.
This code is derived almost entirely from triangle.py.

\end{description}\end{quote}

\end{fulllineitems}

\index{getModel() (in module splat\_model)}

\begin{fulllineitems}
\phantomsection\label{splat_model:splat_model.getModel}\pysiglinewithargsret{\sphinxcode{splat\_model.}\sphinxbfcode{getModel}}{\emph{*args}, \emph{**kwargs}}{}
Redundant routine with loadModel

\end{fulllineitems}

\index{loadInterpolatedModel() (in module splat\_model)}

\begin{fulllineitems}
\phantomsection\label{splat_model:splat_model.loadInterpolatedModel}\pysiglinewithargsret{\sphinxcode{splat\_model.}\sphinxbfcode{loadInterpolatedModel}}{\emph{*args}, \emph{**kwargs}}{}~\begin{quote}\begin{description}
\item[{Purpose}] \leavevmode
Loads interpolated model spectrum based on parameters

\item[{Parameters}] \leavevmode\begin{itemize}
\item {} 
\textbf{\texttt{model}} (\emph{\texttt{optional, default = 'BTSettl2008'}}) -- 
set of models to use; options include:
\begin{itemize}
\item {} 
\emph{`BTSettl2008'}: model set with effective temperature of 400 to 2900 K, surface gravity of 3.5 to 5.5 and metallicity of -3.0 to 0.5
from \href{http://adsabs.harvard.edu/abs/2012RSPTA.370.2765A}{Allard et al. (2012)}

\item {} 
\emph{`burrows06'}: model set with effective temperature of 700 to 2000 K, surface gravity of 4.5 to 5.5, metallicity of -0.5 to 0.5,
and sedimentation efficiency of either 0 or 100 from \href{http://adsabs.harvard.edu/abs/2006ApJ...640.1063B}{Burrows et al. (2006)}

\item {} 
\emph{`morley12'}: model set with effective temperature of 400 to 1300 K, surface gravity of 4.0 to 5.5, metallicity of 0.0
and sedimentation efficiency of 2 to 5 from \href{http://adsabs.harvard.edu/abs/2012ApJ...756..172M}{Morley et al. (2012)}

\item {} 
\emph{`morley14'}: model set with effective temperature of 200 to 450 K, surface gravity of 3.0 to 5.0, metallicity of 0.0
and sedimentation efficiency of 5 from \href{http://adsabs.harvard.edu/abs/2014ApJ...787...78M}{Morley et al. (2014)}

\item {} 
\emph{`saumon12'}: model set with effective temperature of 400 to 1500 K, surface gravity of 3.0 to 5.5 and metallicity of 0.0
from \href{http://adsabs.harvard.edu/abs/2012ApJ...750...74S}{Saumon et al. (2012)}

\item {} 
\emph{`drift'}: model set with effective temperature of 1700 to 3000 K, surface gravity of 5.0 to 5.5 and metallicity of -3.0 to 0.0
from \href{http://adsabs.harvard.edu/abs/2011A\%26A...529A..44W}{Witte et al. (2011)}

\end{itemize}


\item {} 
\textbf{\texttt{local}} (\emph{\texttt{optional, default = True}}) -- read in parameter file locally if True

\item {} 
\textbf{\texttt{url}} (optional, default = `\url{http://pono.ucsd.edu/~adam/splat/}`) -- string of the url to the SPLAT website

\end{itemize}

\item[{Model Parameters}] \leavevmode
Below are parameters that define the model:
\begin{itemize}
\item {} 
\emph{teff}: effective temperature of the model

\item {} 
\emph{logg}: surface gravity of the model

\item {} 
\emph{z}: metallicity of the model

\item {} 
\emph{fsed}: sedimentation efficiency of the model

\item {} 
\emph{cld}: cloud shape function of the model

\item {} 
\emph{kzz}: vertical eddy diffusion coefficient of the model

\item {} 
\emph{slit}: slit weight of the model

\end{itemize}

\end{description}\end{quote}

\end{fulllineitems}

\index{loadModel() (in module splat\_model)}

\begin{fulllineitems}
\phantomsection\label{splat_model:splat_model.loadModel}\pysiglinewithargsret{\sphinxcode{splat\_model.}\sphinxbfcode{loadModel}}{\emph{*args}, \emph{**kwargs}}{}~\begin{quote}\begin{description}
\item[{Purpose}] \leavevmode
Loads up a model spectrum based on a set of input parameters.

\end{description}\end{quote}

The models may be any one of the following listed below.
A Spectrum object with the wavelength and surface fluxes (F\_lambda in units of erg/cm\textasciicircum{}2/s/mu\{m\}) is returned
\begin{quote}\begin{description}
\item[{Parameters}] \leavevmode\begin{itemize}
\item {} 
\textbf{\texttt{model}} -- 
The model set to use; may be one of the following:
\begin{itemize}
\item {} 
\emph{`BTSettl2008'}: default model set from \href{http://adsabs.harvard.edu/abs/2012RSPTA.370.2765A}{Allard et al. (2012)}

\end{itemize}

with effective temperatures of 400 to 2900 K (steps of 100 K); surface gravities of 3.5 to 5.5 in units of cm/s\textasciicircum{}2 (steps of 0.5 dex); and metallicity of -3.0, -2.5, -2.0, -1.5, -1.0, -0.5, 0.0, 0.3, and 0.5 for temperatures greater than 2000 K only;
cloud opacity is fixed in this model, and equilibrium chemistry is assumed. Note that this grid is not completely filled and some gaps have been interpolated (alternate designations: `btsettled','btsettl','allard','allard12')
- \emph{`burrows06'}: model set from \href{http://adsabs.harvard.edu/abs/2006ApJ...640.1063B}{Burrows et al. (2006)}
with effective temperatures of 700 to 2000 K (steps of 50 K); surface gravities of 4.5 to 5.5 in units of cm/s\textasciicircum{}2 (steps of 0.1 dex); metallicity of -0.5, 0.0 and 0.5; and either no clouds or grain size 100 microns (fsed = `nc' or `f100').
equilibrium chemistry is assumed. Note that this grid is not completely filled and some gaps have been interpolated (alternate designations: `burrows','burrows2006')
- \emph{`morley12'}: model set from \href{http://adsabs.harvard.edu/abs/2012ApJ...756..172M}{Morley et al. (2012)}
with effective temperatures of 400 to 1300 K (steps of 50 K); surface gravities of 4.0 to 5.5 in units of cm/s\textasciicircum{}2 (steps of 0.5 dex); and sedimentation efficiency (fsed) of 2, 3, 4 or 5;
metallicity is fixed to solar, equilibrium chemistry is assumed, and there are no clouds associated with this model (alternate designations: `morley2012')
- \emph{`morley14'}: model set from \href{http://adsabs.harvard.edu/abs/2014ApJ...787...78M}{Morley et al. (2014)}
with effective temperatures of 200 to 450 K (steps of 25 K) and surface gravities of 3.0 to 5.0 in units of cm/s\textasciicircum{}2 (steps of 0.5 dex);
metallicity is fixed to solar, equilibrium chemistry is assumed, sedimentation efficiency is fixed at fsed = 5, and cloud coverage fixed at 50\% (alternate designations: `morley2014')
- \emph{`saumon12'}: model set from \href{http://adsabs.harvard.edu/abs/2012ApJ...750...74S}{Saumon et al. (2012)}
with effective temperatures of 400 to 1500 K (steps of 50 K); and surface gravities of 3.0 to 5.5 in units of cm/s\textasciicircum{}2 (steps of 0.5 dex);
metallicity is fixed to solar, equilibrium chemistry is assumed, and no clouds are associated with these models (alternate designations: `saumon','saumon2012')
- \emph{`drift'}: model set from \href{http://adsabs.harvard.edu/abs/2011A\%26A...529A..44W}{Witte et al. (2011)}
with effective temperatures of 1700 to 3000 K (steps of 50 K); surface gravities of 5.0 and 5.5 in units of cm/s\textasciicircum{}2; and metallicities of -3.0 to 0.0 (in steps of 0.5 dex);
cloud opacity is fixed in this model, equilibrium chemistry is assumed (alternate designations: `witte','witte2011','helling')


\item {} 
\textbf{\texttt{local}} (\emph{\texttt{optional, default = True}}) -- read in parameter file locally if True

\item {} 
\textbf{\texttt{online}} (\emph{\texttt{optional, default = False}}) -- read in parameter file online if True

\item {} 
\textbf{\texttt{folder}} (\emph{\texttt{optional, default = '{'}}}) -- string of the folder name containing the model set

\item {} 
\textbf{\texttt{filename}} (\emph{\texttt{optional, default = False}}) -- string of the filename of the desired model

\item {} 
\textbf{\texttt{force}} -- force the filename to be exactly as specified

\item {} 
\textbf{\texttt{url}} (optional, default = `\url{http://pono.ucsd.edu/~adam/splat/}`) -- string of the url to the SPLAT website

\end{itemize}

\item[{Model Parameters}] \leavevmode
The following parameters may be set:
\begin{itemize}
\item {} 
\emph{teff}: effective temperature of the model in K (e.g. teff = 1000)

\item {} 
\emph{logg}: log\_10 of the surface gravity of the model in cm/s\textasciicircum{}2 units (e.g. logg = 5.0)

\item {} 
\emph{z}: log\_10 of metallicity of the model relative to solar metallicity (e.g. z = -0.5)

\item {} 
\emph{fsed}: sedimentation efficiency of the model (e.g. fsed = `f2')

\item {} 
\emph{cld}: cloud shape function of the model (e.g. cld = `f50')

\item {} 
\emph{kzz}: vertical eddy diffusion coefficient of the model (e.g. kzz = 2)

\item {} 
\emph{slit}: slit weight of the model in arcseconds (e.g. slit = 0.3)

\item {} 
\emph{sed}: if set to True, returns a broad-band spectrum spanning 0.3-30 micron (applies only for BTSettl2008 models with Teff \textless{} 2000 K)

\end{itemize}

\item[{Example}] \leavevmode
\begin{Verbatim}[commandchars=\\\{\}]
\PYG{g+gp}{\PYGZgt{}\PYGZgt{}\PYGZgt{} }\PYG{k+kn}{import} \PYG{n+nn}{splat}
\PYG{g+gp}{\PYGZgt{}\PYGZgt{}\PYGZgt{} }\PYG{n}{mdl} \PYG{o}{=} \PYG{n}{splat}\PYG{o}{.}\PYG{n}{loadModel}\PYG{p}{(}\PYG{n}{teff}\PYG{o}{=}\PYG{l+m+mi}{1000}\PYG{p}{,}\PYG{n}{logg}\PYG{o}{=}\PYG{l+m+mf}{5.0}\PYG{p}{)}
\PYG{g+gp}{\PYGZgt{}\PYGZgt{}\PYGZgt{} }\PYG{n}{mdl}\PYG{o}{.}\PYG{n}{info}\PYG{p}{(}\PYG{p}{)}
\PYG{g+go}{     BTSettl2008 model with the following parmeters:}
\PYG{g+go}{     Teff = 1000 K}
\PYG{g+go}{     logg = 5.0 cm/s2}
\PYG{g+go}{     z = 0.0}
\PYG{g+go}{     fsed = nc}
\PYG{g+go}{     cld = nc}
\PYG{g+go}{     kzz = eq}
\PYG{g+go}{     Smoothed to slit width 0.5 arcseconds}
\PYG{g+gp}{\PYGZgt{}\PYGZgt{}\PYGZgt{} }\PYG{n}{mdl} \PYG{o}{=} \PYG{n}{splat}\PYG{o}{.}\PYG{n}{loadModel}\PYG{p}{(}\PYG{n}{teff}\PYG{o}{=}\PYG{l+m+mi}{2500}\PYG{p}{,}\PYG{n}{logg}\PYG{o}{=}\PYG{l+m+mf}{5.0}\PYG{p}{,}\PYG{n}{model}\PYG{o}{=}\PYG{l+s+s1}{\PYGZsq{}}\PYG{l+s+s1}{burrows}\PYG{l+s+s1}{\PYGZsq{}}\PYG{p}{)}
\PYG{g+go}{     Input value for teff = 2500 out of range for model set burrows06}
\PYG{g+go}{     Warning: Creating an empty Spectrum object}
\end{Verbatim}

\end{description}\end{quote}

\end{fulllineitems}

\index{loadModelParameters() (in module splat\_model)}

\begin{fulllineitems}
\phantomsection\label{splat_model:splat_model.loadModelParameters}\pysiglinewithargsret{\sphinxcode{splat\_model.}\sphinxbfcode{loadModelParameters}}{\emph{**kwargs}}{}~\begin{quote}\begin{description}
\item[{Purpose}] \leavevmode
Load up model parameters and check model inputs. Parameters include
effective temperature, surface gravity (expressed as logg), metallicity,
and sedimentation efficiency (for cloudy models only).

\item[{Parameters}] \leavevmode\begin{itemize}
\item {} 
\textbf{\texttt{parameterFile}} (\emph{\texttt{optional, default = 'parameters.txt'}}) -- name of file containing parameters for spectral models

\item {} 
\textbf{\texttt{model}} -- 
set of models to use; options include:
\begin{itemize}
\item {} 
\emph{`BTSettl2008'}: model set with effective temperature of 400 to 2900 K, surface gravity of 3.5 to 5.5 and metallicity of -3.0 to 0.5
from \href{http://adsabs.harvard.edu/abs/2012RSPTA.370.2765A}{Allard et al. (2012)}

\item {} 
\emph{`burrows06'}: model set with effective temperature of 700 to 2000 K, surface gravity of 4.5 to 5.5, metallicity of -0.5 to 0.5,
and sedimentation efficiency of either 0 or 100 from \href{http://adsabs.harvard.edu/abs/2006ApJ...640.1063B}{Burrows et al. (2006)}

\item {} 
\emph{`morley12'}: model set with effective temperature of 400 to 1300 K, surface gravity of 4.0 to 5.5, metallicity of 0.0
and sedimentation efficiency of 2 to 5 from \href{http://adsabs.harvard.edu/abs/2012ApJ...756..172M}{Morley et al. (2012)}

\item {} 
\emph{`morley14'}: model set with effective temperature of 200 to 450 K, surface gravity of 3.0 to 5.0, metallicity of 0.0
and sedimentation efficiency of 5 from \href{http://adsabs.harvard.edu/abs/2014ApJ...787...78M}{Morley et al. (2014)}

\item {} 
\emph{`saumon12'}: model set with effective temperature of 400 to 1500 K, surface gravity of 3.0 to 5.5 and metallicity of 0.0
from \href{http://adsabs.harvard.edu/abs/2012ApJ...750...74S}{Saumon et al. (2012)}

\item {} 
\emph{`drift'}: model set with effective temperature of 1700 to 3000 K, surface gravity of 5.0 to 5.5 and metallicity of -3.0 to 0.0
from \href{http://adsabs.harvard.edu/abs/2011A\%26A...529A..44W}{Witte et al. (2011)}

\end{itemize}


\item {} 
\textbf{\texttt{online}} (\emph{\texttt{optional, default = False}}) -- read in parameter file online if True

\end{itemize}

\end{description}\end{quote}

\end{fulllineitems}

\index{modelFitGrid() (in module splat\_model)}

\begin{fulllineitems}
\phantomsection\label{splat_model:splat_model.modelFitGrid}\pysiglinewithargsret{\sphinxcode{splat\_model.}\sphinxbfcode{modelFitGrid}}{\emph{spec}, \emph{**kwargs}}{}
Model fitting code to grid of models

\end{fulllineitems}

\index{modelFitMCMC() (in module splat\_model)}

\begin{fulllineitems}
\phantomsection\label{splat_model:splat_model.modelFitMCMC}\pysiglinewithargsret{\sphinxcode{splat\_model.}\sphinxbfcode{modelFitMCMC}}{\emph{spec}, \emph{**kwargs}}{}~\begin{quote}\begin{description}
\item[{Purpose}] \leavevmode
Uses Markov chain Monte Carlo method to compare an object with models from a
given set. Returns the best estimate of the effective temperature, surface
gravity, and metallicity. Can also determine the radius of the object by
using these estimates.

\item[{Parameters}] \leavevmode\begin{itemize}
\item {} 
\textbf{\texttt{spec}} -- Spectrum class object, which should contain wave, flux and
noise array elements.

\item {} 
\textbf{\texttt{nsamples}} (\emph{\texttt{optional, default = 1000}}) -- number of Monte Carlo samples

\item {} 
\textbf{\texttt{initial\_cut}} (\emph{\texttt{optional, default = 0.1}}) -- the fraction of the initial steps to be discarded. (e.g., if
\sphinxcode{initial\_cut = 0.2}, the first 20\% of the samples are discarded.)

\item {} 
\textbf{\texttt{burn}} (\emph{\texttt{optional, default = 0.1}}) -- the same as \sphinxcode{initial\_cut}

\item {} 
\textbf{\texttt{set}} (\emph{\texttt{optional, default = 'BTSettl2008'}}) -- 
set of models to use; options include:
\begin{itemize}
\item {} 
\emph{`BTSettl2008'}: model set with effective temperature of 400 to 2900 K, surface gravity of 3.5 to 5.5 and metallicity of -3.0 to 0.5
from \href{http://adsabs.harvard.edu/abs/2012RSPTA.370.2765A}{Allard et al. (2012)}

\item {} 
\emph{`burrows06'}: model set with effective temperature of 700 to 2000 K, surface gravity of 4.5 to 5.5, metallicity of -0.5 to 0.5,
and sedimentation efficiency of either 0 or 100 from \href{http://adsabs.harvard.edu/abs/2006ApJ...640.1063B}{Burrows et al. (2006)}

\item {} 
\emph{`morley12'}: model set with effective temperature of 400 to 1300 K, surface gravity of 4.0 to 5.5, metallicity of 0.0
and sedimentation efficiency of 2 to 5 from \href{http://adsabs.harvard.edu/abs/2012ApJ...756..172M}{Morley et al. (2012)}

\item {} 
\emph{`morley14'}: model set with effective temperature of 200 to 450 K, surface gravity of 3.0 to 5.0, metallicity of 0.0
and sedimentation efficiency of 5 from \href{http://adsabs.harvard.edu/abs/2014ApJ...787...78M}{Morley et al. (2014)}

\item {} 
\emph{`saumon12'}: model set with effective temperature of 400 to 1500 K, surface gravity of 3.0 to 5.5 and metallicity of 0.0
from \href{http://adsabs.harvard.edu/abs/2012ApJ...750...74S}{Saumon et al. (2012)}

\item {} 
\emph{`drift'}: model set with effective temperature of 1700 to 3000 K, surface gravity of 5.0 to 5.5 and metallicity of -3.0 to 0.0
from \href{http://adsabs.harvard.edu/abs/2011A\%26A...529A..44W}{Witte et al. (2011)}

\end{itemize}


\item {} 
\textbf{\texttt{model}} (\emph{\texttt{optional, default = 'BTSettl2008'}}) -- the same as \sphinxcode{set}

\item {} 
\textbf{\texttt{models}} (\emph{\texttt{optional, default = 'BTSettl2008'}}) -- the same as \sphinxcode{set}

\item {} 
\textbf{\texttt{verbose}} (\emph{\texttt{optional, default = False}}) -- give lots of feedback

\item {} 
\textbf{\texttt{mask\_ranges}} (\emph{\texttt{optional, default = {[}{]}}}) -- mask any flux value of \sphinxcode{spec} by specifying the wavelength range.
Must be in microns.

\item {} 
\textbf{\texttt{mask\_telluric}} (\emph{\texttt{optional, default = False}}) -- masks certain wavelengths to avoid effects from telluric absorption

\item {} 
\textbf{\texttt{mask\_standard}} (\emph{\texttt{optional, default = True}}) -- masks wavelengths below 0.8 and above 2.35 microns

\item {} 
\textbf{\texttt{mask}} (\emph{\texttt{optional, default = {[}0, ..., 0{]} for len(sp1.wave)}}) -- mask any flux value of \sphinxcode{spec}; has to be an array with length equal as \sphinxcode{spec} with only 0 (unmask) or 1 (mask).

\item {} 
\textbf{\texttt{radius}} (\emph{\texttt{optional}}) -- calculates and returns radius of object if True

\item {} 
\textbf{\texttt{filename}} (\emph{\texttt{optional}}) -- filename or filename base for output

\item {} 
\textbf{\texttt{filebase}} (\emph{\texttt{optional}}) -- the same as \sphinxcode{filename}

\item {} 
\textbf{\texttt{savestep}} (optional, default = \sphinxcode{nsamples/10}) -- indicate when to save data output (e.g. \sphinxcode{savestep = 10} will save the output every 10 samples)

\item {} 
\textbf{\texttt{dataformat}} (\emph{\texttt{optional, default = 'ascii.csv'}}) -- output data format type

\item {} 
\textbf{\texttt{initial\_guess}} (\emph{\texttt{optional, default = array of random numbers within allowed ranges}}) -- array including initial guess of the effective temperature, surface gravity and metallicity of \sphinxcode{spec}.
Can also set individual guesses of spectral parameters by using \textbf{initial\_temperature} or \textbf{initial\_teff},
\textbf{initial\_gravity} or \textbf{initial\_logg}, and \textbf{initial\_metallicity} or \textbf{initial\_z}.

\item {} 
\textbf{\texttt{ranges}} (\emph{\texttt{optional, default = depends on model set}}) -- array of arrays indicating ranges of the effective temperature, surface gravity and metallicity of the model set.
Can also set individual ranges of spectral parameters by using \textbf{temperature\_range} or \textbf{teff\_range},
\textbf{gravity\_range} or \textbf{logg\_range}, and \textbf{metallicity\_range} or \textbf{z\_range}.

\item {} 
\textbf{\texttt{step\_sizes}} (\emph{\texttt{optional, default = {[}50, 0.25, 0.1{]}}}) -- an array specifying step sizes of spectral parameters. Can also set individual step sizes by using
\textbf{temperature\_step} or \textbf{teff\_step}, \textbf{gravity\_step} or \textbf{logg\_step}, and \textbf{metallicity\_step} or \textbf{z\_step}.

\item {} 
\textbf{\texttt{nonmetallicity}} (\emph{\texttt{optional, default = False}}) -- if True, sets metallicity = 0

\item {} 
\textbf{\texttt{addon}} (\emph{\texttt{optional, default = False}}) -- reads in prior calculation and starts from there. Allowed object types are tables, dictionaries and strings.

\item {} 
\textbf{\texttt{evolutionary\_model}} (\emph{\texttt{optional, default = 'Baraffe'}}) -- 
set of evolutionary models to use. See Brown Dwarf Evolutionary Models page for
more details. Options include:
\begin{itemize}
\item {} 
\emph{`baraffe'}: Evolutionary models from \href{http://arxiv.org/abs/astro-ph/0302293}{Baraffe et al. (2003)}.

\item {} 
\emph{`burrows'}: Evolutionary models from \href{http://adsabs.harvard.edu/abs/1997ApJ...491..856B}{Burrows et al. (1997)}.

\item {} 
\emph{`saumon'}: Evolutionary models from \href{http://adsabs.harvard.edu/abs/2008ApJ...689.1327S}{Saumon \& Marley (2008)}.

\end{itemize}


\item {} 
\textbf{\texttt{emodel}} (\emph{\texttt{optional, default = 'Baraffe'}}) -- the same as \sphinxcode{evolutionary\_model}

\end{itemize}

\item[{Example}] \leavevmode
\end{description}\end{quote}

\begin{Verbatim}[commandchars=\\\{\}]
\PYG{g+gp}{\PYGZgt{}\PYGZgt{}\PYGZgt{} }\PYG{k+kn}{import} \PYG{n+nn}{splat}
\PYG{g+gp}{\PYGZgt{}\PYGZgt{}\PYGZgt{} }\PYG{n}{sp} \PYG{o}{=} \PYG{n}{splat}\PYG{o}{.}\PYG{n}{getSpectrum}\PYG{p}{(}\PYG{n}{shortname}\PYG{o}{=}\PYG{l+s+s1}{\PYGZsq{}}\PYG{l+s+s1}{1047+2124}\PYG{l+s+s1}{\PYGZsq{}}\PYG{p}{)}\PYG{p}{[}\PYG{l+m+mi}{0}\PYG{p}{]}        \PYG{c+c1}{\PYGZsh{} T6.5 radio emitter}
\PYG{g+gp}{\PYGZgt{}\PYGZgt{}\PYGZgt{} }\PYG{n}{spt}\PYG{p}{,} \PYG{n}{spt\PYGZus{}e} \PYG{o}{=} \PYG{n}{splat}\PYG{o}{.}\PYG{n}{classifyByStandard}\PYG{p}{(}\PYG{n}{sp}\PYG{p}{,}\PYG{n}{spt}\PYG{o}{=}\PYG{p}{[}\PYG{l+s+s1}{\PYGZsq{}}\PYG{l+s+s1}{T2}\PYG{l+s+s1}{\PYGZsq{}}\PYG{p}{,}\PYG{l+s+s1}{\PYGZsq{}}\PYG{l+s+s1}{T8}\PYG{l+s+s1}{\PYGZsq{}}\PYG{p}{]}\PYG{p}{)}
\PYG{g+gp}{\PYGZgt{}\PYGZgt{}\PYGZgt{} }\PYG{n}{teff}\PYG{p}{,}\PYG{n}{teff\PYGZus{}e} \PYG{o}{=} \PYG{n}{splat}\PYG{o}{.}\PYG{n}{typeToTeff}\PYG{p}{(}\PYG{n}{spt}\PYG{p}{)}
\PYG{g+gp}{\PYGZgt{}\PYGZgt{}\PYGZgt{} }\PYG{n}{sp}\PYG{o}{.}\PYG{n}{fluxCalibrate}\PYG{p}{(}\PYG{l+s+s1}{\PYGZsq{}}\PYG{l+s+s1}{MKO J}\PYG{l+s+s1}{\PYGZsq{}}\PYG{p}{,}\PYG{n}{splat}\PYG{o}{.}\PYG{n}{typeToMag}\PYG{p}{(}\PYG{n}{spt}\PYG{p}{,}\PYG{l+s+s1}{\PYGZsq{}}\PYG{l+s+s1}{MKO J}\PYG{l+s+s1}{\PYGZsq{}}\PYG{p}{)}\PYG{p}{[}\PYG{l+m+mi}{0}\PYG{p}{]}\PYG{p}{,}\PYG{n}{absolute}\PYG{o}{=}\PYG{k+kc}{True}\PYG{p}{)}
\PYG{g+gp}{\PYGZgt{}\PYGZgt{}\PYGZgt{} }\PYG{n}{table} \PYG{o}{=} \PYG{n}{splat}\PYG{o}{.}\PYG{n}{modelFitMCMC}\PYG{p}{(}\PYG{n}{sp}\PYG{p}{,} \PYG{n}{mask\PYGZus{}standard}\PYG{o}{=}\PYG{k+kc}{True}\PYG{p}{,} \PYG{n}{initial\PYGZus{}guess}\PYG{o}{=}\PYG{p}{[}\PYG{n}{teff}\PYG{p}{,} \PYG{l+m+mf}{5.3}\PYG{p}{,} \PYG{l+m+mf}{0.}\PYG{p}{]}\PYG{p}{,} \PYG{n}{zstep}\PYG{o}{=}\PYG{l+m+mf}{0.1}\PYG{p}{,} \PYG{n}{nsamples}\PYG{o}{=}\PYG{l+m+mi}{100}\PYG{p}{,} \PYG{n}{savestep}\PYG{o}{=}\PYG{l+m+mi}{0}\PYG{p}{,} \PYG{n}{verbose}\PYG{o}{=}\PYG{k+kc}{True}\PYG{p}{)}
\PYG{g+go}{    Trouble with model BTSettl2008 T=1031.61, logg=5.27, z=\PYGZhy{}0.02}
\PYG{g+go}{    At cycle 0: fit = T=1031.61, logg=5.27, z=0.00 with chi2 = 35948.5}
\PYG{g+go}{    Trouble with model BTSettl2008 T=1031.61, logg=5.27, z=\PYGZhy{}0.13}
\PYG{g+go}{    At cycle 1: fit = T=1031.61, logg=5.27, z=0.00 with chi2 = 35948.5}
\PYG{g+go}{                                    .}
\PYG{g+go}{                                    .}
\PYG{g+go}{                                    .}
\PYG{g+go}{                        \PYGZsh{} Skipped a few lines}
\PYG{g+go}{                                    .}
\PYG{g+go}{                                    .}
\PYG{g+go}{                                    .}
\PYG{g+go}{    Trouble with model BTSettl2008 T=973.89, logg=4.95, z=\PYGZhy{}0.17}
\PYG{g+go}{    At cycle 99: fit = T=973.89, logg=4.95, z=0.00 with chi2 = 30569.6}

\PYG{g+go}{    Number of steps = 170}

\PYG{g+go}{    Best Fit parameters:}
\PYG{g+go}{    Lowest chi2 value = 29402.3750247 for 169.0 degrees of freedom}
\PYG{g+go}{    Effective Temperature = 1031.608 (K)}
\PYG{g+go}{    log Surface Gravity = 5.267}
\PYG{g+go}{    Metallicity = 0.000}
\PYG{g+go}{    Radius (relative to Sun) from surface fluxes = 0.103}

\PYG{g+go}{    Median parameters:}
\PYG{g+go}{    Effective Temperature = 1029.322 + 66.535 \PYGZhy{} 90.360 (K)}
\PYG{g+go}{    log Surface Gravity = 5.108 + 0.338 \PYGZhy{} 0.473}
\PYG{g+go}{    Metallicity = 0.000 + 0.000 \PYGZhy{} 0.000}
\PYG{g+go}{    Radius (relative to Sun) from surface fluxes = 0.094 + 0.012 \PYGZhy{} 0.007}


\PYG{g+go}{    fit\PYGZus{}J1047+2124\PYGZus{}BTSettl2008}
\PYG{g+go}{    Quantiles:}
\PYG{g+go}{    [(0.16, 0.087231370556002871), (0.5, 0.09414839610875167), (0.84, 0.10562967101117798)]}
\PYG{g+go}{    Quantiles:}
\PYG{g+go}{    [(0.16, 4.6366512070621884), (0.5, 5.1077094570511488), (0.84, 5.4459108887603094)]}
\PYG{g+go}{    Quantiles:}
\PYG{g+go}{    [(0.16, 938.96254520460286), (0.5, 1029.3222563137401), (0.84, 1095.8574021575118)]}

\PYG{g+go}{    Total time elapsed = 0:01:46.340169}
\PYG{g+gp}{\PYGZgt{}\PYGZgt{}\PYGZgt{} }\PYG{n+nb}{print} \PYG{n}{table}
\PYG{g+go}{         teff          logg      z       radius         chisqr}
\PYG{g+go}{    \PYGZhy{}\PYGZhy{}\PYGZhy{}\PYGZhy{}\PYGZhy{}\PYGZhy{}\PYGZhy{}\PYGZhy{}\PYGZhy{}\PYGZhy{}\PYGZhy{}\PYGZhy{}\PYGZhy{} \PYGZhy{}\PYGZhy{}\PYGZhy{}\PYGZhy{}\PYGZhy{}\PYGZhy{}\PYGZhy{}\PYGZhy{}\PYGZhy{}\PYGZhy{}\PYGZhy{}\PYGZhy{}\PYGZhy{} \PYGZhy{}\PYGZhy{}\PYGZhy{} \PYGZhy{}\PYGZhy{}\PYGZhy{}\PYGZhy{}\PYGZhy{}\PYGZhy{}\PYGZhy{}\PYGZhy{}\PYGZhy{}\PYGZhy{}\PYGZhy{}\PYGZhy{}\PYGZhy{}\PYGZhy{}\PYGZhy{} \PYGZhy{}\PYGZhy{}\PYGZhy{}\PYGZhy{}\PYGZhy{}\PYGZhy{}\PYGZhy{}\PYGZhy{}\PYGZhy{}\PYGZhy{}\PYGZhy{}\PYGZhy{}\PYGZhy{}}
\PYG{g+go}{    1031.60790828 5.26704520744 0.0  0.103152256465 29402.3750247}
\PYG{g+go}{    1031.60790828 5.26704520744 0.0  0.103152256465 29402.3750247}
\PYG{g+go}{              ...           ... ...             ...           ...   \PYGZsh{} Skipped a few lines}
\PYG{g+go}{    938.962545205 5.43505121711 0.0  0.125429265207 43836.3720496}
\PYG{g+go}{    938.962545205 5.43505121711 0.0  0.129294090544 47650.4267022}
\end{Verbatim}

\end{fulllineitems}

\index{reportModelFitResults() (in module splat\_model)}

\begin{fulllineitems}
\phantomsection\label{splat_model:splat_model.reportModelFitResults}\pysiglinewithargsret{\sphinxcode{splat\_model.}\sphinxbfcode{reportModelFitResults}}{\emph{spec}, \emph{t}, \emph{*arg}, \emph{**kwargs}}{}~\begin{quote}\begin{description}
\item[{Purpose}] \leavevmode
Reports the result of model fitting parameters.
Produces triangle plot, best fit model, statistics of parameters
and saves raw data if \sphinxcode{iterative = True}.

\item[{Parameters}] \leavevmode\begin{itemize}
\item {} 
\textbf{\texttt{spec}} -- Spectrum class object, which should contain wave, flux and
noise array elements.

\item {} 
\textbf{\texttt{t}} -- Must be an astropy Table with columns containing parameters fit, and one column for chi-square values (`chisqr').

\item {} 
\textbf{\texttt{evol}} (\emph{\texttt{optional, default = True}}) -- computes the mass, age, temperature, radius, surface gravity, and luminosity
by using various evolutionary model sets. See below for the possible set
options and the Brown Dwarf Evolutionary Models page for more details.

\item {} 
\textbf{\texttt{emodel}} (\emph{\texttt{optional, default = 'Baraffe'}}) -- 
set of evolutionary models to use. See Brown Dwarf Evolutionary Models page for
more details. Options include:
\begin{itemize}
\item {} 
\emph{`baraffe'}: Evolutionary models from \href{http://arxiv.org/abs/astro-ph/0302293}{Baraffe et al. (2003)}.

\item {} 
\emph{`burrows'}: Evolutionary models from \href{http://adsabs.harvard.edu/abs/1997ApJ...491..856B}{Burrows et al. (1997)}.

\item {} 
\emph{`saumon'}: Evolutionary models from \href{http://adsabs.harvard.edu/abs/2008ApJ...689.1327S}{Saumon \& Marley (2008)}.

\end{itemize}


\item {} 
\textbf{\texttt{stats}} (\emph{\texttt{optional, default = True}}) -- if True, prints several statistical values, including number of steps,
best fit parameters, lowest chi2 value, median parameters and key values
along the distribution.

\item {} 
\textbf{\texttt{triangle}} (\emph{\texttt{optional, default = True}}) -- creates a triangle plot, plotting the parameters against each other,
demonstrating areas of high and low chi squared values. Useful for
demonstrating correlations between parameters.

\item {} 
\textbf{\texttt{bestfit}} (\emph{\texttt{optional, default = True}}) -- if True and a best fit model is present in the desired model set, then
plots model against spectrum and saves figure.

\item {} 
\textbf{\texttt{summary}} (\emph{\texttt{optional, default = True}}) -- not yet implemented

\item {} 
\textbf{\texttt{weight}} (\emph{\texttt{optional, default = True}}) -- if True, sets weights for computing key values along the distribution

\item {} 
\textbf{\texttt{filebase}} (\emph{\texttt{optional, default = 'modelfit\_results'}}) -- filename or filename base for output

\item {} 
\textbf{\texttt{stat}} (\emph{\texttt{optional, default = 'chisqr'}}) -- name of the statistics column used in astropy Table \sphinxcode{t}.

\item {} 
\textbf{\texttt{model\_set}} (\emph{\texttt{optional, default = '{'}}}) -- 
desired model set of \sphinxcode{bestfit}; options include:
\begin{itemize}
\item {} 
\emph{`BTSettl2008'}: model set with effective temperature of 400 to 2900 K, surface gravity of 3.5 to 5.5 and metallicity of -3.0 to 0.5
from \href{http://adsabs.harvard.edu/abs/2012RSPTA.370.2765A}{Allard et al. (2012)}

\item {} 
\emph{`burrows06'}: model set with effective temperature of 700 to 2000 K, surface gravity of 4.5 to 5.5, metallicity of -0.5 to 0.5,
and sedimentation efficiency of either 0 or 100 from \href{http://adsabs.harvard.edu/abs/2006ApJ...640.1063B}{Burrows et al. (2006)}

\item {} 
\emph{`morley12'}: model set with effective temperature of 400 to 1300 K, surface gravity of 4.0 to 5.5, metallicity of 0.0
and sedimentation efficiency of 2 to 5 from \href{http://adsabs.harvard.edu/abs/2012ApJ...756..172M}{Morley et al. (2012)}

\item {} 
\emph{`morley14'}: model set with effective temperature of 200 to 450 K, surface gravity of 3.0 to 5.0, metallicity of 0.0
and sedimentation efficiency of 5 from \href{http://adsabs.harvard.edu/abs/2014ApJ...787...78M}{Morley et al. (2014)}

\item {} 
\emph{`saumon12'}: model set with effective temperature of 400 to 1500 K, surface gravity of 3.0 to 5.5 and metallicity of 0.0
from \href{http://adsabs.harvard.edu/abs/2012ApJ...750...74S}{Saumon et al. (2012)}

\item {} 
\emph{`drift'}: model set with effective temperature of 1700 to 3000 K, surface gravity of 5.0 to 5.5 and metallicity of -3.0 to 0.0
from \href{http://adsabs.harvard.edu/abs/2011A\%26A...529A..44W}{Witte et al. (2011)}

\end{itemize}


\item {} 
\textbf{\texttt{mset}} (\emph{\texttt{optional, default = '{'}}}) -- same as \sphinxcode{model\_set}

\item {} 
\textbf{\texttt{mask\_ranges}} (\emph{\texttt{optional, default = {[}{]}}}) -- mask any flux value of \sphinxcode{spec} by specifying the wavelength range. Must be in microns.

\item {} 
\textbf{\texttt{sigma}} (\emph{\texttt{optional, default = 1.}}) -- when printing statistical results, prints the value at \sphinxcode{sigma} standard
deviations away from the mean. Only effective if \sphinxcode{stats = True}.

\item {} 
\textbf{\texttt{iterative}} (\emph{\texttt{optional, default = False}}) -- if True, prints quantitative results and does not plot anything

\end{itemize}

\item[{Example}] \leavevmode
\end{description}\end{quote}

\begin{Verbatim}[commandchars=\\\{\}]
\PYG{g+gp}{\PYGZgt{}\PYGZgt{}\PYGZgt{} }\PYG{k+kn}{import} \PYG{n+nn}{splat}
\PYG{g+gp}{\PYGZgt{}\PYGZgt{}\PYGZgt{} }\PYG{n}{sp} \PYG{o}{=} \PYG{n}{splat}\PYG{o}{.}\PYG{n}{getSpectrum}\PYG{p}{(}\PYG{n}{shortname}\PYG{o}{=}\PYG{l+s+s1}{\PYGZsq{}}\PYG{l+s+s1}{1047+2124}\PYG{l+s+s1}{\PYGZsq{}}\PYG{p}{)}\PYG{p}{[}\PYG{l+m+mi}{0}\PYG{p}{]}        \PYG{c+c1}{\PYGZsh{} T6.5 radio emitter}
\PYG{g+gp}{\PYGZgt{}\PYGZgt{}\PYGZgt{} }\PYG{n}{spt}\PYG{p}{,} \PYG{n}{spt\PYGZus{}e} \PYG{o}{=} \PYG{n}{splat}\PYG{o}{.}\PYG{n}{classifyByStandard}\PYG{p}{(}\PYG{n}{sp}\PYG{p}{,}\PYG{n}{spt}\PYG{o}{=}\PYG{p}{[}\PYG{l+s+s1}{\PYGZsq{}}\PYG{l+s+s1}{T2}\PYG{l+s+s1}{\PYGZsq{}}\PYG{p}{,}\PYG{l+s+s1}{\PYGZsq{}}\PYG{l+s+s1}{T8}\PYG{l+s+s1}{\PYGZsq{}}\PYG{p}{]}\PYG{p}{)}
\PYG{g+gp}{\PYGZgt{}\PYGZgt{}\PYGZgt{} }\PYG{n}{teff}\PYG{p}{,}\PYG{n}{teff\PYGZus{}e} \PYG{o}{=} \PYG{n}{splat}\PYG{o}{.}\PYG{n}{typeToTeff}\PYG{p}{(}\PYG{n}{spt}\PYG{p}{)}
\PYG{g+gp}{\PYGZgt{}\PYGZgt{}\PYGZgt{} }\PYG{n}{sp}\PYG{o}{.}\PYG{n}{fluxCalibrate}\PYG{p}{(}\PYG{l+s+s1}{\PYGZsq{}}\PYG{l+s+s1}{MKO J}\PYG{l+s+s1}{\PYGZsq{}}\PYG{p}{,}\PYG{n}{splat}\PYG{o}{.}\PYG{n}{typeToMag}\PYG{p}{(}\PYG{n}{spt}\PYG{p}{,}\PYG{l+s+s1}{\PYGZsq{}}\PYG{l+s+s1}{MKO J}\PYG{l+s+s1}{\PYGZsq{}}\PYG{p}{)}\PYG{p}{[}\PYG{l+m+mi}{0}\PYG{p}{]}\PYG{p}{,}\PYG{n}{absolute}\PYG{o}{=}\PYG{k+kc}{True}\PYG{p}{)}
\PYG{g+gp}{\PYGZgt{}\PYGZgt{}\PYGZgt{} }\PYG{n}{table} \PYG{o}{=} \PYG{n}{splat}\PYG{o}{.}\PYG{n}{modelFitMCMC}\PYG{p}{(}\PYG{n}{sp}\PYG{p}{,} \PYG{n}{mask\PYGZus{}standard}\PYG{o}{=}\PYG{k+kc}{True}\PYG{p}{,} \PYG{n}{initial\PYGZus{}guess}\PYG{o}{=}\PYG{p}{[}\PYG{n}{teff}\PYG{p}{,} \PYG{l+m+mf}{5.3}\PYG{p}{,} \PYG{l+m+mf}{0.}\PYG{p}{]}\PYG{p}{,} \PYG{n}{zstep}\PYG{o}{=}\PYG{l+m+mf}{0.1}\PYG{p}{,} \PYG{n}{nsamples}\PYG{o}{=}\PYG{l+m+mi}{100}\PYG{p}{,} \PYG{n}{savestep}\PYG{o}{=}\PYG{l+m+mi}{0}\PYG{p}{,} \PYG{n}{verbose}\PYG{o}{=}\PYG{k+kc}{False}\PYG{p}{)}
\PYG{g+gp}{\PYGZgt{}\PYGZgt{}\PYGZgt{} }\PYG{n}{splat}\PYG{o}{.}\PYG{n}{reportModelFitResults}\PYG{p}{(}\PYG{n}{sp}\PYG{p}{,} \PYG{n}{table}\PYG{p}{,} \PYG{n}{evol} \PYG{o}{=} \PYG{k+kc}{True}\PYG{p}{,} \PYG{n}{stats} \PYG{o}{=} \PYG{k+kc}{True}\PYG{p}{,} \PYG{n}{sigma} \PYG{o}{=} \PYG{l+m+mi}{2}\PYG{p}{,} \PYG{n}{triangle} \PYG{o}{=} \PYG{k+kc}{False}\PYG{p}{)}
\PYG{g+go}{    Number of steps = 169}

\PYG{g+go}{    Best Fit parameters:}
\PYG{g+go}{    Lowest chi2 value = 29567.2136599 for 169.0 degrees of freedom}
\PYG{g+go}{    Effective Temperature = 918.641 (K)}
\PYG{g+go}{    log Surface Gravity = 5.211}
\PYG{g+go}{    Metallicity = 0.000}
\PYG{g+go}{    Radius (relative to Sun) from surface fluxes = 0.096}

\PYG{g+go}{    Median parameters:}
\PYG{g+go}{    Effective Temperature = 927.875 + 71.635 \PYGZhy{} 73.237 (K)}
\PYG{g+go}{    log Surface Gravity = 5.210 + 0.283 \PYGZhy{} 0.927}
\PYG{g+go}{    Metallicity = 0.000 + 0.000 \PYGZhy{} 0.000}
\PYG{g+go}{    Radius (relative to Sun) from surface fluxes = 0.108 + 0.015 \PYGZhy{} 0.013}
\end{Verbatim}

\end{fulllineitems}

\begin{itemize}
\item {} 
\DUrole{xref,std,std-ref}{genindex}

\item {} 
\DUrole{xref,std,std-ref}{modindex}

\item {} 
\DUrole{xref,std,std-ref}{search}

\end{itemize}


\section{Brown Dwarf Evolutionary Models}
\label{bdevopar:brown-dwarf-evolutionary-models}\label{bdevopar::doc}\phantomsection\label{bdevopar:module-bdevopar}\index{bdevopar (module)}\index{Parameters (class in bdevopar)}

\begin{fulllineitems}
\phantomsection\label{bdevopar:bdevopar.Parameters}\pysigline{\sphinxstrong{class }\sphinxcode{bdevopar.}\sphinxbfcode{Parameters}}~\begin{quote}\begin{description}
\item[{Description}] \leavevmode
Allows the user to input a list of parameters.

\item[{Returns}] \leavevmode
\begin{Verbatim}[commandchars=\\\{\}]
\PYG{g+gp}{\PYGZgt{}\PYGZgt{}\PYGZgt{} }\PYG{n}{model} \PYG{o}{=} \PYG{n}{ReadModel}\PYG{p}{(}\PYG{l+s+s1}{\PYGZsq{}}\PYG{l+s+s1}{baraffe}\PYG{l+s+s1}{\PYGZsq{}}\PYG{p}{)}
\PYG{g+gp}{\PYGZgt{}\PYGZgt{}\PYGZgt{} }\PYG{n}{m} \PYG{o}{=} \PYG{p}{[}\PYG{l+m+mf}{0.04}\PYG{p}{,}\PYG{l+m+mf}{0.06}\PYG{p}{,}\PYG{l+m+mf}{0.07}\PYG{p}{]}
\PYG{g+gp}{\PYGZgt{}\PYGZgt{}\PYGZgt{} }\PYG{n}{a} \PYG{o}{=} \PYG{p}{[}\PYG{l+m+mf}{0.001}\PYG{p}{,}\PYG{l+m+mf}{0.003}\PYG{p}{,}\PYG{l+m+mf}{0.004}\PYG{p}{]}
\PYG{g+gp}{\PYGZgt{}\PYGZgt{}\PYGZgt{} }\PYG{n}{params} \PYG{o}{=} \PYG{n}{ParamsList}\PYG{p}{(}\PYG{n}{model}\PYG{p}{,}\PYG{n}{masses}\PYG{o}{=}\PYG{n}{m}\PYG{p}{,}\PYG{n}{ages}\PYG{o}{=}\PYG{n}{a}\PYG{p}{)}
\PYG{g+gp}{\PYGZgt{}\PYGZgt{}\PYGZgt{} }\PYG{n+nb}{print} \PYG{n}{params}
\end{Verbatim}

\end{description}\end{quote}

\end{fulllineitems}

\index{Params (class in bdevopar)}

\begin{fulllineitems}
\phantomsection\label{bdevopar:bdevopar.Params}\pysigline{\sphinxstrong{class }\sphinxcode{bdevopar.}\sphinxbfcode{Params}}~\begin{quote}\begin{description}
\item[{Description}] \leavevmode
Checks that the input parameters are valid (e.g user
inputs must be capable of being interpolated).
Interpolation between evolutionary models is done here
as well, and we used the method interp1d from astropy.

\item[{Parameters}] \leavevmode\begin{itemize}
\item {} 
\textbf{\texttt{*args}} -- 
If you're using Saumon's models, then u must specified the
metallicity since there's 5 total: hybrid\_solar, f2\_solar,
nc+0.3, nc-0.3, and nc\_solar.


\item {} 
\textbf{\texttt{*kwargs}} -- 
You must input any of the two following parameters: age,
mass, temperature, luminosity, gravity, or radius.


\end{itemize}

\item[{Returns}] \leavevmode

Returns the interpolated parameters.
\begin{itemize}
\item {} 
\textbf{Examples:}

\begin{Verbatim}[commandchars=\\\{\}]
\PYG{g+gp}{\PYGZgt{}\PYGZgt{}\PYGZgt{} }\PYG{n}{model} \PYG{o}{=} \PYG{n}{ReadModel}\PYG{p}{(}\PYG{l+s+s1}{\PYGZsq{}}\PYG{l+s+s1}{saumon}\PYG{l+s+s1}{\PYGZsq{}}\PYG{p}{,} \PYG{n}{z}\PYG{o}{=}\PYG{l+s+s1}{\PYGZsq{}}\PYG{l+s+s1}{hybrid\PYGZus{}solar}\PYG{l+s+s1}{\PYGZsq{}}\PYG{p}{)}
\PYG{g+gp}{\PYGZgt{}\PYGZgt{}\PYGZgt{} }\PYG{n}{saumon} \PYG{o}{=} \PYG{n}{Parameters}\PYG{p}{(}\PYG{n}{model}\PYG{p}{,} \PYG{n}{mass}\PYG{o}{=}\PYG{l+m+mf}{0.4}\PYG{p}{,} \PYG{n}{age}\PYG{o}{=}\PYG{l+m+mf}{0.01}\PYG{p}{)}
\PYG{g+gp}{\PYGZgt{}\PYGZgt{}\PYGZgt{} }\PYG{n}{model} \PYG{o}{=} \PYG{n}{ReadModel}\PYG{p}{(}\PYG{l+s+s1}{\PYGZsq{}}\PYG{l+s+s1}{saumon}\PYG{l+s+s1}{\PYGZsq{}}\PYG{p}{,} \PYG{n}{z}\PYG{o}{=}\PYG{l+s+s1}{\PYGZsq{}}\PYG{l+s+s1}{nc+0.3}\PYG{l+s+s1}{\PYGZsq{}}\PYG{p}{)}
\PYG{g+gp}{\PYGZgt{}\PYGZgt{}\PYGZgt{} }\PYG{n}{saumon} \PYG{o}{=} \PYG{n}{Parameters}\PYG{p}{(}\PYG{n}{a}\PYG{o}{=}\PYG{l+m+mf}{0.01}\PYG{p}{,} \PYG{n}{mss}\PYG{o}{=}\PYG{l+m+mf}{0.4}\PYG{p}{)}
\PYG{g+gp}{\PYGZgt{}\PYGZgt{}\PYGZgt{} }\PYG{n}{model} \PYG{o}{=} \PYG{n}{ReadModel}\PYG{p}{(}\PYG{l+s+s1}{\PYGZsq{}}\PYG{l+s+s1}{saumon}\PYG{l+s+s1}{\PYGZsq{}}\PYG{p}{,} \PYG{n}{z}\PYG{o}{=}\PYG{l+s+s1}{\PYGZsq{}}\PYG{l+s+s1}{nc\PYGZhy{}0.3}\PYG{l+s+s1}{\PYGZsq{}}\PYG{p}{)}
\PYG{g+gp}{\PYGZgt{}\PYGZgt{}\PYGZgt{} }\PYG{n}{saumon} \PYG{o}{=} \PYG{n}{Parameters}\PYG{p}{(}\PYG{n}{mas}\PYG{o}{=}\PYG{l+m+mf}{0.4}\PYG{p}{,} \PYG{n}{RiUS}\PYG{o}{=}\PYG{l+m+mf}{0.01}\PYG{p}{)}
\PYG{g+gp}{\PYGZgt{}\PYGZgt{}\PYGZgt{} }\PYG{n}{model} \PYG{o}{=} \PYG{n}{ReadModel}\PYG{p}{(}\PYG{l+s+s1}{\PYGZsq{}}\PYG{l+s+s1}{saumon}\PYG{l+s+s1}{\PYGZsq{}}\PYG{p}{,} \PYG{n}{z}\PYG{o}{=}\PYG{l+s+s1}{\PYGZsq{}}\PYG{l+s+s1}{nc\PYGZus{}solar}\PYG{l+s+s1}{\PYGZsq{}}\PYG{p}{)}
\PYG{g+gp}{\PYGZgt{}\PYGZgt{}\PYGZgt{} }\PYG{n}{saumon} \PYG{o}{=} \PYG{n}{Parameters}\PYG{p}{(}\PYG{n}{M}\PYG{o}{=}\PYG{l+m+mf}{0.4}\PYG{p}{,} \PYG{n}{AgE}\PYG{o}{=}\PYG{l+m+mf}{0.01}\PYG{p}{)}
\PYG{g+gp}{\PYGZgt{}\PYGZgt{}\PYGZgt{} }\PYG{n}{model} \PYG{o}{=} \PYG{n}{ReadModel}\PYG{p}{(}\PYG{l+s+s1}{\PYGZsq{}}\PYG{l+s+s1}{saumon}\PYG{l+s+s1}{\PYGZsq{}}\PYG{p}{,} \PYG{n}{z}\PYG{o}{=}\PYG{l+s+s1}{\PYGZsq{}}\PYG{l+s+s1}{f2\PYGZus{}solar}\PYG{l+s+s1}{\PYGZsq{}}\PYG{p}{)}
\PYG{g+gp}{\PYGZgt{}\PYGZgt{}\PYGZgt{} }\PYG{n}{saumon} \PYG{o}{=} \PYG{n}{Parameters}\PYG{p}{(}\PYG{n}{adf}\PYG{o}{=}\PYG{l+m+mf}{0.01}\PYG{p}{,} \PYG{n}{MASSSS}\PYG{o}{=}\PYG{l+m+mf}{0.4}\PYG{p}{)}
\end{Verbatim}

\begin{Verbatim}[commandchars=\\\{\}]
\PYG{g+gp}{\PYGZgt{}\PYGZgt{}\PYGZgt{} }\PYG{n}{model} \PYG{o}{=} \PYG{n}{ReadModel}\PYG{p}{(}\PYG{l+s+s1}{\PYGZsq{}}\PYG{l+s+s1}{baraffe}\PYG{l+s+s1}{\PYGZsq{}}\PYG{p}{,} \PYG{n}{z}\PYG{o}{=}\PYG{l+s+s1}{\PYGZsq{}}\PYG{l+s+s1}{f2\PYGZus{}solar}\PYG{l+s+s1}{\PYGZsq{}}\PYG{p}{)}
\PYG{g+gp}{\PYGZgt{}\PYGZgt{}\PYGZgt{} }\PYG{n}{burrows} \PYG{o}{=} \PYG{n}{Parameters}\PYG{p}{(}\PYG{n}{model}\PYG{p}{,} \PYG{n}{L}\PYG{o}{=}\PYG{l+m+mf}{0.05}\PYG{p}{,} \PYG{n}{Gradfty}\PYG{o}{=}\PYG{l+m+mf}{0.4}\PYG{p}{)}
\PYG{g+gp}{\PYGZgt{}\PYGZgt{}\PYGZgt{} }\PYG{n}{model} \PYG{o}{=} \PYG{n}{ReadModel}\PYG{p}{(}\PYG{l+s+s1}{\PYGZsq{}}\PYG{l+s+s1}{burrows}\PYG{l+s+s1}{\PYGZsq{}}\PYG{p}{,} \PYG{n}{z}\PYG{o}{=}\PYG{l+s+s1}{\PYGZsq{}}\PYG{l+s+s1}{f2\PYGZus{}solar}\PYG{l+s+s1}{\PYGZsq{}}\PYG{p}{)}
\PYG{g+gp}{\PYGZgt{}\PYGZgt{}\PYGZgt{} }\PYG{n}{baraffe} \PYG{o}{=} \PYG{n}{Parameters}\PYG{p}{(}\PYG{n}{model}\PYG{p}{,} \PYG{n}{temp}\PYG{o}{=}\PYG{l+m+mi}{2000}\PYG{p}{,} \PYG{n}{age}\PYG{o}{=}\PYG{l+m+mf}{0.4}\PYG{p}{)}
\end{Verbatim}

\end{itemize}


\end{description}\end{quote}

\begin{notice}{note}{Note:}
Keyword spelling doesn't matter as long as the first
letter is right. Also, ordering of the keywords is not
important. However, metallicites must be spelled exactly
as mentioned aboved--there are no execptions.
\end{notice}

\end{fulllineitems}

\index{ReadModel (class in bdevopar)}

\begin{fulllineitems}
\phantomsection\label{bdevopar:bdevopar.ReadModel}\pysigline{\sphinxstrong{class }\sphinxcode{bdevopar.}\sphinxbfcode{ReadModel}}~\begin{quote}\begin{description}
\item[{Description}] \leavevmode
This class reads in evolutionary models that are defined
in the methods below, and their data is acquired \href{http://pono.ucsd.edu/~adam/splat/EvolutionaryModels/}{here}.
Units are the following: masses are in M/Msun,
luminosities in log L/Lsun, radius in R/Rsun, surface
gravities in log g (cm/s\textasciicircum{}2), temperatures in Kelvin,
and ages in Gyr. Each evolutionary model gives
snapshots of brown dwarfs with different masses and
other physical properties as a function of time.

\item[{Parameters}] \leavevmode
\textbf{\texttt{model}} -- \begin{itemize}
\item {} 
\textbf{baraffe:}
Isochrones from Baraffe et. la models (2003), described in the
following paper: ``Evolutionary models for cool brown dwarfs and
extrasolar giant planets. The case of HD 20945'': \href{http://arxiv.org/abs/astro-ph/0302293}{Here}. Original model's \href{https://perso.ens-lyon.fr/isabelle.baraffe/COND03\_models}{URL}.
Ages (in Gyr) used for interpolation were the following:

\begin{Verbatim}[commandchars=\\\{\}]
\PYG{g+gp}{\PYGZgt{}\PYGZgt{}\PYGZgt{} }\PYG{k+kn}{from} \PYG{n+nn}{bdevopar} \PYG{k}{import} \PYG{o}{*}
\PYG{g+gp}{\PYGZgt{}\PYGZgt{}\PYGZgt{} }\PYG{n+nb}{print} \PYG{n}{ReadModel}\PYG{p}{(}\PYG{l+s+s1}{\PYGZsq{}}\PYG{l+s+s1}{baraffe}\PYG{l+s+s1}{\PYGZsq{}}\PYG{p}{)}\PYG{p}{[}\PYG{l+s+s1}{\PYGZsq{}}\PYG{l+s+s1}{age}\PYG{l+s+s1}{\PYGZsq{}}\PYG{p}{]}
\PYG{g+go}{[0.001, 0.005, 0.01, 0.05, 0.1, 0.12, 0.5, 1.0, 5.0, 10.0]}
\PYG{g+gp}{\PYGZgt{}\PYGZgt{}\PYGZgt{} }\PYG{n+nb}{print} \PYG{n}{ReadModel}\PYG{p}{(}\PYG{l+s+s1}{\PYGZsq{}}\PYG{l+s+s1}{BaRaFfE}\PYG{l+s+s1}{\PYGZsq{}}\PYG{p}{)}\PYG{p}{[}\PYG{l+s+s1}{\PYGZsq{}}\PYG{l+s+s1}{age}\PYG{l+s+s1}{\PYGZsq{}}\PYG{p}{]}
\PYG{g+go}{[0.001, 0.005, 0.01, 0.05, 0.1, 0.12, 0.5, 1.0, 5.0, 10.0]}
\end{Verbatim}

\end{itemize}
\begin{quote}

\begin{notice}{note}{Note:}
Capilatizing some letters or none won't cause any problems.
However, incorrect spelling will raise a NameError, so
please make sure you spell correctly.
\end{notice}
\end{quote}
\begin{itemize}
\item {} 
\textbf{burrows:}

\item {} 
\textbf{saumon:}
Isochrones from Saumon \& Marley models (2008), described in
\href{http://adsabs.harvard.edu/abs/2008ApJ...689.1327S}{paper}.
Original models' URL and the README used to differentiate
between metallicites are found \href{https://laws.lanl.gov/x7/dsaumon/BD\_evolution/}{here.}
Brown dwarfs ages used here are as follows:
\begin{quote}

\begin{Verbatim}[commandchars=\\\{\}]
\PYG{g+gp}{\PYGZgt{}\PYGZgt{}\PYGZgt{} }\PYG{k+kn}{from} \PYG{n+nn}{bdevopar} \PYG{k}{import} \PYG{o}{*}
\PYG{g+gp}{\PYGZgt{}\PYGZgt{}\PYGZgt{} }\PYG{n+nb}{print} \PYG{n}{ReadModel}\PYG{p}{(}\PYG{l+s+s1}{\PYGZsq{}}\PYG{l+s+s1}{saumon}\PYG{l+s+s1}{\PYGZsq{}}\PYG{p}{)}\PYG{p}{[}\PYG{l+s+s1}{\PYGZsq{}}\PYG{l+s+s1}{age}\PYG{l+s+s1}{\PYGZsq{}}\PYG{p}{]}
\PYG{g+go}{[0.003, 0.004, 0.006, 0.008, 0.01, 0.015, 0.02, 0.03, 0.04,}
\PYG{g+go}{0.06, 0.08, 0.1, 0.15, 0.2, 0.3, 0.4, 0.6, 0.8, 1.0, 1.5,}
\PYG{g+go}{2.0, 3.0, 4.0, 6.0, 8.0, 10.0]}
\end{Verbatim}
\begin{quote}\begin{description}
\item[{metallicity}] \leavevmode\begin{itemize}
\item {} 
\textbf{nc\_solar:} The atmosphere model is cloudless with {[}M/H{]}=0

\item {} 
\textbf{nc+0.3:} The atmosphere model is cloudless with {[}M/H{]}=+0.3

\item {} 
\textbf{nc-0.3:} The atmosphere model is cloudless with {[}M/H{]}=-0.3

\item {} 
\textbf{f2\_solar:} Atmosphere model is cloudless Fsed=2) with {[}M/H{]}=0

\item {} 
\textbf{hybrid\_solar} Atmosphere cloudless(Fsed=2 to nc) with {[}M/H{]}=0

\end{itemize}

\begin{Verbatim}[commandchars=\\\{\}]
\PYG{g+gp}{\PYGZgt{}\PYGZgt{}\PYGZgt{} }\PYG{k+kn}{from} \PYG{n+nn}{bdevopar} \PYG{k}{import} \PYG{o}{*}
\PYG{g+gp}{\PYGZgt{}\PYGZgt{}\PYGZgt{} }\PYG{n}{saumon} \PYG{o}{=} \PYG{n}{ReadModel}\PYG{p}{(}\PYG{l+s+s1}{\PYGZsq{}}\PYG{l+s+s1}{saumon}\PYG{l+s+s1}{\PYGZsq{}}\PYG{p}{,}\PYG{n}{metallicity}\PYG{o}{=}\PYG{l+s+s1}{\PYGZsq{}}\PYG{l+s+s1}{nc\PYGZus{}solar}\PYG{l+s+s1}{\PYGZsq{}}\PYG{p}{)}
\PYG{g+gp}{\PYGZgt{}\PYGZgt{}\PYGZgt{} }\PYG{n}{saumon} \PYG{o}{=} \PYG{n}{ReadModel}\PYG{p}{(}\PYG{l+s+s1}{\PYGZsq{}}\PYG{l+s+s1}{saumon}\PYG{l+s+s1}{\PYGZsq{}}\PYG{p}{,}\PYG{n}{metallicity}\PYG{o}{=}\PYG{l+s+s1}{\PYGZsq{}}\PYG{l+s+s1}{f2\PYGZus{}solar}\PYG{l+s+s1}{\PYGZsq{}}\PYG{p}{)}
\PYG{g+gp}{\PYGZgt{}\PYGZgt{}\PYGZgt{} }\PYG{n}{saumon} \PYG{o}{=} \PYG{n}{ReadModel}\PYG{p}{(}\PYG{l+s+s1}{\PYGZsq{}}\PYG{l+s+s1}{saumon}\PYG{l+s+s1}{\PYGZsq{}}\PYG{p}{,}\PYG{n}{metallicity}\PYG{o}{=}\PYG{l+s+s1}{\PYGZsq{}}\PYG{l+s+s1}{hybri\PYGZus{}solar}\PYG{l+s+s1}{\PYGZsq{}}\PYG{p}{)}
\end{Verbatim}

\end{description}\end{quote}
\end{quote}

\end{itemize}


\item[{Return}] \leavevmode
A dictionary where each key maps to a 3 dimensional matrix. The
dimensions are as follows: (Age x Mass x OtherParam). This
method is used in the subclass Parameters, which is
further discussed in the section belowed.

\end{description}\end{quote}

\end{fulllineitems}

\begin{itemize}
\item {} 
\DUrole{xref,std,std-ref}{genindex}

\item {} 
\DUrole{xref,std,std-ref}{modindex}

\item {} 
\DUrole{xref,std,std-ref}{search}

\end{itemize}


\section{Known Bugs}
\label{bugs:known-bugs}\label{bugs::doc}
As of 29 July 2015


\subsection{High priority}
\label{bugs:high-priority}
Currently broken or doesn't exist and should:
\begin{itemize}
\item {} 
Documentation incomplete

\item {} 
MCMC fitting code in splat\_model.py currently not functional (temporary code in place)

\item {} 
\sphinxcode{searchLibrary} needs to be ditched in exchange for SQL

\item {} 
calling \sphinxcode{searchLibrary(spt = sp, logic = “and”)} returns the full database

\item {} 
no \sphinxcode{plotIndices} function

\item {} 
no \sphinxcode{spectralBinary} function

\item {} 
no tabular output of data

\item {} 
need a reddening function

\item {} 
young objects are not properly selected by \sphinxcode{searchLibrary} - these need to be better labeled in source database

\item {} 
need a bibliography tool (feed in bibcode, get out citation information)

\end{itemize}


\subsection{Medium priority}
\label{bugs:medium-priority}
Annoying but functional:
\begin{itemize}
\item {} 
\sphinxcode{plotSpectrum}: add inset plots

\item {} 
\sphinxcode{plotSpectrum}: if you set don't set ``multipage'' then layout doesn't do anything

\item {} 
\sphinxcode{plotSpectrum}: font scaling is not consistent for all devices - add a fontScale parameter

\item {} 
dividing two spectra cannot be plotted - not sure why

\item {} 
need a way to store/read in indices

\item {} 
evolutionary models may not be returning accurate values - this needs to be validated

\item {} 
if an error occurs or you stop a spectrum download midway, that (blank) file persists and causes problems - deleting in advance does not seem to work

\item {} 
not entirely sure EW program is returning correct values

\end{itemize}


\subsection{Low priority}
\label{bugs:low-priority}
Would be nice if these worked better, or some convenient things to add:
\begin{itemize}
\item {} 
\sphinxcode{compareSpectra} and \sphinxcode{plotSpectrum}: would be nice if this plotted the difference spectrum

\item {} 
\sphinxcode{plotSpectrum}: add in alternate feature labeling

\item {} 
\sphinxcode{plotSpectrum}: add more atomic/molecular features

\item {} 
add in subdwarf and gravity standards

\item {} 
add in mean templates (from Kelle Cruz)

\item {} 
model fitting that is grid style (i.e., not MCMC)

\item {} 
visualization for evolutionary model parameters

\end{itemize}
\begin{itemize}
\item {} 
\DUrole{xref,std,std-ref}{genindex}

\item {} 
\DUrole{xref,std,std-ref}{modindex}

\item {} 
\DUrole{xref,std,std-ref}{search}

\end{itemize}


\section{API}
\label{api::doc}\label{api:api}

\subsection{SPLAT Classes}
\label{api:splat-classes}

\subsubsection{Spectrum}
\label{api:spectrum}\index{Spectrum (class in splat)}

\begin{fulllineitems}
\phantomsection\label{api:splat.Spectrum}\pysiglinewithargsret{\sphinxstrong{class }\sphinxcode{splat.}\sphinxbfcode{Spectrum}}{\emph{*args}, \emph{**kwargs}}{}~\begin{quote}\begin{description}
\item[{Description}] \leavevmode
Primary class for containing spectral and source data for SpeX Prism Library.

\item[{Parameters}] \leavevmode\begin{itemize}
\item {} 
\textbf{\texttt{ismodel}} (\emph{\texttt{optional, default = False}}) -- 

\item {} 
\textbf{\texttt{wlabel}} (\emph{\texttt{optional, default = 'Wavelength'}}) -- label of wavelength

\item {} 
\textbf{\texttt{wunit}} (optional, default = \sphinxcode{u.micron}) -- unit in which wavelength is measured

\item {} 
\textbf{\texttt{wunit\_label}} (optional, default = \sphinxtitleref{mu m}) -- label of the unit of wavelength

\item {} 
\textbf{\texttt{flabel}} (optional, default = \(F_{\lambda}\)) -- label of flux density

\item {} 
\textbf{\texttt{fscale}} (\emph{\texttt{optional, default = '{'}}}) -- string describing how flux density is scaled

\item {} 
\textbf{\texttt{funit}} (optional, default = \sphinxcode{u.erg/(u.cm**2 * u.s * u.micron)}) -- unit in which flux density is measured

\item {} 
\textbf{\texttt{funit\_label}} (optional, default = \(erg\;cm^{-2} s^{-1} \mu m^{-1}\)) -- label of the unit of flux density

\item {} 
\textbf{\texttt{resolution}} (\emph{\texttt{optional, default = 150}}) -- 

\item {} 
\textbf{\texttt{slitpixelwidth}} (\emph{\texttt{optional, default = 3.33}}) -- Width of the slit measured in subpixel values.

\item {} 
\textbf{\texttt{slitwidth}} (optional, default = \sphinxcode{slitpixelwidth} * 0.15) -- Actual width of the slit, measured in arc seconds. Default value is the \sphinxcode{slitpixelwidth} multiplied by the spectrograph pixel scale of 0.15 arcseconds.

\item {} 
\textbf{\texttt{header}} (\emph{\texttt{optional, default = Table()}}) -- header info of the spectrum

\item {} 
\textbf{\texttt{filename}} (\emph{\texttt{optional, default = '{'}}}) -- a string containing the spectrum's filename.

\item {} 
\textbf{\texttt{file}} (\emph{\texttt{optional, default = '{'}}}) -- same as filename

\item {} 
\textbf{\texttt{idkey}} (\emph{\texttt{optional, default = False}}) -- spectrum key of the desired spectrum

\end{itemize}

\item[{Example}] \leavevmode
\begin{Verbatim}[commandchars=\\\{\}]
\PYG{g+gp}{\PYGZgt{}\PYGZgt{}\PYGZgt{} }\PYG{k+kn}{import} \PYG{n+nn}{splat}
\PYG{g+gp}{\PYGZgt{}\PYGZgt{}\PYGZgt{} }\PYG{n}{sp} \PYG{o}{=} \PYG{n}{splat}\PYG{o}{.}\PYG{n}{Spectrum}\PYG{p}{(}\PYG{n}{filename}\PYG{o}{=}\PYG{l+s+s1}{\PYGZsq{}}\PYG{l+s+s1}{myspectrum.fits}\PYG{l+s+s1}{\PYGZsq{}}\PYG{p}{)}      \PYG{c+c1}{\PYGZsh{} read in a file}
\PYG{g+gp}{\PYGZgt{}\PYGZgt{}\PYGZgt{} }\PYG{n}{sp} \PYG{o}{=} \PYG{n}{splat}\PYG{o}{.}\PYG{n}{Spectrum}\PYG{p}{(}\PYG{l+s+s1}{\PYGZsq{}}\PYG{l+s+s1}{myspectrum.fits}\PYG{l+s+s1}{\PYGZsq{}}\PYG{p}{)}               \PYG{c+c1}{\PYGZsh{} same}
\PYG{g+gp}{\PYGZgt{}\PYGZgt{}\PYGZgt{} }\PYG{n}{sp} \PYG{o}{=} \PYG{n}{splat}\PYG{o}{.}\PYG{n}{Spectrum}\PYG{p}{(}\PYG{l+m+mi}{10002}\PYG{p}{)}                           \PYG{c+c1}{\PYGZsh{} read in spectrum with idkey = 10002}
\PYG{g+gp}{\PYGZgt{}\PYGZgt{}\PYGZgt{} }\PYG{n}{sp} \PYG{o}{=} \PYG{n}{splat}\PYG{o}{.}\PYG{n}{Spectrum}\PYG{p}{(}\PYG{n}{wave}\PYG{o}{=}\PYG{n}{wavearray}\PYG{p}{,}\PYG{n}{flux}\PYG{o}{=}\PYG{n}{fluxarray}\PYG{p}{)}   \PYG{c+c1}{\PYGZsh{} create objects with wavelength \PYGZam{} flux arrays}
\end{Verbatim}

\end{description}\end{quote}
\index{computeSN() (splat.Spectrum method)}

\begin{fulllineitems}
\phantomsection\label{api:splat.Spectrum.computeSN}\pysiglinewithargsret{\sphinxbfcode{computeSN}}{}{}~\begin{quote}\begin{description}
\item[{Purpose}] \leavevmode
Compute a representative S/N value as the median value of S/N among the top 50\% of flux values

\item[{Output}] \leavevmode
the S/N value

\item[{Example}] \leavevmode
\begin{Verbatim}[commandchars=\\\{\}]
\PYG{g+gp}{\PYGZgt{}\PYGZgt{}\PYGZgt{} }\PYG{k+kn}{import} \PYG{n+nn}{splat}
\PYG{g+gp}{\PYGZgt{}\PYGZgt{}\PYGZgt{} }\PYG{n}{sp} \PYG{o}{=} \PYG{n}{splat}\PYG{o}{.}\PYG{n}{getSpectrum}\PYG{p}{(}\PYG{n}{lucky}\PYG{o}{=}\PYG{k+kc}{True}\PYG{p}{)}\PYG{p}{[}\PYG{l+m+mi}{0}\PYG{p}{]}
\PYG{g+gp}{\PYGZgt{}\PYGZgt{}\PYGZgt{} }\PYG{n}{sp}\PYG{o}{.}\PYG{n}{computeSN}\PYG{p}{(}\PYG{p}{)}
\PYG{g+go}{115.96374031163553}
\end{Verbatim}

\end{description}\end{quote}

\end{fulllineitems}

\index{copy() (splat.Spectrum method)}

\begin{fulllineitems}
\phantomsection\label{api:splat.Spectrum.copy}\pysiglinewithargsret{\sphinxbfcode{copy}}{}{}~\begin{quote}\begin{description}
\item[{Purpose}] \leavevmode
Make a copy of a Spectrum object

\end{description}\end{quote}

\end{fulllineitems}

\index{export() (splat.Spectrum method)}

\begin{fulllineitems}
\phantomsection\label{api:splat.Spectrum.export}\pysiglinewithargsret{\sphinxbfcode{export}}{\emph{*args}, \emph{**kwargs}}{}~\begin{quote}\begin{description}
\item[{Purpose}] \leavevmode
Exports a Spectrum object to either a fits or ascii file, depending on file extension given.  If no filename is explicitly given, the Spectrum.filename attribute is used. If the filename does not include the full path, the file is saved in the current directory.  Spectrum.export and Spectrum.save function in the same manner.

\end{description}\end{quote}
\begin{quote}\begin{description}
\item[{Parameters}] \leavevmode
\textbf{\texttt{filename}} (\emph{\texttt{optional, default = Spectrum.simplefilename}}) -- String specifying the filename to save

\item[{Output}] \leavevmode
An ascii or fits file with the data

\item[{Example}] \leavevmode
\begin{Verbatim}[commandchars=\\\{\}]
\PYG{g+gp}{\PYGZgt{}\PYGZgt{}\PYGZgt{} }\PYG{k+kn}{import} \PYG{n+nn}{splat}
\PYG{g+gp}{\PYGZgt{}\PYGZgt{}\PYGZgt{} }\PYG{n}{sp} \PYG{o}{=} \PYG{n}{splat}\PYG{o}{.}\PYG{n}{getSpectrum}\PYG{p}{(}\PYG{n}{lucky}\PYG{o}{=}\PYG{k+kc}{True}\PYG{p}{)}\PYG{p}{[}\PYG{l+m+mi}{0}\PYG{p}{]}
\PYG{g+gp}{\PYGZgt{}\PYGZgt{}\PYGZgt{} }\PYG{n}{sp}\PYG{o}{.}\PYG{n}{export}\PYG{p}{(}\PYG{l+s+s1}{\PYGZsq{}}\PYG{l+s+s1}{/Users/adam/myspectrum.txt}\PYG{l+s+s1}{\PYGZsq{}}\PYG{p}{)}
\PYG{g+gp}{\PYGZgt{}\PYGZgt{}\PYGZgt{} }\PYG{k+kn}{from} \PYG{n+nn}{astropy}\PYG{n+nn}{.}\PYG{n+nn}{io} \PYG{k}{import} \PYG{n}{ascii}
\PYG{g+gp}{\PYGZgt{}\PYGZgt{}\PYGZgt{} }\PYG{n}{data} \PYG{o}{=} \PYG{n}{ascii}\PYG{o}{.}\PYG{n}{read}\PYG{p}{(}\PYG{l+s+s1}{\PYGZsq{}}\PYG{l+s+s1}{/Users/adam/myspectrum.txt}\PYG{l+s+s1}{\PYGZsq{}}\PYG{p}{,}\PYG{n+nb}{format}\PYG{o}{=}\PYG{l+s+s1}{\PYGZsq{}}\PYG{l+s+s1}{tab}\PYG{l+s+s1}{\PYGZsq{}}\PYG{p}{)}
\PYG{g+gp}{\PYGZgt{}\PYGZgt{}\PYGZgt{} }\PYG{n}{data}
\PYG{g+go}{ \PYGZlt{}Table length=564\PYGZgt{}}
\PYG{g+go}{   wavelength          flux          uncertainty}
\PYG{g+go}{    float64          float64           float64}
\PYG{g+go}{ \PYGZhy{}\PYGZhy{}\PYGZhy{}\PYGZhy{}\PYGZhy{}\PYGZhy{}\PYGZhy{}\PYGZhy{}\PYGZhy{}\PYGZhy{}\PYGZhy{}\PYGZhy{}\PYGZhy{}\PYGZhy{} \PYGZhy{}\PYGZhy{}\PYGZhy{}\PYGZhy{}\PYGZhy{}\PYGZhy{}\PYGZhy{}\PYGZhy{}\PYGZhy{}\PYGZhy{}\PYGZhy{}\PYGZhy{}\PYGZhy{}\PYGZhy{}\PYGZhy{}\PYGZhy{}\PYGZhy{} \PYGZhy{}\PYGZhy{}\PYGZhy{}\PYGZhy{}\PYGZhy{}\PYGZhy{}\PYGZhy{}\PYGZhy{}\PYGZhy{}\PYGZhy{}\PYGZhy{}\PYGZhy{}\PYGZhy{}\PYGZhy{}\PYGZhy{}\PYGZhy{}\PYGZhy{}}
\PYG{g+go}{ 0.645418405533               0.0               nan}
\PYG{g+go}{ 0.647664904594 6.71920214475e\PYGZhy{}16 3.71175052033e\PYGZhy{}16}
\PYG{g+go}{ 0.649897933006 1.26009925777e\PYGZhy{}15 3.85722895842e\PYGZhy{}16}
\PYG{g+go}{ 0.652118623257 7.23781818374e\PYGZhy{}16 3.68178778862e\PYGZhy{}16}
\PYG{g+go}{ 0.654327988625 1.94569566622e\PYGZhy{}15 3.21007116982e\PYGZhy{}16}
\PYG{g+go}{ ...}
\end{Verbatim}

\end{description}\end{quote}

\end{fulllineitems}

\index{flamToFnu() (splat.Spectrum method)}

\begin{fulllineitems}
\phantomsection\label{api:splat.Spectrum.flamToFnu}\pysiglinewithargsret{\sphinxbfcode{flamToFnu}}{}{}~\begin{quote}
\begin{quote}\begin{description}
\item[{Purpose}] \leavevmode
Converts flux density from \(F_{\lambda}\) to :math:{\color{red}\bfseries{}{}`}F\_\{

\end{description}\end{quote}
\end{quote}

u\}{}`, the latter in Jy. This routine changes the underlying Spectrum object. There is no change if the spectrum is already in \(F_{
u}\) units.
\begin{quote}
\begin{quote}\begin{description}
\item[{Example}] \leavevmode
\begin{Verbatim}[commandchars=\\\{\}]
\PYG{g+gp}{\PYGZgt{}\PYGZgt{}\PYGZgt{} }\PYG{k+kn}{import} \PYG{n+nn}{splat}
\PYG{g+gp}{\PYGZgt{}\PYGZgt{}\PYGZgt{} }\PYG{n}{sp} \PYG{o}{=} \PYG{n}{splat}\PYG{o}{.}\PYG{n}{getSpectrum}\PYG{p}{(}\PYG{n}{lucky}\PYG{o}{=}\PYG{k+kc}{True}\PYG{p}{)}\PYG{p}{[}\PYG{l+m+mi}{0}\PYG{p}{]}
\PYG{g+gp}{\PYGZgt{}\PYGZgt{}\PYGZgt{} }\PYG{n}{sp}\PYG{o}{.}\PYG{n}{flamToFnu}\PYG{p}{(}\PYG{p}{)}
\PYG{g+gp}{\PYGZgt{}\PYGZgt{}\PYGZgt{} }\PYG{n}{sp}\PYG{o}{.}\PYG{n}{flux}\PYG{o}{.}\PYG{n}{unit}
\PYG{g+go}{ Unit(\PYGZdq{}Jy\PYGZdq{})}
\end{Verbatim}

\end{description}\end{quote}
\end{quote}

\end{fulllineitems}

\index{fluxCalibrate() (splat.Spectrum method)}

\begin{fulllineitems}
\phantomsection\label{api:splat.Spectrum.fluxCalibrate}\pysiglinewithargsret{\sphinxbfcode{fluxCalibrate}}{\emph{filter}, \emph{mag}, \emph{**kwargs}}{}~\begin{quote}\begin{description}
\item[{Purpose}] \leavevmode
Flux calibrates a spectrum given a filter and a magnitude. The filter must be one of those listed in splat.FILTERS.keys(). It is possible to specifically set the magnitude to be absolute (by default it is apparent).  This function changes the Spectrum object's flux, noise and variance arrays.

\item[{Parameters}] \leavevmode\begin{itemize}
\item {} 
\textbf{\texttt{filter}} (\emph{\texttt{string, default = None}}) -- name of filter

\item {} 
\textbf{\texttt{mag}} (\emph{\texttt{float, default = None}}) -- magnitude to scale too

\item {} 
\textbf{\texttt{absolute}} (\emph{\texttt{Boolean, optional, default = False}}) -- given magnitude is an absolute magnitude

\item {} 
\textbf{\texttt{apparent}} (\emph{\texttt{Boolean, optional, default = False}}) -- given magnitude is an apparent magnitude

\end{itemize}

\item[{Example}] \leavevmode
\begin{Verbatim}[commandchars=\\\{\}]
\PYG{g+gp}{\PYGZgt{}\PYGZgt{}\PYGZgt{} }\PYG{k+kn}{import} \PYG{n+nn}{splat}
\PYG{g+gp}{\PYGZgt{}\PYGZgt{}\PYGZgt{} }\PYG{n}{sp} \PYG{o}{=} \PYG{n}{splat}\PYG{o}{.}\PYG{n}{getSpectrum}\PYG{p}{(}\PYG{n}{lucky}\PYG{o}{=}\PYG{k+kc}{True}\PYG{p}{)}\PYG{p}{[}\PYG{l+m+mi}{0}\PYG{p}{]}
\PYG{g+gp}{\PYGZgt{}\PYGZgt{}\PYGZgt{} }\PYG{n}{sp}\PYG{o}{.}\PYG{n}{fluxCalibrate}\PYG{p}{(}\PYG{l+s+s1}{\PYGZsq{}}\PYG{l+s+s1}{2MASS J}\PYG{l+s+s1}{\PYGZsq{}}\PYG{p}{,}\PYG{l+m+mf}{15.0}\PYG{p}{)}
\PYG{g+gp}{\PYGZgt{}\PYGZgt{}\PYGZgt{} }\PYG{n}{splat}\PYG{o}{.}\PYG{n}{filterMag}\PYG{p}{(}\PYG{n}{sp}\PYG{p}{,}\PYG{l+s+s1}{\PYGZsq{}}\PYG{l+s+s1}{2MASS J}\PYG{l+s+s1}{\PYGZsq{}}\PYG{p}{)}
\PYG{g+go}{ (15.002545668628173, 0.017635234089677564)}
\end{Verbatim}

\end{description}\end{quote}

\end{fulllineitems}

\index{fluxMax() (splat.Spectrum method)}

\begin{fulllineitems}
\phantomsection\label{api:splat.Spectrum.fluxMax}\pysiglinewithargsret{\sphinxbfcode{fluxMax}}{\emph{**kwargs}}{}~\begin{quote}\begin{description}
\item[{Purpose}] \leavevmode
Reports the maximum flux of a Spectrum object ignoring nan's.

\item[{Parameters}] \leavevmode
\textbf{\texttt{maskTelluric}} (\emph{\texttt{optional, default = True}}) -- masks telluric regions

\item[{Output}] \leavevmode
maximum flux (with units)

\item[{Example}] \leavevmode
\begin{Verbatim}[commandchars=\\\{\}]
\PYG{g+gp}{\PYGZgt{}\PYGZgt{}\PYGZgt{} }\PYG{k+kn}{import} \PYG{n+nn}{splat}
\PYG{g+gp}{\PYGZgt{}\PYGZgt{}\PYGZgt{} }\PYG{n}{sp} \PYG{o}{=} \PYG{n}{splat}\PYG{o}{.}\PYG{n}{getSpectrum}\PYG{p}{(}\PYG{n}{lucky}\PYG{o}{=}\PYG{k+kc}{True}\PYG{p}{)}\PYG{p}{[}\PYG{l+m+mi}{0}\PYG{p}{]}
\PYG{g+gp}{\PYGZgt{}\PYGZgt{}\PYGZgt{} }\PYG{n}{sp}\PYG{o}{.}\PYG{n}{normalize}\PYG{p}{(}\PYG{p}{)}
\PYG{g+gp}{\PYGZgt{}\PYGZgt{}\PYGZgt{} }\PYG{n}{sp}\PYG{o}{.}\PYG{n}{fluxMax}\PYG{p}{(}\PYG{p}{)}
\PYG{g+go}{\PYGZlt{}Quantity 1.0 erg / (cm2 micron s)\PYGZgt{}}
\end{Verbatim}

\end{description}\end{quote}

\end{fulllineitems}

\index{fnuToFlam() (splat.Spectrum method)}

\begin{fulllineitems}
\phantomsection\label{api:splat.Spectrum.fnuToFlam}\pysiglinewithargsret{\sphinxbfcode{fnuToFlam}}{}{}~\begin{quote}
\begin{quote}\begin{description}
\item[{Purpose}] \leavevmode
Converts flux density from :math:{\color{red}\bfseries{}{}`}F\_\{

\end{description}\end{quote}
\end{quote}

u\}{}` to \(F_{\lambda}\), the latter in erg/s/cm2/Hz. This routine changes the underlying Spectrum object. There is no change if the spectrum is already in \(F_{\lambda}\) units.
\begin{quote}
\begin{quote}\begin{description}
\item[{Example}] \leavevmode
\begin{Verbatim}[commandchars=\\\{\}]
\PYG{g+gp}{\PYGZgt{}\PYGZgt{}\PYGZgt{} }\PYG{k+kn}{import} \PYG{n+nn}{splat}
\PYG{g+gp}{\PYGZgt{}\PYGZgt{}\PYGZgt{} }\PYG{n}{sp} \PYG{o}{=} \PYG{n}{splat}\PYG{o}{.}\PYG{n}{getSpectrum}\PYG{p}{(}\PYG{n}{lucky}\PYG{o}{=}\PYG{k+kc}{True}\PYG{p}{)}\PYG{p}{[}\PYG{l+m+mi}{0}\PYG{p}{]}
\PYG{g+gp}{\PYGZgt{}\PYGZgt{}\PYGZgt{} }\PYG{n}{sp}\PYG{o}{.}\PYG{n}{flamToFnu}\PYG{p}{(}\PYG{p}{)}
\PYG{g+gp}{\PYGZgt{}\PYGZgt{}\PYGZgt{} }\PYG{n}{sp}\PYG{o}{.}\PYG{n}{flux}\PYG{o}{.}\PYG{n}{unit}
\PYG{g+go}{ Unit(\PYGZdq{}Jy\PYGZdq{})}
\PYG{g+gp}{\PYGZgt{}\PYGZgt{}\PYGZgt{} }\PYG{n}{sp}\PYG{o}{.}\PYG{n}{fnuToFlam}\PYG{p}{(}\PYG{p}{)}
\PYG{g+gp}{\PYGZgt{}\PYGZgt{}\PYGZgt{} }\PYG{n}{sp}\PYG{o}{.}\PYG{n}{flux}\PYG{o}{.}\PYG{n}{unit}
\PYG{g+go}{ Unit(\PYGZdq{}erg / (cm2 micron s)\PYGZdq{})}
\end{Verbatim}

\end{description}\end{quote}
\end{quote}

\end{fulllineitems}

\index{info() (splat.Spectrum method)}

\begin{fulllineitems}
\phantomsection\label{api:splat.Spectrum.info}\pysiglinewithargsret{\sphinxbfcode{info}}{}{}~\begin{quote}\begin{description}
\item[{Purpose}] \leavevmode
Returns a summary of properties for the Spectrum object

\item[{Output}] \leavevmode
Text summary

\item[{Example}] \leavevmode
\begin{Verbatim}[commandchars=\\\{\}]
\PYG{g+gp}{\PYGZgt{}\PYGZgt{}\PYGZgt{} }\PYG{k+kn}{import} \PYG{n+nn}{splat}
\PYG{g+gp}{\PYGZgt{}\PYGZgt{}\PYGZgt{} }\PYG{n}{sp} \PYG{o}{=} \PYG{n}{splat}\PYG{o}{.}\PYG{n}{getSpectrum}\PYG{p}{(}\PYG{n}{lucky}\PYG{o}{=}\PYG{k+kc}{True}\PYG{p}{)}\PYG{p}{[}\PYG{l+m+mi}{0}\PYG{p}{]}
\PYG{g+gp}{\PYGZgt{}\PYGZgt{}\PYGZgt{} }\PYG{n}{sp}\PYG{o}{.}\PYG{n}{info}\PYG{p}{(}\PYG{p}{)}
\PYG{g+go}{ Spectrum of NLTT 184}
\PYG{g+go}{ Observed on 20071012}
\PYG{g+go}{ at an airmass of 1.145}
\PYG{g+go}{ Full source designation is J00054517+0723423}
\PYG{g+go}{ Median S/N = 97.0}
\PYG{g+go}{ SPLAT source key is 10012.0}
\PYG{g+go}{ SPLAT spectrum key is 10857}
\PYG{g+go}{ Data published in Kirkpatrick, J. D. et al. (2010, ApJS, 190, 100\PYGZhy{}146)}
\PYG{g+go}{ History:}
\PYG{g+go}{ Spectrum successfully loaded}
\end{Verbatim}

\end{description}\end{quote}

\end{fulllineitems}

\index{normalize() (splat.Spectrum method)}

\begin{fulllineitems}
\phantomsection\label{api:splat.Spectrum.normalize}\pysiglinewithargsret{\sphinxbfcode{normalize}}{\emph{**kwargs}}{}~\begin{quote}\begin{description}
\item[{Purpose}] \leavevmode
Normalize a spectrum to a maximum value of 1 (in its current units)

\item[{Parameters}] \leavevmode
\textbf{\texttt{waveRange}} (\emph{\texttt{optional, default = None}}) -- choose the wavelength range to normalize; can be a list specifying minimum and maximum or a single number to normalize around a particular point

\item[{Output}] \leavevmode
maximum flux (with units)

\item[{Example}] \leavevmode
\begin{Verbatim}[commandchars=\\\{\}]
\PYG{g+gp}{\PYGZgt{}\PYGZgt{}\PYGZgt{} }\PYG{k+kn}{import} \PYG{n+nn}{splat}
\PYG{g+gp}{\PYGZgt{}\PYGZgt{}\PYGZgt{} }\PYG{n}{sp} \PYG{o}{=} \PYG{n}{splat}\PYG{o}{.}\PYG{n}{getSpectrum}\PYG{p}{(}\PYG{n}{lucky}\PYG{o}{=}\PYG{k+kc}{True}\PYG{p}{)}\PYG{p}{[}\PYG{l+m+mi}{0}\PYG{p}{]}
\PYG{g+gp}{\PYGZgt{}\PYGZgt{}\PYGZgt{} }\PYG{n}{sp}\PYG{o}{.}\PYG{n}{normalize}\PYG{p}{(}\PYG{p}{)}
\PYG{g+gp}{\PYGZgt{}\PYGZgt{}\PYGZgt{} }\PYG{n}{sp}\PYG{o}{.}\PYG{n}{fluxMax}\PYG{p}{(}\PYG{p}{)}
\PYG{g+go}{\PYGZlt{}Quantity 1.0 erg / (cm2 micron s)\PYGZgt{}}
\PYG{g+gp}{\PYGZgt{}\PYGZgt{}\PYGZgt{} }\PYG{n}{sp}\PYG{o}{.}\PYG{n}{normalize}\PYG{p}{(}\PYG{n}{waverange}\PYG{o}{=}\PYG{p}{[}\PYG{l+m+mf}{2.25}\PYG{p}{,}\PYG{l+m+mf}{2.3}\PYG{p}{]}\PYG{p}{)}
\PYG{g+gp}{\PYGZgt{}\PYGZgt{}\PYGZgt{} }\PYG{n}{sp}\PYG{o}{.}\PYG{n}{fluxMax}\PYG{p}{(}\PYG{p}{)}
\PYG{g+go}{\PYGZlt{}Quantity 1.591310977935791 erg / (cm2 micron s)\PYGZgt{}}
\end{Verbatim}

\end{description}\end{quote}

\end{fulllineitems}

\index{plot() (splat.Spectrum method)}

\begin{fulllineitems}
\phantomsection\label{api:splat.Spectrum.plot}\pysiglinewithargsret{\sphinxbfcode{plot}}{\emph{**kwargs}}{}~\begin{quote}\begin{description}
\item[{Purpose}] \leavevmode
calls the plotSpectrum function, by default showing the noise spectrum and zeropoints. See the plotSpectrum API listing for details.

\end{description}\end{quote}
\begin{quote}\begin{description}
\item[{Output}] \leavevmode
A plot of the Spectrum object

\item[{Example}] \leavevmode
\begin{Verbatim}[commandchars=\\\{\}]
\PYG{g+gp}{\PYGZgt{}\PYGZgt{}\PYGZgt{} }\PYG{k+kn}{import} \PYG{n+nn}{splat}
\PYG{g+gp}{\PYGZgt{}\PYGZgt{}\PYGZgt{} }\PYG{n}{sp} \PYG{o}{=} \PYG{n}{splat}\PYG{o}{.}\PYG{n}{getSpectrum}\PYG{p}{(}\PYG{n}{lucky}\PYG{o}{=}\PYG{k+kc}{True}\PYG{p}{)}\PYG{p}{[}\PYG{l+m+mi}{0}\PYG{p}{]}
\PYG{g+gp}{\PYGZgt{}\PYGZgt{}\PYGZgt{} }\PYG{n}{sp}\PYG{o}{.}\PYG{n}{plot}\PYG{p}{(}\PYG{p}{)}
\end{Verbatim}

\end{description}\end{quote}

\end{fulllineitems}

\index{reset() (splat.Spectrum method)}

\begin{fulllineitems}
\phantomsection\label{api:splat.Spectrum.reset}\pysiglinewithargsret{\sphinxbfcode{reset}}{}{}~\begin{quote}\begin{description}
\item[{Purpose}] \leavevmode
Restores a Spectrum to its original read-in state, removing scaling and smoothing. This routine changes the Spectrum object directly and there is no output.

\item[{Example}] \leavevmode
\begin{Verbatim}[commandchars=\\\{\}]
\PYG{g+gp}{\PYGZgt{}\PYGZgt{}\PYGZgt{} }\PYG{k+kn}{import} \PYG{n+nn}{splat}
\PYG{g+gp}{\PYGZgt{}\PYGZgt{}\PYGZgt{} }\PYG{n}{sp} \PYG{o}{=} \PYG{n}{splat}\PYG{o}{.}\PYG{n}{getSpectrum}\PYG{p}{(}\PYG{n}{lucky}\PYG{o}{=}\PYG{k+kc}{True}\PYG{p}{)}\PYG{p}{[}\PYG{l+m+mi}{0}\PYG{p}{]}
\PYG{g+gp}{\PYGZgt{}\PYGZgt{}\PYGZgt{} }\PYG{n}{sp}\PYG{o}{.}\PYG{n}{fluxMax}\PYG{p}{(}\PYG{p}{)}
\PYG{g+go}{\PYGZlt{}Quantity 4.561630292384622e\PYGZhy{}15 erg / (cm2 micron s)\PYGZgt{}}
\PYG{g+gp}{\PYGZgt{}\PYGZgt{}\PYGZgt{} }\PYG{n}{sp}\PYG{o}{.}\PYG{n}{normalize}\PYG{p}{(}\PYG{p}{)}
\PYG{g+gp}{\PYGZgt{}\PYGZgt{}\PYGZgt{} }\PYG{n}{sp}\PYG{o}{.}\PYG{n}{fluxMax}\PYG{p}{(}\PYG{p}{)}
\PYG{g+go}{\PYGZlt{}Quantity 0.9999999403953552 erg / (cm2 micron s)\PYGZgt{}}
\PYG{g+gp}{\PYGZgt{}\PYGZgt{}\PYGZgt{} }\PYG{n}{sp}\PYG{o}{.}\PYG{n}{reset}\PYG{p}{(}\PYG{p}{)}
\PYG{g+gp}{\PYGZgt{}\PYGZgt{}\PYGZgt{} }\PYG{n}{sp}\PYG{o}{.}\PYG{n}{fluxMax}\PYG{p}{(}\PYG{p}{)}
\PYG{g+go}{\PYGZlt{}Quantity 4.561630292384622e\PYGZhy{}15 erg / (cm2 micron s)\PYGZgt{}}
\end{Verbatim}

\end{description}\end{quote}

\end{fulllineitems}

\index{save() (splat.Spectrum method)}

\begin{fulllineitems}
\phantomsection\label{api:splat.Spectrum.save}\pysiglinewithargsret{\sphinxbfcode{save}}{\emph{*args}, \emph{**kwargs}}{}~\begin{quote}\begin{description}
\item[{Purpose}] \leavevmode
Exports a Spectrum object to either a fits or ascii file, depending on file extension given.  If no filename is explicitly given, the Spectrum.filename attribute is used. If the filename does not include the full path, the file is saved in the current directory.  Spectrum.export and Spectrum.save function in the same manner.

\end{description}\end{quote}

\end{fulllineitems}

\index{scale() (splat.Spectrum method)}

\begin{fulllineitems}
\phantomsection\label{api:splat.Spectrum.scale}\pysiglinewithargsret{\sphinxbfcode{scale}}{\emph{factor}, \emph{**kwargs}}{}~\begin{quote}\begin{description}
\item[{Purpose}] \leavevmode
Scales a Spectrum object's flux and noise values by a constant factor. This routine changes the Spectrum object directly.

\item[{Parameters}] \leavevmode
\textbf{\texttt{factor}} (\emph{\texttt{required, default = None}}) -- A floating point number used to scale the Spectrum object

\item[{Output}] \leavevmode
maximum flux (with units)

\item[{Example}] \leavevmode
\begin{Verbatim}[commandchars=\\\{\}]
\PYG{g+gp}{\PYGZgt{}\PYGZgt{}\PYGZgt{} }\PYG{k+kn}{import} \PYG{n+nn}{splat}
\PYG{g+gp}{\PYGZgt{}\PYGZgt{}\PYGZgt{} }\PYG{n}{sp} \PYG{o}{=} \PYG{n}{splat}\PYG{o}{.}\PYG{n}{getSpectrum}\PYG{p}{(}\PYG{n}{lucky}\PYG{o}{=}\PYG{k+kc}{True}\PYG{p}{)}\PYG{p}{[}\PYG{l+m+mi}{0}\PYG{p}{]}
\PYG{g+gp}{\PYGZgt{}\PYGZgt{}\PYGZgt{} }\PYG{n}{sp}\PYG{o}{.}\PYG{n}{fluxMax}\PYG{p}{(}\PYG{p}{)}
\PYG{g+go}{\PYGZlt{}Quantity 1.0577336634332284e\PYGZhy{}14 erg / (cm2 micron s)\PYGZgt{}}
\PYG{g+gp}{\PYGZgt{}\PYGZgt{}\PYGZgt{} }\PYG{n}{sp}\PYG{o}{.}\PYG{n}{computeSN}\PYG{p}{(}\PYG{p}{)}
\PYG{g+go}{124.5198}
\PYG{g+gp}{\PYGZgt{}\PYGZgt{}\PYGZgt{} }\PYG{n}{sp}\PYG{o}{.}\PYG{n}{scale}\PYG{p}{(}\PYG{l+m+mf}{1.e15}\PYG{p}{)}
\PYG{g+gp}{\PYGZgt{}\PYGZgt{}\PYGZgt{} }\PYG{n}{sp}\PYG{o}{.}\PYG{n}{fluxMax}\PYG{p}{(}\PYG{p}{)}
\PYG{g+go}{\PYGZlt{}Quantity 1.0577336549758911 erg / (cm2 micron s)\PYGZgt{}}
\PYG{g+gp}{\PYGZgt{}\PYGZgt{}\PYGZgt{} }\PYG{n}{sp}\PYG{o}{.}\PYG{n}{computeSN}\PYG{p}{(}\PYG{p}{)}
\PYG{g+go}{124.51981}
\end{Verbatim}

\end{description}\end{quote}

\end{fulllineitems}

\index{showHistory() (splat.Spectrum method)}

\begin{fulllineitems}
\phantomsection\label{api:splat.Spectrum.showHistory}\pysiglinewithargsret{\sphinxbfcode{showHistory}}{}{}~\begin{quote}\begin{description}
\item[{Purpose}] \leavevmode
Report history of actions taken on a Spectrum object. This can also be retrieved by printing the attribute Spectrum.history

\item[{Output}] \leavevmode
List of actions taken on spectrum

\item[{Example}] \leavevmode
\begin{Verbatim}[commandchars=\\\{\}]
\PYG{g+gp}{\PYGZgt{}\PYGZgt{}\PYGZgt{} }\PYG{k+kn}{import} \PYG{n+nn}{splat}
\PYG{g+gp}{\PYGZgt{}\PYGZgt{}\PYGZgt{} }\PYG{n}{sp} \PYG{o}{=} \PYG{n}{splat}\PYG{o}{.}\PYG{n}{getSpectrum}\PYG{p}{(}\PYG{n}{lucky}\PYG{o}{=}\PYG{k+kc}{True}\PYG{p}{)}\PYG{p}{[}\PYG{l+m+mi}{0}\PYG{p}{]}
\PYG{g+gp}{\PYGZgt{}\PYGZgt{}\PYGZgt{} }\PYG{n}{sp}\PYG{o}{.}\PYG{n}{normalize}\PYG{p}{(}\PYG{p}{)}
\PYG{g+gp}{\PYGZgt{}\PYGZgt{}\PYGZgt{} }\PYG{n}{sp}\PYG{o}{.}\PYG{n}{fluxCalibrate}\PYG{p}{(}\PYG{l+s+s1}{\PYGZsq{}}\PYG{l+s+s1}{2MASS J}\PYG{l+s+s1}{\PYGZsq{}}\PYG{p}{,}\PYG{l+m+mf}{15.0}\PYG{p}{)}
\PYG{g+gp}{\PYGZgt{}\PYGZgt{}\PYGZgt{} }\PYG{n}{sp}\PYG{o}{.}\PYG{n}{showHistory}\PYG{p}{(}\PYG{p}{)}
\PYG{g+go}{ Spectrum successfully loaded}
\PYG{g+go}{ Spectrum normalized}
\PYG{g+go}{ Flux calibrated with 2MASS J filter to an apparent magnitude of 15.0}
\end{Verbatim}

\end{description}\end{quote}

\end{fulllineitems}

\index{smooth() (splat.Spectrum method)}

\begin{fulllineitems}
\phantomsection\label{api:splat.Spectrum.smooth}\pysiglinewithargsret{\sphinxbfcode{smooth}}{\emph{**kwargs}}{}~\begin{quote}\begin{description}
\item[{Purpose}] \leavevmode
Smoothes a spectrum either by selecting a constant slit width (smooth in spectral dispersion space), pixel width (smooth in pixel space) or resolution (smooth in velocity space). One of these options must be selected for any smoothing to happen. Changes spectrum directly.

\item[{Parameters}] \leavevmode\begin{itemize}
\item {} 
\textbf{\texttt{method}} (\emph{\texttt{optional, default = Hanning}}) -- the type of smoothing window to use. See \url{http://docs.scipy.org/doc/scipy-0.14.0/reference/generated/scipy.signal.get\_window.html} for more details.

\item {} 
\textbf{\texttt{resolution}} (\emph{\texttt{optional, default = None}}) -- Constant resolution to smooth toe(see {\color{red}\bfseries{}smoothResolution\_})

\item {} 
\textbf{\texttt{slitPixelWidth}} (\emph{\texttt{optional, default = None}}) -- Number of pixels to smooth in pixel space (see smoothToSlitPixelWidth)

\item {} 
\textbf{\texttt{slitWidth}} (\emph{\texttt{optional, default = None}}) -- Number of pixels to smooth in angular space (see smoothToPixelWidth)

\end{itemize}

\end{description}\end{quote}
\begin{quote}\begin{description}
\item[{Example}] \leavevmode
\begin{Verbatim}[commandchars=\\\{\}]
\PYG{g+gp}{\PYGZgt{}\PYGZgt{}\PYGZgt{} }\PYG{k+kn}{import} \PYG{n+nn}{splat}
\PYG{g+gp}{\PYGZgt{}\PYGZgt{}\PYGZgt{} }\PYG{n}{sp} \PYG{o}{=} \PYG{n}{splat}\PYG{o}{.}\PYG{n}{getSpectrum}\PYG{p}{(}\PYG{n}{lucky}\PYG{o}{=}\PYG{k+kc}{True}\PYG{p}{)}\PYG{p}{[}\PYG{l+m+mi}{0}\PYG{p}{]}
\PYG{g+gp}{\PYGZgt{}\PYGZgt{}\PYGZgt{} }\PYG{n}{sp}\PYG{o}{.}\PYG{n}{smoothfluxMax}\PYG{p}{(}\PYG{p}{)}
\PYG{g+go}{\PYGZlt{}Quantity 1.0577336634332284e\PYGZhy{}14 erg / (cm2 micron s)\PYGZgt{}}
\PYG{g+gp}{\PYGZgt{}\PYGZgt{}\PYGZgt{} }\PYG{n}{sp}\PYG{o}{.}\PYG{n}{computeSN}\PYG{p}{(}\PYG{p}{)}
\PYG{g+go}{124.5198}
\PYG{g+gp}{\PYGZgt{}\PYGZgt{}\PYGZgt{} }\PYG{n}{sp}\PYG{o}{.}\PYG{n}{scale}\PYG{p}{(}\PYG{l+m+mf}{1.e15}\PYG{p}{)}
\PYG{g+gp}{\PYGZgt{}\PYGZgt{}\PYGZgt{} }\PYG{n}{sp}\PYG{o}{.}\PYG{n}{fluxMax}\PYG{p}{(}\PYG{p}{)}
\PYG{g+go}{\PYGZlt{}Quantity 1.0577336549758911 erg / (cm2 micron s)\PYGZgt{}}
\PYG{g+gp}{\PYGZgt{}\PYGZgt{}\PYGZgt{} }\PYG{n}{sp}\PYG{o}{.}\PYG{n}{computeSN}\PYG{p}{(}\PYG{p}{)}
\PYG{g+go}{124.51981}
\end{Verbatim}

\end{description}\end{quote}

\end{fulllineitems}

\index{smoothToResolution() (splat.Spectrum method)}

\begin{fulllineitems}
\phantomsection\label{api:splat.Spectrum.smoothToResolution}\pysiglinewithargsret{\sphinxbfcode{smoothToResolution}}{\emph{resolution}, \emph{**kwargs}}{}~\begin{quote}\begin{description}
\item[{Purpose}] \leavevmode
Smoothes a spectrum to a constant or resolution (smooth in velocity space). Changes spectrum directly.  Note that no smoothing is done if requested resolution is greater than the current resolution

\item[{Parameters}] \leavevmode\begin{itemize}
\item {} 
\textbf{\texttt{resolution}} (\emph{\texttt{required}}) -- number giving the desired resolution

\item {} 
\textbf{\texttt{method}} (\emph{\texttt{optional, default = Hanning}}) -- the type of smoothing window to use. See \url{http://docs.scipy.org/doc/scipy-0.14.0/reference/generated/scipy.signal.get\_window.html} for more details.

\item {} 
\textbf{\texttt{overscale}} (\emph{\texttt{optional, default = 10.}}) -- used for computing number of samples in the window

\end{itemize}

\item[{Example}] \leavevmode
\begin{Verbatim}[commandchars=\\\{\}]
\PYG{g+gp}{\PYGZgt{}\PYGZgt{}\PYGZgt{} }\PYG{k+kn}{import} \PYG{n+nn}{splat}
\PYG{g+gp}{\PYGZgt{}\PYGZgt{}\PYGZgt{} }\PYG{n}{sp} \PYG{o}{=} \PYG{n}{splat}\PYG{o}{.}\PYG{n}{getSpectrum}\PYG{p}{(}\PYG{n}{lucky}\PYG{o}{=}\PYG{k+kc}{True}\PYG{p}{)}\PYG{p}{[}\PYG{l+m+mi}{0}\PYG{p}{]}
\PYG{g+gp}{\PYGZgt{}\PYGZgt{}\PYGZgt{} }\PYG{n}{sp}\PYG{o}{.}\PYG{n}{resolution}\PYG{p}{(}\PYG{p}{)}
\PYG{g+go}{120}
\PYG{g+gp}{\PYGZgt{}\PYGZgt{}\PYGZgt{} }\PYG{n}{sp}\PYG{o}{.}\PYG{n}{computeSN}\PYG{p}{(}\PYG{p}{)}
\PYG{g+go}{21.550974}
\PYG{g+gp}{\PYGZgt{}\PYGZgt{}\PYGZgt{} }\PYG{n}{sp}\PYG{o}{.}\PYG{n}{smoothToResolution}\PYG{p}{(}\PYG{l+m+mi}{50}\PYG{p}{)}
\PYG{g+gp}{\PYGZgt{}\PYGZgt{}\PYGZgt{} }\PYG{n}{sp}\PYG{o}{.}\PYG{n}{resolution}\PYG{p}{(}\PYG{p}{)}
\PYG{g+go}{50}
\PYG{g+gp}{\PYGZgt{}\PYGZgt{}\PYGZgt{} }\PYG{n}{sp}\PYG{o}{.}\PYG{n}{computeSN}\PYG{p}{(}\PYG{p}{)}
\PYG{g+go}{49.459522314460855}
\end{Verbatim}

\end{description}\end{quote}

\end{fulllineitems}

\index{smoothToSlitPixelWidth() (splat.Spectrum method)}

\begin{fulllineitems}
\phantomsection\label{api:splat.Spectrum.smoothToSlitPixelWidth}\pysiglinewithargsret{\sphinxbfcode{smoothToSlitPixelWidth}}{\emph{width}, \emph{**kwargs}}{}~\begin{quote}\begin{description}
\item[{Purpose}] \leavevmode
Smoothes a spectrum to a constant slit pixel width (smooth in pixel space). Changes spectrum directly.  Note that no smoothing is done if requested width is greater than the current slit width.

\item[{Parameters}] \leavevmode\begin{itemize}
\item {} 
\textbf{\texttt{width}} (\emph{\texttt{required}}) -- number giving the desired smoothing scale in pixels

\item {} 
\textbf{\texttt{method}} (\emph{\texttt{optional, default = Hanning}}) -- the type of smoothing window to use. See \url{http://docs.scipy.org/doc/scipy-0.14.0/reference/generated/scipy.signal.get\_window.html} for more details.

\end{itemize}

\item[{Example}] \leavevmode
\begin{Verbatim}[commandchars=\\\{\}]
\PYG{g+gp}{\PYGZgt{}\PYGZgt{}\PYGZgt{} }\PYG{k+kn}{import} \PYG{n+nn}{splat}
\PYG{g+gp}{\PYGZgt{}\PYGZgt{}\PYGZgt{} }\PYG{n}{sp} \PYG{o}{=} \PYG{n}{splat}\PYG{o}{.}\PYG{n}{getSpectrum}\PYG{p}{(}\PYG{n}{lucky}\PYG{o}{=}\PYG{k+kc}{True}\PYG{p}{)}\PYG{p}{[}\PYG{l+m+mi}{0}\PYG{p}{]}
\PYG{g+gp}{\PYGZgt{}\PYGZgt{}\PYGZgt{} }\PYG{n}{sp}\PYG{o}{.}\PYG{n}{slitpixelwidth}
\PYG{g+go}{3.33}
\PYG{g+gp}{\PYGZgt{}\PYGZgt{}\PYGZgt{} }\PYG{n}{sp}\PYG{o}{.}\PYG{n}{resolution}
\PYG{g+go}{120}
\PYG{g+gp}{\PYGZgt{}\PYGZgt{}\PYGZgt{} }\PYG{n}{sp}\PYG{o}{.}\PYG{n}{computeSN}\PYG{p}{(}\PYG{p}{)}
\PYG{g+go}{105.41789}
\PYG{g+gp}{\PYGZgt{}\PYGZgt{}\PYGZgt{} }\PYG{n}{sp}\PYG{o}{.}\PYG{n}{smoothToSlitPixelWidth}\PYG{p}{(}\PYG{l+m+mi}{10}\PYG{p}{)}
\PYG{g+gp}{\PYGZgt{}\PYGZgt{}\PYGZgt{} }\PYG{n}{sp}\PYG{o}{.}\PYG{n}{slitpixelwidth}
\PYG{g+go}{10}
\PYG{g+gp}{\PYGZgt{}\PYGZgt{}\PYGZgt{} }\PYG{n}{sp}\PYG{o}{.}\PYG{n}{resolution}
\PYG{g+go}{39.96}
\PYG{g+gp}{\PYGZgt{}\PYGZgt{}\PYGZgt{} }\PYG{n}{sp}\PYG{o}{.}\PYG{n}{computeSN}\PYG{p}{(}\PYG{p}{)}
\PYG{g+go}{235.77536310249229}
\end{Verbatim}

\end{description}\end{quote}

\end{fulllineitems}

\index{smoothToSlitWidth() (splat.Spectrum method)}

\begin{fulllineitems}
\phantomsection\label{api:splat.Spectrum.smoothToSlitWidth}\pysiglinewithargsret{\sphinxbfcode{smoothToSlitWidth}}{\emph{width}, \emph{**kwargs}}{}~\begin{quote}\begin{description}
\item[{Purpose}] \leavevmode
Smoothes a spectrum to a constant slit angular width (smooth in dispersion space). Changes spectrum directly.  Note that no smoothing is done if requested width is greater than the current slit width.

\item[{Parameters}] \leavevmode\begin{itemize}
\item {} 
\textbf{\texttt{width}} (\emph{\texttt{required}}) -- number giving the desired smoothing scale in arcseconds

\item {} 
\textbf{\texttt{method}} (\emph{\texttt{optional, default = Hanning}}) -- the type of smoothing window to use. See \url{http://docs.scipy.org/doc/scipy-0.14.0/reference/generated/scipy.signal.get\_window.html} for more details.

\end{itemize}

\item[{Output}] \leavevmode
maximum flux (with units)

\item[{Example}] \leavevmode
\begin{Verbatim}[commandchars=\\\{\}]
\PYG{g+gp}{\PYGZgt{}\PYGZgt{}\PYGZgt{} }\PYG{k+kn}{import} \PYG{n+nn}{splat}
\PYG{g+gp}{\PYGZgt{}\PYGZgt{}\PYGZgt{} }\PYG{n}{sp} \PYG{o}{=} \PYG{n}{splat}\PYG{o}{.}\PYG{n}{getSpectrum}\PYG{p}{(}\PYG{n}{lucky}\PYG{o}{=}\PYG{k+kc}{True}\PYG{p}{)}\PYG{p}{[}\PYG{l+m+mi}{0}\PYG{p}{]}
\PYG{g+gp}{\PYGZgt{}\PYGZgt{}\PYGZgt{} }\PYG{n}{sp}\PYG{o}{.}\PYG{n}{slitwidth}
\PYG{g+go}{0.4995}
\PYG{g+gp}{\PYGZgt{}\PYGZgt{}\PYGZgt{} }\PYG{n}{sp}\PYG{o}{.}\PYG{n}{resolution}
\PYG{g+go}{120}
\PYG{g+gp}{\PYGZgt{}\PYGZgt{}\PYGZgt{} }\PYG{n}{sp}\PYG{o}{.}\PYG{n}{computeSN}\PYG{p}{(}\PYG{p}{)}
\PYG{g+go}{105.41789}
\PYG{g+gp}{\PYGZgt{}\PYGZgt{}\PYGZgt{} }\PYG{n}{sp}\PYG{o}{.}\PYG{n}{smoothToSlitWidth}\PYG{p}{(}\PYG{l+m+mf}{2.0}\PYG{p}{)}
\PYG{g+gp}{\PYGZgt{}\PYGZgt{}\PYGZgt{} }\PYG{n}{sp}\PYG{o}{.}\PYG{n}{slitwidth}
\PYG{g+go}{2.0}
\PYG{g+gp}{\PYGZgt{}\PYGZgt{}\PYGZgt{} }\PYG{n}{sp}\PYG{o}{.}\PYG{n}{resolution}
\PYG{g+go}{29.97}
\PYG{g+gp}{\PYGZgt{}\PYGZgt{}\PYGZgt{} }\PYG{n}{sp}\PYG{o}{.}\PYG{n}{computeSN}\PYG{p}{(}\PYG{p}{)}
\PYG{g+go}{258.87135134070593}
\end{Verbatim}

\end{description}\end{quote}

\end{fulllineitems}

\index{surface() (splat.Spectrum method)}

\begin{fulllineitems}
\phantomsection\label{api:splat.Spectrum.surface}\pysiglinewithargsret{\sphinxbfcode{surface}}{\emph{radius}}{}~\begin{quote}\begin{description}
\item[{Purpose}] \leavevmode
Convert to surface fluxes given a radius, assuming at absolute fluxes

\end{description}\end{quote}

\begin{notice}{note}{Note:}
Unfinished
\end{notice}

\end{fulllineitems}

\index{trim() (splat.Spectrum method)}

\begin{fulllineitems}
\phantomsection\label{api:splat.Spectrum.trim}\pysiglinewithargsret{\sphinxbfcode{trim}}{\emph{range}, \emph{**kwargs}}{}~\begin{quote}\begin{description}
\item[{Purpose}] \leavevmode
Trims a spectrum to be within a certain wavelength range or set of ranges. Data outside of these ranges are excised from the wave, flux and noise arrays. The full spectrum can be restored with the reset() procedure.

\item[{Parameters}] \leavevmode
\textbf{\texttt{range}} -- the range(s) over which the spectrum is retained - a series of nested 2-element arrays

\end{description}\end{quote}
\begin{quote}\begin{description}
\item[{Example}] \leavevmode
\begin{Verbatim}[commandchars=\\\{\}]
\PYG{g+gp}{\PYGZgt{}\PYGZgt{}\PYGZgt{} }\PYG{k+kn}{import} \PYG{n+nn}{splat}
\PYG{g+gp}{\PYGZgt{}\PYGZgt{}\PYGZgt{} }\PYG{n}{sp} \PYG{o}{=} \PYG{n}{splat}\PYG{o}{.}\PYG{n}{getSpectrum}\PYG{p}{(}\PYG{n}{lucky}\PYG{o}{=}\PYG{k+kc}{True}\PYG{p}{)}\PYG{p}{[}\PYG{l+m+mi}{0}\PYG{p}{]}
\PYG{g+gp}{\PYGZgt{}\PYGZgt{}\PYGZgt{} }\PYG{n}{sp}\PYG{o}{.}\PYG{n}{smoothfluxMax}\PYG{p}{(}\PYG{p}{)}
\PYG{g+go}{\PYGZlt{}Quantity 1.0577336634332284e\PYGZhy{}14 erg / (cm2 micron s)\PYGZgt{}}
\PYG{g+gp}{\PYGZgt{}\PYGZgt{}\PYGZgt{} }\PYG{n}{sp}\PYG{o}{.}\PYG{n}{computeSN}\PYG{p}{(}\PYG{p}{)}
\PYG{g+go}{124.5198}
\PYG{g+gp}{\PYGZgt{}\PYGZgt{}\PYGZgt{} }\PYG{n}{sp}\PYG{o}{.}\PYG{n}{scale}\PYG{p}{(}\PYG{l+m+mf}{1.e15}\PYG{p}{)}
\PYG{g+gp}{\PYGZgt{}\PYGZgt{}\PYGZgt{} }\PYG{n}{sp}\PYG{o}{.}\PYG{n}{fluxMax}\PYG{p}{(}\PYG{p}{)}
\PYG{g+go}{\PYGZlt{}Quantity 1.0577336549758911 erg / (cm2 micron s)\PYGZgt{}}
\PYG{g+gp}{\PYGZgt{}\PYGZgt{}\PYGZgt{} }\PYG{n}{sp}\PYG{o}{.}\PYG{n}{computeSN}\PYG{p}{(}\PYG{p}{)}
\PYG{g+go}{124.51981}
\end{Verbatim}

\end{description}\end{quote}

\end{fulllineitems}

\index{waveRange() (splat.Spectrum method)}

\begin{fulllineitems}
\phantomsection\label{api:splat.Spectrum.waveRange}\pysiglinewithargsret{\sphinxbfcode{waveRange}}{}{}~\begin{quote}\begin{description}
\item[{Purpose}] \leavevmode
Return the wavelength range of the current Spectrum object.

\item[{Output}] \leavevmode
2-element array giving minimum and maximum of wavelength range

\item[{Example}] \leavevmode
\begin{Verbatim}[commandchars=\\\{\}]
\PYG{g+gp}{\PYGZgt{}\PYGZgt{}\PYGZgt{} }\PYG{k+kn}{import} \PYG{n+nn}{splat}
\PYG{g+gp}{\PYGZgt{}\PYGZgt{}\PYGZgt{} }\PYG{n}{sp} \PYG{o}{=} \PYG{n}{splat}\PYG{o}{.}\PYG{n}{getSpectrum}\PYG{p}{(}\PYG{n}{lucky}\PYG{o}{=}\PYG{k+kc}{True}\PYG{p}{)}\PYG{p}{[}\PYG{l+m+mi}{0}\PYG{p}{]}
\PYG{g+gp}{\PYGZgt{}\PYGZgt{}\PYGZgt{} }\PYG{n}{sp}\PYG{o}{.}\PYG{n}{slitwidth}
\PYG{g+go}{[\PYGZlt{}Quantity 0.6447611451148987 micron\PYGZgt{}, \PYGZlt{}Quantity 2.5517737865448 micron\PYGZgt{}]}
\end{Verbatim}

\end{description}\end{quote}

\end{fulllineitems}


\end{fulllineitems}



\subsection{SPLAT Routines}
\label{api:splat-routines}

\subsubsection{Data Access}
\label{api:data-access}\index{getSpectrum() (in module splat)}

\begin{fulllineitems}
\phantomsection\label{api:splat.getSpectrum}\pysiglinewithargsret{\sphinxcode{splat.}\sphinxbfcode{getSpectrum}}{\emph{*args}, \emph{**kwargs}}{}~\begin{quote}\begin{description}
\item[{Purpose}] \leavevmode
Gets a spectrum from the SPLAT library using various selection criteria. Calls searchLibrary to select spectra; if any found it routines an array of Spectrum objects, otherwise an empty array. See {\color{red}\bfseries{}splat.searchLibrary\_} for full list of search parameters.

\end{description}\end{quote}
\begin{quote}\begin{description}
\item[{Output}] \leavevmode
An array of Spectrum objects that satisfy the search criteria

\item[{Example}] \leavevmode
\end{description}\end{quote}

\begin{Verbatim}[commandchars=\\\{\}]
\PYG{g+gp}{\PYGZgt{}\PYGZgt{}\PYGZgt{} }\PYG{k+kn}{import} \PYG{n+nn}{splat}
\PYG{g+gp}{\PYGZgt{}\PYGZgt{}\PYGZgt{} }\PYG{n}{sp} \PYG{o}{=} \PYG{n}{splat}\PYG{o}{.}\PYG{n}{getSpectrum}\PYG{p}{(}\PYG{n}{shortname}\PYG{o}{=}\PYG{l+s+s1}{\PYGZsq{}}\PYG{l+s+s1}{1507\PYGZhy{}1627}\PYG{l+s+s1}{\PYGZsq{}}\PYG{p}{)}\PYG{p}{[}\PYG{l+m+mi}{0}\PYG{p}{]}
\PYG{g+go}{    Retrieving 1 file}
\PYG{g+gp}{\PYGZgt{}\PYGZgt{}\PYGZgt{} }\PYG{n}{sparr} \PYG{o}{=} \PYG{n}{splat}\PYG{o}{.}\PYG{n}{getSpectrum}\PYG{p}{(}\PYG{n}{spt}\PYG{o}{=}\PYG{l+s+s1}{\PYGZsq{}}\PYG{l+s+s1}{M7}\PYG{l+s+s1}{\PYGZsq{}}\PYG{p}{)}
\PYG{g+go}{    Retrieving 120 files}
\PYG{g+gp}{\PYGZgt{}\PYGZgt{}\PYGZgt{} }\PYG{n}{sparr} \PYG{o}{=} \PYG{n}{splat}\PYG{o}{.}\PYG{n}{getSpectrum}\PYG{p}{(}\PYG{n}{spt}\PYG{o}{=}\PYG{l+s+s1}{\PYGZsq{}}\PYG{l+s+s1}{T5}\PYG{l+s+s1}{\PYGZsq{}}\PYG{p}{,}\PYG{n}{young}\PYG{o}{=}\PYG{k+kc}{True}\PYG{p}{)}
\PYG{g+go}{    No files match search criteria}
\end{Verbatim}

\end{fulllineitems}

\index{getStandard() (in module splat)}

\begin{fulllineitems}
\phantomsection\label{api:splat.getStandard}\pysiglinewithargsret{\sphinxcode{splat.}\sphinxbfcode{getStandard}}{\emph{spt}, \emph{**kwargs}}{}~\begin{quote}\begin{description}
\item[{Purpose}] \leavevmode
Gets one of the pre-defined spectral standards from the SPLAT library.

\item[{Parameters}] \leavevmode\begin{itemize}
\item {} 
\textbf{\texttt{spt}} (\emph{\texttt{required}}) -- Spectral type of standard desired, either string (`M7') or numberic (17)

\item {} 
\textbf{\texttt{sd}} (\emph{\texttt{optional, default = False}}) -- Set to True to get a subdwarf standard

\item {} 
\textbf{\texttt{esd}} (\emph{\texttt{optional, default = False}}) -- Set to True to get an extreme subdwarf standard

\end{itemize}

\item[{Example}] \leavevmode
\end{description}\end{quote}

\begin{Verbatim}[commandchars=\\\{\}]
\PYG{g+gp}{\PYGZgt{}\PYGZgt{}\PYGZgt{} }\PYG{k+kn}{import} \PYG{n+nn}{splat}
\PYG{g+gp}{\PYGZgt{}\PYGZgt{}\PYGZgt{} }\PYG{n}{sp} \PYG{o}{=} \PYG{n}{splat}\PYG{o}{.}\PYG{n}{getStandard}\PYG{p}{(}\PYG{l+s+s1}{\PYGZsq{}}\PYG{l+s+s1}{M7}\PYG{l+s+s1}{\PYGZsq{}}\PYG{p}{)}\PYG{p}{[}\PYG{l+m+mi}{0}\PYG{p}{]}
\PYG{g+go}{    Spectrum of VB 8}
\PYG{g+gp}{\PYGZgt{}\PYGZgt{}\PYGZgt{} }\PYG{n}{sparr} \PYG{o}{=} \PYG{n}{splat}\PYG{o}{.}\PYG{n}{getStandard}\PYG{p}{(}\PYG{l+s+s1}{\PYGZsq{}}\PYG{l+s+s1}{T5}\PYG{l+s+s1}{\PYGZsq{}}\PYG{p}{,}\PYG{n}{esd}\PYG{o}{=}\PYG{k+kc}{True}\PYG{p}{)}
\PYG{g+go}{    Type esdT5.0 is not in esd standards: try one of the following:}
\PYG{g+go}{    [\PYGZsq{}esdM5.0\PYGZsq{}, \PYGZsq{}esdM7.0\PYGZsq{}, \PYGZsq{}esdM8.5\PYGZsq{}]}
\end{Verbatim}

\end{fulllineitems}

\index{getPhotometry() (in module splat\_db)}

\begin{fulllineitems}
\phantomsection\label{api:splat_db.getPhotometry}\pysiglinewithargsret{\sphinxcode{splat\_db.}\sphinxbfcode{getPhotometry}}{\emph{coordinate}, \emph{**kwargs}}{}~\begin{description}
\item[{Purpose}] \leavevmode
Downloads photometry for a source using astroquery (?)

\end{description}
\begin{quote}\begin{description}
\item[{Note}] \leavevmode
\textbf{Currently not functional}

\item[{Required parameters}] \leavevmode\begin{quote}\begin{description}
\item[{param coordinate}] \leavevmode
astropy Coordinate object with coordinates (RA, Dec) of source to be searched

\end{description}\end{quote}

\item[{Optional parameters}] \leavevmode\begin{quote}\begin{description}
\item[{param radius}] \leavevmode
radius in arcseconds for matching

\item[{type float}] \leavevmode
optional, default = 5.

\item[{param validate}] \leavevmode
Ask for validation from user for matched photometry

\item[{type logical}] \leavevmode
optional, default = False

\item[{param 2MASS}] \leavevmode
Download 2MASS photometry

\item[{type logical}] \leavevmode
optional, default = True

\item[{param SDSS}] \leavevmode
Download 2MASS photometry

\item[{type logical}] \leavevmode
optional, default = True

\item[{param UKIDSS}] \leavevmode
Download 2MASS photometry

\item[{type logical}] \leavevmode
optional, default = True

\item[{param WISE}] \leavevmode
Download 2MASS photometry

\item[{type logical}] \leavevmode
optional, default = True

\item[{param DENIS}] \leavevmode
Download 2MASS photometry

\item[{type logical}] \leavevmode
optional, default = True

\end{description}\end{quote}

\item[{Output}] \leavevmode\begin{itemize}
\item {} 
A table? filename? if new photometry

\end{itemize}

\end{description}\end{quote}

\end{fulllineitems}

\index{searchLibrary() (in module splat\_db)}

\begin{fulllineitems}
\phantomsection\label{api:splat_db.searchLibrary}\pysiglinewithargsret{\sphinxcode{splat\_db.}\sphinxbfcode{searchLibrary}}{\emph{*args}, \emph{**kwargs}}{}~\begin{quote}\begin{description}
\item[{Purpose}] \leavevmode
Search the SpeX database to extract the key reference for that Spectrum

\item[{Parameters}] \leavevmode\begin{itemize}
\item {} 
\textbf{\texttt{name}} (\emph{\texttt{optional}}) -- search by source name (e.g., \sphinxcode{name = 'Gliese 570D'})

\item {} 
\textbf{\texttt{shortname}} (\emph{\texttt{optional}}) -- search be short name (e.g. \sphinxcode{shortname = 'J1457-2124'})

\item {} 
\textbf{\texttt{designation}} (\emph{\texttt{optional}}) -- search by full designation (e.g., \sphinxcode{designation = 'J11040127+1959217'})

\item {} 
\textbf{\texttt{coordinate}} (\emph{\texttt{optional}}) -- search around a coordinate by a radius specified by radius keyword (e.g., \sphinxcode{coordinate = {[}180.,+30.{]}, radius = 10.})

\item {} 
\textbf{\texttt{radius}} (\emph{\texttt{optional, default = 10}}) -- search radius in arcseconds for coordinate search

\item {} 
\textbf{\texttt{spt}} (\emph{\texttt{optional}}) -- search by SpeX spectral type; single value is exact, two-element array gives range (e.g., \sphinxcode{spt = 'M7'} or \sphinxcode{spt = {[}24,39{]}})

\item {} 
\textbf{\texttt{spex\_spt}} (\emph{\texttt{optional}}) -- same as \sphinxcode{spt}

\item {} 
\textbf{\texttt{opt\_spt}} (\emph{\texttt{optional}}) -- same as \sphinxcode{spt} for literature optical spectral types

\item {} 
\textbf{\texttt{nir\_spt}} (\emph{\texttt{optional}}) -- same as \sphinxcode{spt} for literature NIR spectral types

\item {} 
\textbf{\texttt{jmag, hmag, kmag}} (\emph{\texttt{optional}}) -- select based on faint limit or range of J, H or Ks magnitudes (e.g., \sphinxcode{jmag = {[}12,15{]}})

\item {} 
\textbf{\texttt{snr}} (\emph{\texttt{optional}}) -- search on minimum or range of S/N ratios (e.g., \sphinxcode{snr = 30.} or \sphinxcode{snr = {[}50.,100.{]}})

\item {} 
\textbf{\texttt{subdwarf, young, binary, spbinary, red, blue, giant, wd, standard}} (\emph{\texttt{optional}}) -- classes to search on (e.g., \sphinxcode{young = True})

\item {} 
\textbf{\texttt{logic}} (\emph{\texttt{optional, default = 'and'}}) -- search logic, can be \sphinxcode{and} or \sphinxcode{or}

\item {} 
\textbf{\texttt{combine}} (\emph{\texttt{optional, default = 'and'}}) -- same as logic

\item {} 
\textbf{\texttt{date}} (\emph{\texttt{optional}}) -- search by date (e.g., \sphinxcode{date = '20040322'}) or range of dates (e.g., \sphinxcode{date={[}20040301,20040330{]}})

\item {} 
\textbf{\texttt{reference}} (\emph{\texttt{optional}}) -- search by list of references (bibcodes) (e.g., \sphinxcode{reference = '2011ApJS..197...19K'})

\item {} 
\textbf{\texttt{sort}} (\emph{\texttt{optional, default = True}}) -- sort results based on Right Ascension

\item {} 
\textbf{\texttt{list}} (\emph{\texttt{optional, default = False}}) -- if True, return just a list of the data files (can be done with searchLibrary as well)

\item {} 
\textbf{\texttt{lucky}} (\emph{\texttt{optional, default = False}}) -- if True, return one randomly selected spectrum from the selected sample

\item {} 
\textbf{\texttt{output}} (\emph{\texttt{optional, default = 'all'}}) -- returns desired output of selected results

\item {} 
\textbf{\texttt{logic}} -- search logic, can be and{}`{}` or \sphinxcode{or}

\item {} 
\textbf{\texttt{combine}} -- same as logic

\end{itemize}

\item[{Example}] \leavevmode
\end{description}\end{quote}

\begin{Verbatim}[commandchars=\\\{\}]
\PYG{g+gp}{\PYGZgt{}\PYGZgt{}\PYGZgt{} }\PYG{k+kn}{import} \PYG{n+nn}{splat}
\PYG{g+gp}{\PYGZgt{}\PYGZgt{}\PYGZgt{} }\PYG{n+nb}{print} \PYG{n}{SearchLibrary}\PYG{p}{(}\PYG{n}{shortname} \PYG{o}{=} \PYG{l+s+s1}{\PYGZsq{}}\PYG{l+s+s1}{2213\PYGZhy{}2136}\PYG{l+s+s1}{\PYGZsq{}}\PYG{p}{)}
\PYG{g+go}{    DATA\PYGZus{}KEY SOURCE\PYGZus{}KEY    DATA\PYGZus{}FILE     ... SHORTNAME  SELECT\PYGZus{}2}
\PYG{g+go}{    \PYGZhy{}\PYGZhy{}\PYGZhy{}\PYGZhy{}\PYGZhy{}\PYGZhy{}\PYGZhy{}\PYGZhy{} \PYGZhy{}\PYGZhy{}\PYGZhy{}\PYGZhy{}\PYGZhy{}\PYGZhy{}\PYGZhy{}\PYGZhy{}\PYGZhy{}\PYGZhy{} \PYGZhy{}\PYGZhy{}\PYGZhy{}\PYGZhy{}\PYGZhy{}\PYGZhy{}\PYGZhy{}\PYGZhy{}\PYGZhy{}\PYGZhy{}\PYGZhy{}\PYGZhy{}\PYGZhy{}\PYGZhy{}\PYGZhy{}\PYGZhy{} ... \PYGZhy{}\PYGZhy{}\PYGZhy{}\PYGZhy{}\PYGZhy{}\PYGZhy{}\PYGZhy{}\PYGZhy{}\PYGZhy{}\PYGZhy{} \PYGZhy{}\PYGZhy{}\PYGZhy{}\PYGZhy{}\PYGZhy{}\PYGZhy{}\PYGZhy{}\PYGZhy{}}
\PYG{g+go}{       11590      11586 11590\PYGZus{}11586.fits ... J2213\PYGZhy{}2136      1.0}
\PYG{g+go}{       11127      11586 11127\PYGZus{}11586.fits ... J2213\PYGZhy{}2136      1.0}
\PYG{g+go}{       10697      11586 10697\PYGZus{}11586.fits ... J2213\PYGZhy{}2136      1.0}
\PYG{g+go}{       10489      11586 10489\PYGZus{}11586.fits ... J2213\PYGZhy{}2136      1.0}
\PYG{g+gp}{\PYGZgt{}\PYGZgt{}\PYGZgt{} }\PYG{n+nb}{print} \PYG{n}{SearchLibrary}\PYG{p}{(}\PYG{n}{shortname} \PYG{o}{=} \PYG{l+s+s1}{\PYGZsq{}}\PYG{l+s+s1}{2213\PYGZhy{}2136}\PYG{l+s+s1}{\PYGZsq{}}\PYG{p}{,} \PYG{n}{output} \PYG{o}{=} \PYG{l+s+s1}{\PYGZsq{}}\PYG{l+s+s1}{OBSERVATION\PYGZus{}DATE}\PYG{l+s+s1}{\PYGZsq{}}\PYG{p}{)}
\PYG{g+go}{    OBSERVATION\PYGZus{}DATE}
\PYG{g+go}{    \PYGZhy{}\PYGZhy{}\PYGZhy{}\PYGZhy{}\PYGZhy{}\PYGZhy{}\PYGZhy{}\PYGZhy{}\PYGZhy{}\PYGZhy{}\PYGZhy{}\PYGZhy{}\PYGZhy{}\PYGZhy{}\PYGZhy{}\PYGZhy{}}
\PYG{g+go}{            20110908}
\PYG{g+go}{            20080829}
\PYG{g+go}{            20060902}
\PYG{g+go}{            20051017}
\end{Verbatim}

\begin{notice}{note}{Note:}
Note that this is currently only and AND search - need to figure out how to a full SQL style search
\end{notice}

\end{fulllineitems}

\index{fetchDatabase() (in module splat\_db)}

\begin{fulllineitems}
\phantomsection\label{api:splat_db.fetchDatabase}\pysiglinewithargsret{\sphinxcode{splat\_db.}\sphinxbfcode{fetchDatabase}}{\emph{*args}, \emph{**kwargs}}{}~\begin{quote}\begin{description}
\item[{Purpose}] \leavevmode
Get the SpeX Database from either online repository or local drive

\end{description}\end{quote}

\end{fulllineitems}

\index{keySource() (in module splat\_db)}

\begin{fulllineitems}
\phantomsection\label{api:splat_db.keySource}\pysiglinewithargsret{\sphinxcode{splat\_db.}\sphinxbfcode{keySource}}{\emph{keys}, \emph{**kwargs}}{}~\begin{quote}\begin{description}
\item[{Purpose}] \leavevmode
Takes a source key and returns a table with the source information

\item[{Parameters}] \leavevmode
\textbf{\texttt{keys}} -- source key or a list of source keys

\item[{Example}] \leavevmode
\end{description}\end{quote}

\begin{Verbatim}[commandchars=\\\{\}]
\PYG{g+gp}{\PYGZgt{}\PYGZgt{}\PYGZgt{} }\PYG{k+kn}{import} \PYG{n+nn}{splat}
\PYG{g+gp}{\PYGZgt{}\PYGZgt{}\PYGZgt{} }\PYG{n+nb}{print} \PYG{n}{splat}\PYG{o}{.}\PYG{n}{keySource}\PYG{p}{(}\PYG{l+m+mi}{10001}\PYG{p}{)}
\PYG{g+go}{    SOURCE\PYGZus{}KEY           NAME              DESIGNATION    ... NOTE SELECT}
\PYG{g+go}{    \PYGZhy{}\PYGZhy{}\PYGZhy{}\PYGZhy{}\PYGZhy{}\PYGZhy{}\PYGZhy{}\PYGZhy{}\PYGZhy{}\PYGZhy{} \PYGZhy{}\PYGZhy{}\PYGZhy{}\PYGZhy{}\PYGZhy{}\PYGZhy{}\PYGZhy{}\PYGZhy{}\PYGZhy{}\PYGZhy{}\PYGZhy{}\PYGZhy{}\PYGZhy{}\PYGZhy{}\PYGZhy{}\PYGZhy{}\PYGZhy{}\PYGZhy{}\PYGZhy{}\PYGZhy{}\PYGZhy{}\PYGZhy{}\PYGZhy{}\PYGZhy{} \PYGZhy{}\PYGZhy{}\PYGZhy{}\PYGZhy{}\PYGZhy{}\PYGZhy{}\PYGZhy{}\PYGZhy{}\PYGZhy{}\PYGZhy{}\PYGZhy{}\PYGZhy{}\PYGZhy{}\PYGZhy{}\PYGZhy{}\PYGZhy{}\PYGZhy{} ... \PYGZhy{}\PYGZhy{}\PYGZhy{}\PYGZhy{} \PYGZhy{}\PYGZhy{}\PYGZhy{}\PYGZhy{}\PYGZhy{}\PYGZhy{}}
\PYG{g+go}{         10001 SDSS J000013.54+255418.6 J00001354+2554180 ...        True}
\PYG{g+gp}{\PYGZgt{}\PYGZgt{}\PYGZgt{} }\PYG{n+nb}{print} \PYG{n}{splat}\PYG{o}{.}\PYG{n}{keySource}\PYG{p}{(}\PYG{p}{[}\PYG{l+m+mi}{10105}\PYG{p}{,} \PYG{l+m+mi}{10623}\PYG{p}{]}\PYG{p}{)}
\PYG{g+go}{    SOURCE\PYGZus{}KEY          NAME             DESIGNATION    ... NOTE SELECT}
\PYG{g+go}{    \PYGZhy{}\PYGZhy{}\PYGZhy{}\PYGZhy{}\PYGZhy{}\PYGZhy{}\PYGZhy{}\PYGZhy{}\PYGZhy{}\PYGZhy{} \PYGZhy{}\PYGZhy{}\PYGZhy{}\PYGZhy{}\PYGZhy{}\PYGZhy{}\PYGZhy{}\PYGZhy{}\PYGZhy{}\PYGZhy{}\PYGZhy{}\PYGZhy{}\PYGZhy{}\PYGZhy{}\PYGZhy{}\PYGZhy{}\PYGZhy{}\PYGZhy{}\PYGZhy{}\PYGZhy{}\PYGZhy{}\PYGZhy{} \PYGZhy{}\PYGZhy{}\PYGZhy{}\PYGZhy{}\PYGZhy{}\PYGZhy{}\PYGZhy{}\PYGZhy{}\PYGZhy{}\PYGZhy{}\PYGZhy{}\PYGZhy{}\PYGZhy{}\PYGZhy{}\PYGZhy{}\PYGZhy{}\PYGZhy{} ... \PYGZhy{}\PYGZhy{}\PYGZhy{}\PYGZhy{} \PYGZhy{}\PYGZhy{}\PYGZhy{}\PYGZhy{}\PYGZhy{}\PYGZhy{}}
\PYG{g+go}{         10105 2MASSI J0103320+193536 J01033203+1935361 ...        True}
\PYG{g+go}{         10623 SDSS J09002368+2539343 J09002368+2539343 ...        True}
\PYG{g+gp}{\PYGZgt{}\PYGZgt{}\PYGZgt{} }\PYG{n+nb}{print} \PYG{n}{splat}\PYG{o}{.}\PYG{n}{keySource}\PYG{p}{(}\PYG{l+m+mi}{1000001}\PYG{p}{)}
\PYG{g+go}{    No sources found with source key 1000001}
\PYG{g+go}{    False}
\end{Verbatim}

\end{fulllineitems}

\index{keySpectrum() (in module splat\_db)}

\begin{fulllineitems}
\phantomsection\label{api:splat_db.keySpectrum}\pysiglinewithargsret{\sphinxcode{splat\_db.}\sphinxbfcode{keySpectrum}}{\emph{keys}, \emph{**kwargs}}{}~\begin{quote}\begin{description}
\item[{Purpose}] \leavevmode
Takes a spectrum key and returns a table with the spectrum and source information

\item[{Parameters}] \leavevmode
\textbf{\texttt{keys}} -- spectrum key or a list of source keys

\item[{Example}] \leavevmode
\end{description}\end{quote}

\begin{Verbatim}[commandchars=\\\{\}]
\PYG{g+gp}{\PYGZgt{}\PYGZgt{}\PYGZgt{} }\PYG{k+kn}{import} \PYG{n+nn}{splat}
\PYG{g+gp}{\PYGZgt{}\PYGZgt{}\PYGZgt{} }\PYG{n+nb}{print} \PYG{n}{splat}\PYG{o}{.}\PYG{n}{keySpectrum}\PYG{p}{(}\PYG{l+m+mi}{10001}\PYG{p}{)}
\PYG{g+go}{    DATA\PYGZus{}KEY SOURCE\PYGZus{}KEY    DATA\PYGZus{}FILE     ... COMPANION COMPANION\PYGZus{}NAME NOTE\PYGZus{}2}
\PYG{g+go}{    \PYGZhy{}\PYGZhy{}\PYGZhy{}\PYGZhy{}\PYGZhy{}\PYGZhy{}\PYGZhy{}\PYGZhy{} \PYGZhy{}\PYGZhy{}\PYGZhy{}\PYGZhy{}\PYGZhy{}\PYGZhy{}\PYGZhy{}\PYGZhy{}\PYGZhy{}\PYGZhy{} \PYGZhy{}\PYGZhy{}\PYGZhy{}\PYGZhy{}\PYGZhy{}\PYGZhy{}\PYGZhy{}\PYGZhy{}\PYGZhy{}\PYGZhy{}\PYGZhy{}\PYGZhy{}\PYGZhy{}\PYGZhy{}\PYGZhy{}\PYGZhy{} ... \PYGZhy{}\PYGZhy{}\PYGZhy{}\PYGZhy{}\PYGZhy{}\PYGZhy{}\PYGZhy{}\PYGZhy{}\PYGZhy{} \PYGZhy{}\PYGZhy{}\PYGZhy{}\PYGZhy{}\PYGZhy{}\PYGZhy{}\PYGZhy{}\PYGZhy{}\PYGZhy{}\PYGZhy{}\PYGZhy{}\PYGZhy{}\PYGZhy{}\PYGZhy{} \PYGZhy{}\PYGZhy{}\PYGZhy{}\PYGZhy{}\PYGZhy{}\PYGZhy{}}
\PYG{g+go}{       10001      10443 10001\PYGZus{}10443.fits ...}
\PYG{g+gp}{\PYGZgt{}\PYGZgt{}\PYGZgt{} }\PYG{n+nb}{print} \PYG{n}{splat}\PYG{o}{.}\PYG{n}{keySpectrum}\PYG{p}{(}\PYG{p}{[}\PYG{l+m+mi}{10123}\PYG{p}{,} \PYG{l+m+mi}{11298}\PYG{p}{]}\PYG{p}{)}
\PYG{g+go}{    DATA\PYGZus{}KEY SOURCE\PYGZus{}KEY    DATA\PYGZus{}FILE     ... COMPANION COMPANION\PYGZus{}NAME NOTE\PYGZus{}2}
\PYG{g+go}{    \PYGZhy{}\PYGZhy{}\PYGZhy{}\PYGZhy{}\PYGZhy{}\PYGZhy{}\PYGZhy{}\PYGZhy{} \PYGZhy{}\PYGZhy{}\PYGZhy{}\PYGZhy{}\PYGZhy{}\PYGZhy{}\PYGZhy{}\PYGZhy{}\PYGZhy{}\PYGZhy{} \PYGZhy{}\PYGZhy{}\PYGZhy{}\PYGZhy{}\PYGZhy{}\PYGZhy{}\PYGZhy{}\PYGZhy{}\PYGZhy{}\PYGZhy{}\PYGZhy{}\PYGZhy{}\PYGZhy{}\PYGZhy{}\PYGZhy{}\PYGZhy{} ... \PYGZhy{}\PYGZhy{}\PYGZhy{}\PYGZhy{}\PYGZhy{}\PYGZhy{}\PYGZhy{}\PYGZhy{}\PYGZhy{} \PYGZhy{}\PYGZhy{}\PYGZhy{}\PYGZhy{}\PYGZhy{}\PYGZhy{}\PYGZhy{}\PYGZhy{}\PYGZhy{}\PYGZhy{}\PYGZhy{}\PYGZhy{}\PYGZhy{}\PYGZhy{} \PYGZhy{}\PYGZhy{}\PYGZhy{}\PYGZhy{}\PYGZhy{}\PYGZhy{}}
\PYG{g+go}{       11298      10118 11298\PYGZus{}10118.fits ...}
\PYG{g+go}{       10123      10145 10123\PYGZus{}10145.fits ...}
\PYG{g+gp}{\PYGZgt{}\PYGZgt{}\PYGZgt{} }\PYG{n+nb}{print} \PYG{n}{splat}\PYG{o}{.}\PYG{n}{keySpectrum}\PYG{p}{(}\PYG{l+m+mi}{1000001}\PYG{p}{)}
\PYG{g+go}{    No spectra found with spectrum key 1000001}
\PYG{g+go}{    False}
\end{Verbatim}

\end{fulllineitems}



\subsubsection{Spectral Classification}
\label{api:spectral-classification}\index{classifyByIndex() (in module splat)}

\begin{fulllineitems}
\phantomsection\label{api:splat.classifyByIndex}\pysiglinewithargsret{\sphinxcode{splat.}\sphinxbfcode{classifyByIndex}}{\emph{sp}, \emph{*args}, \emph{**kwargs}}{}~\begin{quote}\begin{description}
\item[{Purpose}] \leavevmode
Determine the spectral type and uncertainty for a spectrum
based on indices. Makes use of published index-SpT relations
from \href{http://adsabs.harvard.edu/abs/2001AJ....121.1710R}{Reid et al. (2001)};
\href{http://adsabs.harvard.edu/abs/2001ApJ...552L.147T}{Testi et al. (2001)};
\href{http://adsabs.harvard.edu/abs/2007ApJ...657..511A}{Allers et al. (2007)};
and \href{http://adsabs.harvard.edu/abs/2007ApJ...659..655B}{Burgasser (2007)}. Returns 2-element tuple
containing spectral type (numeric or string) and
uncertainty.

\item[{Parameters}] \leavevmode\begin{itemize}
\item {} 
\textbf{\texttt{sp}} -- Spectrum class object, which should contain wave, flux and
noise array elements.

\item {} 
\textbf{\texttt{set}} (\emph{\texttt{optional, default = 'burgasser'}}) -- 
named set of indices to measure and compute spectral type
\begin{itemize}
\item {} 
\emph{`allers'}: H2O from \href{http://adsabs.harvard.edu/abs/2007ApJ...657..511A}{Allers et al. (2007)}

\item {} 
\emph{`burgasser'}: H2O-J, CH4-J, H2O-H, CH4-H, CH4-K from \href{http://adsabs.harvard.edu/abs/2007ApJ...659..655B}{Burgasser (2007)}

\item {} 
\emph{`reid'}:H2O-A and H2O-B from \href{http://adsabs.harvard.edu/abs/2001AJ....121.1710R}{Reid et al. (2001)}

\item {} 
\emph{`testi'}: sHJ, sKJ, sH2O\_J, sH2O\_H1, sH2O\_H2, sH2O\_K from \href{http://adsabs.harvard.edu/abs/2001ApJ...552L.147T}{Testi et al. (2001)}

\end{itemize}


\item {} 
\textbf{\texttt{string}} (\emph{\texttt{optional, default = False}}) -- return spectral type as a string (uses typeToNum)

\item {} 
\textbf{\texttt{round}} (\emph{\texttt{optional, default = False}}) -- rounds off to nearest 0.5 subtypes

\item {} 
\textbf{\texttt{allmeasures}} (\emph{\texttt{optional, default = False}}) -- Set to True to return all of the index values and individual subtypes

\item {} 
\textbf{\texttt{remeasure}} (\emph{\texttt{optional, default = True}}) -- force remeasurement of indices

\item {} 
\textbf{\texttt{nsamples}} (\emph{\texttt{optional, default = 100}}) -- number of Monte Carlo samples for error computation

\item {} 
\textbf{\texttt{nloop}} (\emph{\texttt{optional, default = 5}}) -- number of testing loops to see if spectral type is within a certain range

\end{itemize}

\item[{Example}] \leavevmode
\end{description}\end{quote}

\begin{Verbatim}[commandchars=\\\{\}]
\PYG{g+gp}{\PYGZgt{}\PYGZgt{}\PYGZgt{} }\PYG{k+kn}{import} \PYG{n+nn}{splat}
\PYG{g+gp}{\PYGZgt{}\PYGZgt{}\PYGZgt{} }\PYG{n}{spc} \PYG{o}{=} \PYG{n}{splat}\PYG{o}{.}\PYG{n}{getSpectrum}\PYG{p}{(}\PYG{n}{shortname}\PYG{o}{=}\PYG{l+s+s1}{\PYGZsq{}}\PYG{l+s+s1}{0559\PYGZhy{}1404}\PYG{l+s+s1}{\PYGZsq{}}\PYG{p}{)}\PYG{p}{[}\PYG{l+m+mi}{0}\PYG{p}{]}
\PYG{g+gp}{\PYGZgt{}\PYGZgt{}\PYGZgt{} }\PYG{n}{splat}\PYG{o}{.}\PYG{n}{classifyByIndex}\PYG{p}{(}\PYG{n}{spc}\PYG{p}{,} \PYG{n}{string}\PYG{o}{=}\PYG{k+kc}{True}\PYG{p}{,} \PYG{n+nb}{set}\PYG{o}{=}\PYG{l+s+s1}{\PYGZsq{}}\PYG{l+s+s1}{burgasser}\PYG{l+s+s1}{\PYGZsq{}}\PYG{p}{,} \PYG{n+nb}{round}\PYG{o}{=}\PYG{k+kc}{True}\PYG{p}{)}
\PYG{g+go}{    (\PYGZsq{}T4.5\PYGZsq{}, 0.2562934083414341)}
\end{Verbatim}

\begin{notice}{note}{Note:}\begin{itemize}
\item {} 
Need to allow output of individual spectral types from individual indices

\end{itemize}
\end{notice}

\end{fulllineitems}

\index{classifyByStandard() (in module splat)}

\begin{fulllineitems}
\phantomsection\label{api:splat.classifyByStandard}\pysiglinewithargsret{\sphinxcode{splat.}\sphinxbfcode{classifyByStandard}}{\emph{sp}, \emph{*args}, \emph{**kwargs}}{}~\begin{quote}\begin{description}
\item[{Purpose}] \leavevmode
Determine the spectral type and uncertainty for a
spectrum by direct comparison to defined spectral standards.
Dwarf standards span M0-T9 and include the standards listed in
\href{http://adsabs.harvard.edu/abs/2006ApJ...637.1067B}{Burgasser et al. (2006)}, \href{http://adsabs.harvard.edu/abs/2010ApJS..190..100K}{Kirkpatrick et al. (2010)} and \href{http://adsabs.harvard.edu/abs/2011ApJ...743...50C}{Cushing et al. (2011)}.
Comparison to subdwarf and extreme subdwarf standards may also be done.
Returns the best
match or an F-test weighted mean and uncertainty. There is an option
to follow the procedure of \href{http://adsabs.harvard.edu/abs/2010ApJS..190..100K}{Kirkpatrick et al. (2010)}, fitting only in
the 0.9-1.4 micron region.

\item[{Output}] \leavevmode
A tuple listing the best match standard and uncertainty based on F-test weighting and systematic uncertainty of 0.5 subtypes

\item[{Parameters}] \leavevmode\begin{itemize}
\item {} 
\textbf{\texttt{sp}} -- Spectrum class object, which should contain wave, flux and
noise array elements.

\item {} 
\textbf{\texttt{sp}} -- required

\item {} 
\textbf{\texttt{sptrange}} (\emph{\texttt{optional, default = {[}'M0','T9'{]}}}) -- Set to the spectral type range over which comparisons should be made, can be a two-element array of strings or numbers

\item {} 
\textbf{\texttt{statistic}} (\emph{\texttt{optional, default = 'chisqr'}}) -- 
string defining which statistic to use in comparison; available options are:
\begin{itemize}
\item {} 
\emph{`chisqr'}: compare by computing chi squared value (requires spectra with noise values)

\item {} 
\emph{`stddev'}: compare by computing standard deviation

\item {} 
\emph{`stddev\_norm'}: compare by computing normalized standard deviation

\item {} 
\emph{`absdev'}: compare by computing absolute deviation

\end{itemize}


\item {} 
\textbf{\texttt{method}} (\emph{\texttt{optional, default = '{'}}}) -- 
set to \sphinxcode{'kirkpatrick'} to follow the \href{http://adsabs.harvard.edu/abs/2010ApJS..190..100K}{Kirkpatrick et al. (2010)} method, fitting only to the 0.9-1.4 micron band


\item {} 
\textbf{\texttt{best}} (\emph{\texttt{optional, default = True}}) -- Set to True to return the best fit standard type

\item {} 
\textbf{\texttt{average}} (\emph{\texttt{optional, default = False}}) -- Set to True to return an chi-square weighted type only

\item {} 
\textbf{\texttt{compareto}} (\emph{\texttt{optional, default = None}}) -- Set to the single standard (string or number) you want to compare to

\item {} 
\textbf{\texttt{plot}} (\emph{\texttt{optional, default = False}}) -- Set to True to generate a plot comparing best fit template to source; can also set keywords associated with plotSpectrum routine

\item {} 
\textbf{\texttt{string}} (\emph{\texttt{optional, default = True}}) -- return spectral type as a string

\item {} 
\textbf{\texttt{verbose}} (\emph{\texttt{optional, default = False}}) -- Set to True to give extra feedback

\end{itemize}

\end{description}\end{quote}

Users can also set keyword parameters defined in plotSpectrum and compareSpectra routine.
\begin{quote}\begin{description}
\item[{Example}] \leavevmode
\end{description}\end{quote}

\begin{Verbatim}[commandchars=\\\{\}]
\PYG{g+gp}{\PYGZgt{}\PYGZgt{}\PYGZgt{} }\PYG{k+kn}{import} \PYG{n+nn}{splat}
\PYG{g+gp}{\PYGZgt{}\PYGZgt{}\PYGZgt{} }\PYG{n}{sp} \PYG{o}{=} \PYG{n}{splat}\PYG{o}{.}\PYG{n}{getSpectrum}\PYG{p}{(}\PYG{n}{lucky}\PYG{o}{=}\PYG{k+kc}{True}\PYG{p}{)}\PYG{p}{[}\PYG{l+m+mi}{0}\PYG{p}{]}
\PYG{g+gp}{\PYGZgt{}\PYGZgt{}\PYGZgt{} }\PYG{n}{result} \PYG{o}{=} \PYG{n}{splat}\PYG{o}{.}\PYG{n}{classifyByStandard}\PYG{p}{(}\PYG{n}{sp}\PYG{p}{,}\PYG{n}{verbose}\PYG{o}{=}\PYG{k+kc}{True}\PYG{p}{)}
\PYG{g+go}{    Using dwarf standards}
\PYG{g+go}{    Type M3.0: statistic = 5763368.10355, scale = 0.000144521824721}
\PYG{g+go}{    Type M2.0: statistic = 5613862.67356, scale = 0.000406992798674}
\PYG{g+go}{    Type T8.0: statistic = 18949835.2087, scale = 9.70960919364}
\PYG{g+go}{    Type T9.0: statistic = 21591485.163, scale = 29.1529786804}
\PYG{g+go}{    Type L8.0: statistic = 3115605.62687, scale = 1.36392504072}
\PYG{g+go}{    Type L9.0: statistic = 2413450.79206, scale = 0.821131769522}
\PYG{g+go}{    ...}
\PYG{g+go}{    Best match to L1.0 spectral standard}
\PYG{g+go}{    Best spectral type = L1.0+/\PYGZhy{}0.5}
\PYG{g+gp}{\PYGZgt{}\PYGZgt{}\PYGZgt{} }\PYG{n}{result}
\PYG{g+go}{    (\PYGZsq{}L1.0\PYGZsq{}, 0.5)}
\PYG{g+gp}{\PYGZgt{}\PYGZgt{}\PYGZgt{} }\PYG{n}{splat}\PYG{o}{.}\PYG{n}{classifyByStandard}\PYG{p}{(}\PYG{n}{sp}\PYG{p}{,}\PYG{n}{sd}\PYG{o}{=}\PYG{k+kc}{True}\PYG{p}{,}\PYG{n}{average}\PYG{o}{=}\PYG{k+kc}{True}\PYG{p}{)}
\PYG{g+go}{    (\PYGZsq{}sdL0.0:\PYGZsq{}, 1.8630159149200021)}
\end{Verbatim}

\end{fulllineitems}

\index{classifyByTemplate() (in module splat)}

\begin{fulllineitems}
\phantomsection\label{api:splat.classifyByTemplate}\pysiglinewithargsret{\sphinxcode{splat.}\sphinxbfcode{classifyByTemplate}}{\emph{sp}, \emph{*args}, \emph{**kwargs}}{}~\begin{quote}\begin{description}
\item[{Purpose}] \leavevmode
Determine the spectral type and uncertainty for a
spectrum by direct comparison to a large set of spectra in
the library. Returns a dictionary with the best spectral type (F-test weighted mean and
uncertainty), and arrays for the N best-matching Spectrum objects, scale factors, spectral types and comparison statistics.
There is an option to follow the procedure of
\href{http://adsabs.harvard.edu/abs/2010ApJS..190..100K}{Kirkpatrick et al. (2010)},
fitting only in the 0.9-1.4 micron region.
It is strongly encouraged that users winnow down the templates used in the comparison
by selecting templates using the searchLibrary options or optionally the \sphinxcode{set} parameter.

\item[{Output}] \leavevmode
A dictionary containing the following keys:
\begin{itemize}
\item {} 
\textbf{result}: a tuple containing the spectral type and its uncertainty based on F-test statistic

\item {} 
\textbf{statistic}: array of N best statistical comparison values

\item {} 
\textbf{scale}: array of N best optimal scale factors

\item {} 
\textbf{spectra}: array of N best Spectrum objects

\item {} 
\textbf{spt}: array of N best spectral types

\end{itemize}

\item[{Parameters}] \leavevmode\begin{itemize}
\item {} 
\textbf{\texttt{sp}} -- Spectrum class object, which should contain wave, flux and
noise array elements.

\item {} 
\textbf{\texttt{sp}} -- required

\item {} 
\textbf{\texttt{statistic}} (\emph{\texttt{optional, default = 'chisqr'}}) -- 
string defining which statistic to use in comparison; available options are:
\begin{itemize}
\item {} 
\emph{`chisqr'}: compare by computing chi squared value (requires spectra with noise values)

\item {} 
\emph{`stddev'}: compare by computing standard deviation

\item {} 
\emph{`stddev\_norm'}: compare by computing normalized standard deviation

\item {} 
\emph{`absdev'}: compare by computing absolute deviation

\end{itemize}


\item {} 
\textbf{\texttt{select}} (\emph{\texttt{optional, default = '{'}}}) -- 
string defining which spectral template set you want to compare to; several options which can be combined:
\begin{itemize}
\item {} 
\emph{m dwarf}: fit to M dwarfs only

\item {} 
\emph{l dwarf}: fit to M dwarfs only

\item {} 
\emph{t dwarf}: fit to M dwarfs only

\item {} 
\emph{vlm}: fit to M7-T9 dwarfs

\item {} 
\emph{optical}: only optical classifications

\item {} 
\emph{high sn}: median S/N greater than 100

\item {} 
\emph{young}: only young/low surface gravity dwarfs

\item {} 
\emph{companion}: only companion dwarfs

\item {} 
\emph{subdwarf}: only subdwarfs

\item {} 
\emph{single}: only dwarfs not indicated a binaries

\item {} 
\emph{spectral binaries}: only dwarfs indicated to be spectral binaries

\item {} 
\emph{standard}: only spectral standards (Note: use classifyByStandard instead)

\end{itemize}


\item {} 
\textbf{\texttt{method}} (\emph{\texttt{optional, default = '{'}}}) -- 
set to \sphinxcode{'kirkpatrick'} to follow the \href{http://adsabs.harvard.edu/abs/2010ApJS..190..100K}{Kirkpatrick et al. (2010)} method, fitting only to the 0.9-1.4 micron band


\item {} 
\textbf{\texttt{best}} (\emph{\texttt{optional, default = False}}) -- Set to True to return only the best fit template type

\item {} 
\textbf{\texttt{nbest}} (\emph{\texttt{optional, default = 1}}) -- Set to the number of best fitting spectra to return

\item {} 
\textbf{\texttt{maxtemplates}} (\emph{\texttt{optional, default = 100}}) -- Set to the maximum number of templates that should be fit

\item {} 
\textbf{\texttt{force}} (\emph{\texttt{optional, default = False}}) -- By default, classifyByTemplate won't proceed if you have more than 100 templates; set this parameter to True to ignore that constraint

\item {} 
\textbf{\texttt{plot}} (\emph{\texttt{optional, default = False}}) -- Set to True to generate a plot comparing best fit template to source; can also set keywords associated with plotSpectrum routine

\item {} 
\textbf{\texttt{string}} (\emph{\texttt{optional, default = True}}) -- return spectral type as a string

\item {} 
\textbf{\texttt{verbose}} (\emph{\texttt{optional, default = False}}) -- give lots of feedback

\end{itemize}

\end{description}\end{quote}

Users can also set keyword parameters defined in plotSpectrum and searchLibrary routines
\begin{quote}\begin{description}
\item[{Example}] \leavevmode
\end{description}\end{quote}

\begin{Verbatim}[commandchars=\\\{\}]
\PYG{g+gp}{\PYGZgt{}\PYGZgt{}\PYGZgt{} }\PYG{k+kn}{import} \PYG{n+nn}{splat}
\PYG{g+gp}{\PYGZgt{}\PYGZgt{}\PYGZgt{} }\PYG{n}{sp} \PYG{o}{=} \PYG{n}{splat}\PYG{o}{.}\PYG{n}{getSpectrum}\PYG{p}{(}\PYG{n}{shortname}\PYG{o}{=}\PYG{l+s+s1}{\PYGZsq{}}\PYG{l+s+s1}{1507\PYGZhy{}1627}\PYG{l+s+s1}{\PYGZsq{}}\PYG{p}{)}\PYG{p}{[}\PYG{l+m+mi}{0}\PYG{p}{]}
\PYG{g+gp}{\PYGZgt{}\PYGZgt{}\PYGZgt{} }\PYG{n}{result} \PYG{o}{=} \PYG{n}{splat}\PYG{o}{.}\PYG{n}{classifyByTemplate}\PYG{p}{(}\PYG{n}{sp}\PYG{p}{,}\PYG{n}{string}\PYG{o}{=}\PYG{k+kc}{True}\PYG{p}{,}\PYG{n}{spt}\PYG{o}{=}\PYG{p}{[}\PYG{l+m+mi}{24}\PYG{p}{,}\PYG{l+m+mi}{26}\PYG{p}{]}\PYG{p}{,}\PYG{n}{nbest}\PYG{o}{=}\PYG{l+m+mi}{5}\PYG{p}{)}
\PYG{g+go}{    Too many templates (171) for classifyByTemplate; set force=True to override this}
\PYG{g+gp}{\PYGZgt{}\PYGZgt{}\PYGZgt{} }\PYG{n}{result} \PYG{o}{=} \PYG{n}{splat}\PYG{o}{.}\PYG{n}{classifyByTemplate}\PYG{p}{(}\PYG{n}{sp}\PYG{p}{,}\PYG{n}{string}\PYG{o}{=}\PYG{k+kc}{True}\PYG{p}{,}\PYG{n}{spt}\PYG{o}{=}\PYG{p}{[}\PYG{l+m+mi}{24}\PYG{p}{,}\PYG{l+m+mi}{26}\PYG{p}{]}\PYG{p}{,}\PYG{n}{snr}\PYG{o}{=}\PYG{l+m+mi}{50}\PYG{p}{,}\PYG{n}{nbest}\PYG{o}{=}\PYG{l+m+mi}{5}\PYG{p}{)}
\PYG{g+go}{    Comparing to 98 templates}
\PYG{g+go}{    LHS 102B L5.0 10488.1100432 11.0947838116}
\PYG{g+go}{    2MASSI J0013578\PYGZhy{}223520 L4.0 7037.37441677 136.830522173}
\PYG{g+go}{    SDSS J001608.44\PYGZhy{}004302.3 L5.5 15468.6209466 274.797693706}
\PYG{g+go}{    2MASSI J0028394+150141 L4.5 63696.1897668 187.266152375}
\PYG{g+go}{    ...}
\PYG{g+go}{    Best match = DENIS\PYGZhy{}P J153941.96\PYGZhy{}052042.4 with spectral type L4:}
\PYG{g+go}{    Mean spectral type = L4.5+/\PYGZhy{}0.718078660103}
\PYG{g+gp}{\PYGZgt{}\PYGZgt{}\PYGZgt{} }\PYG{n}{result}
\PYG{g+go}{    \PYGZob{}\PYGZsq{}result\PYGZsq{}: (\PYGZsq{}L4.5\PYGZsq{}, 0.71807866010293797),}
\PYG{g+go}{     \PYGZsq{}scale\PYGZsq{}: [3.0379089778408642e\PYGZhy{}14,}
\PYG{g+go}{      96.534933767992072,}
\PYG{g+go}{      3.812718429200959,}
\PYG{g+go}{      2.9878801833735986e\PYGZhy{}14,}
\PYG{g+go}{      3.0353579048704484e\PYGZhy{}14],}
\PYG{g+go}{     \PYGZsq{}spectra\PYGZsq{}: [Spectrum of DENIS\PYGZhy{}P J153941.96\PYGZhy{}052042.4,}
\PYG{g+go}{      Spectrum of 2MASSI J0443058\PYGZhy{}320209,}
\PYG{g+go}{      Spectrum of SDSSp J053951.99\PYGZhy{}005902.0,}
\PYG{g+go}{      Spectrum of 2MASSI J1104012+195921,}
\PYG{g+go}{      Spectrum of 2MASS J17502484\PYGZhy{}0016151],}
\PYG{g+go}{     \PYGZsq{}spt\PYGZsq{}: [24.0, 25.0, 25.0, 24.0, 25.5],}
\PYG{g+go}{     \PYGZsq{}statistic\PYGZsq{}: [\PYGZlt{}Quantity 2108.997879536768\PYGZgt{},}
\PYG{g+go}{      \PYGZlt{}Quantity 2205.640664932956\PYGZgt{},}
\PYG{g+go}{      \PYGZlt{}Quantity 2279.316858783139\PYGZgt{},}
\PYG{g+go}{      \PYGZlt{}Quantity 2579.0089210846527\PYGZgt{},}
\PYG{g+go}{      \PYGZlt{}Quantity 2684.003187310027\PYGZgt{}]\PYGZcb{}}
\end{Verbatim}

\end{fulllineitems}

\index{classifyGravity() (in module splat)}

\begin{fulllineitems}
\phantomsection\label{api:splat.classifyGravity}\pysiglinewithargsret{\sphinxcode{splat.}\sphinxbfcode{classifyGravity}}{\emph{sp}, \emph{*args}, \emph{**kwargs}}{}~\begin{quote}\begin{description}
\item[{Purpose}] \leavevmode
Determine the gravity classification of a brown dwarf using the method of \href{http://adsabs.harvard.edu/abs/2013ApJ...772...79A}{Allers \& Liu (2013)}.

\item[{Parameters}] \leavevmode\begin{itemize}
\item {} 
\textbf{\texttt{sp}} (\emph{\texttt{required}}) -- Spectrum class object, which should contain wave, flux and
noise array elements. Must be between M6.0 and L7.0.

\item {} 
\textbf{\texttt{spt}} (\emph{\texttt{optional, default = False}}) -- spectral type of \sphinxcode{sp}. Must be between M6.0 and L7.0

\item {} 
\textbf{\texttt{indices}} (\emph{\texttt{optional, default = 'allers'}}) -- specify indices set using \sphinxcode{measureIndexSet}.

\item {} 
\textbf{\texttt{plot}} (\emph{\texttt{optional, default = False}}) -- Set to True to plot sources against closest dwarf spectral standard

\item {} 
\textbf{\texttt{allscores}} (\emph{\texttt{optional, default = False}}) -- Set to True to return a dictionary containing the gravity scores from individual indices

\item {} 
\textbf{\texttt{verbose}} (\emph{\texttt{optional, default = False}}) -- Give feedback while computing

\end{itemize}

\item[{Output}] \leavevmode
Either a string specifying the gravity classification or a dictionary specifying the gravity scores for each index

\item[{Example}] \leavevmode
\end{description}\end{quote}

\begin{Verbatim}[commandchars=\\\{\}]
\PYG{g+gp}{\PYGZgt{}\PYGZgt{}\PYGZgt{} }\PYG{k+kn}{import} \PYG{n+nn}{splat}
\PYG{g+gp}{\PYGZgt{}\PYGZgt{}\PYGZgt{} }\PYG{n}{sp} \PYG{o}{=} \PYG{n}{splat}\PYG{o}{.}\PYG{n}{getSpectrum}\PYG{p}{(}\PYG{n}{shortname}\PYG{o}{=}\PYG{l+s+s1}{\PYGZsq{}}\PYG{l+s+s1}{1507\PYGZhy{}1627}\PYG{l+s+s1}{\PYGZsq{}}\PYG{p}{)}\PYG{p}{[}\PYG{l+m+mi}{0}\PYG{p}{]}
\PYG{g+gp}{\PYGZgt{}\PYGZgt{}\PYGZgt{} }\PYG{n}{splat}\PYG{o}{.}\PYG{n}{classifyGravity}\PYG{p}{(}\PYG{n}{sp}\PYG{p}{)}
\PYG{g+go}{    FLD\PYGZhy{}G}
\PYG{g+gp}{\PYGZgt{}\PYGZgt{}\PYGZgt{} }\PYG{n}{result} \PYG{o}{=} \PYG{n}{splat}\PYG{o}{.}\PYG{n}{classifyGravity}\PYG{p}{(}\PYG{n}{sp}\PYG{p}{,} \PYG{n}{allscores} \PYG{o}{=} \PYG{k+kc}{True}\PYG{p}{,} \PYG{n}{verbose}\PYG{o}{=}\PYG{k+kc}{True}\PYG{p}{)}
\PYG{g+go}{    Gravity Classification:}
\PYG{g+go}{        SpT = L4.0}
\PYG{g+go}{        VO\PYGZhy{}z: 1.012+/\PYGZhy{}0.029 =\PYGZgt{} 0.0}
\PYG{g+go}{        FeH\PYGZhy{}z: 1.299+/\PYGZhy{}0.031 =\PYGZgt{} 1.0}
\PYG{g+go}{        H\PYGZhy{}cont: 0.859+/\PYGZhy{}0.032 =\PYGZgt{} 0.0}
\PYG{g+go}{        KI\PYGZhy{}J: 1.114+/\PYGZhy{}0.038 =\PYGZgt{} 1.0}
\PYG{g+go}{        Gravity Class = FLD\PYGZhy{}G}
\PYG{g+gp}{\PYGZgt{}\PYGZgt{}\PYGZgt{} }\PYG{n}{result}
\PYG{g+go}{    \PYGZob{}\PYGZsq{}FeH\PYGZhy{}z\PYGZsq{}: 1.0,}
\PYG{g+go}{     \PYGZsq{}H\PYGZhy{}cont\PYGZsq{}: 0.0,}
\PYG{g+go}{     \PYGZsq{}KI\PYGZhy{}J\PYGZsq{}: 1.0,}
\PYG{g+go}{     \PYGZsq{}VO\PYGZhy{}z\PYGZsq{}: 0.0,}
\PYG{g+go}{     \PYGZsq{}gravity\PYGZus{}class\PYGZsq{}: \PYGZsq{}FLD\PYGZhy{}G\PYGZsq{},}
\PYG{g+go}{     \PYGZsq{}score\PYGZsq{}: 0.5,}
\PYG{g+go}{     \PYGZsq{}spt\PYGZsq{}: \PYGZsq{}L4.0\PYGZsq{}\PYGZcb{}}
\end{Verbatim}

\end{fulllineitems}

\index{initiateStandards() (in module splat)}

\begin{fulllineitems}
\phantomsection\label{api:splat.initiateStandards}\pysiglinewithargsret{\sphinxcode{splat.}\sphinxbfcode{initiateStandards}}{\emph{**kwargs}}{}~\begin{quote}\begin{description}
\item[{Purpose}] \leavevmode
Initiates the spectral standards in the SpeX library. By default this loads the dwarfs standards, but you can also specify loading of subdwarf and extreme subdwarf standards as well. Once loaded, these standards remain in memory.

\item[{Parameters}] \leavevmode\begin{itemize}
\item {} 
\textbf{\texttt{sd}} (\emph{\texttt{optional, default = False}}) -- Set equal to True to load subdwarf standards

\item {} 
\textbf{\texttt{esd}} (\emph{\texttt{optional, default = False}}) -- Set equal to True to load extreme subdwarf standards

\end{itemize}

\item[{Example}] \leavevmode
\end{description}\end{quote}

\begin{Verbatim}[commandchars=\\\{\}]
\PYG{g+gp}{\PYGZgt{}\PYGZgt{}\PYGZgt{} }\PYG{k+kn}{import} \PYG{n+nn}{splat}
\PYG{g+gp}{\PYGZgt{}\PYGZgt{}\PYGZgt{} }\PYG{n}{splat}\PYG{o}{.}\PYG{n}{initiateStandards}\PYG{p}{(}\PYG{p}{)}
\PYG{g+gp}{\PYGZgt{}\PYGZgt{}\PYGZgt{} }\PYG{n}{splat}\PYG{o}{.}\PYG{n}{SPEX\PYGZus{}STDS}\PYG{p}{[}\PYG{l+s+s1}{\PYGZsq{}}\PYG{l+s+s1}{M5.0}\PYG{l+s+s1}{\PYGZsq{}}\PYG{p}{]}
\PYG{g+go}{Spectrum of Gl51}
\end{Verbatim}

\end{fulllineitems}

\index{metallicity() (in module splat)}

\begin{fulllineitems}
\phantomsection\label{api:splat.metallicity}\pysiglinewithargsret{\sphinxcode{splat.}\sphinxbfcode{metallicity}}{\emph{sp}, \emph{**kwargs}}{}~\begin{quote}\begin{description}
\item[{Purpose}] \leavevmode
Metallicity measurement using Na I and Ca I lines and H2O-K2 index as described in \href{http://adsabs.harvard.edu/abs/2012ApJ...748...93R}{Rojas-Ayala et al.(2012)}

\item[{Parameters}] \leavevmode\begin{itemize}
\item {} 
\textbf{\texttt{sp}} -- Spectrum class object, which should contain wave, flux and noise array elements

\item {} 
\textbf{\texttt{nsamples}} (\emph{\texttt{optional, default = 100}}) -- number of Monte Carlo samples for error computation

\end{itemize}

\item[{Example}] \leavevmode
\end{description}\end{quote}

\begin{Verbatim}[commandchars=\\\{\}]
\PYG{g+gp}{\PYGZgt{}\PYGZgt{}\PYGZgt{} }\PYG{k+kn}{import} \PYG{n+nn}{splat}
\PYG{g+gp}{\PYGZgt{}\PYGZgt{}\PYGZgt{} }\PYG{n}{sp} \PYG{o}{=} \PYG{n}{splat}\PYG{o}{.}\PYG{n}{getSpectrum}\PYG{p}{(}\PYG{n}{shortname}\PYG{o}{=}\PYG{l+s+s1}{\PYGZsq{}}\PYG{l+s+s1}{0559\PYGZhy{}1404}\PYG{l+s+s1}{\PYGZsq{}}\PYG{p}{)}\PYG{p}{[}\PYG{l+m+mi}{0}\PYG{p}{]}
\PYG{g+gp}{\PYGZgt{}\PYGZgt{}\PYGZgt{} }\PYG{n+nb}{print} \PYG{n}{splat}\PYG{o}{.}\PYG{n}{metallicity}\PYG{p}{(}\PYG{n}{sp}\PYG{p}{)}
\PYG{g+go}{    (\PYGZhy{}0.50726104530066363, 0.24844773591243882)}
\end{Verbatim}

\end{fulllineitems}



\subsubsection{Spectrophotometry}
\label{api:spectrophotometry}\index{filterInfo() (in module splat)}

\begin{fulllineitems}
\phantomsection\label{api:splat.filterInfo}\pysiglinewithargsret{\sphinxcode{splat.}\sphinxbfcode{filterInfo}}{}{}~\begin{quote}\begin{description}
\item[{Purpose}] \leavevmode
Prints out the current list of filters in the SPLAT reference library.

\end{description}\end{quote}

\end{fulllineitems}

\index{filterProperties() (in module splat)}

\begin{fulllineitems}
\phantomsection\label{api:splat.filterProperties}\pysiglinewithargsret{\sphinxcode{splat.}\sphinxbfcode{filterProperties}}{\emph{filter}, \emph{**kwargs}}{}~\begin{quote}\begin{description}
\item[{Purpose}] \leavevmode
Returns a dictionary containing key parameters for a particular filter.

\item[{Parameters}] \leavevmode\begin{itemize}
\item {} 
\textbf{\texttt{filter}} (\emph{\texttt{required}}) -- name of filter, must be one of the specifed filters given by splat.FILTERS.keys()

\item {} 
\textbf{\texttt{verbose}} (\emph{\texttt{optional, default = True}}) -- print out information about filter to screen

\end{itemize}

\item[{Example}] \leavevmode
\end{description}\end{quote}

\begin{Verbatim}[commandchars=\\\{\}]
\PYG{g+gp}{\PYGZgt{}\PYGZgt{}\PYGZgt{} }\PYG{k+kn}{import} \PYG{n+nn}{splat}
\PYG{g+gp}{\PYGZgt{}\PYGZgt{}\PYGZgt{} }\PYG{n}{data} \PYG{o}{=} \PYG{n}{splat}\PYG{o}{.}\PYG{n}{filterProperties}\PYG{p}{(}\PYG{l+s+s1}{\PYGZsq{}}\PYG{l+s+s1}{2MASS J}\PYG{l+s+s1}{\PYGZsq{}}\PYG{p}{)}
\PYG{g+go}{Filter 2MASS J: 2MASS J\PYGZhy{}band}
\PYG{g+go}{Zeropoint = 1594.0 Jy}
\PYG{g+go}{Pivot point: = 1.252 micron}
\PYG{g+go}{FWHM = 0.323 micron}
\PYG{g+go}{Wavelength range = 1.066 to 1.442 micron}
\PYG{g+gp}{\PYGZgt{}\PYGZgt{}\PYGZgt{} }\PYG{n}{data} \PYG{o}{=} \PYG{n}{splat}\PYG{o}{.}\PYG{n}{filterProperties}\PYG{p}{(}\PYG{l+s+s1}{\PYGZsq{}}\PYG{l+s+s1}{2MASS X}\PYG{l+s+s1}{\PYGZsq{}}\PYG{p}{)}
\PYG{g+go}{Filter 2MASS X not among the available filters:}
\PYG{g+go}{  2MASS H: 2MASS H\PYGZhy{}band}
\PYG{g+go}{  2MASS J: 2MASS J\PYGZhy{}band}
\PYG{g+go}{  2MASS KS: 2MASS Ks\PYGZhy{}band}
\PYG{g+go}{  BESSEL I: Bessel I\PYGZhy{}band}
\PYG{g+go}{  FOURSTAR H: FOURSTAR H\PYGZhy{}band}
\PYG{g+go}{  FOURSTAR H LONG: FOURSTAR H long}
\PYG{g+go}{  FOURSTAR H SHORT: FOURSTAR H short}
\PYG{g+go}{  ...}
\end{Verbatim}

\end{fulllineitems}

\index{filterMag() (in module splat)}

\begin{fulllineitems}
\phantomsection\label{api:splat.filterMag}\pysiglinewithargsret{\sphinxcode{splat.}\sphinxbfcode{filterMag}}{\emph{sp}, \emph{filter}, \emph{*args}, \emph{**kwargs}}{}~\begin{quote}\begin{description}
\item[{Purpose}] \leavevmode
Determine the photometric magnitude of a source based on its
spectrum. Spectral fluxes are convolved with the filter profile specified by
the \sphinxcode{filter} input.  By default this filter is also
convolved with a model of Vega to extract Vega magnitudes,
but the user can also specify AB magnitudes, photon flux or
energy flux.

\item[{Parameters}] \leavevmode\begin{itemize}
\item {} 
\textbf{\texttt{sp}} (\emph{\texttt{required}}) -- Spectrum class object, which should contain wave, flux and
noise array elements.

\item {} 
\textbf{\texttt{filter}} (\emph{\texttt{required}}) -- String giving name of filter, which can either be one of the predefined filters listed in splat.FILTERS.keys() or a custom filter name

\item {} 
\textbf{\texttt{custom}} (\emph{\texttt{optional, default = None}}) -- A 2 x N vector array specifying the wavelengths and transmissions for a custom filter

\item {} 
\textbf{\texttt{notch}} (\emph{\texttt{optional, default = None}}) -- A 2 element array that specifies the lower and upper wavelengths for a notch filter (100\% transmission within, 0\% transmission without)

\item {} 
\textbf{\texttt{vega}} (\emph{\texttt{optional, default = True}}) -- compute Vega magnitudes

\item {} 
\textbf{\texttt{ab}} (\emph{\texttt{optional, default = False}}) -- compute AB magnitudes

\item {} 
\textbf{\texttt{energy}} (\emph{\texttt{optional, default = False}}) -- compute energy flux

\item {} 
\textbf{\texttt{photon}} (\emph{\texttt{optional, default = False}}) -- compute photon flux

\item {} 
\textbf{\texttt{filterFolder}} (\emph{\texttt{optional, default = splat.FILTER\_FOLDER}}) -- folder containing the filter transmission files

\item {} 
\textbf{\texttt{vegaFile}} (\emph{\texttt{optional, default = vega\_kurucz.txt}}) -- name of file containing Vega flux file, must be within \sphinxcode{filterFolder}

\item {} 
\textbf{\texttt{nsamples}} (\emph{\texttt{optional, default = 100}}) -- number of samples to use in Monte Carlo error estimation

\item {} 
\textbf{\texttt{info}} (\emph{\texttt{optional, default = False}}) -- List the predefined filter names available

\end{itemize}

\item[{Example}] \leavevmode
\end{description}\end{quote}

\begin{Verbatim}[commandchars=\\\{\}]
\PYG{g+gp}{\PYGZgt{}\PYGZgt{}\PYGZgt{} }\PYG{k+kn}{import} \PYG{n+nn}{splat}
\PYG{g+gp}{\PYGZgt{}\PYGZgt{}\PYGZgt{} }\PYG{n}{sp} \PYG{o}{=} \PYG{n}{splat}\PYG{o}{.}\PYG{n}{getSpectrum}\PYG{p}{(}\PYG{n}{shortname}\PYG{o}{=}\PYG{l+s+s1}{\PYGZsq{}}\PYG{l+s+s1}{1507\PYGZhy{}1627}\PYG{l+s+s1}{\PYGZsq{}}\PYG{p}{)}\PYG{p}{[}\PYG{l+m+mi}{0}\PYG{p}{]}
\PYG{g+gp}{\PYGZgt{}\PYGZgt{}\PYGZgt{} }\PYG{n}{sp}\PYG{o}{.}\PYG{n}{fluxCalibrate}\PYG{p}{(}\PYG{l+s+s1}{\PYGZsq{}}\PYG{l+s+s1}{2MASS J}\PYG{l+s+s1}{\PYGZsq{}}\PYG{p}{,}\PYG{l+m+mf}{14.5}\PYG{p}{)}
\PYG{g+gp}{\PYGZgt{}\PYGZgt{}\PYGZgt{} }\PYG{n}{splat}\PYG{o}{.}\PYG{n}{filterMag}\PYG{p}{(}\PYG{n}{sp}\PYG{p}{,}\PYG{l+s+s1}{\PYGZsq{}}\PYG{l+s+s1}{MKO J}\PYG{l+s+s1}{\PYGZsq{}}\PYG{p}{)}
\PYG{g+go}{    (14.345894376898123, 0.027596454828421831)}
\end{Verbatim}

\end{fulllineitems}



\subsubsection{Other Spectral Analyses}
\label{api:other-spectral-analyses}\index{compareSpectra() (in module splat)}

\begin{fulllineitems}
\phantomsection\label{api:splat.compareSpectra}\pysiglinewithargsret{\sphinxcode{splat.}\sphinxbfcode{compareSpectra}}{\emph{sp1}, \emph{sp2}, \emph{*args}, \emph{**kwargs}}{}~\begin{quote}\begin{description}
\item[{Purpose}] \leavevmode
Compare two spectra against each other using a pre-selected statistic. Returns the value of the desired statistic as well as the optimal scale factor. Minimum possible value for statistic is 1.e-9.

\item[{Parameters}] \leavevmode\begin{itemize}
\item {} 
\textbf{\texttt{sp1}} (\emph{\texttt{required}}) -- First spectrum class object, which sets the wavelength scale

\item {} 
\textbf{\texttt{sp2}} (\emph{\texttt{required}}) -- Second spectrum class object, interpolated onto the wavelength scale of sp1

\item {} 
\textbf{\texttt{statistic}} (\emph{\texttt{optional, default = 'chisqr'}}) -- 
string defining which statistic to use in comparison; available options are:
\begin{itemize}
\item {} 
\emph{`chisqr'}: compare by computing chi squared value (requires spectra with noise values)

\item {} 
\emph{`stddev'}: compare by computing standard deviation

\item {} 
\emph{`stddev\_norm'}: compare by computing normalized standard deviation

\item {} 
\emph{`absdev'}: compare by computing absolute deviation

\end{itemize}


\item {} 
\textbf{\texttt{fit\_ranges}} (\emph{\texttt{optional, default = {[}0.65,2.45{]}}}) -- 2-element array or nested array of 2-element arrays specifying the wavelength ranges to be used for the fit, assumed to be measured in microns. This is effectively the opposite of mask\_ranges.

\item {} 
\textbf{\texttt{weights}} (\emph{\texttt{optional, default = {[}1, ..., 1{]} for len(sp1.wave)}}) -- Array specifying the weights for individual wavelengths; must be an array with length equal to the wavelength scale of \sphinxcode{sp1}; need not be normalized

\item {} 
\textbf{\texttt{mask\_ranges}} (\emph{\texttt{optional, default = None}}) -- Multi-vector array setting wavelength boundaries for masking data, assumed to be in microns

\item {} 
\textbf{\texttt{mask}} (\emph{\texttt{optional, default = {[}0, ..., 0{]} for len(sp1.wave)}}) -- Array specifiying which wavelengths to mask; must be an array with length equal to the wavelength scale of \sphinxcode{sp1} with only 0 (OK) or 1 (mask).

\item {} 
\textbf{\texttt{mask\_telluric}} (\emph{\texttt{optional, default = False}}) -- Set to True to mask pre-defined telluric absorption regions

\item {} 
\textbf{\texttt{mask\_standard}} (\emph{\texttt{optional, default = False}}) -- Like \sphinxcode{mask\_telluric}, with a slightly tighter cut of 0.80-2.35 micron

\item {} 
\textbf{\texttt{novar2}} (\emph{\texttt{optional, default = True}}) -- Set to True to compute statistic without considering variance of \sphinxcode{sp2}

\item {} 
\textbf{\texttt{plot}} (\emph{\texttt{optional, default = False}}) -- Set to True to plot \sphinxcode{sp1} with scaled \sphinxcode{sp2} and difference spectrum overlaid

\item {} 
\textbf{\texttt{verbose}} (\emph{\texttt{optional, default = False}}) -- Set to True to report things as you're going along

\end{itemize}

\item[{Example}] \leavevmode
\end{description}\end{quote}

\begin{Verbatim}[commandchars=\\\{\}]
\PYG{g+gp}{\PYGZgt{}\PYGZgt{}\PYGZgt{} }\PYG{k+kn}{import} \PYG{n+nn}{splat}
\PYG{g+gp}{\PYGZgt{}\PYGZgt{}\PYGZgt{} }\PYG{k+kn}{import} \PYG{n+nn}{numpy}
\PYG{g+gp}{\PYGZgt{}\PYGZgt{}\PYGZgt{} }\PYG{n}{sp1} \PYG{o}{=} \PYG{n}{splat}\PYG{o}{.}\PYG{n}{getSpectrum}\PYG{p}{(}\PYG{n}{shortname} \PYG{o}{=} \PYG{l+s+s1}{\PYGZsq{}}\PYG{l+s+s1}{2346\PYGZhy{}3153}\PYG{l+s+s1}{\PYGZsq{}}\PYG{p}{)}\PYG{p}{[}\PYG{l+m+mi}{0}\PYG{p}{]}
\PYG{g+go}{    Retrieving 1 file}
\PYG{g+gp}{\PYGZgt{}\PYGZgt{}\PYGZgt{} }\PYG{n}{sp2} \PYG{o}{=} \PYG{n}{splat}\PYG{o}{.}\PYG{n}{getSpectrum}\PYG{p}{(}\PYG{n}{shortname} \PYG{o}{=} \PYG{l+s+s1}{\PYGZsq{}}\PYG{l+s+s1}{1421+1827}\PYG{l+s+s1}{\PYGZsq{}}\PYG{p}{)}\PYG{p}{[}\PYG{l+m+mi}{0}\PYG{p}{]}
\PYG{g+go}{    Retrieving 1 file}
\PYG{g+gp}{\PYGZgt{}\PYGZgt{}\PYGZgt{} }\PYG{n}{sp1}\PYG{o}{.}\PYG{n}{normalize}\PYG{p}{(}\PYG{p}{)}
\PYG{g+gp}{\PYGZgt{}\PYGZgt{}\PYGZgt{} }\PYG{n}{sp2}\PYG{o}{.}\PYG{n}{normalize}\PYG{p}{(}\PYG{p}{)}
\PYG{g+gp}{\PYGZgt{}\PYGZgt{}\PYGZgt{} }\PYG{n}{splat}\PYG{o}{.}\PYG{n}{compareSpectra}\PYG{p}{(}\PYG{n}{sp1}\PYG{p}{,} \PYG{n}{sp2}\PYG{p}{,} \PYG{n}{statistic}\PYG{o}{=}\PYG{l+s+s1}{\PYGZsq{}}\PYG{l+s+s1}{chisqr}\PYG{l+s+s1}{\PYGZsq{}}\PYG{p}{)}
\PYG{g+go}{    (\PYGZlt{}Quantity 19927.74527822856\PYGZgt{}, 0.94360732593223595)}
\PYG{g+gp}{\PYGZgt{}\PYGZgt{}\PYGZgt{} }\PYG{n}{splat}\PYG{o}{.}\PYG{n}{compareSpectra}\PYG{p}{(}\PYG{n}{sp1}\PYG{p}{,} \PYG{n}{sp2}\PYG{p}{,} \PYG{n}{statistic}\PYG{o}{=}\PYG{l+s+s1}{\PYGZsq{}}\PYG{l+s+s1}{stddev}\PYG{l+s+s1}{\PYGZsq{}}\PYG{p}{)}
\PYG{g+go}{    (\PYGZlt{}Quantity 3.0237604611215705 erg2 / (cm4 micron2 s2)\PYGZgt{}, 0.98180983971456637)}
\PYG{g+gp}{\PYGZgt{}\PYGZgt{}\PYGZgt{} }\PYG{n}{splat}\PYG{o}{.}\PYG{n}{compareSpectra}\PYG{p}{(}\PYG{n}{sp1}\PYG{p}{,} \PYG{n}{sp2}\PYG{p}{,} \PYG{n}{statistic}\PYG{o}{=}\PYG{l+s+s1}{\PYGZsq{}}\PYG{l+s+s1}{absdev}\PYG{l+s+s1}{\PYGZsq{}}\PYG{p}{)}
\PYG{g+go}{    (\PYGZlt{}Quantity 32.99816249949072 erg / (cm2 micron s)\PYGZgt{}, 0.98155779612333172)}
\PYG{g+gp}{\PYGZgt{}\PYGZgt{}\PYGZgt{} }\PYG{n}{splat}\PYG{o}{.}\PYG{n}{compareSpectra}\PYG{p}{(}\PYG{n}{sp1}\PYG{p}{,} \PYG{n}{sp2}\PYG{p}{,} \PYG{n}{statistic}\PYG{o}{=}\PYG{l+s+s1}{\PYGZsq{}}\PYG{l+s+s1}{chisqr}\PYG{l+s+s1}{\PYGZsq{}}\PYG{p}{,} \PYG{n}{novar2}\PYG{o}{=}\PYG{k+kc}{False}\PYG{p}{)}
\PYG{g+go}{    (\PYGZlt{}Quantity 17071.690727945213\PYGZgt{}, 0.94029474635786015)}
\end{Verbatim}

\end{fulllineitems}

\index{measureIndex() (in module splat)}

\begin{fulllineitems}
\phantomsection\label{api:splat.measureIndex}\pysiglinewithargsret{\sphinxcode{splat.}\sphinxbfcode{measureIndex}}{\emph{sp}, \emph{*args}, \emph{**kwargs}}{}~\begin{quote}\begin{description}
\item[{Purpose}] \leavevmode
Measure an index on a spectrum based on defined methodology
measure method can be mean, median, integrate
index method can be ratio = 1/2, valley = 1-2/3, OTHERS
output is index value and uncertainty

\end{description}\end{quote}

\end{fulllineitems}

\index{measureIndexSet() (in module splat)}

\begin{fulllineitems}
\phantomsection\label{api:splat.measureIndexSet}\pysiglinewithargsret{\sphinxcode{splat.}\sphinxbfcode{measureIndexSet}}{\emph{sp}, \emph{**kwargs}}{}~\begin{quote}\begin{description}
\item[{Purpose}] \leavevmode
Measures indices of \sphinxcode{sp} from specified sets. Returns dictionary of indices.

\item[{Parameters}] \leavevmode\begin{itemize}
\item {} 
\textbf{\texttt{sp}} -- Spectrum class object, which should contain wave, flux and noise array elements

\item {} 
\textbf{\texttt{set}} (\emph{\texttt{optional, default = 'burgasser'}}) -- 
string defining which indices set you want to use; options include:
\begin{itemize}
\item {} 
\emph{bardalez}: H2O-J, CH4-J, H2O-H, CH4-H, H2O-K, CH4-K, K-J, H-dip, K-slope, J-slope, J-curve, H-bump, H2O-Y from \href{http://adsabs.harvard.edu/abs/2014ApJ...794..143B}{Bardalez Gagliuffi et al. (2014)}

\item {} 
\emph{burgasser}: H2O-J, CH4-J, H2O-H, CH4-H, H2O-K, CH4-K, K-J from \href{http://adsabs.harvard.edu/abs/2006ApJ...637.1067B}{Burgasser et al. (2006)}

\item {} 
\emph{tokunaga}: K1, K2 from \href{http://adsabs.harvard.edu/abs/1999AJ....117.1010T}{Tokunaga \& Kobayashi (1999)}

\item {} 
\emph{reid}: H2O-A, H2O-B from \href{http://adsabs.harvard.edu/abs/2001AJ....121.1710R}{Reid et al. (2001)}

\item {} 
\emph{geballe}: H2O-1.2, H2O-1.5, CH4-2.2 from \href{http://adsabs.harvard.edu/abs/2002ApJ...564..466G}{Geballe et al. (2002)}

\item {} 
\emph{allers}: H2O, FeH-z, VO-z, FeH-J, KI-J, H-cont from \href{http://adsabs.harvard.edu/abs/2007ApJ...657..511A}{Allers et al. (2007)}, \href{http://adsabs.harvard.edu/abs/2013ApJ...772...79A}{Allers \& Liu (2013)}

\item {} 
\emph{testi}: sHJ, sKJ, sH2O-J, sH2O-H1, sH2O-H2, sH2O-K from \href{http://adsabs.harvard.edu/abs/2001ApJ...552L.147T}{Testi et al. (2001)}

\item {} 
\emph{slesnick}: H2O-1, H2O-2, FeH from \href{http://adsabs.harvard.edu/abs/2004ApJ...610.1045S}{Slesnick et al. (2004)}

\item {} 
\emph{mclean}: H2OD from \href{http://adsabs.harvard.edu/abs/2003ApJ...596..561M}{McLean et al. (2003)}

\item {} 
\emph{rojas}: H2O-K2 from \href{http://adsabs.harvard.edu/abs/2012ApJ...748...93R}{Rojas-Ayala et al.(2012)}

\end{itemize}


\end{itemize}

\item[{Example}] \leavevmode
\end{description}\end{quote}

\begin{Verbatim}[commandchars=\\\{\}]
\PYG{g+gp}{\PYGZgt{}\PYGZgt{}\PYGZgt{} }\PYG{k+kn}{import} \PYG{n+nn}{splat}
\PYG{g+gp}{\PYGZgt{}\PYGZgt{}\PYGZgt{} }\PYG{n}{sp} \PYG{o}{=} \PYG{n}{splat}\PYG{o}{.}\PYG{n}{getSpectrum}\PYG{p}{(}\PYG{n}{shortname}\PYG{o}{=}\PYG{l+s+s1}{\PYGZsq{}}\PYG{l+s+s1}{1555+0954}\PYG{l+s+s1}{\PYGZsq{}}\PYG{p}{)}\PYG{p}{[}\PYG{l+m+mi}{0}\PYG{p}{]}
\PYG{g+gp}{\PYGZgt{}\PYGZgt{}\PYGZgt{} }\PYG{n+nb}{print} \PYG{n}{splat}\PYG{o}{.}\PYG{n}{measureIndexSet}\PYG{p}{(}\PYG{n}{sp}\PYG{p}{,} \PYG{n+nb}{set} \PYG{o}{=} \PYG{l+s+s1}{\PYGZsq{}}\PYG{l+s+s1}{reid}\PYG{l+s+s1}{\PYGZsq{}}\PYG{p}{)}
\PYG{g+go}{    \PYGZob{}\PYGZsq{}H2O\PYGZhy{}B\PYGZsq{}: (1.0531856077273236, 0.0045092074790538221), \PYGZsq{}H2O\PYGZhy{}A\PYGZsq{}: (0.89673318593633422, 0.0031278302105038594)\PYGZcb{}}
\end{Verbatim}

\end{fulllineitems}

\index{measureEW() (in module splat)}

\begin{fulllineitems}
\phantomsection\label{api:splat.measureEW}\pysiglinewithargsret{\sphinxcode{splat.}\sphinxbfcode{measureEW}}{\emph{sp}, \emph{*args}, \emph{**kwargs}}{}~\begin{quote}\begin{description}
\item[{Purpose}] \leavevmode
Measures equivalent widths (EWs) of specified lines

\item[{Parameters}] \leavevmode\begin{itemize}
\item {} 
\textbf{\texttt{sp}} -- Spectrum class object, which should contain wave, flux and noise array elements

\item {} 
\textbf{\texttt{args}} -- wavelength arrays. Needs at least two arrays to measure line and continuum regions.

\item {} 
\textbf{\texttt{nonoise}} (\emph{\texttt{optional, default = '{'}}}) -- 

\item {} 
\textbf{\texttt{line}} -- 

\end{itemize}

\end{description}\end{quote}

\end{fulllineitems}

\index{measureEWSet() (in module splat)}

\begin{fulllineitems}
\phantomsection\label{api:splat.measureEWSet}\pysiglinewithargsret{\sphinxcode{splat.}\sphinxbfcode{measureEWSet}}{\emph{sp}, \emph{*args}, \emph{**kwargs}}{}~\begin{quote}\begin{description}
\item[{Purpose}] \leavevmode
Measures equivalent widths (EWs) of lines from specified sets. Returns dictionary of indices.

\item[{Parameters}] \leavevmode\begin{itemize}
\item {} 
\textbf{\texttt{sp}} -- Spectrum class object, which should contain wave, flux and noise array elements

\item {} 
\textbf{\texttt{set}} (\emph{\texttt{optional, default = 'rojas'}}) -- 
string defining which EW measurement set you want to use; options include:
\begin{itemize}
\item {} 
\emph{rojas}: EW measures from \href{http://adsabs.harvard.edu/abs/2012ApJ...748...93R}{Rojas-Ayala et al. (2012)};
uses Na I 2.206/2.209 Ca I 2.26 micron lines.

\end{itemize}


\end{itemize}

\item[{Example}] \leavevmode
\end{description}\end{quote}

\begin{Verbatim}[commandchars=\\\{\}]
\PYG{g+gp}{\PYGZgt{}\PYGZgt{}\PYGZgt{} }\PYG{k+kn}{import} \PYG{n+nn}{splat}
\PYG{g+gp}{\PYGZgt{}\PYGZgt{}\PYGZgt{} }\PYG{n}{sp} \PYG{o}{=} \PYG{n}{splat}\PYG{o}{.}\PYG{n}{getSpectrum}\PYG{p}{(}\PYG{n}{shortname}\PYG{o}{=}\PYG{l+s+s1}{\PYGZsq{}}\PYG{l+s+s1}{1555+0954}\PYG{l+s+s1}{\PYGZsq{}}\PYG{p}{)}\PYG{p}{[}\PYG{l+m+mi}{0}\PYG{p}{]}
\PYG{g+gp}{\PYGZgt{}\PYGZgt{}\PYGZgt{} }\PYG{n+nb}{print} \PYG{n}{splat}\PYG{o}{.}\PYG{n}{measureEWSet}\PYG{p}{(}\PYG{n}{sp}\PYG{p}{,} \PYG{n+nb}{set} \PYG{o}{=} \PYG{l+s+s1}{\PYGZsq{}}\PYG{l+s+s1}{rojas}\PYG{l+s+s1}{\PYGZsq{}}\PYG{p}{)}
\PYG{g+go}{    \PYGZob{}\PYGZsq{}Na I 2.206/2.209\PYGZsq{}: (1.7484002652013144, 0.23332441577025356), \PYGZsq{}Ca I 2.26\PYGZsq{}: (1.3742491939667159, 0.24867705962337672), \PYGZsq{}names\PYGZsq{}: [\PYGZsq{}Na I 2.206/2.209\PYGZsq{}, \PYGZsq{}Ca I 2.26\PYGZsq{}], \PYGZsq{}reference\PYGZsq{}: \PYGZsq{}EW measures from Rojas\PYGZhy{}Ayala et al. (2012)\PYGZsq{}\PYGZcb{}}
\end{Verbatim}

\end{fulllineitems}



\subsubsection{Source Analysis}
\label{api:source-analysis}\index{typeToMag() (in module splat)}

\begin{fulllineitems}
\phantomsection\label{api:splat.typeToMag}\pysiglinewithargsret{\sphinxcode{splat.}\sphinxbfcode{typeToMag}}{\emph{spt}, \emph{filt}, \emph{**kwargs}}{}~\begin{quote}\begin{description}
\item[{Purpose}] \leavevmode
Takes a spectral type and a filter, and returns absolute magnitude

\item[{Parameters}] \leavevmode\begin{itemize}
\item {} 
\textbf{\texttt{spt}} -- string or integer of the spectral type

\item {} 
\textbf{\texttt{filter}} -- filter of the absolute magnitude. Options are MKO K, MKO H, MKO J, MKO Y, MKO LP, 2MASS J, 2MASS K, or 2MASS H

\item {} 
\textbf{\texttt{nsamples}} (\emph{\texttt{optional, default = 100}}) -- number of Monte Carlo samples for error computation

\item {} 
\textbf{\texttt{unc}} (\emph{\texttt{optional, default = 0.}}) -- uncertainty of \sphinxcode{spt}

\item {} 
\textbf{\texttt{ref}} (\emph{\texttt{optional, default = 'dupuy'}}) -- 
Abs Mag/SpT relation used to compute the absolute magnitude. Options are:
\begin{itemize}
\item {} 
\emph{burgasser}: Abs Mag/SpT relation from \href{http://adsabs.harvard.edu/abs/2007ApJ...659..655B}{Burgasser (2007)}.
Allowed spectral type range is L0 to T8, and allowed filters are MKO K.

\item {} 
\emph{faherty}: Abs Mag/SpT relation from \href{http://adsabs.harvard.edu/abs/2012ApJ...752...56F}{Faherty et al. (2012)}.
Allowed spectral type range is L0 to T8, and allowed filters are MKO J, MKO H and MKO K.

\item {} 
\emph{dupuy}: Abs Mag/SpT relation from \href{http://adsabs.harvard.edu/abs/2012ApJS..201...19D}{Dupuy \& Liu (2012)}.
Allowed spectral type range is M6 to T9, and allowed filters are MKO J, MKO Y, MKO H, MKO K, MKO LP, 2MASS J, 2MASS H, and 2MASS K.

\item {} 
\emph{filippazzo}: Abs Mag/SpT relation from Filippazzo et al. (2015). Allowed spectral type range is M6 to T9, and allowed filters are 2MASS J and WISE W2.

\end{itemize}


\end{itemize}

\item[{Example}] \leavevmode
\begin{Verbatim}[commandchars=\\\{\}]
\PYG{g+gp}{\PYGZgt{}\PYGZgt{}\PYGZgt{} }\PYG{k+kn}{import} \PYG{n+nn}{splat}
\PYG{g+gp}{\PYGZgt{}\PYGZgt{}\PYGZgt{} }\PYG{n+nb}{print} \PYG{n}{splat}\PYG{o}{.}\PYG{n}{typeToMag}\PYG{p}{(}\PYG{l+s+s1}{\PYGZsq{}}\PYG{l+s+s1}{L3}\PYG{l+s+s1}{\PYGZsq{}}\PYG{p}{,} \PYG{l+s+s1}{\PYGZsq{}}\PYG{l+s+s1}{2MASS J}\PYG{l+s+s1}{\PYGZsq{}}\PYG{p}{)}
\PYG{g+go}{    (12.730064813273996, 0.4)}
\PYG{g+gp}{\PYGZgt{}\PYGZgt{}\PYGZgt{} }\PYG{n+nb}{print} \PYG{n}{splat}\PYG{o}{.}\PYG{n}{typeToMag}\PYG{p}{(}\PYG{l+m+mi}{21}\PYG{p}{,} \PYG{l+s+s1}{\PYGZsq{}}\PYG{l+s+s1}{MKO K}\PYG{l+s+s1}{\PYGZsq{}}\PYG{p}{,} \PYG{n}{ref} \PYG{o}{=} \PYG{l+s+s1}{\PYGZsq{}}\PYG{l+s+s1}{burgasser}\PYG{l+s+s1}{\PYGZsq{}}\PYG{p}{)}
\PYG{g+go}{    (10.705292820099999, 0.26)}
\PYG{g+gp}{\PYGZgt{}\PYGZgt{}\PYGZgt{} }\PYG{n+nb}{print} \PYG{n}{splat}\PYG{o}{.}\PYG{n}{typeToMag}\PYG{p}{(}\PYG{l+m+mi}{24}\PYG{p}{,} \PYG{l+s+s1}{\PYGZsq{}}\PYG{l+s+s1}{2MASS J}\PYG{l+s+s1}{\PYGZsq{}}\PYG{p}{,} \PYG{n}{ref} \PYG{o}{=} \PYG{l+s+s1}{\PYGZsq{}}\PYG{l+s+s1}{faherty}\PYG{l+s+s1}{\PYGZsq{}}\PYG{p}{)}
\PYG{g+go}{    Invalid filter given for Abs Mag/SpT relation from Faherty et al. (2012)}
\PYG{g+go}{    (nan, nan)}
\PYG{g+gp}{\PYGZgt{}\PYGZgt{}\PYGZgt{} }\PYG{n+nb}{print} \PYG{n}{splat}\PYG{o}{.}\PYG{n}{typeToMag}\PYG{p}{(}\PYG{l+s+s1}{\PYGZsq{}}\PYG{l+s+s1}{M0}\PYG{l+s+s1}{\PYGZsq{}}\PYG{p}{,} \PYG{l+s+s1}{\PYGZsq{}}\PYG{l+s+s1}{2MASS H}\PYG{l+s+s1}{\PYGZsq{}}\PYG{p}{,} \PYG{n}{ref} \PYG{o}{=} \PYG{l+s+s1}{\PYGZsq{}}\PYG{l+s+s1}{dupuy}\PYG{l+s+s1}{\PYGZsq{}}\PYG{p}{)}
\PYG{g+go}{    Spectral Type is out of range for Abs Mag/SpT relation from Dupuy \PYGZam{} Liu (2012) Abs Mag/SpT relation}
\PYG{g+go}{    (nan, nan)}
\end{Verbatim}

\end{description}\end{quote}

\end{fulllineitems}

\index{typeToTeff() (in module splat)}

\begin{fulllineitems}
\phantomsection\label{api:splat.typeToTeff}\pysiglinewithargsret{\sphinxcode{splat.}\sphinxbfcode{typeToTeff}}{\emph{input}, \emph{**kwargs}}{}~\begin{quote}\begin{description}
\item[{Purpose}] \leavevmode
Returns an effective temperature (Teff) and its uncertainty for a given spectral type

\item[{Parameters}] \leavevmode\begin{itemize}
\item {} 
\textbf{\texttt{input}} -- Spectral type; can be a number or a string from 0 (K0) and 49.0 (Y9).

\item {} 
\textbf{\texttt{uncertainty}} (\emph{\texttt{optional, default = 0.001}}) -- uncertainty of spectral type

\item {} 
\textbf{\texttt{unc}} (\emph{\texttt{optional, default = 0.001}}) -- same as \sphinxcode{uncertainty}

\item {} 
\textbf{\texttt{spt\_e}} (\emph{\texttt{optional, default = 0.001}}) -- same as \sphinxcode{uncertainty}

\item {} 
\textbf{\texttt{ref}} (\emph{\texttt{optional, default = 'stephens2009'}}) -- 
Teff/SpT relation used to compute the effective temperature. Options are:
\begin{itemize}
\item {} 
\emph{golimowski}: Teff/SpT relation from \href{http://adsabs.harvard.edu/abs/2004AJ....127.3516G}{Golimowski et al. (2004)}.
Allowed spectral type range is M6 to T8.

\item {} 
\emph{looper}: Teff/SpT relation from \href{http://adsabs.harvard.edu/abs/2008ApJ...685.1183L}{Looper et al. (2008)}.
Allowed spectral type range is L0 to T8.

\item {} 
\emph{stephens}: Teff/SpT relation from \href{http://adsabs.harvard.edu/abs/2009ApJ...702..154S}{Stephens et al. (2009)}.
Allowed spectral type range is M6 to T8 and uses alternate coefficients for L3 to T8.

\item {} 
\emph{marocco}: Teff/SpT relation from \href{http://adsabs.harvard.edu/abs/2013AJ....146..161M}{Marocco et al. (2013)}.
Allowed spectral type range is M7 to T8.

\item {} 
\emph{filippazzo}: Teff/SpT relation from Filippazzo et al. (2015). Allowed spectral type range is M6 to T9.

\end{itemize}


\item {} 
\textbf{\texttt{set}} (\emph{\texttt{optional, default = 'stephens2009'}}) -- same as \sphinxcode{ref}

\item {} 
\textbf{\texttt{method}} (\emph{\texttt{optional, default = 'stephens2009'}}) -- same as \sphinxcode{ref}

\item {} 
\textbf{\texttt{nsamples}} (\emph{\texttt{optional, default = 100}}) -- number of samples to use in Monte Carlo error estimation

\end{itemize}

\item[{Example}] \leavevmode
\begin{Verbatim}[commandchars=\\\{\}]
\PYG{g+gp}{\PYGZgt{}\PYGZgt{}\PYGZgt{} }\PYG{k+kn}{import} \PYG{n+nn}{splat}
\PYG{g+gp}{\PYGZgt{}\PYGZgt{}\PYGZgt{} }\PYG{n+nb}{print} \PYG{n}{splat}\PYG{o}{.}\PYG{n}{typeToTeff}\PYG{p}{(}\PYG{l+m+mi}{20}\PYG{p}{)}
\PYG{g+go}{    (2233.4796740905499, 100.00007874571999)}
\PYG{g+gp}{\PYGZgt{}\PYGZgt{}\PYGZgt{} }\PYG{n+nb}{print} \PYG{n}{splat}\PYG{o}{.}\PYG{n}{typeToTeff}\PYG{p}{(}\PYG{l+m+mi}{20}\PYG{p}{,} \PYG{n}{unc} \PYG{o}{=} \PYG{l+m+mf}{0.3}\PYG{p}{,} \PYG{n}{ref} \PYG{o}{=} \PYG{l+s+s1}{\PYGZsq{}}\PYG{l+s+s1}{golimowski}\PYG{l+s+s1}{\PYGZsq{}}\PYG{p}{)}
\PYG{g+go}{    (2305.7500497902788, 127.62548366132124)}
\end{Verbatim}

\end{description}\end{quote}

\end{fulllineitems}

\index{estimateDistance() (in module splat)}

\begin{fulllineitems}
\phantomsection\label{api:splat.estimateDistance}\pysiglinewithargsret{\sphinxcode{splat.}\sphinxbfcode{estimateDistance}}{\emph{*args}, \emph{**kwargs}}{}~\begin{quote}\begin{description}
\item[{Purpose}] \leavevmode
Takes the apparent magnitude and either takes or determines the absolute
magnitude, then uses the magnitude/distance relation to estimate the
distance to the object in parsecs. Returns estimated distance and
uncertainty in parsecs

\item[{Parameters}] \leavevmode\begin{itemize}
\item {} 
\textbf{\texttt{sp}} -- Spectrum class object, which should be flux calibrated to its empirical apparent magnitude

\item {} 
\textbf{\texttt{mag}} (\emph{\texttt{optional, default = False}}) -- apparent magnitude of \sphinxcode{sp}

\item {} 
\textbf{\texttt{mag\_unc}} (\emph{\texttt{optional, default = 0}}) -- uncertainty of the apparent magnitude

\item {} 
\textbf{\texttt{absmag}} (\emph{\texttt{optional, default = False}}) -- absolute magnitude of \sphinxcode{sp}

\item {} 
\textbf{\texttt{absmag\_unc}} (\emph{\texttt{optional, default = 0}}) -- uncertainty of the absolute magnitude

\item {} 
\textbf{\texttt{spt}} (\emph{\texttt{optional, default = False}}) -- spectral type of \sphinxcode{sp}

\item {} 
\textbf{\texttt{spt\_e}} (\emph{\texttt{optional, default = 0}}) -- uncertainty of the spectral type

\item {} 
\textbf{\texttt{nsamples}} (\emph{\texttt{optional, default = 100}}) -- number of samples to use in Monte Carlo error estimation

\item {} 
\textbf{\texttt{filter}} (\emph{\texttt{optional, default = False}}) -- 
Name of filter, must be one of the following:
\begin{itemize}
\item {} 
`2MASS J', `2MASS H', `2MASS Ks'

\item {} 
`MKO J', `MKO H', `MKO K', MKO Kp', `MKO Ks'

\item {} 
`NICMOS F090M', `NICMOS F095N', `NICMOS F097N', `NICMOS F108N'

\item {} 
`NICMOS F110M', `NICMOS F110W', `NICMOS F113N', `NICMOS F140W'

\item {} 
`NICMOS F145M', `NICMOS F160W', `NICMOS F164N', `NICMOS F165M'

\item {} 
`NICMOS F166N', `NICMOS F170M', `NICMOS F187N', `NICMOS F190N'

\item {} 
`NIRC2 J', `NIRC2 H', `NIRC2 Kp', `NIRC2 Ks'

\item {} 
`WIRC J', `WIRC H', `WIRC K', `WIRC CH4S', `WIRC CH4L'

\item {} 
`WIRC CO', `WIRC PaBeta', `WIRC BrGamma', `WIRC Fe2'

\item {} 
`WISE W1', `WISE W2'

\end{itemize}


\end{itemize}

\item[{Example}] \leavevmode
\end{description}\end{quote}

\begin{Verbatim}[commandchars=\\\{\}]
\PYG{g+gp}{\PYGZgt{}\PYGZgt{}\PYGZgt{} }\PYG{k+kn}{import} \PYG{n+nn}{splat}
\PYG{g+gp}{\PYGZgt{}\PYGZgt{}\PYGZgt{} }\PYG{n}{sp} \PYG{o}{=} \PYG{n}{splat}\PYG{o}{.}\PYG{n}{getSpectrum}\PYG{p}{(}\PYG{n}{shortname}\PYG{o}{=}\PYG{l+s+s1}{\PYGZsq{}}\PYG{l+s+s1}{1555+0954}\PYG{l+s+s1}{\PYGZsq{}}\PYG{p}{)}\PYG{p}{[}\PYG{l+m+mi}{0}\PYG{p}{]}
\PYG{g+gp}{\PYGZgt{}\PYGZgt{}\PYGZgt{} }\PYG{n+nb}{print} \PYG{n}{splat}\PYG{o}{.}\PYG{n}{estimateDistance}\PYG{p}{(}\PYG{n}{sp}\PYG{p}{)}
\PYG{g+go}{    Please specify the filter used to determine the apparent magnitude}
\PYG{g+go}{    (nan, nan)}
\PYG{g+gp}{\PYGZgt{}\PYGZgt{}\PYGZgt{} }\PYG{n+nb}{print} \PYG{n}{splat}\PYG{o}{.}\PYG{n}{estimateDistance}\PYG{p}{(}\PYG{n}{sp}\PYG{p}{,} \PYG{n}{mag} \PYG{o}{=} \PYG{l+m+mf}{12.521}\PYG{p}{,} \PYG{n}{mag\PYGZus{}unc} \PYG{o}{=} \PYG{l+m+mf}{0.022}\PYG{p}{,} \PYG{n}{absmag} \PYG{o}{=} \PYG{l+m+mf}{7.24}\PYG{p}{,} \PYG{n}{absmag\PYGZus{}unc} \PYG{o}{=} \PYG{l+m+mf}{0.50}\PYG{p}{,} \PYG{n}{spt} \PYG{o}{=} \PYG{l+s+s1}{\PYGZsq{}}\PYG{l+s+s1}{M3}\PYG{l+s+s1}{\PYGZsq{}}\PYG{p}{)}
\PYG{g+go}{    (116.36999172188771, 33.124820555524224)}
\end{Verbatim}

\end{fulllineitems}



\subsubsection{Conversion Routines}
\label{api:conversion-routines}\index{caldateToDate() (in module splat)}

\begin{fulllineitems}
\phantomsection\label{api:splat.caldateToDate}\pysiglinewithargsret{\sphinxcode{splat.}\sphinxbfcode{caldateToDate}}{\emph{d}}{}~\begin{quote}\begin{description}
\item[{Purpose}] \leavevmode
Convert from numeric date to calendar date, and vice-versa.

\item[{Parameters}] \leavevmode
\textbf{\texttt{d}} -- A numeric date of the format `20050412', or a date in the
calendar format `2005 Jun 12'

\item[{Example}] \leavevmode
\begin{Verbatim}[commandchars=\\\{\}]
\PYG{g+gp}{\PYGZgt{}\PYGZgt{}\PYGZgt{} }\PYG{k+kn}{import} \PYG{n+nn}{splat}
\PYG{g+gp}{\PYGZgt{}\PYGZgt{}\PYGZgt{} }\PYG{n}{caldate} \PYG{o}{=} \PYG{n}{splat}\PYG{o}{.}\PYG{n}{dateToCaldate}\PYG{p}{(}\PYG{l+s+s1}{\PYGZsq{}}\PYG{l+s+s1}{20050612}\PYG{l+s+s1}{\PYGZsq{}}\PYG{p}{)}
\PYG{g+gp}{\PYGZgt{}\PYGZgt{}\PYGZgt{} }\PYG{n+nb}{print} \PYG{n}{caldate}
\PYG{g+go}{2005 Jun 12}
\PYG{g+gp}{\PYGZgt{}\PYGZgt{}\PYGZgt{} }\PYG{n}{date} \PYG{o}{=} \PYG{n}{splat}\PYG{o}{.}\PYG{n}{caldateToDate}\PYG{p}{(}\PYG{l+s+s1}{\PYGZsq{}}\PYG{l+s+s1}{2005 June 12}\PYG{l+s+s1}{\PYGZsq{}}\PYG{p}{)}
\PYG{g+gp}{\PYGZgt{}\PYGZgt{}\PYGZgt{} }\PYG{n+nb}{print} \PYG{n}{date}
\PYG{g+go}{20050612}
\end{Verbatim}

\end{description}\end{quote}

\end{fulllineitems}

\index{dateToCaldate() (in module splat)}

\begin{fulllineitems}
\phantomsection\label{api:splat.dateToCaldate}\pysiglinewithargsret{\sphinxcode{splat.}\sphinxbfcode{dateToCaldate}}{\emph{date}}{}~\begin{quote}\begin{description}
\item[{Purpose}] \leavevmode
Converts numeric date to calendar date

\item[{Parameters}] \leavevmode
\textbf{\texttt{date}} -- String in the form `YYYYMMDD'

\item[{Output}] \leavevmode
Date in format YYYY MMM DD

\item[{Example}] \leavevmode
\end{description}\end{quote}

\begin{Verbatim}[commandchars=\\\{\}]
\PYG{g+gp}{\PYGZgt{}\PYGZgt{}\PYGZgt{} }\PYG{k+kn}{import} \PYG{n+nn}{splat}
\PYG{g+gp}{\PYGZgt{}\PYGZgt{}\PYGZgt{} }\PYG{n}{splat}\PYG{o}{.}\PYG{n}{dateToCaldate}\PYG{p}{(}\PYG{l+s+s1}{\PYGZsq{}}\PYG{l+s+s1}{19940523}\PYG{l+s+s1}{\PYGZsq{}}\PYG{p}{)}
\PYG{g+go}{    1994 May 23}
\end{Verbatim}

\end{fulllineitems}

\index{coordinateToDesignation() (in module splat)}

\begin{fulllineitems}
\phantomsection\label{api:splat.coordinateToDesignation}\pysiglinewithargsret{\sphinxcode{splat.}\sphinxbfcode{coordinateToDesignation}}{\emph{c}}{}~\begin{quote}\begin{description}
\item[{Purpose}] \leavevmode
Converts right ascension and declination into a designation string

\item[{Parameters}] \leavevmode
\textbf{\texttt{c}} -- RA and Dec coordinate to be converted; can be a SkyCoord object with units of degrees,
a list with RA and Dec in degrees, or a string with RA measured in hour
angles and Dec in degrees

\item[{Output}] \leavevmode
Designation string

\item[{Example}] \leavevmode
\end{description}\end{quote}

\begin{Verbatim}[commandchars=\\\{\}]
\PYG{g+gp}{\PYGZgt{}\PYGZgt{}\PYGZgt{} }\PYG{k+kn}{import} \PYG{n+nn}{splat}
\PYG{g+gp}{\PYGZgt{}\PYGZgt{}\PYGZgt{} }\PYG{k+kn}{from} \PYG{n+nn}{astropy}\PYG{n+nn}{.}\PYG{n+nn}{coordinates} \PYG{k}{import} \PYG{n}{SkyCoord}
\PYG{g+gp}{\PYGZgt{}\PYGZgt{}\PYGZgt{} }\PYG{n}{c} \PYG{o}{=} \PYG{n}{SkyCoord}\PYG{p}{(}\PYG{l+m+mf}{238.86}\PYG{p}{,} \PYG{l+m+mf}{9.90}\PYG{p}{,} \PYG{n}{unit}\PYG{o}{=}\PYG{l+s+s2}{\PYGZdq{}}\PYG{l+s+s2}{deg}\PYG{l+s+s2}{\PYGZdq{}}\PYG{p}{)}
\PYG{g+gp}{\PYGZgt{}\PYGZgt{}\PYGZgt{} }\PYG{n+nb}{print} \PYG{n}{splat}\PYG{o}{.}\PYG{n}{coordinateToDesignation}\PYG{p}{(}\PYG{n}{c}\PYG{p}{)}
\PYG{g+go}{    J15552640+0954000}
\PYG{g+gp}{\PYGZgt{}\PYGZgt{}\PYGZgt{} }\PYG{n+nb}{print} \PYG{n}{splat}\PYG{o}{.}\PYG{n}{coordinateToDesignation}\PYG{p}{(}\PYG{p}{[}\PYG{l+m+mf}{238.86}\PYG{p}{,} \PYG{l+m+mf}{9.90}\PYG{p}{]}\PYG{p}{)}
\PYG{g+go}{    J15552640+0954000}
\PYG{g+gp}{\PYGZgt{}\PYGZgt{}\PYGZgt{} }\PYG{n+nb}{print} \PYG{n}{splat}\PYG{o}{.}\PYG{n}{coordinateToDesignation}\PYG{p}{(}\PYG{l+s+s1}{\PYGZsq{}}\PYG{l+s+s1}{15:55:26.4 +09:54:00.0}\PYG{l+s+s1}{\PYGZsq{}}\PYG{p}{)}
\PYG{g+go}{    J15552640+0954000}
\end{Verbatim}

\end{fulllineitems}

\index{designationToCoordinate() (in module splat)}

\begin{fulllineitems}
\phantomsection\label{api:splat.designationToCoordinate}\pysiglinewithargsret{\sphinxcode{splat.}\sphinxbfcode{designationToCoordinate}}{\emph{value}, \emph{**kwargs}}{}~\begin{quote}\begin{description}
\item[{Purpose}] \leavevmode
Convert a designation srtring into a RA, Dec tuple or ICRS SkyCoord objects (default)

\item[{Parameters}] \leavevmode\begin{itemize}
\item {} 
\textbf{\texttt{value}} (\emph{\texttt{required}}) -- Designation string with RA measured in hour angles and Dec in degrees

\item {} 
\textbf{\texttt{icrs}} (\emph{\texttt{optional, defualt = True}}) -- returns astropy SkyCoord coordinate in ICRS frame if \sphinxcode{True}

\end{itemize}

\item[{Output}] \leavevmode
Coordinate, either as {[}RA, Dec{]} or SkyCoord object

\item[{Example}] \leavevmode
\end{description}\end{quote}

\begin{Verbatim}[commandchars=\\\{\}]
\PYG{g+gp}{\PYGZgt{}\PYGZgt{}\PYGZgt{} }\PYG{k+kn}{import} \PYG{n+nn}{splat}
\PYG{g+gp}{\PYGZgt{}\PYGZgt{}\PYGZgt{} }\PYG{n}{splat}\PYG{o}{.}\PYG{n}{designationToCoordinate}\PYG{p}{(}\PYG{l+s+s1}{\PYGZsq{}}\PYG{l+s+s1}{J1555264+0954120}\PYG{l+s+s1}{\PYGZsq{}}\PYG{p}{)}
\PYG{g+go}{\PYGZlt{}SkyCoord (ICRS): (ra, dec) in deg}
\PYG{g+go}{    (238.8585, 9.90333333)\PYGZgt{}}
\end{Verbatim}

\end{fulllineitems}

\index{designationToShortName() (in module splat)}

\begin{fulllineitems}
\phantomsection\label{api:splat.designationToShortName}\pysiglinewithargsret{\sphinxcode{splat.}\sphinxbfcode{designationToShortName}}{\emph{value}}{}~\begin{quote}\begin{description}
\item[{Purpose}] \leavevmode
Produce a shortened version of designation

\item[{Parameters}] \leavevmode
\textbf{\texttt{value}} (\emph{\texttt{required}}) -- Designation string with RA measured in hour angles and Dec in degrees

\item[{Output}] \leavevmode
Shorthand designation string

\item[{Example}] \leavevmode
\end{description}\end{quote}

\begin{Verbatim}[commandchars=\\\{\}]
\PYG{g+gp}{\PYGZgt{}\PYGZgt{}\PYGZgt{} }\PYG{k+kn}{import} \PYG{n+nn}{splat}
\PYG{g+gp}{\PYGZgt{}\PYGZgt{}\PYGZgt{} }\PYG{n+nb}{print} \PYG{n}{splat}\PYG{o}{.}\PYG{n}{designationToShortName}\PYG{p}{(}\PYG{l+s+s1}{\PYGZsq{}}\PYG{l+s+s1}{J1555264+0954120}\PYG{l+s+s1}{\PYGZsq{}}\PYG{p}{)}
\PYG{g+go}{    J1555+0954}
\end{Verbatim}

\end{fulllineitems}

\index{typeToNum() (in module splat)}

\begin{fulllineitems}
\phantomsection\label{api:splat.typeToNum}\pysiglinewithargsret{\sphinxcode{splat.}\sphinxbfcode{typeToNum}}{\emph{input}, \emph{**kwargs}}{}~\begin{quote}\begin{description}
\item[{Purpose}] \leavevmode
Converts between string and numeric spectral types, and vise versa.

\item[{Parameters}] \leavevmode\begin{itemize}
\item {} 
\textbf{\texttt{input}} -- Spectral type to convert. Can convert a number or a string from 0 (K0) and 49.0 (Y9).

\item {} 
\textbf{\texttt{error}} (\emph{\texttt{optional, default = '{'}}}) -- magnitude of uncertainty. `:' for uncertainty \textgreater{} 1 and `::' for uncertainty \textgreater{} 2.

\item {} 
\textbf{\texttt{uncertainty}} (\emph{\texttt{optional, default = 0}}) -- uncertainty of spectral type

\item {} 
\textbf{\texttt{subclass}} (\emph{\texttt{optional, default = '{'}}}) -- 
subclass of object. Options include:
\begin{itemize}
\item {} 
\emph{sd}: object is a subdwarf

\item {} 
\emph{esd}: object is an extreme subdwarf

\item {} 
\emph{usd}: object is an ultra subdwarf

\end{itemize}


\item {} 
\textbf{\texttt{lumclass}} (\emph{\texttt{optional, default = '{'}}}) -- luminosity class of object represented by roman numerals

\item {} 
\textbf{\texttt{ageclass}} (\emph{\texttt{optional, default = '{'}}}) -- age class of object

\item {} 
\textbf{\texttt{colorclass}} (\emph{\texttt{optional, default = '{'}}}) -- color class of object

\item {} 
\textbf{\texttt{peculiar}} (\emph{\texttt{optional, default = False}}) -- if object is peculiar or not

\end{itemize}

\end{description}\end{quote}
\begin{quote}\begin{description}
\item[{Example}] \leavevmode
\begin{Verbatim}[commandchars=\\\{\}]
\PYG{g+gp}{\PYGZgt{}\PYGZgt{}\PYGZgt{} }\PYG{k+kn}{import} \PYG{n+nn}{splat}
\PYG{g+gp}{\PYGZgt{}\PYGZgt{}\PYGZgt{} }\PYG{n+nb}{print} \PYG{n}{splat}\PYG{o}{.}\PYG{n}{typeToNum}\PYG{p}{(}\PYG{l+m+mi}{30}\PYG{p}{)}
\PYG{g+go}{    T0.0}
\PYG{g+gp}{\PYGZgt{}\PYGZgt{}\PYGZgt{} }\PYG{n+nb}{print} \PYG{n}{splat}\PYG{o}{.}\PYG{n}{typeToNum}\PYG{p}{(}\PYG{l+s+s1}{\PYGZsq{}}\PYG{l+s+s1}{T0.0}\PYG{l+s+s1}{\PYGZsq{}}\PYG{p}{)}
\PYG{g+go}{    30.0}
\PYG{g+gp}{\PYGZgt{}\PYGZgt{}\PYGZgt{} }\PYG{n+nb}{print} \PYG{n}{splat}\PYG{o}{.}\PYG{n}{typeToNum}\PYG{p}{(}\PYG{l+m+mi}{27}\PYG{p}{,} \PYG{n}{peculiar} \PYG{o}{=} \PYG{k+kc}{True}\PYG{p}{,} \PYG{n}{uncertainty} \PYG{o}{=} \PYG{l+m+mf}{1.2}\PYG{p}{,} \PYG{n}{lumclass} \PYG{o}{=} \PYG{l+s+s1}{\PYGZsq{}}\PYG{l+s+s1}{II}\PYG{l+s+s1}{\PYGZsq{}}\PYG{p}{)}
\PYG{g+go}{    L7.0IIp:}
\PYG{g+gp}{\PYGZgt{}\PYGZgt{}\PYGZgt{} }\PYG{n+nb}{print} \PYG{n}{splat}\PYG{o}{.}\PYG{n}{typeToNum}\PYG{p}{(}\PYG{l+m+mi}{50}\PYG{p}{)}
\PYG{g+go}{    Spectral type number must be between 0 (K0) and 49.0 (Y9)}
\PYG{g+go}{    nan}
\end{Verbatim}

\end{description}\end{quote}

\end{fulllineitems}

\index{properCoordinates() (in module splat)}

\begin{fulllineitems}
\phantomsection\label{api:splat.properCoordinates}\pysiglinewithargsret{\sphinxcode{splat.}\sphinxbfcode{properCoordinates}}{\emph{c}}{}~\begin{quote}\begin{description}
\item[{Purpose}] \leavevmode
Converts various coordinate forms to the proper SkyCoord format. Convertible forms include lists and strings.

\item[{Parameters}] \leavevmode
\textbf{\texttt{c}} -- coordinate to be converted. Can be a list (ra, dec) or a string.

\item[{Example}] \leavevmode
\end{description}\end{quote}

\begin{Verbatim}[commandchars=\\\{\}]
\PYG{g+gp}{\PYGZgt{}\PYGZgt{}\PYGZgt{} }\PYG{k+kn}{import} \PYG{n+nn}{splat}
\PYG{g+gp}{\PYGZgt{}\PYGZgt{}\PYGZgt{} }\PYG{n+nb}{print} \PYG{n}{splat}\PYG{o}{.}\PYG{n}{properCoordinates}\PYG{p}{(}\PYG{p}{[}\PYG{l+m+mf}{104.79}\PYG{p}{,} \PYG{l+m+mf}{25.06}\PYG{p}{]}\PYG{p}{)}
\PYG{g+go}{    \PYGZlt{}SkyCoord (ICRS): ra=104.79 deg, dec=25.06 deg\PYGZgt{}}
\PYG{g+gp}{\PYGZgt{}\PYGZgt{}\PYGZgt{} }\PYG{n+nb}{print} \PYG{n}{splat}\PYG{o}{.}\PYG{n}{properCoordinates}\PYG{p}{(}\PYG{l+s+s1}{\PYGZsq{}}\PYG{l+s+s1}{06:59:09.60 +25:03:36.0}\PYG{l+s+s1}{\PYGZsq{}}\PYG{p}{)}
\PYG{g+go}{    \PYGZlt{}SkyCoord (ICRS): ra=104.79 deg, dec=25.06 deg\PYGZgt{}}
\PYG{g+gp}{\PYGZgt{}\PYGZgt{}\PYGZgt{} }\PYG{n+nb}{print} \PYG{n}{splat}\PYG{o}{.}\PYG{n}{properCoordinates}\PYG{p}{(}\PYG{l+s+s1}{\PYGZsq{}}\PYG{l+s+s1}{J06590960+2503360}\PYG{l+s+s1}{\PYGZsq{}}\PYG{p}{)}
\PYG{g+go}{    \PYGZlt{}SkyCoord (ICRS): ra=104.79 deg, dec=25.06 deg\PYGZgt{}}
\end{Verbatim}

\end{fulllineitems}

\index{isNumber() (in module splat)}

\begin{fulllineitems}
\phantomsection\label{api:splat.isNumber}\pysiglinewithargsret{\sphinxcode{splat.}\sphinxbfcode{isNumber}}{\emph{s}}{}~\begin{quote}\begin{description}
\item[{Purpose}] \leavevmode
Checks if something is a number.

\item[{Parameters}] \leavevmode
\textbf{\texttt{s}} (\emph{\texttt{required}}) -- object to be checked

\item[{Output}] \leavevmode
True or False

\item[{Example}] \leavevmode
\end{description}\end{quote}

\begin{Verbatim}[commandchars=\\\{\}]
\PYG{g+gp}{\PYGZgt{}\PYGZgt{}\PYGZgt{} }\PYG{k+kn}{import} \PYG{n+nn}{splat}
\PYG{g+gp}{\PYGZgt{}\PYGZgt{}\PYGZgt{} }\PYG{n+nb}{print} \PYG{n}{splat}\PYG{o}{.}\PYG{n}{isNumber}\PYG{p}{(}\PYG{l+m+mi}{3}\PYG{p}{)}
\PYG{g+go}{    True}
\PYG{g+gp}{\PYGZgt{}\PYGZgt{}\PYGZgt{} }\PYG{n+nb}{print} \PYG{n}{splat}\PYG{o}{.}\PYG{n}{isNumber}\PYG{p}{(}\PYG{l+s+s1}{\PYGZsq{}}\PYG{l+s+s1}{hello}\PYG{l+s+s1}{\PYGZsq{}}\PYG{p}{)}
\PYG{g+go}{    False}
\end{Verbatim}

\end{fulllineitems}



\subsubsection{Plotting Routines}
\label{api:plotting-routines}\index{plotSpectrum() (in module splat\_plot)}

\begin{fulllineitems}
\phantomsection\label{api:splat_plot.plotSpectrum}\pysiglinewithargsret{\sphinxcode{splat\_plot.}\sphinxbfcode{plotSpectrum}}{\emph{*args}, \emph{**kwargs}}{}~\begin{quote}\begin{description}
\item[{Purpose}] \leavevmode
\sphinxcode{Primary plotting program for Spectrum objects.}

\end{description}\end{quote}

:Input
Spectrum objects, either sequentially, in list, or in list of lists
\begin{itemize}
\item {} 
Spec1, Spec2, ...: plot multiple spectra together, or separately if multiplot = True

\item {} 
{[}Spec1, Spec2, ...{]}: plot multiple spectra together, or separately if multiplot = True

\item {} 
{[}{[}Spec1, Spec2{]}, {[}Spec3, Spec4{]}, ..{]}: plot multiple sets of spectra (multiplot forced to be True)

\end{itemize}

:Parameters
title = `'
\begin{quote}

string giving plot title
\end{quote}
\begin{description}
\item[{xrange = {[}0.85,2.42{]}:}] \leavevmode
plot range for wavelength axis

\item[{yrange = {[}-0.02,1.2{]}*fluxMax:}] \leavevmode
plot range for wavelength axis

\item[{xlabel:}] \leavevmode
wavelength axis label; by default set by wlabel and wunit keywords in first spectrum object

\item[{ylabel:}] \leavevmode
flux axis label; by default set by fscale, flabel and funit keywords in first spectrum object

\item[{features:}] \leavevmode
a list of strings indicating chemical features to label on the spectra
options include H2O, CH4, CO, TiO, VO, FeH, H2, HI, KI, NaI, SB (for spectral binary)

\item[{mdwarf, ldwarf, tdwarf, young, binary = False:}] \leavevmode
add in features characteristic of these classes

\item[{telluric = False:}] \leavevmode
mark telluric absorption features

\item[{legend, legends, label or labels:}] \leavevmode
list of strings providing legend-style labels for each spectrum plotted

\item[{legendLocation or labelLocation = `upper right':}] \leavevmode
place of legend; options are `upper left', `center middle', `lower right' (variations thereof) and `outside'

\item[{legendfontscale = 1:}] \leavevmode
sets the scale factor for the legend fontsize (defaults to fontscale)

\item[{grid = False:}] \leavevmode
add a grid

\item[{stack = 0:}] \leavevmode
set to a numerical offset to stack spectra on top of each other

\item[{zeropoint = {[}0,...{]}:}] \leavevmode
list of offsets for each spectrum, giving finer control than stack

\item[{showZero = True:}] \leavevmode
plot the zeropoint(s) of the spectra

\item[{comparison:}] \leavevmode
a comparison Spectrum to compare in each plot, useful for common reference standard

\item[{noise, showNoise or uncertainty = False:}] \leavevmode
plot the uncertainty for each spectrum

\item[{residual = False:}] \leavevmode
plots the residual between two spectra

\item[{color or colors:}] \leavevmode
color of plot lines; by default all black

\item[{colorUnc or colorsUnc:}] \leavevmode
color of uncertainty lines; by default same as line color but reduced opacity

\item[{colorScheme or colorMap:}] \leavevmode
color map to apply based on matplotlib colormaps;
see \url{http://matplotlib.org/api/pyplot\_summary.html?highlight=colormaps\#matplotlib.pyplot.colormaps}

\item[{linestyle:}] \leavevmode
line style of plot lines; by default all solid

\item[{fontscale = 1:}] \leavevmode
sets a scale factor for the fontsize

\item[{inset = False:}] \leavevmode
place an inset panel showing a close up region of the spectral data

\item[{inset\_xrange = False:}] \leavevmode
wavelength range for inset panel

\item[{inset\_position = {[}0.65,0.60,0.20,0.20{]}}] \leavevmode
position of inset planet in normalized units, in order left, bottom, width, height

\item[{inset\_features = False}] \leavevmode
list of features to label in inset plot

\item[{file or filename or output:}] \leavevmode
filename or filename base for output

\item[{filetype = `pdf':}] \leavevmode
output filetype, generally determined from filename

\item[{multiplot = False:}] \leavevmode
creates multiple plots, depending on format of input (optional)

\item[{multipage = False:}] \leavevmode
spreads plots across multiple pages; output file format must be PDF
if not set and plots span multiple pages, these pages are output sequentially as separate files

\item[{layout or multilayout = {[}1,1{]}:}] \leavevmode
defines how multiple plots are laid out on a page

\item[{figsize:}] \leavevmode
set the figure size; set to default size if not indicated

\item[{interactive = False:}] \leavevmode
if plotting to window, set this to make window interactive

\end{description}
\begin{quote}\begin{description}
\item[{Example 1}] \leavevmode
A simple view of a random spectrum
\textgreater{}\textgreater{}\textgreater{} import splat
\textgreater{}\textgreater{}\textgreater{} spc = splat.getSpectrum(spt = `T5', lucky=True){[}0{]}
\textgreater{}\textgreater{}\textgreater{} spc.plot()                       \# this automatically generates a ``quicklook'' plot
\textgreater{}\textgreater{}\textgreater{} splat.plotSpectrum(spc)          \# does the same thing
\textgreater{}\textgreater{}\textgreater{} splat.plotSpectrum(spc,uncertainty=True,tdwarf=True)     \# show the spectrum uncertainty and T dwarf absorption features

\item[{Example 2}] \leavevmode\begin{description}
\item[{Viewing a set of spectra for a given object}] \leavevmode
In this case we'll look at all of the spectra of TWA 30B in the library, sorted by year and compared to the first epoch data
This is an example of using multiplot and multipage

\end{description}

\begin{Verbatim}[commandchars=\\\{\}]
\PYG{g+gp}{\PYGZgt{}\PYGZgt{}\PYGZgt{} }\PYG{n}{splist} \PYG{o}{=} \PYG{n}{splat}\PYG{o}{.}\PYG{n}{getSpectrum}\PYG{p}{(}\PYG{n}{name} \PYG{o}{=} \PYG{l+s+s1}{\PYGZsq{}}\PYG{l+s+s1}{TWA 30B}\PYG{l+s+s1}{\PYGZsq{}}\PYG{p}{)}         \PYG{c+c1}{\PYGZsh{} get all spectra of TWA 30B}
\PYG{g+gp}{\PYGZgt{}\PYGZgt{}\PYGZgt{} }\PYG{n}{junk} \PYG{o}{=} \PYG{p}{[}\PYG{n}{sp}\PYG{o}{.}\PYG{n}{normalize}\PYG{p}{(}\PYG{p}{)} \PYG{k}{for} \PYG{n}{sp} \PYG{o+ow}{in} \PYG{n}{splist}\PYG{p}{]}             \PYG{c+c1}{\PYGZsh{} normalize the spectra}
\PYG{g+gp}{\PYGZgt{}\PYGZgt{}\PYGZgt{} }\PYG{n}{dates} \PYG{o}{=} \PYG{p}{[}\PYG{n}{sp}\PYG{o}{.}\PYG{n}{date} \PYG{k}{for} \PYG{n}{sp} \PYG{o+ow}{in} \PYG{n}{splist}\PYG{p}{]}                   \PYG{c+c1}{\PYGZsh{} observation dates}
\PYG{g+gp}{\PYGZgt{}\PYGZgt{}\PYGZgt{} }\PYG{n}{spsort} \PYG{o}{=} \PYG{p}{[}\PYG{n}{s} \PYG{k}{for} \PYG{p}{(}\PYG{n}{s}\PYG{p}{,}\PYG{n}{d}\PYG{p}{)} \PYG{o+ow}{in} \PYG{n+nb}{sorted}\PYG{p}{(}\PYG{n+nb}{zip}\PYG{p}{(}\PYG{n}{dates}\PYG{p}{,}\PYG{n}{splis}\PYG{p}{)}\PYG{p}{)}\PYG{p}{]}   \PYG{c+c1}{\PYGZsh{} sort spectra by dates}
\PYG{g+gp}{\PYGZgt{}\PYGZgt{}\PYGZgt{} }\PYG{n}{dates}\PYG{o}{.}\PYG{n}{sort}\PYG{p}{(}\PYG{p}{)}                                         \PYG{c+c1}{\PYGZsh{} don\PYGZsq{}t forget to sort dates!}
\PYG{g+gp}{\PYGZgt{}\PYGZgt{}\PYGZgt{} }\PYG{n}{splat}\PYG{o}{.}\PYG{n}{plotSpectrum}\PYG{p}{(}\PYG{n}{spsort}\PYG{p}{,}\PYG{n}{multiplot}\PYG{o}{=}\PYG{k+kc}{True}\PYG{p}{,}\PYG{n}{layout}\PYG{o}{=}\PYG{p}{[}\PYG{l+m+mi}{2}\PYG{p}{,}\PYG{l+m+mi}{2}\PYG{p}{]}\PYG{p}{,}\PYG{n}{multipage}\PYG{o}{=}\PYG{k+kc}{True}\PYG{p}{,}\PYGZbs{}   \PYG{c+c1}{\PYGZsh{} here\PYGZsq{}s our plot statement}
\PYG{g+go}{    comparison=spsort[0],uncertainty=True,mdwarf=True,telluric=True,legends=dates,           legendLocation=\PYGZsq{}lower left\PYGZsq{},output=\PYGZsq{}TWA30B.pdf\PYGZsq{})}
\end{Verbatim}

\item[{Example 3}] \leavevmode\begin{description}
\item[{Display the spectra sequence of L dwarfs}] \leavevmode
This example uses the list of standard files contained in SPLAT, and illustrates the stack feature

\end{description}

\begin{Verbatim}[commandchars=\\\{\}]
\PYG{g+gp}{\PYGZgt{}\PYGZgt{}\PYGZgt{} }\PYG{n}{spt} \PYG{o}{=} \PYG{p}{[}\PYG{n}{splat}\PYG{o}{.}\PYG{n}{typeToNum}\PYG{p}{(}\PYG{n}{i}\PYG{o}{+}\PYG{l+m+mi}{20}\PYG{p}{)} \PYG{k}{for} \PYG{n}{i} \PYG{o+ow}{in} \PYG{n+nb}{range}\PYG{p}{(}\PYG{l+m+mi}{10}\PYG{p}{)}\PYG{p}{]} \PYG{c+c1}{\PYGZsh{} generate list of L spectral types}
\PYG{g+gp}{\PYGZgt{}\PYGZgt{}\PYGZgt{} }\PYG{n}{splat}\PYG{o}{.}\PYG{n}{initiateStandards}\PYG{p}{(}\PYG{p}{)}                        \PYG{c+c1}{\PYGZsh{} initiate standards}
\PYG{g+gp}{\PYGZgt{}\PYGZgt{}\PYGZgt{} }\PYG{n}{splist} \PYG{o}{=} \PYG{p}{[}\PYG{n}{splat}\PYG{o}{.}\PYG{n}{SPEX\PYGZus{}STDS}\PYG{p}{[}\PYG{n}{s}\PYG{p}{]} \PYG{k}{for} \PYG{n}{s} \PYG{o+ow}{in} \PYG{n}{spt}\PYG{p}{]}       \PYG{c+c1}{\PYGZsh{} extact just L dwarfs}
\PYG{g+gp}{\PYGZgt{}\PYGZgt{}\PYGZgt{} }\PYG{n}{junk} \PYG{o}{=} \PYG{p}{[}\PYG{n}{sp}\PYG{o}{.}\PYG{n}{normalize}\PYG{p}{(}\PYG{p}{)} \PYG{k}{for} \PYG{n}{sp} \PYG{o+ow}{in} \PYG{n}{splist}\PYG{p}{]}         \PYG{c+c1}{\PYGZsh{} normalize the spectra}
\PYG{g+gp}{\PYGZgt{}\PYGZgt{}\PYGZgt{} }\PYG{n}{labels} \PYG{o}{=} \PYG{p}{[}\PYG{n}{sp}\PYG{o}{.}\PYG{n}{shortname} \PYG{k}{for} \PYG{n}{sp} \PYG{o+ow}{in} \PYG{n}{splist}\PYG{p}{]}         \PYG{c+c1}{\PYGZsh{} set labels to be names}
\PYG{g+gp}{\PYGZgt{}\PYGZgt{}\PYGZgt{} }\PYG{n}{splat}\PYG{o}{.}\PYG{n}{plotSpectrum}\PYG{p}{(}\PYG{n}{splist}\PYG{p}{,}\PYG{n}{figsize}\PYG{o}{=}\PYG{p}{[}\PYG{l+m+mi}{10}\PYG{p}{,}\PYG{l+m+mi}{20}\PYG{p}{]}\PYG{p}{,}\PYG{n}{labels}\PYG{o}{=}\PYG{n}{labels}\PYG{p}{,}\PYG{n}{stack}\PYG{o}{=}\PYG{l+m+mf}{0.5}\PYG{p}{,}\PYGZbs{}  \PYG{c+c1}{\PYGZsh{} here\PYGZsq{}s our plot statement}
\PYG{g+go}{    colorScheme=\PYGZsq{}copper\PYGZsq{},legendLocation=\PYGZsq{}outside\PYGZsq{},telluric=True,output=\PYGZsq{}lstandards.pdf\PYGZsq{})}
\end{Verbatim}

\end{description}\end{quote}

\end{fulllineitems}

\index{plotBatch() (in module splat\_plot)}

\begin{fulllineitems}
\phantomsection\label{api:splat_plot.plotBatch}\pysiglinewithargsret{\sphinxcode{splat\_plot.}\sphinxbfcode{plotBatch}}{\emph{*args}, \emph{**kwargs}}{}~\begin{quote}\begin{description}
\item[{Purpose}] \leavevmode
Plots a batch of spectra into a 2x2 set of PDF files, with options of overplotting comparison spectra, including best-match spectral standards.

\item[{Parameters}] \leavevmode\begin{itemize}
\item {} 
\textbf{\texttt{input}} (\emph{\texttt{required}}) -- A single or list of Spectrum objects or filenames, or the glob search string for a set of files (e.g., `/Data/myspectra/{\color{red}\bfseries{}*}.fits').

\item {} 
\textbf{\texttt{output}} (\emph{\texttt{optional, default = 'spectra\_plot.pdf'}}) -- Filename for PDF file output; full path should be include if not saving to current directory

\item {} 
\textbf{\texttt{comparisons}} (\emph{\texttt{optional, default = None}}) -- list of Spectrum objects or filenames for comparison spectra. If comparisons list is shorter than source list, then last comparison source will be repeated. If the comparisons list is longer, the list will be truncated.

\item {} 
\textbf{\texttt{classify}} (\emph{\texttt{optional, default = False}}) -- 
Set to True to classify sources based on comparison to MLT spectral standards following the method of \href{http://adsabs.harvard.edu/abs/2010ApJS..190..100K}{Kirkpatrick et al. (2010)}. This option normalizes the spectra by default


\item {} 
\textbf{\texttt{normalize}} (\emph{\texttt{optional, default = False}}) -- Set to True to normalize source and (if passed) comparison spectra.

\item {} 
\textbf{\texttt{legend}} (\emph{\texttt{optional, default = displays file name for source and object name for comparison (if passed)}}) -- Set to list of legends for plots. The number of legends should equal the number of sources and comparisons (if passed) in an alternating sequence. T

\end{itemize}

\end{description}\end{quote}

Relevant parameters for plotSpectrum may also be passed
\begin{quote}\begin{description}
\item[{Example}] \leavevmode
\begin{Verbatim}[commandchars=\\\{\}]
\PYG{g+gp}{\PYGZgt{}\PYGZgt{}\PYGZgt{} }\PYG{k+kn}{import} \PYG{n+nn}{glob}\PYG{o}{,} \PYG{n+nn}{splat}
\PYG{g+gp}{\PYGZgt{}\PYGZgt{}\PYGZgt{} }\PYG{n}{files} \PYG{o}{=} \PYG{n}{glob}\PYG{o}{.}\PYG{n}{glob}\PYG{p}{(}\PYG{l+s+s1}{\PYGZsq{}}\PYG{l+s+s1}{/home/mydata/*.fits}\PYG{l+s+s1}{\PYGZsq{}}\PYG{p}{)}
\PYG{g+gp}{\PYGZgt{}\PYGZgt{}\PYGZgt{} }\PYG{n}{sp} \PYG{o}{=} \PYG{n}{splat}\PYG{o}{.}\PYG{n}{plotBatch}\PYG{p}{(}\PYG{n}{files}\PYG{p}{,}\PYG{n}{classify}\PYG{o}{=}\PYG{k+kc}{True}\PYG{p}{,}\PYG{n}{output}\PYG{o}{=}\PYG{l+s+s1}{\PYGZsq{}}\PYG{l+s+s1}{comparison.pdf}\PYG{l+s+s1}{\PYGZsq{}}\PYG{p}{)}
\PYG{g+gp}{\PYGZgt{}\PYGZgt{}\PYGZgt{} }\PYG{n}{sp} \PYG{o}{=} \PYG{n}{splat}\PYG{o}{.}\PYG{n}{plotBatch}\PYG{p}{(}\PYG{l+s+s1}{\PYGZsq{}}\PYG{l+s+s1}{/home/mydata/*.fits}\PYG{l+s+s1}{\PYGZsq{}}\PYG{p}{,}\PYG{n}{classify}\PYG{o}{=}\PYG{k+kc}{True}\PYG{p}{,}\PYG{n}{output}\PYG{o}{=}\PYG{l+s+s1}{\PYGZsq{}}\PYG{l+s+s1}{comparison.pdf}\PYG{l+s+s1}{\PYGZsq{}}\PYG{p}{)}
\PYG{g+gp}{\PYGZgt{}\PYGZgt{}\PYGZgt{} }\PYG{n}{sp} \PYG{o}{=} \PYG{n}{splat}\PYG{o}{.}\PYG{n}{plotBatch}\PYG{p}{(}\PYG{p}{[}\PYG{n}{splat}\PYG{o}{.}\PYG{n}{Spectrum}\PYG{p}{(}\PYG{n}{file}\PYG{o}{=}\PYG{n}{f}\PYG{p}{)} \PYG{k}{for} \PYG{n}{f} \PYG{o+ow}{in} \PYG{n}{files}\PYG{p}{]}\PYG{p}{,}\PYG{n}{classify}\PYG{o}{=}\PYG{k+kc}{True}\PYG{p}{,}\PYG{n}{output}\PYG{o}{=}\PYG{l+s+s1}{\PYGZsq{}}\PYG{l+s+s1}{comparison.pdf}\PYG{l+s+s1}{\PYGZsq{}}\PYG{p}{)}
\PYG{g+go}{All three of these commands produce the same result}
\end{Verbatim}

\end{description}\end{quote}

\end{fulllineitems}

\index{plotSequence() (in module splat\_plot)}

\begin{fulllineitems}
\phantomsection\label{api:splat_plot.plotSequence}\pysiglinewithargsret{\sphinxcode{splat\_plot.}\sphinxbfcode{plotSequence}}{\emph{*args}, \emph{**kwargs}}{}~\begin{quote}\begin{description}
\item[{Purpose}] \leavevmode
Compares a spectrum to a sequence of standards a batch of spectra into a 2x2 set of PDF files, with options of overplotting comparison spectra, including best-match spectral standards.

\item[{Parameters}] \leavevmode\begin{itemize}
\item {} 
\textbf{\texttt{input}} (\emph{\texttt{required}}) -- A single or list of Spectrum objects or filenames, or the glob search string for a set of files (e.g., `/Data/myspectra/{\color{red}\bfseries{}*}.fits').

\item {} 
\textbf{\texttt{type\_range}} (\emph{\texttt{optional, default = 2}}) -- Number of subtypes to consider above and below best-fit spectral type

\item {} 
\textbf{\texttt{spt}} (\emph{\texttt{optional, default = None}}) -- Default spectral type for source; this input skips classifyByStandard

\item {} 
\textbf{\texttt{output}} (\emph{\texttt{optional, default = None (screen display)}}) -- Filename for output; full path should be include if not saving to current directory. If blank, plot is shown on screen

\end{itemize}

\end{description}\end{quote}

Relevant parameters for plotSpectrum may also be passed
\begin{quote}\begin{description}
\item[{Example}] \leavevmode
\begin{Verbatim}[commandchars=\\\{\}]
\PYG{g+gp}{\PYGZgt{}\PYGZgt{}\PYGZgt{} }\PYG{k+kn}{import} \PYG{n+nn}{glob}\PYG{o}{,} \PYG{n+nn}{splat}
\PYG{g+gp}{\PYGZgt{}\PYGZgt{}\PYGZgt{} }\PYG{n}{files} \PYG{o}{=} \PYG{n}{glob}\PYG{o}{.}\PYG{n}{glob}\PYG{p}{(}\PYG{l+s+s1}{\PYGZsq{}}\PYG{l+s+s1}{/home/mydata/*.fits}\PYG{l+s+s1}{\PYGZsq{}}\PYG{p}{)}
\PYG{g+gp}{\PYGZgt{}\PYGZgt{}\PYGZgt{} }\PYG{n}{sp} \PYG{o}{=} \PYG{n}{splat}\PYG{o}{.}\PYG{n}{plotBatch}\PYG{p}{(}\PYG{n}{files}\PYG{p}{,}\PYG{n}{classify}\PYG{o}{=}\PYG{k+kc}{True}\PYG{p}{,}\PYG{n}{output}\PYG{o}{=}\PYG{l+s+s1}{\PYGZsq{}}\PYG{l+s+s1}{comparison.pdf}\PYG{l+s+s1}{\PYGZsq{}}\PYG{p}{)}
\PYG{g+gp}{\PYGZgt{}\PYGZgt{}\PYGZgt{} }\PYG{n}{sp} \PYG{o}{=} \PYG{n}{splat}\PYG{o}{.}\PYG{n}{plotBatch}\PYG{p}{(}\PYG{l+s+s1}{\PYGZsq{}}\PYG{l+s+s1}{/home/mydata/*.fits}\PYG{l+s+s1}{\PYGZsq{}}\PYG{p}{,}\PYG{n}{classify}\PYG{o}{=}\PYG{k+kc}{True}\PYG{p}{,}\PYG{n}{output}\PYG{o}{=}\PYG{l+s+s1}{\PYGZsq{}}\PYG{l+s+s1}{comparison.pdf}\PYG{l+s+s1}{\PYGZsq{}}\PYG{p}{)}
\PYG{g+gp}{\PYGZgt{}\PYGZgt{}\PYGZgt{} }\PYG{n}{sp} \PYG{o}{=} \PYG{n}{splat}\PYG{o}{.}\PYG{n}{plotBatch}\PYG{p}{(}\PYG{p}{[}\PYG{n}{splat}\PYG{o}{.}\PYG{n}{Spectrum}\PYG{p}{(}\PYG{n}{file}\PYG{o}{=}\PYG{n}{f}\PYG{p}{)} \PYG{k}{for} \PYG{n}{f} \PYG{o+ow}{in} \PYG{n}{files}\PYG{p}{]}\PYG{p}{,}\PYG{n}{classify}\PYG{o}{=}\PYG{k+kc}{True}\PYG{p}{,}\PYG{n}{output}\PYG{o}{=}\PYG{l+s+s1}{\PYGZsq{}}\PYG{l+s+s1}{comparison.pdf}\PYG{l+s+s1}{\PYGZsq{}}\PYG{p}{)}
\PYG{g+go}{All three of these commands produce the same result}
\end{Verbatim}

\end{description}\end{quote}

\end{fulllineitems}

\index{plotSED() (in module splat\_plot)}

\begin{fulllineitems}
\phantomsection\label{api:splat_plot.plotSED}\pysiglinewithargsret{\sphinxcode{splat\_plot.}\sphinxbfcode{plotSED}}{\emph{*args}, \emph{**kwargs}}{}~\begin{quote}\begin{description}
\item[{Purpose}] \leavevmode
\sphinxcode{Plot SED photometry with SpeX spectrum.}

\end{description}\end{quote}

Not currently implemented

\end{fulllineitems}

\index{plotIndices() (in module splat\_plot)}

\begin{fulllineitems}
\phantomsection\label{api:splat_plot.plotIndices}\pysiglinewithargsret{\sphinxcode{splat\_plot.}\sphinxbfcode{plotIndices}}{\emph{*args}, \emph{**kwargs}}{}~\begin{quote}\begin{description}
\item[{Purpose}] \leavevmode
\sphinxcode{Plot index-index plots.}

\end{description}\end{quote}

Not currently implemented

\end{fulllineitems}



\subsubsection{Modeling Routines}
\label{api:modeling-routines}\index{getModel() (in module splat\_model)}

\begin{fulllineitems}
\phantomsection\label{api:splat_model.getModel}\pysiglinewithargsret{\sphinxcode{splat\_model.}\sphinxbfcode{getModel}}{\emph{*args}, \emph{**kwargs}}{}
Redundant routine with loadModel

\end{fulllineitems}

\index{loadModel() (in module splat\_model)}

\begin{fulllineitems}
\phantomsection\label{api:splat_model.loadModel}\pysiglinewithargsret{\sphinxcode{splat\_model.}\sphinxbfcode{loadModel}}{\emph{*args}, \emph{**kwargs}}{}~\begin{quote}\begin{description}
\item[{Purpose}] \leavevmode
Loads up a model spectrum based on a set of input parameters.

\end{description}\end{quote}

The models may be any one of the following listed below.
A Spectrum object with the wavelength and surface fluxes (F\_lambda in units of erg/cm\textasciicircum{}2/s/mu\{m\}) is returned
\begin{quote}\begin{description}
\item[{Parameters}] \leavevmode\begin{itemize}
\item {} 
\textbf{\texttt{model}} -- 
The model set to use; may be one of the following:
\begin{itemize}
\item {} 
\emph{`BTSettl2008'}: default model set from \href{http://adsabs.harvard.edu/abs/2012RSPTA.370.2765A}{Allard et al. (2012)}

\end{itemize}

with effective temperatures of 400 to 2900 K (steps of 100 K); surface gravities of 3.5 to 5.5 in units of cm/s\textasciicircum{}2 (steps of 0.5 dex); and metallicity of -3.0, -2.5, -2.0, -1.5, -1.0, -0.5, 0.0, 0.3, and 0.5 for temperatures greater than 2000 K only;
cloud opacity is fixed in this model, and equilibrium chemistry is assumed. Note that this grid is not completely filled and some gaps have been interpolated (alternate designations: `btsettled','btsettl','allard','allard12')
- \emph{`burrows06'}: model set from \href{http://adsabs.harvard.edu/abs/2006ApJ...640.1063B}{Burrows et al. (2006)}
with effective temperatures of 700 to 2000 K (steps of 50 K); surface gravities of 4.5 to 5.5 in units of cm/s\textasciicircum{}2 (steps of 0.1 dex); metallicity of -0.5, 0.0 and 0.5; and either no clouds or grain size 100 microns (fsed = `nc' or `f100').
equilibrium chemistry is assumed. Note that this grid is not completely filled and some gaps have been interpolated (alternate designations: `burrows','burrows2006')
- \emph{`morley12'}: model set from \href{http://adsabs.harvard.edu/abs/2012ApJ...756..172M}{Morley et al. (2012)}
with effective temperatures of 400 to 1300 K (steps of 50 K); surface gravities of 4.0 to 5.5 in units of cm/s\textasciicircum{}2 (steps of 0.5 dex); and sedimentation efficiency (fsed) of 2, 3, 4 or 5;
metallicity is fixed to solar, equilibrium chemistry is assumed, and there are no clouds associated with this model (alternate designations: `morley2012')
- \emph{`morley14'}: model set from \href{http://adsabs.harvard.edu/abs/2014ApJ...787...78M}{Morley et al. (2014)}
with effective temperatures of 200 to 450 K (steps of 25 K) and surface gravities of 3.0 to 5.0 in units of cm/s\textasciicircum{}2 (steps of 0.5 dex);
metallicity is fixed to solar, equilibrium chemistry is assumed, sedimentation efficiency is fixed at fsed = 5, and cloud coverage fixed at 50\% (alternate designations: `morley2014')
- \emph{`saumon12'}: model set from \href{http://adsabs.harvard.edu/abs/2012ApJ...750...74S}{Saumon et al. (2012)}
with effective temperatures of 400 to 1500 K (steps of 50 K); and surface gravities of 3.0 to 5.5 in units of cm/s\textasciicircum{}2 (steps of 0.5 dex);
metallicity is fixed to solar, equilibrium chemistry is assumed, and no clouds are associated with these models (alternate designations: `saumon','saumon2012')
- \emph{`drift'}: model set from \href{http://adsabs.harvard.edu/abs/2011A\%26A...529A..44W}{Witte et al. (2011)}
with effective temperatures of 1700 to 3000 K (steps of 50 K); surface gravities of 5.0 and 5.5 in units of cm/s\textasciicircum{}2; and metallicities of -3.0 to 0.0 (in steps of 0.5 dex);
cloud opacity is fixed in this model, equilibrium chemistry is assumed (alternate designations: `witte','witte2011','helling')


\item {} 
\textbf{\texttt{local}} (\emph{\texttt{optional, default = True}}) -- read in parameter file locally if True

\item {} 
\textbf{\texttt{online}} (\emph{\texttt{optional, default = False}}) -- read in parameter file online if True

\item {} 
\textbf{\texttt{folder}} (\emph{\texttt{optional, default = '{'}}}) -- string of the folder name containing the model set

\item {} 
\textbf{\texttt{filename}} (\emph{\texttt{optional, default = False}}) -- string of the filename of the desired model

\item {} 
\textbf{\texttt{force}} -- force the filename to be exactly as specified

\item {} 
\textbf{\texttt{url}} (optional, default = `\url{http://pono.ucsd.edu/~adam/splat/}`) -- string of the url to the SPLAT website

\end{itemize}

\item[{Model Parameters}] \leavevmode
The following parameters may be set:
\begin{itemize}
\item {} 
\emph{teff}: effective temperature of the model in K (e.g. teff = 1000)

\item {} 
\emph{logg}: log\_10 of the surface gravity of the model in cm/s\textasciicircum{}2 units (e.g. logg = 5.0)

\item {} 
\emph{z}: log\_10 of metallicity of the model relative to solar metallicity (e.g. z = -0.5)

\item {} 
\emph{fsed}: sedimentation efficiency of the model (e.g. fsed = `f2')

\item {} 
\emph{cld}: cloud shape function of the model (e.g. cld = `f50')

\item {} 
\emph{kzz}: vertical eddy diffusion coefficient of the model (e.g. kzz = 2)

\item {} 
\emph{slit}: slit weight of the model in arcseconds (e.g. slit = 0.3)

\item {} 
\emph{sed}: if set to True, returns a broad-band spectrum spanning 0.3-30 micron (applies only for BTSettl2008 models with Teff \textless{} 2000 K)

\end{itemize}

\item[{Example}] \leavevmode
\begin{Verbatim}[commandchars=\\\{\}]
\PYG{g+gp}{\PYGZgt{}\PYGZgt{}\PYGZgt{} }\PYG{k+kn}{import} \PYG{n+nn}{splat}
\PYG{g+gp}{\PYGZgt{}\PYGZgt{}\PYGZgt{} }\PYG{n}{mdl} \PYG{o}{=} \PYG{n}{splat}\PYG{o}{.}\PYG{n}{loadModel}\PYG{p}{(}\PYG{n}{teff}\PYG{o}{=}\PYG{l+m+mi}{1000}\PYG{p}{,}\PYG{n}{logg}\PYG{o}{=}\PYG{l+m+mf}{5.0}\PYG{p}{)}
\PYG{g+gp}{\PYGZgt{}\PYGZgt{}\PYGZgt{} }\PYG{n}{mdl}\PYG{o}{.}\PYG{n}{info}\PYG{p}{(}\PYG{p}{)}
\PYG{g+go}{     BTSettl2008 model with the following parmeters:}
\PYG{g+go}{     Teff = 1000 K}
\PYG{g+go}{     logg = 5.0 cm/s2}
\PYG{g+go}{     z = 0.0}
\PYG{g+go}{     fsed = nc}
\PYG{g+go}{     cld = nc}
\PYG{g+go}{     kzz = eq}
\PYG{g+go}{     Smoothed to slit width 0.5 arcseconds}
\PYG{g+gp}{\PYGZgt{}\PYGZgt{}\PYGZgt{} }\PYG{n}{mdl} \PYG{o}{=} \PYG{n}{splat}\PYG{o}{.}\PYG{n}{loadModel}\PYG{p}{(}\PYG{n}{teff}\PYG{o}{=}\PYG{l+m+mi}{2500}\PYG{p}{,}\PYG{n}{logg}\PYG{o}{=}\PYG{l+m+mf}{5.0}\PYG{p}{,}\PYG{n}{model}\PYG{o}{=}\PYG{l+s+s1}{\PYGZsq{}}\PYG{l+s+s1}{burrows}\PYG{l+s+s1}{\PYGZsq{}}\PYG{p}{)}
\PYG{g+go}{     Input value for teff = 2500 out of range for model set burrows06}
\PYG{g+go}{     Warning: Creating an empty Spectrum object}
\end{Verbatim}

\end{description}\end{quote}

\end{fulllineitems}



\subsubsection{Evolutionary Model Routines}
\label{api:evolutionary-model-routines}

\subsubsection{EUCLID Analysis Routines}
\label{api:euclid-analysis-routines}\index{spexToEuclid() (in module splat\_euclid)}

\begin{fulllineitems}
\phantomsection\label{api:splat_euclid.spexToEuclid}\pysiglinewithargsret{\sphinxcode{splat\_euclid.}\sphinxbfcode{spexToEuclid}}{\emph{sp}}{}~\begin{quote}\begin{description}
\item[{Purpose}] \leavevmode
Convert a SpeX file into EUCLID form, using the resolution and wavelength coverage
defined from the Euclid Red Book (\href{http://sci.esa.int/euclid/48983-euclid-definition-study-report-esa-sre-2011-12/}{Laurijs et al. 2011}). This function changes the input Spectrum
objects, which can be restored by the Spectrum.reset() method.

\item[{Parameters}] \leavevmode
\textbf{\texttt{sp}} -- Spectrum class object, which should contain wave, flux and noise array elements

\item[{Example}] \leavevmode
\begin{Verbatim}[commandchars=\\\{\}]
\PYG{g+gp}{\PYGZgt{}\PYGZgt{}\PYGZgt{} }\PYG{k+kn}{import} \PYG{n+nn}{splat}
\PYG{g+gp}{\PYGZgt{}\PYGZgt{}\PYGZgt{} }\PYG{n}{sp} \PYG{o}{=} \PYG{n}{splat}\PYG{o}{.}\PYG{n}{getSpectrum}\PYG{p}{(}\PYG{n}{lucky}\PYG{o}{=}\PYG{k+kc}{True}\PYG{p}{)}\PYG{p}{[}\PYG{l+m+mi}{0}\PYG{p}{]}                                \PYG{c+c1}{\PYGZsh{} grab a random file}
\PYG{g+gp}{\PYGZgt{}\PYGZgt{}\PYGZgt{} }\PYG{n}{splat}\PYG{o}{.}\PYG{n}{spexToEuclid}\PYG{p}{(}\PYG{n}{sp}\PYG{p}{)}
\PYG{g+gp}{\PYGZgt{}\PYGZgt{}\PYGZgt{} }\PYG{n+nb}{min}\PYG{p}{(}\PYG{n}{sp}\PYG{o}{.}\PYG{n}{wave}\PYG{p}{)}\PYG{p}{,} \PYG{n+nb}{max}\PYG{p}{(}\PYG{n}{sp}\PYG{o}{.}\PYG{n}{wave}\PYG{p}{)}
\PYG{g+go}{         (\PYGZlt{}Quantity 1.25 micron\PYGZgt{}, \PYGZlt{}Quantity 1.8493000000000364 micron\PYGZgt{})}
\PYG{g+gp}{\PYGZgt{}\PYGZgt{}\PYGZgt{} }\PYG{n}{sp}\PYG{o}{.}\PYG{n}{history}
\PYG{g+go}{         [{}`{}`\PYGZsq{}Spectrum successfully loaded{}`{}`\PYGZsq{},}
\PYG{g+go}{          {}`{}`\PYGZsq{}Converted to EUCLID format{}`{}`\PYGZsq{}]}
\PYG{g+gp}{\PYGZgt{}\PYGZgt{}\PYGZgt{} }\PYG{n}{sp}\PYG{o}{.}\PYG{n}{reset}\PYG{p}{(}\PYG{p}{)}
\PYG{g+gp}{\PYGZgt{}\PYGZgt{}\PYGZgt{} }\PYG{n+nb}{min}\PYG{p}{(}\PYG{n}{sp}\PYG{o}{.}\PYG{n}{wave}\PYG{p}{)}\PYG{p}{,} \PYG{n+nb}{max}\PYG{p}{(}\PYG{n}{sp}\PYG{o}{.}\PYG{n}{wave}\PYG{p}{)}
\PYG{g+go}{         (\PYGZlt{}Quantity 0.6454827785491943 micron\PYGZgt{}, \PYGZlt{}Quantity 2.555659770965576 micron\PYGZgt{})}
\end{Verbatim}

\end{description}\end{quote}

\end{fulllineitems}

\index{addEuclidNoise() (in module splat\_euclid)}

\begin{fulllineitems}
\phantomsection\label{api:splat_euclid.addEuclidNoise}\pysiglinewithargsret{\sphinxcode{splat\_euclid.}\sphinxbfcode{addEuclidNoise}}{\emph{sp}}{}~\begin{quote}\begin{description}
\item[{Purpose}] \leavevmode
Adds Gaussian noise to a EUCLID-formatted spectrum assuming a constant noise
model of 3e-15 erg/s/cm2/micron (as extrapolated from the Euclid Red Book;
Laurijs et al. 2011 \textless{}\url{http://sci.esa.int/euclid/48983-euclid-definition-study-report-esa-sre-2011-12/}\textgreater{}{}`\_).
Note that noise is added to both flux and (in quadrature) variance. This function creates a
new Spectrum object so as not to corrupt the original data.

\item[{Parameters}] \leavevmode
\textbf{\texttt{sp}} -- Spectrum class object, which should contain wave, flux and noise array elements

\item[{Output}] \leavevmode
Spectrum object with Euclid noise added in

\item[{Example}] \leavevmode
\begin{Verbatim}[commandchars=\\\{\}]
\PYG{g+gp}{\PYGZgt{}\PYGZgt{}\PYGZgt{} }\PYG{k+kn}{import} \PYG{n+nn}{splat}
\PYG{g+gp}{\PYGZgt{}\PYGZgt{}\PYGZgt{} }\PYG{n}{sp} \PYG{o}{=} \PYG{n}{splat}\PYG{o}{.}\PYG{n}{getSpectrum}\PYG{p}{(}\PYG{n}{lucky}\PYG{o}{=}\PYG{k+kc}{True}\PYG{p}{)}\PYG{p}{[}\PYG{l+m+mi}{0}\PYG{p}{]}                                \PYG{c+c1}{\PYGZsh{} grab a random file}
\PYG{g+gp}{\PYGZgt{}\PYGZgt{}\PYGZgt{} }\PYG{n}{splat}\PYG{o}{.}\PYG{n}{spexToEuclid}\PYG{p}{(}\PYG{n}{sp}\PYG{p}{)}
\PYG{g+gp}{\PYGZgt{}\PYGZgt{}\PYGZgt{} }\PYG{n}{sp}\PYG{o}{.}\PYG{n}{normalize}\PYG{p}{(}\PYG{p}{)}
\PYG{g+gp}{\PYGZgt{}\PYGZgt{}\PYGZgt{} }\PYG{n}{sp}\PYG{o}{.}\PYG{n}{scale}\PYG{p}{(}\PYG{l+m+mf}{1.e\PYGZhy{}14}\PYG{p}{)}
\PYG{g+gp}{\PYGZgt{}\PYGZgt{}\PYGZgt{} }\PYG{n}{sp}\PYG{o}{.}\PYG{n}{computeSN}\PYG{p}{(}\PYG{p}{)}
\PYG{g+go}{         115.96374031163553}
\PYG{g+gp}{\PYGZgt{}\PYGZgt{}\PYGZgt{} }\PYG{n}{sp\PYGZus{}noisy} \PYG{o}{=} \PYG{n}{splat}\PYG{o}{.}\PYG{n}{addEculidNoise}\PYG{p}{(}\PYG{n}{sp}\PYG{p}{)}
\PYG{g+gp}{\PYGZgt{}\PYGZgt{}\PYGZgt{} }\PYG{n}{sp\PYGZus{}noisy}\PYG{o}{.}\PYG{n}{computeSN}\PYG{p}{(}\PYG{p}{)}
\PYG{g+go}{         3.0847209519763172}
\end{Verbatim}

\end{description}\end{quote}

\end{fulllineitems}



\subsubsection{Other Calculation Routines}
\label{api:other-calculation-routines}\index{test() (in module splat)}

\begin{fulllineitems}
\phantomsection\label{api:splat.test}\pysiglinewithargsret{\sphinxcode{splat.}\sphinxbfcode{test}}{}{}~\begin{quote}\begin{description}
\item[{Purpose}] \leavevmode
Tests the SPLAT Code

\item[{Checks the following}] \leavevmode\begin{itemize}
\item {} 
If you are online and can see the SPLAT website

\item {} 
If you have access to unpublished spectra

\item {} 
If you can search for and load a spectrum

\item {} 
If \sphinxcode{searchLibrary} functions properly

\item {} 
If index measurement routines functions properly

\item {} 
If classification routines function properly

\item {} 
If \sphinxcode{typeToTeff} functions properly

\item {} 
If flux calibration and normalization function properly

\item {} 
If \sphinxcode{loadModel} functions properly

\item {} 
If \sphinxcode{compareSpectra} functions properly

\item {} 
If \sphinxcode{plotSpectrum} functions properly

\end{itemize}

\end{description}\end{quote}

\end{fulllineitems}

\index{distributionStats() (in module splat\_model)}

\begin{fulllineitems}
\phantomsection\label{api:splat_model.distributionStats}\pysiglinewithargsret{\sphinxcode{splat\_model.}\sphinxbfcode{distributionStats}}{\emph{x, q={[}0.16, 0.5, 0.84{]}, weights=None, sigma=None, **kwargs}}{}~\begin{quote}\begin{description}
\item[{Purpose}] \leavevmode
Find key values along distributions based on quantile steps.
This code is derived almost entirely from triangle.py.

\end{description}\end{quote}

\end{fulllineitems}



\subsubsection{BibTeX Routines}
\label{api:bibtex-routines}\index{getBibTex() (in module splat\_db)}

\begin{fulllineitems}
\phantomsection\label{api:splat_db.getBibTex}\pysiglinewithargsret{\sphinxcode{splat\_db.}\sphinxbfcode{getBibTex}}{\emph{bibcode}, \emph{**kwargs}}{}~\begin{description}
\item[{Purpose}] \leavevmode\begin{description}
\item[{Takes a bibcode and returns a dictionary containing the bibtex information; looks either in internal SPLAT}] \leavevmode
or user-supplied bibfile, or seeks online. If nothing found, gives a soft warning and returns False

\end{description}

\end{description}
\begin{quote}\begin{description}
\item[{Note}] \leavevmode
\textbf{Currently not functional}

\item[{Required parameters}] \leavevmode\begin{quote}\begin{description}
\item[{param bibcode}] \leavevmode
Bibcode string to look up (e.g., `2014ApJ...787..126L')

\end{description}\end{quote}

\item[{Optional parameters}] \leavevmode\begin{quote}\begin{description}
\item[{param biblibrary}] \leavevmode
Filename for biblibrary to use in place of SPLAT internal one

\item[{type string}] \leavevmode
optional, default = `'

\item[{param online}] \leavevmode
If True, go directly online; if False, do not try to go online

\item[{type logical}] \leavevmode
optional, default = null

\end{description}\end{quote}

\item[{Output}] \leavevmode\begin{itemize}
\item {} 
A dictionary containing the bibtex fields, or False if not found

\end{itemize}

\end{description}\end{quote}

\end{fulllineitems}



\subsubsection{I/O Routines}
\label{api:i-o-routines}\index{checkOnline() (in module splat\_db)}

\begin{fulllineitems}
\phantomsection\label{api:splat_db.checkOnline}\pysiglinewithargsret{\sphinxcode{splat\_db.}\sphinxbfcode{checkOnline}}{\emph{*args}}{}~\begin{quote}\begin{description}
\item[{Purpose}] \leavevmode
Checks if SPLAT's URL is accessible from your machine--
that is, checks if you and the host are online. Alternately
checks if a given filename is present locally or online

\item[{Example}] \leavevmode
\begin{Verbatim}[commandchars=\\\{\}]
\PYG{g+gp}{\PYGZgt{}\PYGZgt{}\PYGZgt{} }\PYG{k+kn}{import} \PYG{n+nn}{splat}
\PYG{g+gp}{\PYGZgt{}\PYGZgt{}\PYGZgt{} }\PYG{n}{splat}\PYG{o}{.}\PYG{n}{checkOnline}\PYG{p}{(}\PYG{p}{)}
\PYG{g+go}{True  \PYGZsh{} SPLAT\PYGZsq{}s URL was detected.}
\PYG{g+gp}{\PYGZgt{}\PYGZgt{}\PYGZgt{} }\PYG{n}{splat}\PYG{o}{.}\PYG{n}{checkOnline}\PYG{p}{(}\PYG{p}{)}
\PYG{g+go}{False \PYGZsh{} SPLAT\PYGZsq{}s URL was not detected.}
\PYG{g+gp}{\PYGZgt{}\PYGZgt{}\PYGZgt{} }\PYG{n}{splat}\PYG{o}{.}\PYG{n}{checkOnline}\PYG{p}{(}\PYG{l+s+s1}{\PYGZsq{}}\PYG{l+s+s1}{SpectralModels/BTSettl08/parameters.txt}\PYG{l+s+s1}{\PYGZsq{}}\PYG{p}{)}
\PYG{g+go}{\PYGZsq{}\PYGZsq{} \PYGZsh{} Could not find this online file.}
\end{Verbatim}

\end{description}\end{quote}

\end{fulllineitems}

\index{checkAccess() (in module splat\_db)}

\begin{fulllineitems}
\phantomsection\label{api:splat_db.checkAccess}\pysiglinewithargsret{\sphinxcode{splat\_db.}\sphinxbfcode{checkAccess}}{\emph{**kwargs}}{}~\begin{quote}\begin{description}
\item[{Purpose}] \leavevmode
Checks if user has access to unpublished spectra in SPLAT library.

\item[{Example}] \leavevmode
\begin{Verbatim}[commandchars=\\\{\}]
\PYG{g+gp}{\PYGZgt{}\PYGZgt{}\PYGZgt{} }\PYG{k+kn}{import} \PYG{n+nn}{splat}
\PYG{g+gp}{\PYGZgt{}\PYGZgt{}\PYGZgt{} }\PYG{n+nb}{print} \PYG{n}{splat}\PYG{o}{.}\PYG{n}{checkAccess}\PYG{p}{(}\PYG{p}{)}
\PYG{g+go}{True}
\end{Verbatim}

\item[{Note}] \leavevmode
Must have the file .splat\_access in your home directory with the correct passcode to use.

\end{description}\end{quote}

\end{fulllineitems}

\index{checkFile() (in module splat\_db)}

\begin{fulllineitems}
\phantomsection\label{api:splat_db.checkFile}\pysiglinewithargsret{\sphinxcode{splat\_db.}\sphinxbfcode{checkFile}}{\emph{filename}, \emph{**kwargs}}{}~\begin{quote}\begin{description}
\item[{Purpose}] \leavevmode
Checks if a spectrum file exists in the SPLAT's library.

\item[{Parameters}] \leavevmode
\textbf{\texttt{filename}} -- A string containing the spectrum's filename.

\item[{Example}] \leavevmode
\begin{Verbatim}[commandchars=\\\{\}]
\PYG{g+gp}{\PYGZgt{}\PYGZgt{}\PYGZgt{} }\PYG{k+kn}{import} \PYG{n+nn}{splat}
\PYG{g+gp}{\PYGZgt{}\PYGZgt{}\PYGZgt{} }\PYG{n}{spectrum1} \PYG{o}{=} \PYG{l+s+s1}{\PYGZsq{}}\PYG{l+s+s1}{spex\PYGZus{}prism\PYGZus{}1315+2334\PYGZus{}110404.fits}\PYG{l+s+s1}{\PYGZsq{}}
\PYG{g+gp}{\PYGZgt{}\PYGZgt{}\PYGZgt{} }\PYG{n+nb}{print} \PYG{n}{splat}\PYG{o}{.}\PYG{n}{checkFile}\PYG{p}{(}\PYG{n}{spectrum1}\PYG{p}{)}
\PYG{g+go}{True}
\PYG{g+gp}{\PYGZgt{}\PYGZgt{}\PYGZgt{} }\PYG{n}{spectrum2} \PYG{o}{=} \PYG{l+s+s1}{\PYGZsq{}}\PYG{l+s+s1}{fake\PYGZus{}name.fits}\PYG{l+s+s1}{\PYGZsq{}}
\PYG{g+gp}{\PYGZgt{}\PYGZgt{}\PYGZgt{} }\PYG{n+nb}{print} \PYG{n}{splat}\PYG{o}{.}\PYG{n}{checkFile}\PYG{p}{(}\PYG{n}{spectrum2}\PYG{p}{)}
\PYG{g+go}{False}
\end{Verbatim}

\end{description}\end{quote}

\end{fulllineitems}


\emph{Search}
\begin{itemize}
\item {} 
\DUrole{xref,std,std-ref}{genindex}

\item {} 
\DUrole{xref,std,std-ref}{modindex}

\item {} 
\DUrole{xref,std,std-ref}{search}

\end{itemize}

\emph{Search}
\begin{itemize}
\item {} 
\DUrole{xref,std,std-ref}{genindex}

\item {} 
\DUrole{xref,std,std-ref}{modindex}

\item {} 
\DUrole{xref,std,std-ref}{search}

\end{itemize}


\renewcommand{\indexname}{Python Module Index}
\begin{theindex}
\def\bigletter#1{{\Large\sffamily#1}\nopagebreak\vspace{1mm}}
\bigletter{b}
\item {\texttt{bdevopar}}, \pageref{bdevopar:module-bdevopar}
\indexspace
\bigletter{s}
\item {\texttt{splat\_model}}, \pageref{splat_model:module-splat_model}
\end{theindex}

\renewcommand{\indexname}{Index}
\printindex
\end{document}
